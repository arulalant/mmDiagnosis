% Generated by Sphinx.
\def\sphinxdocclass{report}
\documentclass[letterpaper,10pt,english]{sphinxmanual}
\usepackage[utf8]{inputenc}
\DeclareUnicodeCharacter{00A0}{\nobreakspace}
\usepackage{cmap}
\usepackage[T1]{fontenc}
\usepackage{babel}
\usepackage{times}
\usepackage[Bjarne]{fncychap}
\usepackage{longtable}
\usepackage{sphinx}
\usepackage{multirow}


\title{mmDiagnosis Documentation}
\date{April 13, 2015}
\release{1a}
\author{Arulalan.T, Dr.Krishna AchutaRao, Dilipkumar.R}
\newcommand{\sphinxlogo}{}
\renewcommand{\releasename}{Release}
\makeindex

\makeatletter
\def\PYG@reset{\let\PYG@it=\relax \let\PYG@bf=\relax%
    \let\PYG@ul=\relax \let\PYG@tc=\relax%
    \let\PYG@bc=\relax \let\PYG@ff=\relax}
\def\PYG@tok#1{\csname PYG@tok@#1\endcsname}
\def\PYG@toks#1+{\ifx\relax#1\empty\else%
    \PYG@tok{#1}\expandafter\PYG@toks\fi}
\def\PYG@do#1{\PYG@bc{\PYG@tc{\PYG@ul{%
    \PYG@it{\PYG@bf{\PYG@ff{#1}}}}}}}
\def\PYG#1#2{\PYG@reset\PYG@toks#1+\relax+\PYG@do{#2}}

\expandafter\def\csname PYG@tok@gd\endcsname{\def\PYG@tc##1{\textcolor[rgb]{0.63,0.00,0.00}{##1}}}
\expandafter\def\csname PYG@tok@gu\endcsname{\let\PYG@bf=\textbf\def\PYG@tc##1{\textcolor[rgb]{0.50,0.00,0.50}{##1}}}
\expandafter\def\csname PYG@tok@gt\endcsname{\def\PYG@tc##1{\textcolor[rgb]{0.00,0.27,0.87}{##1}}}
\expandafter\def\csname PYG@tok@gs\endcsname{\let\PYG@bf=\textbf}
\expandafter\def\csname PYG@tok@gr\endcsname{\def\PYG@tc##1{\textcolor[rgb]{1.00,0.00,0.00}{##1}}}
\expandafter\def\csname PYG@tok@cm\endcsname{\let\PYG@it=\textit\def\PYG@tc##1{\textcolor[rgb]{0.25,0.50,0.56}{##1}}}
\expandafter\def\csname PYG@tok@vg\endcsname{\def\PYG@tc##1{\textcolor[rgb]{0.73,0.38,0.84}{##1}}}
\expandafter\def\csname PYG@tok@m\endcsname{\def\PYG@tc##1{\textcolor[rgb]{0.13,0.50,0.31}{##1}}}
\expandafter\def\csname PYG@tok@mh\endcsname{\def\PYG@tc##1{\textcolor[rgb]{0.13,0.50,0.31}{##1}}}
\expandafter\def\csname PYG@tok@cs\endcsname{\def\PYG@tc##1{\textcolor[rgb]{0.25,0.50,0.56}{##1}}\def\PYG@bc##1{\setlength{\fboxsep}{0pt}\colorbox[rgb]{1.00,0.94,0.94}{\strut ##1}}}
\expandafter\def\csname PYG@tok@ge\endcsname{\let\PYG@it=\textit}
\expandafter\def\csname PYG@tok@vc\endcsname{\def\PYG@tc##1{\textcolor[rgb]{0.73,0.38,0.84}{##1}}}
\expandafter\def\csname PYG@tok@il\endcsname{\def\PYG@tc##1{\textcolor[rgb]{0.13,0.50,0.31}{##1}}}
\expandafter\def\csname PYG@tok@go\endcsname{\def\PYG@tc##1{\textcolor[rgb]{0.20,0.20,0.20}{##1}}}
\expandafter\def\csname PYG@tok@cp\endcsname{\def\PYG@tc##1{\textcolor[rgb]{0.00,0.44,0.13}{##1}}}
\expandafter\def\csname PYG@tok@gi\endcsname{\def\PYG@tc##1{\textcolor[rgb]{0.00,0.63,0.00}{##1}}}
\expandafter\def\csname PYG@tok@gh\endcsname{\let\PYG@bf=\textbf\def\PYG@tc##1{\textcolor[rgb]{0.00,0.00,0.50}{##1}}}
\expandafter\def\csname PYG@tok@ni\endcsname{\let\PYG@bf=\textbf\def\PYG@tc##1{\textcolor[rgb]{0.84,0.33,0.22}{##1}}}
\expandafter\def\csname PYG@tok@nl\endcsname{\let\PYG@bf=\textbf\def\PYG@tc##1{\textcolor[rgb]{0.00,0.13,0.44}{##1}}}
\expandafter\def\csname PYG@tok@nn\endcsname{\let\PYG@bf=\textbf\def\PYG@tc##1{\textcolor[rgb]{0.05,0.52,0.71}{##1}}}
\expandafter\def\csname PYG@tok@no\endcsname{\def\PYG@tc##1{\textcolor[rgb]{0.38,0.68,0.84}{##1}}}
\expandafter\def\csname PYG@tok@na\endcsname{\def\PYG@tc##1{\textcolor[rgb]{0.25,0.44,0.63}{##1}}}
\expandafter\def\csname PYG@tok@nb\endcsname{\def\PYG@tc##1{\textcolor[rgb]{0.00,0.44,0.13}{##1}}}
\expandafter\def\csname PYG@tok@nc\endcsname{\let\PYG@bf=\textbf\def\PYG@tc##1{\textcolor[rgb]{0.05,0.52,0.71}{##1}}}
\expandafter\def\csname PYG@tok@nd\endcsname{\let\PYG@bf=\textbf\def\PYG@tc##1{\textcolor[rgb]{0.33,0.33,0.33}{##1}}}
\expandafter\def\csname PYG@tok@ne\endcsname{\def\PYG@tc##1{\textcolor[rgb]{0.00,0.44,0.13}{##1}}}
\expandafter\def\csname PYG@tok@nf\endcsname{\def\PYG@tc##1{\textcolor[rgb]{0.02,0.16,0.49}{##1}}}
\expandafter\def\csname PYG@tok@si\endcsname{\let\PYG@it=\textit\def\PYG@tc##1{\textcolor[rgb]{0.44,0.63,0.82}{##1}}}
\expandafter\def\csname PYG@tok@s2\endcsname{\def\PYG@tc##1{\textcolor[rgb]{0.25,0.44,0.63}{##1}}}
\expandafter\def\csname PYG@tok@vi\endcsname{\def\PYG@tc##1{\textcolor[rgb]{0.73,0.38,0.84}{##1}}}
\expandafter\def\csname PYG@tok@nt\endcsname{\let\PYG@bf=\textbf\def\PYG@tc##1{\textcolor[rgb]{0.02,0.16,0.45}{##1}}}
\expandafter\def\csname PYG@tok@nv\endcsname{\def\PYG@tc##1{\textcolor[rgb]{0.73,0.38,0.84}{##1}}}
\expandafter\def\csname PYG@tok@s1\endcsname{\def\PYG@tc##1{\textcolor[rgb]{0.25,0.44,0.63}{##1}}}
\expandafter\def\csname PYG@tok@gp\endcsname{\let\PYG@bf=\textbf\def\PYG@tc##1{\textcolor[rgb]{0.78,0.36,0.04}{##1}}}
\expandafter\def\csname PYG@tok@sh\endcsname{\def\PYG@tc##1{\textcolor[rgb]{0.25,0.44,0.63}{##1}}}
\expandafter\def\csname PYG@tok@ow\endcsname{\let\PYG@bf=\textbf\def\PYG@tc##1{\textcolor[rgb]{0.00,0.44,0.13}{##1}}}
\expandafter\def\csname PYG@tok@sx\endcsname{\def\PYG@tc##1{\textcolor[rgb]{0.78,0.36,0.04}{##1}}}
\expandafter\def\csname PYG@tok@bp\endcsname{\def\PYG@tc##1{\textcolor[rgb]{0.00,0.44,0.13}{##1}}}
\expandafter\def\csname PYG@tok@c1\endcsname{\let\PYG@it=\textit\def\PYG@tc##1{\textcolor[rgb]{0.25,0.50,0.56}{##1}}}
\expandafter\def\csname PYG@tok@kc\endcsname{\let\PYG@bf=\textbf\def\PYG@tc##1{\textcolor[rgb]{0.00,0.44,0.13}{##1}}}
\expandafter\def\csname PYG@tok@c\endcsname{\let\PYG@it=\textit\def\PYG@tc##1{\textcolor[rgb]{0.25,0.50,0.56}{##1}}}
\expandafter\def\csname PYG@tok@mf\endcsname{\def\PYG@tc##1{\textcolor[rgb]{0.13,0.50,0.31}{##1}}}
\expandafter\def\csname PYG@tok@err\endcsname{\def\PYG@bc##1{\setlength{\fboxsep}{0pt}\fcolorbox[rgb]{1.00,0.00,0.00}{1,1,1}{\strut ##1}}}
\expandafter\def\csname PYG@tok@kd\endcsname{\let\PYG@bf=\textbf\def\PYG@tc##1{\textcolor[rgb]{0.00,0.44,0.13}{##1}}}
\expandafter\def\csname PYG@tok@ss\endcsname{\def\PYG@tc##1{\textcolor[rgb]{0.32,0.47,0.09}{##1}}}
\expandafter\def\csname PYG@tok@sr\endcsname{\def\PYG@tc##1{\textcolor[rgb]{0.14,0.33,0.53}{##1}}}
\expandafter\def\csname PYG@tok@mo\endcsname{\def\PYG@tc##1{\textcolor[rgb]{0.13,0.50,0.31}{##1}}}
\expandafter\def\csname PYG@tok@mi\endcsname{\def\PYG@tc##1{\textcolor[rgb]{0.13,0.50,0.31}{##1}}}
\expandafter\def\csname PYG@tok@kn\endcsname{\let\PYG@bf=\textbf\def\PYG@tc##1{\textcolor[rgb]{0.00,0.44,0.13}{##1}}}
\expandafter\def\csname PYG@tok@o\endcsname{\def\PYG@tc##1{\textcolor[rgb]{0.40,0.40,0.40}{##1}}}
\expandafter\def\csname PYG@tok@kr\endcsname{\let\PYG@bf=\textbf\def\PYG@tc##1{\textcolor[rgb]{0.00,0.44,0.13}{##1}}}
\expandafter\def\csname PYG@tok@s\endcsname{\def\PYG@tc##1{\textcolor[rgb]{0.25,0.44,0.63}{##1}}}
\expandafter\def\csname PYG@tok@kp\endcsname{\def\PYG@tc##1{\textcolor[rgb]{0.00,0.44,0.13}{##1}}}
\expandafter\def\csname PYG@tok@w\endcsname{\def\PYG@tc##1{\textcolor[rgb]{0.73,0.73,0.73}{##1}}}
\expandafter\def\csname PYG@tok@kt\endcsname{\def\PYG@tc##1{\textcolor[rgb]{0.56,0.13,0.00}{##1}}}
\expandafter\def\csname PYG@tok@sc\endcsname{\def\PYG@tc##1{\textcolor[rgb]{0.25,0.44,0.63}{##1}}}
\expandafter\def\csname PYG@tok@sb\endcsname{\def\PYG@tc##1{\textcolor[rgb]{0.25,0.44,0.63}{##1}}}
\expandafter\def\csname PYG@tok@k\endcsname{\let\PYG@bf=\textbf\def\PYG@tc##1{\textcolor[rgb]{0.00,0.44,0.13}{##1}}}
\expandafter\def\csname PYG@tok@se\endcsname{\let\PYG@bf=\textbf\def\PYG@tc##1{\textcolor[rgb]{0.25,0.44,0.63}{##1}}}
\expandafter\def\csname PYG@tok@sd\endcsname{\let\PYG@it=\textit\def\PYG@tc##1{\textcolor[rgb]{0.25,0.44,0.63}{##1}}}

\def\PYGZbs{\char`\\}
\def\PYGZus{\char`\_}
\def\PYGZob{\char`\{}
\def\PYGZcb{\char`\}}
\def\PYGZca{\char`\^}
\def\PYGZam{\char`\&}
\def\PYGZlt{\char`\<}
\def\PYGZgt{\char`\>}
\def\PYGZsh{\char`\#}
\def\PYGZpc{\char`\%}
\def\PYGZdl{\char`\$}
\def\PYGZhy{\char`\-}
\def\PYGZsq{\char`\'}
\def\PYGZdq{\char`\"}
\def\PYGZti{\char`\~}
% for compatibility with earlier versions
\def\PYGZat{@}
\def\PYGZlb{[}
\def\PYGZrb{]}
\makeatother

\begin{document}

\maketitle
\tableofcontents
\phantomsection\label{index::doc}


Contents:


\chapter{Getting started}
\label{getting_started:getting-started}\label{getting_started::doc}\label{getting_started:id1}\label{getting_started:welcome-to-mmdiagnosis-s-documentation}

\section{Installing uv-cdat in system}
\label{getting_started:installing-uv-cdat-in-system}
To install uv-cdat in your system, please follow the following links.

\href{http://uv-cdat.llnl.gov/wiki/CDATCMakeBuild}{http://uv-cdat.llnl.gov/wiki/CDATCMakeBuild}

\href{http://uv-cdat.llnl.gov/wiki/CDATBuild}{http://uv-cdat.llnl.gov/wiki/CDATBuild}


\section{Installing the Diagnosisutils}
\label{getting_started:installing-the-diagnosisutils}
Have to write it


\section{Installing the Diagnosis}
\label{getting_started:installing-the-diagnosis}
Have to write it


\section{Setup and Configuration}
\label{getting_started:setup-and-configuration}
Have to write it


\chapter{Documentation of \textbf{diagnosisutils} source code}
\label{diagnosisutils:documentation-of-diagnosisutils-source-code}\label{diagnosisutils:diagnosisutils}\label{diagnosisutils::doc}
The diagnosisutils package contains the {\hyperref[diagnosisutils:data-access-utils]{Data Access Utils}}, {\hyperref[diagnosisutils:time-axis-utils]{Time Axis Utils}}, and {\hyperref[diagnosisutils:plot-utils]{Plot Utils}} modules.


\section{Data Access Utils}
\label{diagnosisutils:data-access-utils}
This data access utils uses the {\hyperref[diagnosisutils:time-axis-utils]{Time Axis Utils}} and {\color{red}\bfseries{}{}`Days Utils{}`\_} .

The {\hyperref[diagnosisutils:xml-data-access]{xml\_data\_access}} module help us to access the all the grib files through single object.

Basically all the grib files are pointed into xml dom by cdscan command.

Then we can the xml files through uv-cdat.

Right now we are generating 8 xml files to access analysis grib files and 7 forecasts grib files.

In the {\hyperref[diagnosisutils:xml-data-access]{xml\_data\_access}} module, we are finding which xml needs to access depends upon the user inputs (Type, hour) in the functions of this module.

And once one xml object has initiated then through out that program execute session, it will remains exists and use when ever user needs it.


\subsection{xml\_data\_access}
\label{diagnosisutils:xml-data-access}\label{diagnosisutils:id1}\phantomsection\label{diagnosisutils:module-xml_data_access}\index{xml\_data\_access (module)}\phantomsection\label{diagnosisutils:module-xml_data_access}\index{xml\_data\_access (module)}\index{GribXmlAccess (class in xml\_data\_access)}

\begin{fulllineitems}
\phantomsection\label{diagnosisutils:xml_data_access.GribXmlAccess}\pysiglinewithargsret{\strong{class }\code{xml\_data\_access.}\bfcode{GribXmlAccess}}{\emph{XmlDir}}{}
xml access methods
\index{closeXmlObjs() (xml\_data\_access.GribXmlAccess method)}

\begin{fulllineitems}
\phantomsection\label{diagnosisutils:xml_data_access.GribXmlAccess.closeXmlObjs}\pysiglinewithargsret{\bfcode{closeXmlObjs}}{}{}
{\hyperref[diagnosisutils:xml_data_access.GribXmlAccess.closeXmlObjs]{\code{closeXmlObjs()}}}: close all the opened xml file objects by
cdms2. If we called this method, it will check all the 8 xml objects
are either opened or not. If that is opened by cdms2 means, it will
close that file object properly. We must call this method for the
safety purpose.
\begin{description}
\item[{..note:: If we called this method, at the end of the program then}] \leavevmode
it should be optimized one. If this method called at any inter
mediate level means, then again it need to create the xml object.

\end{description}

Written By: Arulalan.T

Date : 10.09.2011

\end{fulllineitems}

\index{findPartners() (xml\_data\_access.GribXmlAccess method)}

\begin{fulllineitems}
\phantomsection\label{diagnosisutils:xml_data_access.GribXmlAccess.findPartners}\pysiglinewithargsret{\bfcode{findPartners}}{\emph{Type}, \emph{date}, \emph{hour=None}, \emph{returnType='c'}}{}
{\hyperref[diagnosisutils:xml_data_access.GribXmlAccess.findPartners]{\code{findPartners()}}}: To find the partners of the any particular
day anl or any particular day and hour of the fcst.
Each fcst file(day) has its truth anl file(day).
i.e. today 24 hour fcst file's partner is tomorrow's truth anl file.
today 48 hour fcst file's partner is the day after tomorrow's
truth anl file. Keep going on the fcst vc anl files.

Same concept for anl files partner but in reverse concept.
Today's truth anl file's partners are yesterdays' 24 hour fcst file,
day before yesterday's 48 hour fcst file and keep going backward ...

This what we are calling as the partners of anl and fcst files.
For present fcst hours partner is future anl file and for present anl
partners are the past fcst hours files.

Condition :
\begin{quote}

if `f' as passed then hour is mandatory one
else `a' as passed then hour is optional one.
returnType either `c' or `s'
\end{quote}

Inputs :
\begin{quote}

Type = `f' or `a' or `o' i.e fcst or anl or obs file
date must be cdtime.comptime object or its string formate
hour is like 24 multiples in case availability of the fcst files
\end{quote}

Outputs :
\begin{quote}
\begin{quote}

If `f' has passed this method returns a corresponding partner
of the anlysis date in cdtime.comptime object
If `a' or `o' has passed this method returns a dictionary.
It contains the availability of the fcst hours as key and its
corresponding fcst date in cdtime.comptime object as value of
the dict.

we can get the return date as yyyymmdd string formate by
passing returnType = `s'
\end{quote}

Usage :
\begin{quote}
\begin{description}
\item[{example 1 :}] \leavevmode
\begin{Verbatim}[commandchars=\\\{\}]
\PYG{g+gp}{\PYGZgt{}\PYGZgt{}\PYGZgt{} }\PYG{n}{findPartners}\PYG{p}{(}\PYG{l+s}{\PYGZsq{}}\PYG{l+s}{f}\PYG{l+s}{\PYGZsq{}}\PYG{p}{,}\PYG{l+s}{\PYGZsq{}}\PYG{l+s}{2010\PYGZhy{}5\PYGZhy{}25}\PYG{l+s}{\PYGZsq{}}\PYG{p}{,}\PYG{l+m+mi}{24}\PYG{p}{)}
\PYG{g+go}{    2010\PYGZhy{}5\PYGZhy{}26 0:0:0.0}
\end{Verbatim}
\begin{quote}

\begin{notice}{note}{Note:}
The passed date in comptime in string type.
\end{notice}
\end{quote}

\begin{Verbatim}[commandchars=\\\{\}]
\PYG{g+gp}{\PYGZgt{}\PYGZgt{}\PYGZgt{} }\PYG{n}{findPartners}\PYG{p}{(}\PYG{l+s}{\PYGZsq{}}\PYG{l+s}{f}\PYG{l+s}{\PYGZsq{}}\PYG{p}{,}\PYG{n}{cdtime}\PYG{o}{.}\PYG{n}{comptime}\PYG{p}{(}\PYG{l+m+mi}{2010}\PYG{p}{,}\PYG{l+m+mi}{5}\PYG{p}{,}\PYG{l+m+mi}{25}\PYG{p}{)}\PYG{p}{,}\PYG{l+m+mi}{24}\PYG{p}{)}
\PYG{g+go}{    2010\PYGZhy{}5\PYGZhy{}26 0:0:0.0}
\end{Verbatim}
\begin{quote}

\begin{notice}{note}{Note:}
The passed date in comptime object itself.
\end{notice}
\end{quote}

\begin{Verbatim}[commandchars=\\\{\}]
\PYG{g+gp}{\PYGZgt{}\PYGZgt{}\PYGZgt{} }\PYG{n}{findPartners}\PYG{p}{(}\PYG{l+s}{\PYGZsq{}}\PYG{l+s}{a}\PYG{l+s}{\PYGZsq{}}\PYG{p}{,}\PYG{l+s}{\PYGZsq{}}\PYG{l+s}{2010\PYGZhy{}5\PYGZhy{}26}\PYG{l+s}{\PYGZsq{}}\PYG{p}{)}
\PYG{g+go}{    \PYGZob{}24: 2010\PYGZhy{}5\PYGZhy{}25 0:0:0.0\PYGZcb{}}
\end{Verbatim}
\begin{quote}

\begin{notice}{note}{Note:}
Returns dictionary which contains key as hour and its
corresponding date
\end{notice}
\end{quote}

\item[{example 2 :}] \leavevmode
\begin{Verbatim}[commandchars=\\\{\}]
\PYG{g+gp}{\PYGZgt{}\PYGZgt{}\PYGZgt{} }\PYG{n}{findPartners}\PYG{p}{(}\PYG{l+s}{\PYGZsq{}}\PYG{l+s}{f}\PYG{l+s}{\PYGZsq{}}\PYG{p}{,}\PYG{l+s}{\PYGZsq{}}\PYG{l+s}{2010\PYGZhy{}5\PYGZhy{}25}\PYG{l+s}{\PYGZsq{}}\PYG{p}{,}\PYG{l+m+mi}{72}\PYG{p}{)}
\PYG{g+go}{    2010\PYGZhy{}5\PYGZhy{}28 0:0:0.0}
\end{Verbatim}

\begin{Verbatim}[commandchars=\\\{\}]
\PYG{g+gp}{\PYGZgt{}\PYGZgt{}\PYGZgt{} }\PYG{n}{findPartners}\PYG{p}{(}\PYG{l+s}{\PYGZsq{}}\PYG{l+s}{a}\PYG{l+s}{\PYGZsq{}}\PYG{p}{,}\PYG{l+s}{\PYGZsq{}}\PYG{l+s}{2010\PYGZhy{}6\PYGZhy{}1}\PYG{l+s}{\PYGZsq{}}\PYG{p}{,} \PYG{n}{returnType} \PYG{o}{=}\PYG{l+s}{\PYGZsq{}}\PYG{l+s}{s}\PYG{l+s}{\PYGZsq{}}\PYG{p}{)}
\PYG{g+go}{     \PYGZob{}24: \PYGZsq{}20100531\PYGZsq{},}
\PYG{g+go}{      48: \PYGZsq{}20100530\PYGZsq{},}
\PYG{g+go}{      72: \PYGZsq{}20100529\PYGZsq{},}
\PYG{g+go}{      96: \PYGZsq{}20100528\PYGZsq{},}
\PYG{g+go}{     120: \PYGZsq{}20100527\PYGZsq{},}
\PYG{g+go}{     144: \PYGZsq{}20100526\PYGZsq{},}
\PYG{g+go}{     168: \PYGZsq{}20100525\PYGZsq{}\PYGZcb{}}
\end{Verbatim}

\begin{notice}{note}{Note:}
Depends upon the availability of the fcst and anl
files, it should return partner date
\end{notice}

\item[{example 3 :}] \leavevmode
\begin{Verbatim}[commandchars=\\\{\}]
\PYG{g+gp}{\PYGZgt{}\PYGZgt{}\PYGZgt{} }\PYG{n}{findPartners}\PYG{p}{(}\PYG{l+s}{\PYGZsq{}}\PYG{l+s}{a}\PYG{l+s}{\PYGZsq{}}\PYG{p}{,}\PYG{l+s}{\PYGZsq{}}\PYG{l+s}{20100601}\PYG{l+s}{\PYGZsq{}}\PYG{p}{,}\PYG{l+m+mi}{144}\PYG{p}{)}
\PYG{g+go}{    2010\PYGZhy{}5\PYGZhy{}26 0:0:0.0}
\end{Verbatim}


\strong{See also:}


If not available for the passed hour means
it should return None



\end{description}
\end{quote}
\end{quote}

Written by : Arulalan.T

Date : 03.04.2011

\end{fulllineitems}

\index{getData() (xml\_data\_access.GribXmlAccess method)}

\begin{fulllineitems}
\phantomsection\label{diagnosisutils:xml_data_access.GribXmlAccess.getData}\pysiglinewithargsret{\bfcode{getData}}{\emph{var}, \emph{Type}, \emph{date}, \emph{hour=None}, \emph{level='all'}, \emph{**latlonregion}}{}~\begin{description}
\item[{{\hyperref[diagnosisutils:xml_data_access.GribXmlAccess.getData]{\code{getData()}}}: It can extract either the data of a single date or}] \leavevmode
range of dates. It depends up on the input of the date argument.
Finally it should return MV2 variable.

\item[{Condition :}] \leavevmode
date is either tuple or string.
level is optional. level takes default `all'.
if level passed, it must be belongs to the data variable
hour is must when Type arg should be `f' (fcst) to choose
xml object.
Pass either (lat,lon) or region.

\item[{Inputs :}] \leavevmode\begin{quote}

Type - either `a'{[}analysis{]} or `f'{[}forecast{]} or `o'{[}observation{]}
var,level must be belongs to the data file
key word arg lat,lon or region should be passed
date formate must one of the followings
\begin{description}
\item[{date formate 1:}] \leavevmode
date = (startdate, enddate)
here startdate and enddate must be like cdtime.comptime formate.

\item[{date formate 2:}] \leavevmode
date = (startdate)

\item[{date formate 3:}] \leavevmode
date = `startdate' or date = `date'

\item[{eg for date input :}] \leavevmode
date = (`2010-5-1',`2010-6-30')
date = (`2010-5-30')
date = `2010-5-30'

\end{description}
\end{quote}

Outputs :
\begin{quote}

If user passed single date in the date argument, then it should
return the data of that particular date as single MV2 variable.

If user passed start and enddate in the date argument,
then it should return the data for the range of dates as
single MV2 variable with time axis.
\end{quote}

\end{description}

Written by: Arulalan.T

Date: 10.05.2011

\end{fulllineitems}

\index{getDataPartners() (xml\_data\_access.GribXmlAccess method)}

\begin{fulllineitems}
\phantomsection\label{diagnosisutils:xml_data_access.GribXmlAccess.getDataPartners}\pysiglinewithargsret{\bfcode{getDataPartners}}{\emph{var}, \emph{Type}, \emph{date}, \emph{hour=None}, \emph{level='all'}, \emph{orginData=0}, \emph{datePriority='o'}, \emph{**latlonregion}}{}
{\hyperref[diagnosisutils:xml_data_access.GribXmlAccess.getDataPartners]{\code{getDataPartners()}}}: It can extract either the orginDate with its
partnersData or it can extract only the partnersData without its
orginData for a single date or range of dates.

It depends up on the input of the orginData, datePriority,date args.
Finally it should return partnersData and/or orginData as MV2 variable

Condition :
\begin{quote}

date is either tuple or string.
level is optional. level takes default `all'.
if level passed, it must be belongs to the data variable
hour is must when Type arg should be `f' (fcst) to select xml object.
hour is must when range of date passed, even thogut Type arg should
be `a'(anl) or `o'(obs),to choose one fcst xml object along with hour.
Pass either (lat,lon) or regionself.
\end{quote}

Inputs:
\begin{quote}

Type - either `a'{[}analysis{]} or `f'{[}forecast{]} or `o'{[}observation{]}
var,level must be belongs to the data file
orginData - either 0 or 1. 0 means it shouldnot return the
\begin{quote}
\begin{description}
\item[{orginData as single MV2 var.}] \leavevmode
1 means it should return both the orginData and its
partnersData as two seperate MV2 vars.

\end{description}
\end{quote}
\begin{description}
\item[{datePriority - either `o' or `p'. `o' means passed date is with}] \leavevmode\begin{quote}

respect to orginData. According to this
orginData's date, it should return its partnersData.
\end{quote}
\begin{description}
\item[{`p' means passed date is with respect to partnersData.}] \leavevmode
According to this partnersData's date, it should
return its orginData.

\end{description}

\end{description}

key word arg lat,lon or region should be passed
\begin{description}
\item[{date formate 1:}] \leavevmode
date = (startdate,enddate)
here startdate and enddate must be like cdtime.comptime formate.

\item[{date formate 2:}] \leavevmode
date = (startdate)

\item[{date formate 3:}] \leavevmode
date = `startdate' or date = `date'

\item[{eg for date input :}] \leavevmode\begin{description}
\item[{date = (`2010-5-1',`2010-6-30')}] \leavevmode
date = (`2010-5-30')
date = `2010-5-30'

\end{description}

\end{description}
\end{quote}

Outputs:
\begin{quote}

If user passed single date in the date argument, then it should
return the data of that particular date
(both orginData \& partnersData) as a single MV2 variable.

If user passed start and enddate in the date argument, then it
should return the data (both orginData \& partnersData)
for the range of dates as a single MV2 variable with time axis.
\end{quote}

Usage:
\begin{quote}

\begin{notice}{note}{Note:}
if `a'(anl) file is orginData means `f'(fcst) files are
its partnersData and vice versa.
\end{notice}
\begin{description}
\item[{example1:}] \leavevmode
\begin{Verbatim}[commandchars=\\\{\}]
\PYG{g+gp}{\PYGZgt{}\PYGZgt{}\PYGZgt{} }\PYG{n}{a}\PYG{p}{,}\PYG{n}{b} \PYG{o}{=} \PYG{n}{getDataPartners}\PYG{p}{(}\PYG{n}{var} \PYG{o}{=} \PYG{l+s}{\PYGZsq{}}\PYG{l+s}{U component of wind}\PYG{l+s}{\PYGZsq{}}\PYG{p}{,}\PYG{n}{Type} \PYG{o}{=} \PYG{l+s}{\PYGZsq{}}\PYG{l+s}{a}\PYG{l+s}{\PYGZsq{}}\PYG{p}{,}
\PYG{g+go}{       date = \PYGZsq{}2010\PYGZhy{}6\PYGZhy{}5\PYGZsq{},hour = None,level = \PYGZsq{}all\PYGZsq{},orginData = 1,}
\PYG{g+go}{       datePriority = \PYGZsq{}o\PYGZsq{}, lat=(\PYGZhy{}90,90),lon=(0,359.5))}
\end{Verbatim}

a is orginData. i.e. anl. its timeAxis date is `2010-6-5'.
b is partnersData. i.e. fcst. its 24 hour fcst date w.r.t
orginData is `2010-6-4'. 48 hour is `2010-6-3'.

Depends upon the availability of date of fcst files,it should
return the data.
In NCMRWF2010 model, it should return maximum of 7 days fcst.

If we will specify any hour in the same eg, that should return
only that hour fcst file data instead of returning all the
available fcst hours data.

\item[{example2:}] \leavevmode
\begin{Verbatim}[commandchars=\\\{\}]
\PYG{g+gp}{\PYGZgt{}\PYGZgt{}\PYGZgt{} }\PYG{n}{a}\PYG{p}{,}\PYG{n}{b} \PYG{o}{=} \PYG{n}{getDataPartners}\PYG{p}{(}\PYG{n}{var} \PYG{o}{=} \PYG{l+s}{\PYGZsq{}}\PYG{l+s}{U component of wind}\PYG{l+s}{\PYGZsq{}}\PYG{p}{,}\PYG{n}{Type} \PYG{o}{=} \PYG{l+s}{\PYGZsq{}}\PYG{l+s}{f}\PYG{l+s}{\PYGZsq{}}\PYG{p}{,}
\PYG{g+go}{         date = \PYGZsq{}2010\PYGZhy{}6\PYGZhy{}5\PYGZsq{},hour = 24,level = \PYGZsq{}all\PYGZsq{},orginData = 1,}
\PYG{g+go}{         datePriority = \PYGZsq{}o\PYGZsq{}, lat=(\PYGZhy{}90,90),lon=(0,359.5))}
\end{Verbatim}

a is orginData. i.e.fcst 24 hour.its timeAxis date is `2010-6-5'.
b is partnersData. i.e. anl. its anl date w.r.t orginData is
`2010-6-6'.

\item[{example3:}] \leavevmode
\begin{Verbatim}[commandchars=\\\{\}]
\PYG{g+gp}{\PYGZgt{}\PYGZgt{}\PYGZgt{} }\PYG{n}{b} \PYG{o}{=} \PYG{n}{getDataPartners}\PYG{p}{(}\PYG{n}{var} \PYG{o}{=} \PYG{l+s}{\PYGZsq{}}\PYG{l+s}{U component of wind}\PYG{l+s}{\PYGZsq{}}\PYG{p}{,}\PYG{n}{Type} \PYG{o}{=} \PYG{l+s}{\PYGZsq{}}\PYG{l+s}{f}\PYG{l+s}{\PYGZsq{}}\PYG{p}{,}
\PYG{g+go}{        date = \PYGZsq{}2010\PYGZhy{}6\PYGZhy{}5\PYGZsq{},hour = 24,level = \PYGZsq{}all\PYGZsq{}, orginData = 0,}
\PYG{g+go}{        datePriority = \PYGZsq{}o\PYGZsq{}, lat=(\PYGZhy{}90,90),lon=(0,359.5))}
\end{Verbatim}

b is partnersData. i.e. anl. its anl date w.r.t orginData is
`2010-6-6'. No orginData. Because we passed orginData as 0.

\item[{example4:}] \leavevmode
\begin{Verbatim}[commandchars=\\\{\}]
\PYG{g+gp}{\PYGZgt{}\PYGZgt{}\PYGZgt{} }\PYG{n}{a}\PYG{p}{,}\PYG{n}{b} \PYG{o}{=} \PYG{n}{getDataPartners}\PYG{p}{(}\PYG{n}{var} \PYG{o}{=} \PYG{l+s}{\PYGZsq{}}\PYG{l+s}{U component of wind}\PYG{l+s}{\PYGZsq{}}\PYG{p}{,}\PYG{n}{Type} \PYG{o}{=} \PYG{l+s}{\PYGZsq{}}\PYG{l+s}{f}\PYG{l+s}{\PYGZsq{}}\PYG{p}{,}
\PYG{g+go}{         date = \PYGZsq{}2010\PYGZhy{}6\PYGZhy{}5\PYGZsq{},hour = 24,level = \PYGZsq{}all\PYGZsq{},orginData = 1,}
\PYG{g+go}{         datePriority = \PYGZsq{}p\PYGZsq{}, lat=(\PYGZhy{}90,90),lon=(0,359.5))}
\end{Verbatim}

a is orginData. i.e. fcst 24 hour.its timeAxis date is `2010-6-6'.
b is partnersData. i.e. anl. its anl date w.r.t orginData is
`2010-6-5'.
we can compare this eg4 with eg2. In this we passed datePriority
as `p'. So the passed date as set to the partnersData and
orginData's date has shifted to the next day.

\item[{example5:}] \leavevmode
\begin{Verbatim}[commandchars=\\\{\}]
\PYG{g+gp}{\PYGZgt{}\PYGZgt{}\PYGZgt{} }\PYG{n}{a}\PYG{p}{,}\PYG{n}{b} \PYG{o}{=} \PYG{n}{getDataPartners}\PYG{p}{(}\PYG{n}{var} \PYG{o}{=} \PYG{l+s}{\PYGZsq{}}\PYG{l+s}{U component of wind}\PYG{l+s}{\PYGZsq{}}\PYG{p}{,}\PYG{n}{Type} \PYG{o}{=} \PYG{l+s}{\PYGZsq{}}\PYG{l+s}{a}\PYG{l+s}{\PYGZsq{}}\PYG{p}{,}
\PYG{g+go}{         date = (\PYGZsq{}2010\PYGZhy{}6\PYGZhy{}5\PYGZsq{},\PYGZsq{}2010\PYGZhy{}6\PYGZhy{}6\PYGZsq{}),hour = 24,level = \PYGZsq{}all\PYGZsq{},}
\PYG{g+go}{    orginData = 1,datePriority = \PYGZsq{}o\PYGZsq{}, lat=(\PYGZhy{}90,90),lon=(0,359.5))}
\end{Verbatim}
\begin{quote}

\begin{notice}{note}{Note:}
Even though we passed `a' Type, we must choose the
hour option to select the fcst file, since we are
passing the range of dates.
\end{notice}
\end{quote}
\begin{description}
\item[{a is orginData. i.e. anl. its timeAxis size is 2.}] \leavevmode
date are `2010-6-5' and `2010-6-6'.

\end{description}

b is partnersData. i.e. fcst 24 hour data.
its timeAxis size is 2. date w.r.t orginData are `2010-6-4' and
`2010-6-5'.

a's `2010-6-5' has partner is b's `2010-6-4'.i.e.orginData(anl)
partners is partnersData's (fcst).

same concept for the remains day.
a's `2010-6-6' has partner is b's `2010-6-5'.

\item[{example6:}] \leavevmode
\begin{Verbatim}[commandchars=\\\{\}]
\PYG{g+gp}{\PYGZgt{}\PYGZgt{}\PYGZgt{} }\PYG{n}{a}\PYG{p}{,}\PYG{n}{b} \PYG{o}{=} \PYG{n}{getDataPartners}\PYG{p}{(}\PYG{n}{var} \PYG{o}{=} \PYG{l+s}{\PYGZsq{}}\PYG{l+s}{U component of wind}\PYG{l+s}{\PYGZsq{}}\PYG{p}{,}\PYG{n}{Type} \PYG{o}{=} \PYG{l+s}{\PYGZsq{}}\PYG{l+s}{a}\PYG{l+s}{\PYGZsq{}}\PYG{p}{,}
\PYG{g+go}{          date = (\PYGZsq{}2010\PYGZhy{}6\PYGZhy{}5\PYGZsq{},\PYGZsq{}2010\PYGZhy{}6\PYGZhy{}6\PYGZsq{}),hour = 24,level = \PYGZsq{}all\PYGZsq{},}
\PYG{g+go}{    orginData = 1,datePriority = \PYGZsq{}p\PYGZsq{}, lat=(\PYGZhy{}90,90),lon=(0,359.5))}
\end{Verbatim}
\begin{quote}

\begin{notice}{note}{Note:}
Even though we passed `a' Type, we must choose the
hour option to select the fcst file, since we are
passing the range of dates.
\end{notice}
\end{quote}

a is orginData. i.e. anl. its timeAxis size is 2.
date are `2010-6-6' and `2010-6-7'.

b is partnersData. i.e. fcst 24 hour data.
its timeAxis size is 2. date w.r.t orginData are `2010-6-5' and
`2010-6-6'.

a's `2010-6-6' has partner is b's `2010-6-5'.i.e.orginData(anl)
partners is partnersData's (fcst).

same concept for the remains day.
a's `2010-6-7' has partner is b's `2010-6-6'.
we can compare this eg6 with eg5.In this we passed datePriority
as `p'. So the passed date as set to the partnersData and
orginData's date has shifted towards the next days.

\end{description}
\end{quote}

Written by: Arulalan.T

Date: 27.05.2011

\end{fulllineitems}

\index{getMonthAvgData() (xml\_data\_access.GribXmlAccess method)}

\begin{fulllineitems}
\phantomsection\label{diagnosisutils:xml_data_access.GribXmlAccess.getMonthAvgData}\pysiglinewithargsret{\bfcode{getMonthAvgData}}{\emph{var}, \emph{Type}, \emph{level}, \emph{month}, \emph{year}, \emph{calendarName=None}, \emph{hour=None}, \emph{**latlonregion}}{}~\begin{description}
\item[{{\hyperref[diagnosisutils:xml_data_access.GribXmlAccess.getMonthAvgData]{\code{getMonthAvgData()}}}: It returns the average of the given month}] \leavevmode
data (all days in the month) for the passed variable options

\item[{Condition :}] \leavevmode
calendarName,level are optional.
level takes default `all' if not pass arg for it.
hour is must when Type arg should be `f' (fcst) to select xml object.
Pass either (lat,lon) or region.

\item[{Inputs :}] \leavevmode
Type - either `a' or `f' or `o'
var,level must be belongs to the data file
month may be even in 3 char like `apr' or `April' or `aPRiL' or
like any month
year must be passed as integer
calendarName default None, it takes cdtime.DefaultCalendar
key word arg lat,lon or region should be passed

\item[{Outputs :}] \leavevmode
It should return the average of the whole month data for the given
vars as MV2 variable

\end{description}

Usage :
\begin{quote}
\begin{description}
\item[{example :}] \leavevmode
\begin{Verbatim}[commandchars=\\\{\}]
\PYG{g+gp}{\PYGZgt{}\PYGZgt{}\PYGZgt{} }\PYG{n}{getMonthAvgData}\PYG{p}{(}\PYG{n}{var} \PYG{o}{=} \PYG{l+s}{\PYGZdq{}}\PYG{l+s}{Geopotential Height}\PYG{l+s}{\PYGZdq{}}\PYG{p}{,}\PYG{n}{Type} \PYG{o}{=} \PYG{l+s}{\PYGZsq{}}\PYG{l+s}{f}\PYG{l+s}{\PYGZsq{}}\PYG{p}{,}
\PYG{g+go}{            level = \PYGZsq{}all\PYGZsq{}, month = \PYGZsq{}july\PYGZsq{},year=2010 , hour = 24,}
\PYG{g+go}{             region = AIR ) \PYGZsh{}lat=(\PYGZhy{}90,90),lon=(0,359.5)}
\end{Verbatim}

returns dataAvg of one month data as single MV2 variable

\end{description}
\end{quote}

Written by: Arulalan.T

Date: 29.04.2011

\end{fulllineitems}

\index{getRainfallData() (xml\_data\_access.GribXmlAccess method)}

\begin{fulllineitems}
\phantomsection\label{diagnosisutils:xml_data_access.GribXmlAccess.getRainfallData}\pysiglinewithargsret{\bfcode{getRainfallData}}{\emph{date=None}, \emph{level='all'}, \emph{**latlonregion}}{}~\begin{description}
\item[{{\hyperref[diagnosisutils:xml_data_access.GribXmlAccess.getRainfallData]{\code{getRainfallData()}}}: Extract the rainfall data from the xml file}] \leavevmode
which has created by the cdscan command.

\item[{Inputs}] \leavevmode{[}var takes from the instance member of GribAccess class{]}\begin{description}
\item[{if we passed date,level then it should return data accordingly}] \leavevmode
By default level takes `all' levels.
Pass either (lat,lon) or region keyword arg

\end{description}

\item[{Condition}] \leavevmode{[}we must set the member variable called rainfallXmlPath{]}
and (rainfallXmlVar or rainfallModel), then only we can
access this method. rainfallXmlVar is the obeservation
rainfall variable name to access the data. OR
rainfallModel is the model name, which has set in the
global variable names settings. By Using it, this method
should get the observation rainfall variable name.

\end{description}

Written by : Arulalan.T

Date : 29.05.2011

\end{fulllineitems}

\index{getRainfallDataPartners() (xml\_data\_access.GribXmlAccess method)}

\begin{fulllineitems}
\phantomsection\label{diagnosisutils:xml_data_access.GribXmlAccess.getRainfallDataPartners}\pysiglinewithargsret{\bfcode{getRainfallDataPartners}}{\emph{date}, \emph{hour=None}, \emph{level='all'}, \emph{orginData=1}, \emph{datePriority='o'}, \emph{**latlonregion}}{}~\begin{description}
\item[{{\hyperref[diagnosisutils:xml_data_access.GribXmlAccess.getRainfallDataPartners]{\code{getRainfallDataPartners()}}}: It returns the rainfall data \& its}] \leavevmode
partners data(fcst is the partner of the observation.
i.e. rainfall) as MV2 vars

\item[{Condition:}] \leavevmode
startdate is must. enddata is optional one.
If both startdate and enddate has passed means, it should
return the rainfall data and partnersData within that range.

\end{description}

Inputs:
\begin{quote}
\begin{description}
\item[{orginData - either 0 or 1. 0 means it shouldnot return the}] \leavevmode\begin{description}
\item[{orginData as single MV2 var.}] \leavevmode
1 means it should return both the orginData and its
partnersData as two seperate MV2 vars.

\end{description}

\item[{datePriority - either `o' or `p'. `o' means passed date is with}] \leavevmode
respect to orginData. According to this orginData's
date, it should return its partnersData.
`p' means passed date is with respect to partnersData.
\begin{quote}

According to this partnersData's date, it should
return its orginData.
\end{quote}

\end{description}

key word arg lat,lon or region should be passed
By default hour is None and level is `all'.

self.rainfallXmlPath is mandatory one when you choosed orginData
is 1. you must set the rainfall xml path to the rainfallXmlPath.

self.rainfallModel is the model name, which has set in the global
variables settings, to get the model fcst and its obeservation
variable name to access the data. It is mandatory one.
\begin{description}
\item[{date formate 1:}] \leavevmode
date = (startdate,enddate)
here startdate and enddate must be like cdtime.comptime formate.

\item[{date formate 2:}] \leavevmode
date = (startdate)

\item[{date formate 3:}] \leavevmode
date = `startdate' or date = `date'

\item[{eg for date input :}] \leavevmode\begin{quote}

date = (`2010-5-1',`2010-6-30')
date = (`2010-5-30')
date = `2010-5-30'
\end{quote}

By default skipdays as 1 takes place. User cant override till now.

\end{description}
\end{quote}

Outputs :
\begin{quote}

If user passed single date in the date argument, then it should
return the data of that particular date
(both orginData \& partnersData) as MV2 variable.

If user passed start and enddate in the date argument, then it
should return the data (both orginData \& partnersData) for the
range of dates as MV2 variable with time axis.
\end{quote}

Usage :
\begin{quote}
\begin{quote}

\begin{notice}{note}{Note:}
if `r'(observation) file is orginData means `f'(fcst)
files are its partnersData.
\end{notice}
\end{quote}
\begin{description}
\item[{example1:}] \leavevmode
\begin{Verbatim}[commandchars=\\\{\}]
\PYG{g+gp}{\PYGZgt{}\PYGZgt{}\PYGZgt{} }\PYG{n}{a}\PYG{p}{,}\PYG{n}{b} \PYG{o}{=} \PYG{n}{getRainfallDataPartners}\PYG{p}{(}\PYG{n}{date} \PYG{o}{=} \PYG{l+s}{\PYGZsq{}}\PYG{l+s}{2010\PYGZhy{}6\PYGZhy{}5}\PYG{l+s}{\PYGZsq{}}\PYG{p}{,}\PYG{n}{hour} \PYG{o}{=} \PYG{n+nb+bp}{None}\PYG{p}{,}
\PYG{g+go}{                 level = \PYGZsq{}all\PYGZsq{},orginData = 1,datePriority = \PYGZsq{}o\PYGZsq{},}
\PYG{g+go}{                                     lat=(\PYGZhy{}90,90),lon=(0,359.5))}
\end{Verbatim}

a is orginData. i.e. rainfall observation.
its timeAxis date is `2010-6-5'.

b is partnersData. i.e. fcst. its 24 hour fcst date w.r.t
orginData is `2010-6-4'. 48 hour is `2010-6-3'.

Depends upon the availability of date of fcst files,it should
return the data.
In NCMRWF2010 model, it should return maximum of 7 days fcst.

If we will specify any hour in the same eg, that should return
only that hour fcst file data instead of returning all the
available fcst hours data.

\item[{example2:}] \leavevmode
\begin{Verbatim}[commandchars=\\\{\}]
\PYG{g+gp}{\PYGZgt{}\PYGZgt{}\PYGZgt{} }\PYG{n}{a}\PYG{p}{,}\PYG{n}{b} \PYG{o}{=} \PYG{n}{getRainfallDataPartners}\PYG{p}{(}\PYG{n}{date} \PYG{o}{=} \PYG{l+s}{\PYGZsq{}}\PYG{l+s}{2010\PYGZhy{}6\PYGZhy{}5}\PYG{l+s}{\PYGZsq{}}\PYG{p}{,}\PYG{n}{hour} \PYG{o}{=} \PYG{l+m+mi}{24}\PYG{p}{,}
\PYG{g+go}{                 level = \PYGZsq{}all\PYGZsq{},orginData = 1,datePriority = \PYGZsq{}o\PYGZsq{},}
\PYG{g+go}{                                     lat=(\PYGZhy{}90,90),lon=(0,359.5))}
\end{Verbatim}

a is orginData. i.e. rainfall observation.
its timeAxis date is `2010-6-5'.
b is partnersData. i.e. fcst 24 hour. its fcst date w.r.t
orginData is `2010-6-6'.

\item[{example3:}] \leavevmode
\begin{Verbatim}[commandchars=\\\{\}]
\PYG{g+gp}{\PYGZgt{}\PYGZgt{}\PYGZgt{} }\PYG{n}{b} \PYG{o}{=} \PYG{n}{getRainfallDataPartners}\PYG{p}{(}\PYG{n}{date} \PYG{o}{=} \PYG{l+s}{\PYGZsq{}}\PYG{l+s}{2010\PYGZhy{}6\PYGZhy{}5}\PYG{l+s}{\PYGZsq{}}\PYG{p}{,}\PYG{n}{hour} \PYG{o}{=} \PYG{l+m+mi}{24}\PYG{p}{,}
\PYG{g+go}{                 level = \PYGZsq{}all\PYGZsq{},orginData = 0,datePriority = \PYGZsq{}o\PYGZsq{},}
\PYG{g+go}{                                     lat=(\PYGZhy{}90,90),lon=(0,359.5))}
\end{Verbatim}

b is partnersData. i.e. fcst. its fcst date w.r.t orginData
is `2010-6-6'.  No orginData. Because we passed orginData as 0.

\item[{example4:}] \leavevmode
\begin{Verbatim}[commandchars=\\\{\}]
\PYG{g+gp}{\PYGZgt{}\PYGZgt{}\PYGZgt{} }\PYG{n}{a}\PYG{p}{,}\PYG{n}{b} \PYG{o}{=} \PYG{n}{getRainfallDataPartnerss}\PYG{p}{(}\PYG{n}{date} \PYG{o}{=} \PYG{l+s}{\PYGZsq{}}\PYG{l+s}{2010\PYGZhy{}6\PYGZhy{}5}\PYG{l+s}{\PYGZsq{}}\PYG{p}{,}\PYG{n}{hour} \PYG{o}{=} \PYG{l+m+mi}{24}\PYG{p}{,}
\PYG{g+go}{                 level = \PYGZsq{}all\PYGZsq{},orginData = 1,datePriority = \PYGZsq{}p\PYGZsq{},}
\PYG{g+go}{                                     lat=(\PYGZhy{}90,90),lon=(0,359.5))}
\end{Verbatim}

a is orginData. i.e. rainfall observation.
its timeAxis date is `2010-6-6'.

b is partnersData. i.e. fcst 24 hour. its fcst date w.r.t
orginData is `2010-6-5'.  we can compare this eg4 with eg2.

In this we passed datePriority as `p'. So the passed date as
set to the partnersData and orginData's date has shifted to
the next day.

\item[{example5:}] \leavevmode
\begin{Verbatim}[commandchars=\\\{\}]
\PYG{g+gp}{\PYGZgt{}\PYGZgt{}\PYGZgt{} }\PYG{n}{a}\PYG{p}{,}\PYG{n}{b} \PYG{o}{=} \PYG{n}{getRainfallDataPartners}\PYG{p}{(}\PYG{n}{date} \PYG{o}{=} \PYG{p}{(}\PYG{l+s}{\PYGZsq{}}\PYG{l+s}{2010\PYGZhy{}6\PYGZhy{}5}\PYG{l+s}{\PYGZsq{}}\PYG{p}{,}\PYG{l+s}{\PYGZsq{}}\PYG{l+s}{2010\PYGZhy{}6\PYGZhy{}6}\PYG{l+s}{\PYGZsq{}}\PYG{p}{)}\PYG{p}{,}
\PYG{g+go}{                          hour = 24,level = \PYGZsq{}all\PYGZsq{},orginData = 1,}
\PYG{g+go}{                 datePriority = \PYGZsq{}o\PYGZsq{}, lat=(\PYGZhy{}90,90),lon=(0,359.5))}
\end{Verbatim}
\begin{quote}

\begin{notice}{note}{Note:}
We must choose the hour option to select the fcst
file, since we are passing the range of dates.
\end{notice}
\end{quote}

a is orginData.i.e.rainfall observation.its timeAxis size is 2.
date are `2010-6-5' and `2010-6-6'.

b is partnersData.i.e.fcst 24 hour data.its timeAxis size is 2.
date w.r.t orginData are `2010-6-4' and `2010-6-5'.

a's `2010-6-5' has partner is b's `2010-6-4'. i.e.
orginData(rainfall observation) partners is partnersData(fcst)

same concept for the remains day.
a's `2010-6-6' has partner is b's `2010-6-5'.

\item[{example6:}] \leavevmode
\begin{Verbatim}[commandchars=\\\{\}]
\PYG{g+gp}{\PYGZgt{}\PYGZgt{}\PYGZgt{} }\PYG{n}{a}\PYG{p}{,}\PYG{n}{b} \PYG{o}{=} \PYG{n}{getRainfallDataPartners}\PYG{p}{(}\PYG{n}{date} \PYG{o}{=} \PYG{p}{(}\PYG{l+s}{\PYGZsq{}}\PYG{l+s}{2010\PYGZhy{}6\PYGZhy{}5}\PYG{l+s}{\PYGZsq{}}\PYG{p}{,}\PYG{l+s}{\PYGZsq{}}\PYG{l+s}{2010\PYGZhy{}6\PYGZhy{}6}\PYG{l+s}{\PYGZsq{}}\PYG{p}{)}\PYG{p}{,}
\PYG{g+go}{                         hour = 24,level = \PYGZsq{}all\PYGZsq{},orginData = 1,}
\PYG{g+go}{                 datePriority = \PYGZsq{}p\PYGZsq{}, lat=(\PYGZhy{}90,90),lon=(0,359.5))}
\end{Verbatim}
\begin{quote}
\end{quote}

a is orginData. i.e. rainfall observation.
its timeAxis size is 2. date are `2010-6-6' and `2010-6-7'.

b is partnersData. i.e. fcst 24 hour data.
its timeAxis size is 2. date w.r.t orginData are `2010-6-5'
and `2010-6-6'.

a's `2010-6-6' has partner is b's `2010-6-5'. i.e.
orginData(rainfall observation) partners is partnersData(fcst)

same concept for the remains day.
a's `2010-6-7' has partner is b's `2010-6-6'. we can compare
this eg6 with eg5. In this we passed datePriority as `p'.
So the passed date as set to the partnersData and orginData's
date has shifted towards the next days.

\end{description}
\end{quote}

Written by: Arulalan.T

Date: 29.05.2011

\end{fulllineitems}

\index{getXmlPath() (xml\_data\_access.GribXmlAccess method)}

\begin{fulllineitems}
\phantomsection\label{diagnosisutils:xml_data_access.GribXmlAccess.getXmlPath}\pysiglinewithargsret{\bfcode{getXmlPath}}{\emph{Type}, \emph{hour=None}}{}
{\hyperref[diagnosisutils:xml_data_access.GribXmlAccess.getXmlPath]{\code{getXmlPath()}}}: To get the xml's absolute path.
\begin{description}
\item[{Inputs}] \leavevmode{[}Type is either `a' or `o' or `f' or `r'{]}
hour is mandatory when you pass `f' or `r'.
`a' - analysis, `f' - forecast,
`o' - observation, `r' - reference.

\end{description}

Written by : Arulalan.T

Date : 21.08.2011

\end{fulllineitems}

\index{listvariable() (xml\_data\_access.GribXmlAccess method)}

\begin{fulllineitems}
\phantomsection\label{diagnosisutils:xml_data_access.GribXmlAccess.listvariable}\pysiglinewithargsret{\bfcode{listvariable}}{\emph{Type}, \emph{hour=None}}{}
:func:'listvariable': By passing Type and/or hour args to this method,
it will return the listvariable method of the appropriate xml file.

Returns the listvariable of cdms2 open object method result

\end{fulllineitems}


\end{fulllineitems}



\section{Time Axis Utils}
\label{diagnosisutils:time-axis-utils}
This {\hyperref[diagnosisutils:timeutils]{timeutils}} module helps us to generate our own time axis, correct existing time axis bounds and generate bounds.

Here we used inbuilt methods of cdtime and cdutil module of uv-cdat.


\subsection{timeutils}
\label{diagnosisutils:module-timeutils}\label{diagnosisutils:timeutils}\index{timeutils (module)}\phantomsection\label{diagnosisutils:module-timeutils}\index{timeutils (module)}

\section{Plot Utils}
\label{diagnosisutils:plot-utils}
The {\hyperref[diagnosisutils:plot]{plot}} module has the properties to plot the vcs vector with some default template look out.

User can control the reference point, scale of arrow marks of the vector plot.


\subsection{plot}
\label{diagnosisutils:plot}\label{diagnosisutils:module-plot}\index{plot (module)}\index{reference\_std\_dev (class in plot)}

\begin{fulllineitems}
\phantomsection\label{diagnosisutils:plot.reference_std_dev}\pysigline{\strong{class }\code{plot.}\bfcode{reference\_std\_dev}}
This class implements the type: standard deviation of a reference variable.
It is just a float value with one method that will be used in the computation of
a test variable RMS.
\index{compute\_RMS\_function() (plot.reference\_std\_dev method)}

\begin{fulllineitems}
\phantomsection\label{diagnosisutils:plot.reference_std_dev.compute_RMS_function}\pysiglinewithargsret{\bfcode{compute\_RMS\_function}}{\emph{s}, \emph{R}}{}
Compute and return the centered-pattern RMS of a test variable
from its standard-deviation and correlation.
\begin{description}
\item[{Input:}] \leavevmode
self:   reference-variable standard deviation
s:      test variable standard deviation
R:      test-variable correlation with reference variable

\end{description}

\end{fulllineitems}


\end{fulllineitems}

\index{vectorPlot() (in module plot)}

\begin{fulllineitems}
\phantomsection\label{diagnosisutils:plot.vectorPlot}\pysiglinewithargsret{\code{plot.}\bfcode{vectorPlot}}{\emph{u}, \emph{v}, \emph{name}, \emph{path=None}, \emph{reference=20.0}, \emph{scale=1}, \emph{interval=1}, \emph{svg=1}, \emph{png=0}, \emph{latlabel='lat5'}, \emph{lonlabel='lon5'}, \emph{style='portrait'}}{}
{\hyperref[diagnosisutils:plot.vectorPlot]{\code{vectorPlot()}}}: Plotting the vector with some default preferences.
\begin{description}
\item[{Input}] \leavevmode{[}u - u variable{]}
v - v variable
name - name to plot on the top of the vcs
path - path to save as the image file.
reference - vector reference. Default it takes 20.0 (i.e 2 degree)
scale - scaling of the arrow mark in vector plot
interval - slicing the data to reduce the density (noise)
\begin{quote}

in the vector plot, with respect to the interval.
Default it takes 1. (i.e. doesnot affect the u \& v)
\end{quote}

svg - to save image as svg
png - to save image as png

\item[{Condition}] \leavevmode{[}u must be `u variable' and v must be `v variable'.{]}
name must pass to set the name on the vector vcs
path is not passed means, it takes current workig directory
reference must be float.
interval not be 0.

\end{description}

Usage : using this function, user can plot the vector.
\begin{quote}

user can control the reference point of the vector, and scale
length of the arrow marks in plot.

Also can control the u and v data shape by interval.

filename should be generated from the `name' passed by the user,
just replacing the space into underscore `\_'.

if svg and png passed 1, the image will be saved with these
extensions in the filename.
\end{quote}

Written By : Arulalan.T

Date : 26.07.2011

\end{fulllineitems}



\section{More}
\label{diagnosisutils:more}
More utilities will be added and optimized in near future.


\chapter{Documentation of \textbf{diagnosis} source code}
\label{diagnosis:documentation-of-diagnosis-source-code}\label{diagnosis::doc}\label{diagnosis:diagnosis}
The diagnosis package contains the following modules.
\begin{itemize}
\item {} 
{\hyperref[diagnosis:daily-progress]{Daily Progress}}

\item {} 
{\hyperref[diagnosis:monthly-progress]{Monthly Progress}}

\item {} 
{\hyperref[diagnosis:seasonly-progress]{Seasonly Progress}}

\item {} 
{\hyperref[diagnosis:statistical-scores]{Statistical Scores}}

\item {} 
{\hyperref[diagnosis:more]{More}}

\end{itemize}


\section{Daily Progress}
\label{diagnosis:daily-progress}
The daily progress of {\hyperref[diagnosis:diagnosis]{diagnosis}} are listed below.


\subsection{Anomaly}
\label{diagnosis:anomaly}\begin{quote}

Have to update the code. Will do it soon.
\end{quote}


\section{Monthly Progress}
\label{diagnosis:monthly-progress}
The monthly progress of {\hyperref[diagnosis:diagnosis]{diagnosis}} are listed below.
\begin{itemize}
\item {} 
{\hyperref[diagnosis:month-mean]{Month Mean}}

\item {} 
{\hyperref[diagnosis:month-anomaly]{Month Anomaly}}

\item {} 
{\hyperref[diagnosis:month-fcst-sys-error]{Month Fcst Sys Error}}

\end{itemize}

These monthly progress will be automated.


\subsection{Month Mean}
\label{diagnosis:month-mean}
The word \emph{Mean} means average of the data. The average will be taken over the month time axis is called month mean.

The below script \emph{compute\_month\_mean.py} should explain more how we are implementing monthly mean and generating the nc files.
\phantomsection\label{diagnosis:module-compute_month_mean}\index{compute\_month\_mean (module)}\phantomsection\label{diagnosis:module-compute_month_mean.py}\index{compute\_month\_mean.py (module)}\index{genMonthMeanDirs() (in module compute\_month\_mean)}

\begin{fulllineitems}
\phantomsection\label{diagnosis:compute_month_mean.genMonthMeanDirs}\pysiglinewithargsret{\code{compute\_month\_mean.}\bfcode{genMonthMeanDirs}}{\emph{modelname}, \emph{modelpath}, \emph{modelhour}}{}~\begin{description}
\item[{\code{genMonthMeanDirs()}}] \leavevmode{[}It should generate the directory structure{]}
whenever it needs. It reads the timeAxis information of the
model data xml file(which is updating it by cdscan), and once
the full months is completed, then it should check either that
month directory is empty or not.
\begin{description}
\item[{case 1: If that directory is empty means, it should call the}] \leavevmode
function called \emph{genMonthMeanFiles}, to calculate
the mean analysis and anomaly for that month and should
store the processed files in side that directory.

\item[{case 2: If that directory is non empty means,}] \leavevmode
\textbf{**have to update***}

\end{description}

\item[{Inputs}] \leavevmode{[}modelname is the model data name, which will become part of the{]}
directory structure.
modelpath is the absolute path of data where the model xml files
are located.
climatolgyyear is the year of climatolgy data.
climregridpath is the absolute path of the climatolgy regridded
path w.r.t to this model data resolution (both horizontal and
vertical)
climpfilename is the climatolgy Partial File Name to combine the
this passed name with (at the end) of the climatolgy var name to
open the climatolgy files.

\item[{Outputs}] \leavevmode{[}It should create the directory structure in the processfilesPath{]}
and create the processed nc files.

\end{description}

Written By : Arulalan.T

Date : 01.12.2011

\end{fulllineitems}

\index{genMonthMeanFiles() (in module compute\_month\_mean)}

\begin{fulllineitems}
\phantomsection\label{diagnosis:compute_month_mean.genMonthMeanFiles}\pysiglinewithargsret{\code{compute\_month\_mean.}\bfcode{genMonthMeanFiles}}{\emph{meanMonthPath}, \emph{monthdate}, \emph{year}, \emph{typehour}, \emph{**model}}{}~\begin{description}
\item[{\code{genMonthMeanFiles()}}] \leavevmode{[}It should calculate monthly mean analysis \&{]}
monthly mean forecast hours value for the month (of year). Finally
stores it as nc files in corresponding directory path which are
passed in this function args.

\item[{Inputs}] \leavevmode{[}meanMonthPath is the absolute path where the processed month mean{]}
analysis \& fcst hour nc files are going to store.
monthdate (which contains monthname, startdate \& enddate) and
year are the inputs to extract the monthly data.
typehour is tuple which has the type key character and fcst hour
to create sub directories inside mean directory.

\end{description}

KWargs: modelName, modelXmlPath, modelXmlObj
\begin{quote}

modelName is the model data name which will become part of the
process nc files name.
modelPath is the absolute path of data where the model xml files
are located.
modelXmlObj is an instance of the GribXmlAccess class instance.
If we are passing modelXmlObj means, it will be optimized one
when we calls this same function for same model for different
months.

We can pass either modelXmlPath or modelXmlObj KWarg is enough.
\end{quote}
\begin{description}
\item[{Outputs}] \leavevmode{[}It should create monthly mean analysis and monthly mean forecast{]}
hours for all the available variables in the vars.txt file \&
store it as nc file formate in the proper directories structure
(modelname, process name, year, month and then
{[}Analysis or hours{]} hierarchy).

\end{description}

Written By : Arulalan.T

Date : 08.09.2011
Updated : 06.12.2011

\end{fulllineitems}



\subsection{Month Anomaly}
\label{diagnosis:month-anomaly}
Hello This is math test. Remove me ! $a^2 + b^2 a\_b = c^2$.
\begin{gather}
\begin{split}(a + b)^2 = a^2 + 2ab + b^2\end{split}\notag\\\begin{split}(a - b)^2 = a^2 - 2ab a\_b + b^2\end{split}\notag
\end{gather}
Anomaly means the difference between the model analysis and climatology.

Monthly Anomaly : Take the difference between the model analysis data of the particular month and the climatology data of the corresponding month.

Anomaly = Analysis - Climatology

The below script \emph{compute\_month\_anomaly.py} should explain more how we are implementing monthly anomaly and generating the nc files.
\phantomsection\label{diagnosis:module-compute_month_anomaly}\index{compute\_month\_anomaly (module)}\phantomsection\label{diagnosis:module-compute_month_anomaly.py}\index{compute\_month\_anomaly.py (module)}\index{genMonthAnomalyDirs() (in module compute\_month\_anomaly)}

\begin{fulllineitems}
\phantomsection\label{diagnosis:compute_month_anomaly.genMonthAnomalyDirs}\pysiglinewithargsret{\code{compute\_month\_anomaly.}\bfcode{genMonthAnomalyDirs}}{\emph{modelname}, \emph{modelpath}, \emph{climregridpath}, \emph{climpfilename}, \emph{climatologyyear}}{}~\begin{description}
\item[{\code{genMonthAnomalyDirs()}}] \leavevmode{[}It should generate the directory structure{]}
whenever it needs. It reads the timeAxis information of the
model data xml file(which is updating it by cdscan), and once
the full months is completed, then it should check either that
month directory is empty or not.
\begin{description}
\item[{case 1: If that directory is empty means, it should call the}] \leavevmode
function called \emph{genMonthAnomalyFiles}, to calculate
the mean analysis and anomaly for that month and should
store the processed files in side that directory.

\item[{case 2: If that directory is non empty means,}] \leavevmode
\textbf{**have to update***}

\end{description}

\item[{Inputs}] \leavevmode{[}modelname is the model data name, which will become part of the{]}
directory structure.
modelpath is the absolute path of data where the model xml files
are located.
climregridpath is the absolute path of the climatolgy regridded
path w.r.t to this model data resolution (both horizontal and
vertical)
climpfilename is the climatolgy Partial File Name to combine the
this passed name with (at the end) of the climatolgy var name to
open the climatolgy files.
climatolgyyear is the year of climatolgy data.

\item[{Outputs}] \leavevmode{[}It should create the directory structure in the processfilesPath{]}
and create the processed nc files.

\end{description}

Written By : Arulalan.T

Date : 01.12.2011

\end{fulllineitems}

\index{genMonthAnomalyFiles() (in module compute\_month\_anomaly)}

\begin{fulllineitems}
\phantomsection\label{diagnosis:compute_month_anomaly.genMonthAnomalyFiles}\pysiglinewithargsret{\code{compute\_month\_anomaly.}\bfcode{genMonthAnomalyFiles}}{\emph{meanAnomalyPath}, \emph{meanAnalysisPath}, \emph{climRegridPath}, \emph{climPFileName}, \emph{climatologyYear}, \emph{monthdate}, \emph{year}, \emph{**model}}{}~\begin{description}
\item[{\code{genMonthAnomalyFiles()}}] \leavevmode{[}It should calculate monthly mean anomaly{]}
from the monthly mean analysis and monthly mean climatolgy,
for the month (of year) and process it. Finally
stores it as nc files in corresponding directory path which are
passed in this function args.

\item[{Inputs}] \leavevmode{[}meanAnomalyPath is the absolute path where the processed mean{]}
anomaly nc files are going to store.
meanAnalysisPath is the absolute path where the processed mean
analysis nc files were already stored.
climRegridPath is the absolute path where the regridded monthly
mean climatologies (w.r.t the model vertical resolution)
nc files were already stored.
climPFileName is the partial nc filename of the climatolgy.
climatologyYear is the year of the climatolgy to access it.
monthdate (which contains monthname, startdate \& enddate) and
year are the inputs to extract the monthly data.

\end{description}

KWargs: modelName, modelXmlPath, modelXmlObj
\begin{quote}

modelName is the model data name which will become part of the
process nc files name.
modelPath is the absolute path of data where the model xml files
are located.
modelXmlObj is an instance of the GribXmlAccess class instance.
If we are passing modelXmlObj means, it will be optimized one
when we calls this same function for same model for different
months.

We can pass either modelXmlPath or modelXmlObj KWarg is enough.
\end{quote}
\begin{description}
\item[{Outputs}] \leavevmode{[}It should create mean anomaly for the particular variables which{]}
are all set the clim\_var option in the vars.txt file. Finally
store it as nc file formate in the proper directories structure
(modelname, process name, year and then month hierarchy).

\end{description}

Written By : Arulalan.T

Date : 08.09.2011
Updated : 07.12.2011

\end{fulllineitems}



\subsection{Month Fcst Sys Error}
\label{diagnosis:month-fcst-sys-error}
Forecast Systematic Error means the difference between the model forecast hour data and model analysis.

This also called as \emph{Fcst Sys Err}.

Month Fcst Sys Error : Take the difference between the model forecast hour data of the particular month and the model analysis data of the same month.

Fcst Sys Err = Model Fcst Hour Data - Model Analysis

The below script \emph{compute\_month\_fcst\_sys\_error.py} should explain more how we are implementing monthly fcst sys err and generating the nc files.
\phantomsection\label{diagnosis:module-compute_month_fcst_sys_error}\index{compute\_month\_fcst\_sys\_error (module)}\phantomsection\label{diagnosis:module-compute_month_fcst_sys_error.py}\index{compute\_month\_fcst\_sys\_error.py (module)}\index{genMonthFcstSysErrDirs() (in module compute\_month\_fcst\_sys\_error)}

\begin{fulllineitems}
\phantomsection\label{diagnosis:compute_month_fcst_sys_error.genMonthFcstSysErrDirs}\pysiglinewithargsret{\code{compute\_month\_fcst\_sys\_error.}\bfcode{genMonthFcstSysErrDirs}}{\emph{modelname}, \emph{modelpath}, \emph{modelhour}}{}~\begin{description}
\item[{\code{genMonthFcstSysErrDirs()}}] \leavevmode{[}It should generate the directory structure{]}
whenever it needs. It reads the timeAxis information of the
model data xml file(which is updating it by cdscan), and once
the full months is completed, then it should check either that
month directory is empty or not.
\begin{description}
\item[{case 1: If that directory is empty means, it should call the}] \leavevmode
function called \emph{genMonthFcstSysErrFiles}, to calculate
the mean analysis and fcstsyserr for that month and should
store the processed files in side that directory.

\item[{case 2: If that directory is non empty means,}] \leavevmode
\textbf{**have to update***}

\end{description}

\item[{Inputs}] \leavevmode{[}modelname is the model data name, which will become part of the{]}
directory structure.
modelpath is the absolute path of data where the model xml files
are located.
modelhour is the list of model data hours, which will become
part of the directory structure.

\item[{Outputs}] \leavevmode{[}It should create the directory structure in the processfilesPath{]}
and create the processed nc files.

\end{description}

Written By : Arulalan.T

Date : 08.12.2011

\end{fulllineitems}

\index{genMonthFcstSysErrFiles() (in module compute\_month\_fcst\_sys\_error)}

\begin{fulllineitems}
\phantomsection\label{diagnosis:compute_month_fcst_sys_error.genMonthFcstSysErrFiles}\pysiglinewithargsret{\code{compute\_month\_fcst\_sys\_error.}\bfcode{genMonthFcstSysErrFiles}}{\emph{meanFcstSysErrPath}, \emph{meanPath}, \emph{monthdate}, \emph{year}, \emph{modelhour}, \emph{**model}}{}~\begin{description}
\item[{\code{genMonthFcstSysErrFiles()}}] \leavevmode{[}It should calculate mean analysis \&{]}
fcstsyserr for the passed month (of year) and process it. Finally
stores it as nc files in corresponding directory path which are
passed in this function args.

\item[{Inputs}] \leavevmode{[}meanFcstSysErrPath is the absolute path where the processed mean{]}
fcstsyserr nc files are going to store.
meanPath is the absolute path (partial path) where the processed
monthly mean analysis and fcst hour nc files were stored already.
monthdate (which contains monthname, startdate \& enddate) and
year are the inputs to extract the monthly data.
modelhour is the list of model data hours, which will become
part of the directory structure.

\end{description}

KWargs: modelName, modelXmlPath, modelXmlObj
\begin{quote}

modelName is the model data name which will become part of the
process nc files name.
modelPath is the absolute path of data where the model xml files
are located.
modelXmlObj is an instance of the GribXmlAccess class instance.
If we are passing modelXmlObj means, it will be optimized one
when we calls this same function for same model for different
months.

We can pass either modelXmlPath or modelXmlObj KWarg is enough.
\end{quote}
\begin{description}
\item[{Outputs}] \leavevmode{[}It should create mean forecast systematic error for all the{]}
available variables in the vars.txt file \& store it as nc file
formate in the proper directories structure
(modelname, process name, year, month and then hours hierarchy).

\end{description}

Written By : Arulalan.T

Date : 08.09.2011
Updated : 08.12.2011

\end{fulllineitems}



\section{Seasonly Progress}
\label{diagnosis:seasonly-progress}
The seasonly progress of {\hyperref[diagnosis:diagnosis]{diagnosis}} are listed below.
\begin{itemize}
\item {} 
{\hyperref[diagnosis:season-mean]{Season Mean}}

\item {} 
{\hyperref[diagnosis:collect-season-fcst-rainfall]{Collect Season Fcst Rainfall}}

\item {} 
{\hyperref[diagnosis:region-statistical-score]{Region Statistical Score}}

\item {} 
{\hyperref[diagnosis:season-statistical-score-spatial-distribution]{Season Statistical Score Spatial Distribution}}

\end{itemize}

These season progress will be automated with respect to the given season as input in the configure.txt.

The word \emph{season} means the consecutive months.

For eg: JJAS contains June, July, August, September.


\subsection{Season Mean}
\label{diagnosis:season-mean}
The word \emph{Mean} means average of the data. The average will be taken over the season time axis is called season mean.

The below script \emph{compute\_season\_mean.py} should explain more how we are implementing seasonly mean and generating the nc files.
\phantomsection\label{diagnosis:module-compute_season_mean}\index{compute\_season\_mean (module)}\phantomsection\label{diagnosis:module-compute_season_mean.py}\index{compute\_season\_mean.py (module)}\index{genMeanAnlFcstErrDirs() (in module compute\_season\_mean)}

\begin{fulllineitems}
\phantomsection\label{diagnosis:compute_season_mean.genMeanAnlFcstErrDirs}\pysiglinewithargsret{\code{compute\_season\_mean.}\bfcode{genMeanAnlFcstErrDirs}}{\emph{modelname}, \emph{modelpath}, \emph{modelhour}}{}~\begin{description}
\item[{\code{genMeanAnlFcstErrDirs()}}] \leavevmode{[}It should create the directory structure{]}
whenever it needs. It reads the timeAxis information of the
model data xml file(which is updating it by cdscan), and once
the full seasonal months are completed, then it should check
either that season directory is empty or not.
\begin{description}
\item[{case 1: If that directory is empty means, it should call the}] \leavevmode
function called \emph{genSeasonMeanFiles}, to calculate
the mean analysis and fcstsyserr for that season and should
store the processed files in side that directory.

\item[{case 2: If that directory is non empty means,}] \leavevmode
\textbf{**have to update***}

\end{description}

\item[{Inputs}] \leavevmode{[}modelname is the model data name, which will become part of the{]}
directory structure.
modelpath is the absolute path of data where the model xml files
are located.
modelhour is the list of model data hours, which will become
part of the directory structure.

\item[{Outputs}] \leavevmode{[}It should create the directory structure in the processfilesPath{]}
and create the processed nc files.

\end{description}

Written By : Arulalan.T

Date : 08.12.2011

\end{fulllineitems}

\index{genSeasonMeanFiles() (in module compute\_season\_mean)}

\begin{fulllineitems}
\phantomsection\label{diagnosis:compute_season_mean.genSeasonMeanFiles}\pysiglinewithargsret{\code{compute\_season\_mean.}\bfcode{genSeasonMeanFiles}}{\emph{meanSeasonPath}, \emph{meanMonthPath}, \emph{seasonName}, \emph{seasonMonthDate}, \emph{year}, \emph{Type}, \emph{**model}}{}~\begin{description}
\item[{\code{genSeasonMeanFiles()}}] \leavevmode{[}It should calculate the seasonly mean for{]}
either analysis or forecast systematic error. It can be choosed by
the Type argment. Finally stores it as nc files in corresponding
directory path which are passed in this function args.

\item[{Inputs}] \leavevmode{[}meanSeasonPath is the absolute path where the processed season{]}
mean analysis or forecast systematic error nc files are going to
store. Inside the fcst hours directories will be created in this
path, if needed.

meanMonthPath is the absolute path where the processed monthly
mean analysis or monthly mean fcstsyserr nc files were stored,
already.
seasonName is the name of the season.
seasonMonthDate(list of months which contains monthname,
startdate \& enddate) for the season.
year is the part of the directory structure.
Type is either `a' for analysis or `f' for fcstsyserr.

\end{description}

KWargs: modelName, modelXmlPath, modelXmlObj
\begin{quote}

modelName is the model data name which will become part of the
process nc files name.
modelHour is the model hours as list, which will become part of
the directory structure.
modelPath is the absolute path of data where the model xml files
are located.
modelXmlObj is an instance of the GribXmlAccess class instance.
If we are passing modelXmlObj means, it will be optimized one
when we calls this same function for same model for different
months.

We can pass either modelXmlPath or modelXmlObj KWarg is enough.
\end{quote}
\begin{description}
\item[{Process}] \leavevmode{[}This function should compute the seasonly mean for analysis and{]}
fcstsyserr by just opening the monthly mena analysis/fcstsyserr
nc files (according to the season's months) and multiply the
monthly mean into its weights value. So that monthly mean data
should become monthly full data (not mean). Then add it together
for the season of months. Finally takes the average by just
divide the whole season data by sum of monthly mean weights.
\begin{quote}

So it should simplify our life, just extracting data which
\end{quote}

timeAxis length is 4, for 4 months in season. (eg JJAS).

\item[{Outputs}] \leavevmode{[}It should create seasonly mean analysis and forecast systematic{]}
error for all the available variables in the vars.txt file,
and store it as nc file in the proper directory structure
(modelname, process name, year, season, and/or hours hierarchy).

\end{description}

Written By : Arulalan.T

Date : 08.12.2011

\end{fulllineitems}



\subsection{Collect Season Fcst Rainfall}
\label{diagnosis:collect-season-fcst-rainfall}
This script \emph{collect\_season\_fcst\_rainfall.py} should collect the whole season forecast hourly rainfall with respect to the hours of season.
Finally it should create the nc files for every fcst hours that contains the fcst rainfall with needed time axis to compute the further process.
\phantomsection\label{diagnosis:module-collect_season_fcst_rainfall}\index{collect\_season\_fcst\_rainfall (module)}\phantomsection\label{diagnosis:module-collect_season_fcst_rainfall.py}\index{collect\_season\_fcst\_rainfall.py (module)}\begin{description}
\item[{Written by: Dileepkumar R}] \leavevmode
JRF- IIT DELHI

\end{description}

Date: 23.06.2011

Updated By : Arulalan.T
Date: 14.09.2011
Date: 19.10.2011
\index{createSeaonFcstRainfallData() (in module collect\_season\_fcst\_rainfall)}

\begin{fulllineitems}
\phantomsection\label{diagnosis:collect_season_fcst_rainfall.createSeaonFcstRainfallData}\pysiglinewithargsret{\code{collect\_season\_fcst\_rainfall.}\bfcode{createSeaonFcstRainfallData}}{\emph{modelname}, \emph{modelpath}, \emph{modelhour}, \emph{rainfallPath}, \emph{rainfallXmlName=None}}{}~\begin{description}
\item[{\code{createSeaonFcstRainfallData()}: It should create model hours forecast}] \leavevmode
rainfall data nc files, in side the `StatiScore' directory of
processfilesPath in hierarchy structure. The fcst rainfall
timeAxis are in partners timeAxis w.r.t observation rainfall and
fcst hours.

\end{description}

\end{fulllineitems}



\subsection{Region Statistical Score}
\label{diagnosis:region-statistical-score}
The below script \emph{compute\_region\_statistical\_score.py} should compute the various statistical scores regional wise and make the nc files.

Here regions are `Central India', `Peninsular India', etc. That is region name/variable defines the latitude and longitude range.
\phantomsection\label{diagnosis:module-compute_region_statistical_score}\index{compute\_region\_statistical\_score (module)}\phantomsection\label{diagnosis:module-compute_region_statistical_score.py}\index{compute\_region\_statistical\_score.py (module)}\begin{description}
\item[{Example:}] \leavevmode
Let `ts'(Threat Score) is a statistical score,
we are calculate this for different regions.

\item[{Written by: Dileepkumar R}] \leavevmode
JRF- IIT DELHI

\end{description}

Date: 02.08.2011;

Updated By : Arulalan.T
Date : 16.09.2011
\index{genStatisticalScore() (in module compute\_region\_statistical\_score)}

\begin{fulllineitems}
\phantomsection\label{diagnosis:compute_region_statistical_score.genStatisticalScore}\pysiglinewithargsret{\code{compute\_region\_statistical\_score.}\bfcode{genStatisticalScore}}{\emph{modelname}, \emph{modelhour}, \emph{seasonName}, \emph{year}, \emph{procStatiSeason}, \emph{procRegion}, \emph{plotCSV}, \emph{rainfallPath}, \emph{rainfallXmlName=None}}{}~\begin{description}
\item[{\code{genStatisticalScore()}}] \leavevmode{[}It should compute the statistical scores{]}
like '' Threat Score, Equitable Threat Score, Accuracy(Hit Rate),
Bias Score, Probability Of Detection, False Alarm Rate, Odds Ratio,
Probability Of False Detection, Kuipers Skill Score, Log Odd Ratio,
Heidke Skill Score, Odd Ratio Skill Score, \& Extreame Dependency Score''
by accessing the observation and forecast data.

It should compute the statistical scores for different regions.

Finally it should store the scores variable in both csv and nc files
in appropriate directory hierarchy structure.
\begin{description}
\item[{..note:: We are replacing the -ve values with zeros of both the}] \leavevmode
observation and fcst data to make correct statistical scores.

\end{description}

\item[{Inputs}] \leavevmode{[}modelname, modelhour, seasonName, year are helps to generate the{]}
path. procStatiSeason is the partial path of process statistical
score season path.
procRegion is an absolute path to store the nc files
plotCSV is an absolute path to store the csv files.
rainfallPath is the path of the observation rainfall.
rainfallXmlName is the name of the xml file name, it is an
optional one. By default it takes `rainfall\_regrided.xml'

\item[{Outputs}] \leavevmode{[}It should store the computed statistical scores for all the{]}
regions and store it as both ncfile and csv files in the
appropriate directory hierarchy structure.

\end{description}

\end{fulllineitems}

\index{genStatisticalScoreDirs() (in module compute\_region\_statistical\_score)}

\begin{fulllineitems}
\phantomsection\label{diagnosis:compute_region_statistical_score.genStatisticalScoreDirs}\pysiglinewithargsret{\code{compute\_region\_statistical\_score.}\bfcode{genStatisticalScoreDirs}}{\emph{modelname}, \emph{modelhour}, \emph{rainfallPath}, \emph{rainfallXmlName=None}}{}~\begin{quote}\begin{description}
\item[{Func }] \leavevmode
\emph{genStatisticalScoreDirs} : It should generate the appropriate
directory hierarchy structure for `StatiScore' in both the processfiles
path and plotgraph path. In plotgraph path, it should create `CSV'
directory to store the `statistical scores' in csv file formate.

This function should call the \emph{genStatisticalScore} function to
compute and statistical score.

\end{description}\end{quote}
\begin{description}
\item[{Inputs}] \leavevmode{[}modelname and modelhour are the part of the directory hierarchy{]}
structure.
rainfallPath is the path of the observation rainfall.
rainfallXmlName is the name of the xml file name.

\end{description}

\end{fulllineitems}



\subsection{Season Statistical Score Spatial Distribution}
\label{diagnosis:season-statistical-score-spatial-distribution}
The below script \emph{compute\_season\_stati\_score\_spatial\_distribution.py} should compute the various statistical scores w.r.t each \& every lat, lon
location of particular region.
\phantomsection\label{diagnosis:module-compute_season_stati_score_spatial_distribution}\index{compute\_season\_stati\_score\_spatial\_distribution (module)}\phantomsection\label{diagnosis:module-compute_season_stati_score_spatial_distribution.py}\index{compute\_season\_stati\_score\_spatial\_distribution.py (module)}\begin{description}
\item[{Example:}] \leavevmode
Let `TS'(Threat Score) is a statistical score,
we are calculate this spatially.

\item[{Written by: Dileepkumar R}] \leavevmode
JRF- IIT DELHI

\end{description}

Date: 02.09.2011;

Updated By : Arulalan.T
Date : 17.09.2011
Date : 06.10.2011
\index{genStatisticalScorePath() (in module compute\_season\_stati\_score\_spatial\_distribution)}

\begin{fulllineitems}
\phantomsection\label{diagnosis:compute_season_stati_score_spatial_distribution.genStatisticalScorePath}\pysiglinewithargsret{\code{compute\_season\_stati\_score\_spatial\_distribution.}\bfcode{genStatisticalScorePath}}{\emph{modelname}, \emph{modelhour}, \emph{rainfallPath}, \emph{rainfallXmlName=None}}{}
\code{genStatisticalScorePath()}: It should make the existing path of
process files statistical score. Also if that is correct path means, it
should calls the function \emph{genStatisticalScoreSpatialDistribution} to
compute the statistical score in spatially distributed way.
\begin{description}
\item[{Inputs}] \leavevmode{[}modelname and modelhour are the part of the directory hierarchy{]}
structure.

\end{description}

\end{fulllineitems}

\index{genStatisticalScoreSpatialDistribution() (in module compute\_season\_stati\_score\_spatial\_distribution)}

\begin{fulllineitems}
\phantomsection\label{diagnosis:compute_season_stati_score_spatial_distribution.genStatisticalScoreSpatialDistribution}\pysiglinewithargsret{\code{compute\_season\_stati\_score\_spatial\_distribution.}\bfcode{genStatisticalScoreSpatialDistribution}}{\emph{modelname}, \emph{modelhour}, \emph{season}, \emph{year}, \emph{statiSeasonPath}, \emph{rainfallPath}, \emph{rainfallXmlName=None}, \emph{lat=None}, \emph{lon=None}}{}~\begin{quote}\begin{description}
\item[{Func }] \leavevmode
\emph{genStatisticalScoreSpatialDistribution} : It should compute the
statistical scores like '' Threat Score, Equitable Threat Score,
Accuracy(Hit Rate), Bias Score, Probability Of Detection, Odds Ratio,
False Alarm Rate, Probability Of False Detection, Kuipers Skill Score,
Log Odd Ratio, Heidke Skill Score, Odd Ratio Skill Score, \&
Extreame Dependency Score'' in spatially distributed way (i.e compute
scores in each and every lat \& lon points) by accessing the
observation and forecast data.

\end{description}\end{quote}
\begin{description}
\item[{Inputs}] \leavevmode{[}modelname, modelhour, season, year are helps to generate the{]}
path. statiSeasonPath is the partial path of process statistical
score season path.
rainfallPath is the path of the observation rainfall.
rainfallXmlName is the name of the xml file name, it is an
optional one. By default it takes `rainfall\_regrided.xml'.
lat, lon takes tuple args. If we passed it, then the model lat,
lon should be shrinked according to the passed lat,lon.
Some times it may helpful to do statistical score in spatially
distributed in particular region among the global lat,lon.

\item[{Outputs}] \leavevmode{[}It should store the computed statistical scores in spatially{]}
distributed way for all the modelhour(s) as nc files in the
appropriate directory hierarchy structure.

\end{description}

\end{fulllineitems}



\section{Diagnosis Plots}
\label{diagnosis:diagnosis-plots}
The diagnosis plots of {\hyperref[diagnosis:diagnosis]{diagnosis}} are listed below.
\begin{itemize}
\item {} 
{\hyperref[diagnosis:winds-plots]{Winds Plots}}

\item {} 
{\hyperref[diagnosis:iso-plots]{Iso Plots}}

\item {} 
{\hyperref[diagnosis:statistical-score-bar-plots]{Statistical Score Bar Plots}}

\item {} 
{\hyperref[diagnosis:statistical-score-spatial-distribution-plots]{Statistical Score Spatial Distribution Plots}}

\end{itemize}


\subsection{Winds Plots}
\label{diagnosis:winds-plots}
Generate the vector plots using U and V component of the wind data.
The below script \emph{generate\_winds\_plots.py} should generates the wind plots and save it as either png or jpg or svg in the suitable directory for month and season wise.
\phantomsection\label{diagnosis:module-generate_winds_plots}\index{generate\_winds\_plots (module)}\index{editVectorPlot() (in module generate\_winds\_plots)}

\begin{fulllineitems}
\phantomsection\label{diagnosis:generate_winds_plots.editVectorPlot}\pysiglinewithargsret{\code{generate\_winds\_plots.}\bfcode{editVectorPlot}}{\emph{modelname}, \emph{processtype}, \emph{year}, \emph{monthseason}, \emph{hour=None}, \emph{level=None}, \emph{region=None}, \emph{reference=20}, \emph{scale=1}, \emph{interval=4}, \emph{svg=0}, \emph{png=1}, \emph{latlabel='lat5'}, \emph{lonlabel='lon5'}, \emph{outpath=None}}{}~\begin{description}
\item[{{\hyperref[diagnosis:generate_winds_plots.editVectorPlot]{\code{editVectorPlot()}}}}] \leavevmode{[}To edit/reproduce any particular vector plots by{]}
passing modelname, processtype, year, month/season, hour, level(s),
region, reference point of vector, scale of vector arrow markers,
interval of the U \& V datasets, svg \& png options.

\item[{Inputs}] \leavevmode{[}modelname is the part of the directory structure.{]}
processtype is any one of the processes.for eg : `Mean Analysis',
`Mean Fcst' or `Anomaly' or `FcstSysErr', etc.,
year is year in string type.
monthseason is either month name or season name. It should find
out either it is month or season and make the correct path.
hour is the hour string which is the part of the directory
structure only for `FcstSysErr' and `Mean Fcst' process type.

level is either single level, or list of levels or `all'.
`all' means, it takes all the availableLevels from the variables.
level value must be int, float only. Not be string, other than
`all' keyword.
region is the region variable which should cut particular region
shape from the global data. By default it is None, i.e. takes
global data region itself.

reference is the reference points to be plotted in the vcs
vector plot. It must be float only.
scale is the length of the arrow markers in the vcs vector plot.
interval is the integer value, to split the U and V datasets,
to make clear view of the vector plot. By default it takes 4.

svg is the flag. If flag is set, then the vector plot should be
saved as svg formate. By default it is 0.

png is the flag. If flag is set, then the vector plot should be
saved as png formate. By default it is 1.

outpath is the absolute path, where the generated plots should be
stored. By default, it is None, that is it should save in the
appropirate plotsgraphs directory, which is generated by this
function.

\end{description}

Written By : Arulalan.T

Date : 11.09.2011
Updated: 11.12.2011

\end{fulllineitems}

\index{genMonthAnomalyDirs() (in module generate\_winds\_plots)}

\begin{fulllineitems}
\phantomsection\label{diagnosis:generate_winds_plots.genMonthAnomalyDirs}\pysiglinewithargsret{\code{generate\_winds\_plots.}\bfcode{genMonthAnomalyDirs}}{\emph{modelName}, \emph{availableMonths}}{}~\begin{description}
\item[{{\hyperref[diagnosis:generate_winds_plots.genMonthAnomalyDirs]{\code{genMonthAnomalyDirs()}}}: It should generate the directory hierarichy}] \leavevmode
structure of month anomaly in the plotsgraphspath. And calls the
function genVectorPlots to make vector plots and save it inside the
appropirate directory, by reading the u, v nc files of the appropirate
process month anomaly files path.

\item[{Inputs}] \leavevmode{[}modelName is the one of the directories name.{]}
availableMonths is the dictionary which is generated by fully
available months from the timeAxis.

\item[{..note:: It should takes the levels which is set in the global config}] \leavevmode
file, and generate the vector plots to those levels only.

\end{description}

Written By : Arulalan.T

Date : 11.09.2011
Updated: 11.12.2011

\end{fulllineitems}

\index{genMonthMeanDirs() (in module generate\_winds\_plots)}

\begin{fulllineitems}
\phantomsection\label{diagnosis:generate_winds_plots.genMonthMeanDirs}\pysiglinewithargsret{\code{generate\_winds\_plots.}\bfcode{genMonthMeanDirs}}{\emph{modelName}, \emph{availableMonths}}{}~\begin{description}
\item[{{\hyperref[diagnosis:generate_winds_plots.genMonthMeanDirs]{\code{genMonthMeanDirs()}}}: It should generate the directory hierarichy}] \leavevmode
structure of month mean in the plotsgraphspath. And calls the
function genVectorPlots to make vector plots and save it inside the
appropirate directory, by reading the u, v nc files of the appropirate
process month mean files path.

\item[{Inputs}] \leavevmode{[}modelName is the one of the directories name.{]}
availableMonths is the dictionary which is generated by fully
available months from the timeAxis.

\item[{..note:: It should takes the levels which is set in the global config}] \leavevmode
file, and generate the vector plots to those levels only.

\end{description}

Written By : Arulalan.T

Date : 11.09.2011
Updated: 11.12.2011

\end{fulllineitems}

\index{genSeasonFcstSysErrDirs() (in module generate\_winds\_plots)}

\begin{fulllineitems}
\phantomsection\label{diagnosis:generate_winds_plots.genSeasonFcstSysErrDirs}\pysiglinewithargsret{\code{generate\_winds\_plots.}\bfcode{genSeasonFcstSysErrDirs}}{\emph{modelName}, \emph{modelHour}, \emph{availableMonths}}{}~\begin{description}
\item[{{\hyperref[diagnosis:generate_winds_plots.genSeasonFcstSysErrDirs]{\code{genSeasonFcstSysErrDirs()}}}: It should generate the directory hierarichy}] \leavevmode
structure of season fcstsyserr in the plotsgraphspath. And calls the
function genVectorPlots to make vector plots and save it inside the
appropirate directory, by reading the xml file of the appropirate
process season fcstsyserr files path.

\item[{Inputs}] \leavevmode{[}modelName is the one of the directories name.{]}
modelHour is the one of the directories name.
availableMonths is the dictionary which is generated by fully
available months from the timeAxis.

\item[{..note:: It should takes the levels which is set in the global config}] \leavevmode
file, and generate the vector plots to those levels only.

\end{description}

Written By : Arulalan.T

Date : 11.09.2011
Updated: 11.12.2011

\end{fulllineitems}

\index{genSeasonMeanDirs() (in module generate\_winds\_plots)}

\begin{fulllineitems}
\phantomsection\label{diagnosis:generate_winds_plots.genSeasonMeanDirs}\pysiglinewithargsret{\code{generate\_winds\_plots.}\bfcode{genSeasonMeanDirs}}{\emph{modelName}, \emph{availableMonths}}{}~\begin{description}
\item[{{\hyperref[diagnosis:generate_winds_plots.genSeasonMeanDirs]{\code{genSeasonMeanDirs()}}}: It should generate the directory hierarichy}] \leavevmode
structure of season mean in the plotsgraphspath. And calls the
function genVectorPlots to make vector plots and save it inside the
appropirate directory, by reading the xml file of the appropirate
process season mean files path.

\item[{Inputs}] \leavevmode{[}modelName is the one of the directories name.{]}
availableMonths is the dictionary which is generated by fully
available months from the timeAxis.

\item[{..note:: It should takes the levels which is set in the global config}] \leavevmode
file, and generate the vector plots to those levels only.

\end{description}

Written By : Arulalan.T

Date : 11.09.2011
Updated: 11.12.2011

\end{fulllineitems}

\index{genVectorPlots() (in module generate\_winds\_plots)}

\begin{fulllineitems}
\phantomsection\label{diagnosis:generate_winds_plots.genVectorPlots}\pysiglinewithargsret{\code{generate\_winds\_plots.}\bfcode{genVectorPlots}}{\emph{uvar}, \emph{vvar}, \emph{upath=None}, \emph{vpath=None}, \emph{xmlpath=None}, \emph{outpath=None}, \emph{month=None}, \emph{date=None}, \emph{level=None}, \emph{region=None}, \emph{reference=20.0}, \emph{scale=1}, \emph{interval=4}, \emph{svg=0}, \emph{png=1}, \emph{latlabel='lat5'}, \emph{lonlabel='lon5'}, \emph{style='portrait'}}{}~\begin{description}
\item[{{\hyperref[diagnosis:generate_winds_plots.genVectorPlots]{\code{genVectorPlots()}}}: It should generate the vector plots in vcs}] \leavevmode
background and save it as png(by default) inside the outpath,
with some default vector properties like reference, scale and
interval.

\item[{Inputs}] \leavevmode{[}uvar is the `u' variable name{]}
vvar is the `v' variable name
upath is the `u' nc file absolute path.
vpath is the `v' nc file absolute path.
xmlpath is the xml file absolute path which must contains the
u and v vars.
outpath is the absolute path, where the generated plots should be
stored. By default, it is None. It means, it should save in the
current working directory itself.
level is either single level, or list of levels or `all'.
`all' means, it takes all the availableLevels from the variables.
level value must be int, float only. Not be string, other than
`all' keyword.
region is the region variable which should cut particular region
shape from the global data. By default it is None, i.e. takes
global data region itself.

reference is the reference points to be plotted in the vcs
vector plot. It must be float only.
scale is the length of the arrow markers in the vcs vector plot.
interval is the integer value, to split the U and V datasets,
to make clear view of the vector plot. By default it takes 4.

svg is the flag. If flag is set, then the vector plot should be
saved as svg formate. By default it is 0.

png is the flag. If flag is set, then the vector plot should be
saved as png formate. By default it is 1.

\item[{Condition}] \leavevmode{[}If we passed xmlpath, then we no need to pass upath and vpath{]}
args. uvar and vvar must be available in the passed filepath.

\end{description}

Written By : Arulalan.T

Date : 11.09.2011
Updated: 11.12.2011

\end{fulllineitems}

\index{getProcessPath() (in module generate\_winds\_plots)}

\begin{fulllineitems}
\phantomsection\label{diagnosis:generate_winds_plots.getProcessPath}\pysiglinewithargsret{\code{generate\_winds\_plots.}\bfcode{getProcessPath}}{\emph{modelname}, \emph{processtype}, \emph{year}, \emph{monthseason}, \emph{hour=None}}{}~\begin{description}
\item[{{\hyperref[diagnosis:generate_winds_plots.getProcessPath]{\code{getProcessPath()}}}: By passing fewer args and get the correct and}] \leavevmode
absolute path of the process files, which generated by automated or
manual for the purpose of this diagnosis.

\item[{Inputs}] \leavevmode{[}modelname is the part of the directory structure.{]}
processtype is any one of the processes.for eg : `Mean Analysis',
or `Mean Fcst' or `Anomaly' or `FcstSysErr', etc.,
year is year in string type.
monthseason is either month name or season name. It should find
out either it is month or season and make the correct path.
hour is the hour string which is the part of the directory
structure only for `FcstSysErr' and `Mean Fcst' process type.

\item[{Outputs}] \leavevmode{[}Return the absolute path of the process files, only if that{]}
directory is exists. Other wise it raise error.

\item[{To Do}] \leavevmode{[}Need to decide about, either it should raise error, or it should{]}
return None, if wrong args passed or process directory doesnot
exists.

\end{description}

Written By : Arulalan.T

Date : 11.09.2011
Updated: 11.12.2011

\end{fulllineitems}



\subsection{Iso Plots}
\label{diagnosis:iso-plots}
Generate the iso plots for the iso variables which are all set in the `vars.txt' file.

The below script \emph{generate\_iso\_plots} should generates the following kind of plots.
\begin{itemize}
\item {} 
iso line

\item {} 
iso fill

\item {} 
iso fill line

\end{itemize}

Finally save the generated iso plots as either png or jpg or svg in the suitable directory for month and season wise.
\phantomsection\label{diagnosis:module-generate_iso_plots}\index{generate\_iso\_plots (module)}\index{genIsoFillLinePlots() (in module generate\_iso\_plots)}

\begin{fulllineitems}
\phantomsection\label{diagnosis:generate_iso_plots.genIsoFillLinePlots}\pysiglinewithargsret{\code{generate\_iso\_plots.}\bfcode{genIsoFillLinePlots}}{\emph{var}, \emph{key}, \emph{isoLevels}, \emph{xmlpath=None}, \emph{path=None}, \emph{outpath=None}, \emph{month=None}, \emph{date=None}, \emph{level='all'}, \emph{region=None}, \emph{svg=0}, \emph{png=1}}{}~\begin{description}
\item[{{\hyperref[diagnosis:generate_iso_plots.genIsoFillLinePlots]{\code{genIsoFillLinePlots()}}}: It should generate the isoFillLine plots in}] \leavevmode
vcs background and save it as png(by default) inside the outpath,
with isoLevels (passed by user) and isoColors (default).

\item[{Inputs}] \leavevmode{[}var is the variable name.{]}
key to identify, it is which variable to make plot name.
isoLevels is the levels to plot isoFillLine and set the legend
levels in vcs.
xmlpath is the xml file absolute path.
path is the nc file absolute path.
pass any one (path or xmlpath)

outpath is the absolute path, where the generated plots should be
stored. By default, it is None. It means, it should save in the
current working directory itself.
level is either single level, or list of levels or `all'.
`all' means, it takes all the availableLevels from the variables.
level value must be int, float only. Not be string, other than
`all' keyword.
region is the region variable which should cut particular region
shape from the global data. By default it is None, i.e. takes
global data region itself.

svg is the flag. If flag is set, then the vector plot should be
saved as svg formate. By default it is 0.

png is the flag. If flag is set, then the vector plot should be
saved as png formate. By default it is 1.

\end{description}

Written By : Arulalan.T

Date : 21.09.2011
Updated: 12.12.2011

\end{fulllineitems}

\index{genIsoLinePlots() (in module generate\_iso\_plots)}

\begin{fulllineitems}
\phantomsection\label{diagnosis:generate_iso_plots.genIsoLinePlots}\pysiglinewithargsret{\code{generate\_iso\_plots.}\bfcode{genIsoLinePlots}}{\emph{var}, \emph{key}, \emph{xmlpath=None}, \emph{path=None}, \emph{outpath=None}, \emph{month=None}, \emph{date=None}, \emph{level='all'}, \emph{region=None}, \emph{svg=0}, \emph{png=1}}{}~\begin{description}
\item[{{\hyperref[diagnosis:generate_iso_plots.genIsoLinePlots]{\code{genIsoLinePlots()}}}: It should generate the isoLine plots in}] \leavevmode
vcs background and save it as png(by default) inside the outpath,
with isoLevels (find out by this method) and isoColors (default).

\item[{Inputs}] \leavevmode{[}var is the variable name.{]}\begin{quote}

key to identify, it is which variable to make plot name.
xmlpath is the xml file absolute path.
path is the nc file absolute path.
pass any one (path or xmlpath)

outpath is the absolute path, where the generated plots should be
stored. By default, it is None. It means, it should save in the
current working directory itself.
level is either single level, or list of levels or `all'.
`all' means, it takes all the availableLevels from the variables.
level value must be int, float only. Not be string, other than
`all' keyword.
region is the region variable which should cut particular region
shape from the global data. By default it is None, i.e. takes
global data region itself.

svg is the flag. If flag is set, then the vector plot should be
saved as svg formate. By default it is 0.

png is the flag. If flag is set, then the vector plot should be
saved as png formate. By default it is 1.
\end{quote}
\begin{description}
\item[{..note:: This function should find out the isoLevels to set in the}] \leavevmode
isoline plot and its legend in vcs. IsoLevels is the range of min and
max of all the levels data min and max.

\end{description}

\end{description}

Written By : Arulalan.T

Date : 21.09.2011
Updated: 12.12.2011

\end{fulllineitems}

\index{genSeasonFcstSysErrDirs() (in module generate\_iso\_plots)}

\begin{fulllineitems}
\phantomsection\label{diagnosis:generate_iso_plots.genSeasonFcstSysErrDirs}\pysiglinewithargsret{\code{generate\_iso\_plots.}\bfcode{genSeasonFcstSysErrDirs}}{\emph{modelName}, \emph{modelHour}, \emph{availableMonths}, \emph{plotLevel}}{}~\begin{description}
\item[{{\hyperref[diagnosis:generate_iso_plots.genSeasonFcstSysErrDirs]{\code{genSeasonFcstSysErrDirs()}}}: It should generate the directory hierarichy}] \leavevmode
structure of season fcstsyserr in the plotsgraphspath. And calls the
function genIsoFillLinePlots to make `isofillline' plots and save it
inside the appropirate directory, by reading the xml file of the
appropirate process season fcstsyserr files path.

It should plot for all the vars in the `isovars' which has set in the
global `vars.txt' file.

To plot isoFillLinePlot, this function should find out the isoLevels
for all the variables of all the levels and all the hours.
\begin{description}
\item[{isoLevels}] \leavevmode{[}It is a range of levels which is from the min (of data of{]}
all the hours and levels), to the max (of data of all the hours and
levels) to set the levels in the vcs plot and legend.

\end{description}

\item[{Inputs}] \leavevmode{[}modelName is the one of the directories name.{]}
modelHour is the one of the directories name.
availableMonths is the dictionary which is generated by fully
available months from the timeAxis.

\item[{..note:: It should takes the levels which is set in the global config}] \leavevmode
file, and generate the `IsoFillLine' plots to those levels only.

\end{description}

Written By : Arulalan.T

Date : 20.09.2011
Updated: 12.12.2011

\end{fulllineitems}

\index{genSeasonMeanDirs() (in module generate\_iso\_plots)}

\begin{fulllineitems}
\phantomsection\label{diagnosis:generate_iso_plots.genSeasonMeanDirs}\pysiglinewithargsret{\code{generate\_iso\_plots.}\bfcode{genSeasonMeanDirs}}{\emph{modelName}, \emph{availableMonths}, \emph{plotLevel}}{}~\begin{description}
\item[{{\hyperref[diagnosis:generate_iso_plots.genSeasonMeanDirs]{\code{genSeasonMeanDirs()}}}: It should generate the directory hierarichy}] \leavevmode
structure of season mean in the plotsgraphspath. And calls the
function genIsoLinePlots to make `isoline' plots and save it inside the
appropirate directory, by reading the xml file of the appropirate
process season mean files path.

\item[{Inputs}] \leavevmode{[}modelName is the one of the directories name.{]}
availableMonths is the dictionary which is generated by fully
available months from the timeAxis.

\item[{..note:: It should takes the levels which is set in the global config}] \leavevmode
file and generate the `isoline' plots to those levels only.

\end{description}

Written By : Arulalan.T

Date : 20.09.2011
Updated: 12.12.2011

\end{fulllineitems}



\subsection{Statistical Score Bar Plots}
\label{diagnosis:statistical-score-bar-plots}
The script \emph{generate\_statistical\_score\_bars.py} should generate the bar plots w.r.t the statistical scores for different region \& fcst hours.
Finally save the generated bar plots as either png or jpg or svg in the suitable directory for season wise.
\phantomsection\label{diagnosis:module-generate_statistical_score_bars}\index{generate\_statistical\_score\_bars (module)}\phantomsection\label{diagnosis:module-generate_statistical_score_bars.py}\index{generate\_statistical\_score\_bars.py (module)}
Date : 04.08.2011

Updated on : 28.09.2011
\index{genBarDiagrams() (in module generate\_statistical\_score\_bars)}

\begin{fulllineitems}
\phantomsection\label{diagnosis:generate_statistical_score_bars.genBarDiagrams}\pysiglinewithargsret{\code{generate\_statistical\_score\_bars.}\bfcode{genBarDiagrams}}{\emph{var}, \emph{path}, \emph{hours}, \emph{outpath=None}, \emph{bargap=0.28}, \emph{barwidth=0.8}, \emph{yticdiff=0.25}}{}~\begin{description}
\item[{\code{genBarDiagrams()}: It should generate the least directory hierarichy}] \leavevmode
structure of season statiscore in the plotsgraphspath by score name.
It will plots score values in xmgrace as bar diagram and save it
inside the appropirate directory, by reading the nc file of the
appropirate process season Region statiscore files path.

It should plot for all the vars of that statiscore nc files.

\item[{Inputs}] \leavevmode{[}var is the variable name. If var is `all' means, then it should{]}
plot the bar diagram for all the available variables in the passed
path nc or xml file.

path is an absolute nc or xml file path.

outpath is the path to store the images. If it is None means, it
should create the least (plotname)directory in the current
directory path itself and save it.

bargap is the value of the gap ratio in between each bars of each
threshold in xaxis of score bar diagram.

barwidth is the width of the each bar in xaxis of the score
bar diagram.

yticdiff is the difference of the tic levels in y axis of the
bar diagram.

\end{description}

Written By : Dileep Kumar.R, Arulalan.T

Updated on : 28.09.2011

\end{fulllineitems}

\index{genSeasonStatiScoreDirs() (in module generate\_statistical\_score\_bars)}

\begin{fulllineitems}
\phantomsection\label{diagnosis:generate_statistical_score_bars.genSeasonStatiScoreDirs}\pysiglinewithargsret{\code{generate\_statistical\_score\_bars.}\bfcode{genSeasonStatiScoreDirs}}{\emph{modelName}, \emph{modelHour}, \emph{availableMonths}}{}~\begin{description}
\item[{\code{genSeasonStatiScoreDirs()}: It should generate the directory hierarichy}] \leavevmode
structure of season statiscore in the plotsgraphspath. And calls the
function genIsoFillPlots to make `isofill' plots and save it
inside the appropirate directory, by reading the nc file of the
appropirate process season hour statiscore files path.

It should plot for all the vars of that statiscore spatial distributed
nc files.

\item[{Inputs}] \leavevmode{[}modelName is the one of the directories name.{]}
modelHour is the one of the directories name.
availableMonths is the dictionary which is generated by fully
available months from the timeAxis.

\end{description}

Written By : Arulalan.T

Date : 27.09.2011

\end{fulllineitems}



\subsection{Statistical Score Spatial Distribution Plots}
\label{diagnosis:statistical-score-spatial-distribution-plots}
The script \emph{generate\_stati\_score\_spatial\_distribution\_plots.py} should generate the iso fill plots w.r.t the statistical scores for different fcst hours.
Finally save the generated iso fill plots as either png or jpg or svg in the suitable directory for season wise.
\phantomsection\label{diagnosis:module-generate_stati_score_spatial_distribution_plots}\index{generate\_stati\_score\_spatial\_distribution\_plots (module)}\index{genIsoFillPlots() (in module generate\_stati\_score\_spatial\_distribution\_plots)}

\begin{fulllineitems}
\phantomsection\label{diagnosis:generate_stati_score_spatial_distribution_plots.genIsoFillPlots}\pysiglinewithargsret{\code{generate\_stati\_score\_spatial\_distribution\_plots.}\bfcode{genIsoFillPlots}}{\emph{var}, \emph{path}, \emph{outpath=None}, \emph{region=None}, \emph{svg=0}, \emph{png=1}}{}~\begin{description}
\item[{{\hyperref[diagnosis:generate_stati_score_spatial_distribution_plots.genIsoFillPlots]{\code{genIsoFillPlots()}}}: It should generate the directory least hierarichy}] \leavevmode
structure of season statiscore in the plotsgraphspath,by the plotname.

It should plot for all the vars of that statiscore spatial distributed
nc files.

\item[{Inputs}] \leavevmode{[}var is the variable name. If var is `all' means, then it should{]}\begin{quote}

plot the isofill for all the available variables in the passed
path nc or xml file.

path is an absolute nc or xml file path.

outpath is the path to store the images. If it is None means, it
should create the least (plotname)directory in the current
directory path itself and save it.

region to extract the region from the var data.

if svg is 1, then plot should be saved as svg.
if png is 1, then plot should be saved as png.
\end{quote}
\begin{description}
\item[{..note:: isoLevels and isoColors are set inbuilt (some default) range of}] \leavevmode
levels and colors with respect to the variable name of statistical
scores.

\end{description}

\end{description}

Written By : Dileep Kumar.R, Arulalan.T

Date : 26.09.2011

\end{fulllineitems}

\index{genSeasonStatiScoreDirs() (in module generate\_stati\_score\_spatial\_distribution\_plots)}

\begin{fulllineitems}
\phantomsection\label{diagnosis:generate_stati_score_spatial_distribution_plots.genSeasonStatiScoreDirs}\pysiglinewithargsret{\code{generate\_stati\_score\_spatial\_distribution\_plots.}\bfcode{genSeasonStatiScoreDirs}}{\emph{modelName}, \emph{modelHour}, \emph{availableMonths}}{}~\begin{description}
\item[{{\hyperref[diagnosis:generate_stati_score_spatial_distribution_plots.genSeasonStatiScoreDirs]{\code{genSeasonStatiScoreDirs()}}}: It should generate the directory hierarichy}] \leavevmode
structure of season statiscore in the plotsgraphspath. And calls the
function genIsoFillPlots to make `isofill' plots and save it
inside the appropirate directory, by reading the nc file of the
appropirate process season hour statiscore files path.

It should plot for all the vars of that statiscore spatial distributed
nc files.

\item[{Inputs}] \leavevmode{[}modelName is the one of the directories name.{]}
modelHour is the one of the directories name.
availableMonths is the dictionary which is generated by fully
available months from the timeAxis.

\end{description}

Written By : Arulalan.T

Date : 26.09.2011

\end{fulllineitems}



\section{Statistical Scores}
\label{diagnosis:statistical-scores}

\subsection{Contigency Table \& Related Statistical Scores}
\label{diagnosis:contigency-table-related-statistical-scores}
The module \emph{ctgfunction.py} should helps to calculate the contigency table and its related statistical scores.

For eg : Threat Score, Bias Score, Proabability of detection and more.
\phantomsection\label{diagnosis:module-ctgfunction}\index{ctgfunction (module)}\index{accuracy() (in module ctgfunction)}

\begin{fulllineitems}
\phantomsection\label{diagnosis:ctgfunction.accuracy}\pysiglinewithargsret{\code{ctgfunction.}\bfcode{accuracy}}{\emph{obs=None}, \emph{fcst=None}, \emph{th=None}, \emph{**ctg}}{}~\begin{description}
\item[{{\hyperref[diagnosis:ctgfunction.accuracy]{\code{accuracy()}}}:Hit Rate ,the most direct and intuitive measure of the}] \leavevmode\begin{quote}

accuracy of categorical forecasts is hit rate. The average correspond-
ence between individual nforecasts and the events they predict. Scalar
measures  of accuracy are meant to summarize,in a single number, the
overall quality of a set of forecasts. Can be mislead, since it is
heavily influenced by the most common category, usually ``no event''
in the case of rare weather.
\begin{quote}
\begin{description}
\item[{Accuracy= (a+d)/(a+b+c+d); `a' -hits, `b'-false alarm,}] \leavevmode
`c'-misses, \& `d'- correct negatives

\end{description}
\end{quote}
\end{quote}
\begin{description}
\item[{Inputs: obs- the observed values has to be a numpy array(or whatever}] \leavevmode\begin{quote}

you decide)
\end{quote}

fcst - the forecast values
th  - the threshold value for which the contingency table needs
\begin{quote}

to be created (floating point value please!!)
\end{quote}

By default obs, fcst, th are None. Instead of passing obs, fcst,
and th values, you can pass `ctg\_table' kwarg as 2x2 matrix value.

\item[{Outputs:}] \leavevmode
Range: 0 to 1

Perfect Score: 1

\item[{Reference: ``Statistical Methods in the Atmospheric Sciences'',}] \leavevmode
Daniel S Wilks, ACADEMIC PRESS(Page No:236-240)

\end{description}

Links : \href{http://www.cawcr.gov.au/projects/verification/}{http://www.cawcr.gov.au/projects/verification/}
\begin{description}
\item[{Written by: Dileepkumar R,}] \leavevmode
JRF, IIT Delhi

\end{description}

Date: 24/02/2011

\end{description}

\end{fulllineitems}

\index{bias\_score() (in module ctgfunction)}

\begin{fulllineitems}
\phantomsection\label{diagnosis:ctgfunction.bias_score}\pysiglinewithargsret{\code{ctgfunction.}\bfcode{bias\_score}}{\emph{obs=None}, \emph{fcst=None}, \emph{th=None}, \emph{**ctg}}{}~\begin{description}
\item[{{\hyperref[diagnosis:ctgfunction.bias_score]{\code{bias\_score()}}}: Bias score(frequency bias)-Measures the correspondence}] \leavevmode
between the average forecast and the average observed value of the
predictand. This is different from accuracy, which measures the
average correspondence between individual pairs of forecasts and
observations.
\begin{quote}

Bias= (a+b)/(a+c); `a' -hits, `b'-false alarm, \& `c'-misses
\end{quote}

\item[{Inputs: obs- the observed values has to be a numpy array(or whatever}] \leavevmode\begin{quote}

you decide)
\end{quote}

fcst - the forecast values
th  - the threshold value for which the contingency table needs
\begin{quote}

to be created (floating point value please!!)
\end{quote}

By default obs, fcst, th are None. Instead of passing obs, fcst,
and th values, you can pass `ctg\_table' kwarg as 2x2 matrix value.

\item[{Outputs:}] \leavevmode
Range: 0 to infinity

Perfect score: 1

\item[{Reference: ``Statistical Methods in the Atmospheric Sciences'',}] \leavevmode
Daniel S Wilks, ACADEMIC PRESS(Page No: 241)

\end{description}

Links : \href{http://www.cawcr.gov.au/projects/verification/}{http://www.cawcr.gov.au/projects/verification/}
\begin{description}
\item[{Written by: Dileepkumar R,}] \leavevmode
JRF, IIT Delhi

\end{description}

Date: 24/02/2011

\end{fulllineitems}

\index{contingency\_table\_2x2() (in module ctgfunction)}

\begin{fulllineitems}
\phantomsection\label{diagnosis:ctgfunction.contingency_table_2x2}\pysiglinewithargsret{\code{ctgfunction.}\bfcode{contingency\_table\_2x2}}{\emph{obs}, \emph{fcst}, \emph{th}}{}~\begin{description}
\item[{{\hyperref[diagnosis:ctgfunction.contingency_table_2x2]{\code{contingency\_table\_2x2()}}}:Creates the 2x2 contigency table useful for}] \leavevmode
forecast verification. From 2x2 condigency table we can find thse
statistical scores such as Hit Rate(HR), Bias(BS),
Threat Score(TS), Odds Ratio(ODR)...etc

\item[{Inputs: obs- the observed values has to be a numpy array(or whatever}] \leavevmode\begin{quote}

you decide)
\end{quote}

fcst - the forecast values
th  - the threshold value for which the contingency table needs
\begin{quote}

to be created (floating point value please!!)
\end{quote}

\item[{Outputs: A list of values {[}a, b, c, d{]}}] \leavevmode\begin{description}
\item[{where a = No of values such that both observed and}] \leavevmode
predicted \textgreater{} threshold

\item[{b = No of values such that observed is \textless{} threshold and}] \leavevmode
predicted \textgreater{} threshold

\item[{c = No of values such that observed is \textgreater{} threshold and}] \leavevmode
predicted \textless{} threshold

\item[{d = No of values such that both observed and}] \leavevmode
predicted \textless{} threshold

\end{description}

\end{description}

Usage:
\begin{description}
\item[{example:}] \leavevmode
From the contingency table the following statistics can be
calculated.
\textgreater{}\textgreater{}\textgreater{} HR = (a+d)/(a+b+c+d)
\textgreater{}\textgreater{}\textgreater{} ETS = a/(a+b+c)
\textgreater{}\textgreater{}\textgreater{} BS = (a+b)/(a+c)
\textgreater{}\textgreater{}\textgreater{} ODR = (a*d)/(b*c)

\item[{Reference: ``Statistical Methods in the Atmospheric Sciences'',}] \leavevmode
Daniel S Wilks, ACADEMIC PRESS

\end{description}

Links: \href{http://www.cawcr.gov.au/projects/verification/\#Atger\_2001}{http://www.cawcr.gov.au/projects/verification/\#Atger\_2001}
\begin{description}
\item[{Written by: Dileepkumar R,}] \leavevmode
JRF, IIT Delhi

\end{description}

Date: 24/02/2011

\end{fulllineitems}

\index{eds() (in module ctgfunction)}

\begin{fulllineitems}
\phantomsection\label{diagnosis:ctgfunction.eds}\pysiglinewithargsret{\code{ctgfunction.}\bfcode{eds}}{\emph{obs=None}, \emph{fcst=None}, \emph{th=None}, \emph{**ctg}}{}~\begin{description}
\item[{:func:'eds':Extreme dependency score, converges to 2n-1 as event}] \leavevmode
frequency approaches 0, where n is a parameter describing how
fast the hit rate converges to zero for rarer events. EDS is
independent of bias, so should be presented together with the
frequency bias.
\begin{quote}

EDS=\{2Log{[}(a+c)/(a+b+c+d){]}/Log(a/(a+b+c+d))\}-1; `a' -hits,
`b'-false alarm, `c'-misses, \& `d'- correct negatives
\end{quote}

\item[{Inputs: obs- the observed values has to be a numpy array(or whatever}] \leavevmode\begin{quote}

you decide)
\end{quote}

fcst - the forecast values
th  - the threshold value for which the contingency table needs
\begin{quote}

to be created (floating point value please!!)
\end{quote}

By default obs, fcst, th are None. Instead of passing obs, fcst,
and th values, you can pass `ctg\_table' kwarg as 2x2 matrix value.

\end{description}

Outputs:
\begin{quote}

Range: -1 to 1, 0 indicate no skill.

Perfect Score: 1
\end{quote}
\begin{description}
\item[{Reference: {[}1{]}Stephenson D.B., B. Casati, C.A.T. Ferro and}] \leavevmode
C.A. Wilson, 2008: The extreme dependency score:
a non-vanishing measure for forecasts of rare events.
\begin{quote}

Meteorol. Appl., 15, 41-50.
\end{quote}

\end{description}

Link:    \href{http://www.cawcr.gov.au/projects/verification}{http://www.cawcr.gov.au/projects/verification}.
\begin{description}
\item[{Written by: Dileepkumar R,}] \leavevmode
JRF, IIT Delhi

\end{description}

Date: 24/02/2011

\end{fulllineitems}

\index{ets() (in module ctgfunction)}

\begin{fulllineitems}
\phantomsection\label{diagnosis:ctgfunction.ets}\pysiglinewithargsret{\code{ctgfunction.}\bfcode{ets}}{\emph{obs=None}, \emph{fcst=None}, \emph{th=None}, \emph{**ctg}}{}~\begin{description}
\item[{{\hyperref[diagnosis:ctgfunction.ets]{\code{ets()}}}:Equitable threat score (Gilbert skill score), the number of}] \leavevmode
forecasts of the event correct by chance, `a\_random',is determined
by assuming that the forecasts are totally independent of the obs-
ervations, and forecast will match the observation only by chance.
This is one form of an unskilled forecast, which can be generated
by just guessing what will happen.
\begin{quote}

ETS=(a-a\_random)/(a+c+b-a\_random)
a\_random={[}(a+c)(a+b){]}/(a+b+c+d); `a' -hits, `b'-false alarm,
\begin{quote}

`c'-misses, \& `d'- correct negatives
\end{quote}
\end{quote}

\item[{Inputs: obs- the observed values has to be a numpy array(or whatever}] \leavevmode\begin{quote}

you decide)
\end{quote}

fcst - the forecast values
th  - the threshold value for which the contingency table needs
\begin{quote}

to be created (floating point value please!!)
\end{quote}

By default obs, fcst, th are None. Instead of passing obs, fcst,
and th values, you can pass `ctg\_table' kwarg as 2x2 matrix value.

\end{description}

Outputs:
\begin{quote}

Range: -1/3 to 1, 0 indicate no skill.

Perfect Score: 1
\end{quote}

Links: \href{http://www.cawcr.gov.au/projects/verification/}{http://www.cawcr.gov.au/projects/verification/}
\begin{description}
\item[{Written by: Dileepkumar R,}] \leavevmode
JRF, IIT Delhi

\end{description}

Date: 24/02/2011

\end{fulllineitems}

\index{far() (in module ctgfunction)}

\begin{fulllineitems}
\phantomsection\label{diagnosis:ctgfunction.far}\pysiglinewithargsret{\code{ctgfunction.}\bfcode{far}}{\emph{obs=None}, \emph{fcst=None}, \emph{th=None}, \emph{**ctg}}{}~\begin{description}
\item[{{\hyperref[diagnosis:ctgfunction.far]{\code{far()}}}:False alarm ratio(FAR)-Proportion of forecast events that fail}] \leavevmode
to materialize.
\begin{quote}

FAR= b/(a+b); `a' -hits, \& `b'-false alarm
\end{quote}

\item[{Inputs: obs- the observed values has to be a numpy array(or whatever}] \leavevmode\begin{quote}

you decide)
\end{quote}

fcst - the forecast values
th  - the threshold value for which the contingency table needs
\begin{quote}

to be created (floating point value please!!)
\end{quote}

By default obs, fcst, th are None. Instead of passing obs, fcst,
and th values, you can pass `ctg\_table' kwarg as 2x2 matrix value.

\end{description}

Outputs:
\begin{quote}

Range: 0 to 1

Perfect Score: 0
\end{quote}
\begin{description}
\item[{Reference: ``Statistical Methods in the Atmospheric Sciences'',}] \leavevmode
Daniel S Wilks, ACADEMIC PRESS(Page No: 240-241)

\end{description}

Links: \href{http://www.cawcr.gov.au/projects/verification/}{http://www.cawcr.gov.au/projects/verification/}
\begin{description}
\item[{Written by: Dileepkumar R,}] \leavevmode
JRF, IIT Delhi

\end{description}

Date: 24/02/2011

\end{fulllineitems}

\index{hss() (in module ctgfunction)}

\begin{fulllineitems}
\phantomsection\label{diagnosis:ctgfunction.hss}\pysiglinewithargsret{\code{ctgfunction.}\bfcode{hss}}{\emph{obs=None}, \emph{fcst=None}, \emph{th=None}, \emph{**ctg}}{}~\begin{description}
\item[{{\hyperref[diagnosis:ctgfunction.hss]{\code{hss()}}}:Heidke skill score (Cohen's k), the reference accuracy}] \leavevmode
measure in the Heidke score is the hit rate that would be
achieved by random forecasts, subject to the constraint that
the marginal distributions of forecasts and observations
characterizing the contingency table for the random forecasts,
P(Yi) and p(oj), are the same as the marginal distributions in
the actual verification data set.
\begin{quote}

HSS= 2.( a d - bc)/{[}(a + c)(c + d) + (a + b)(b + d){]};
\end{quote}

`a' -hits, `b'-false alarm, `c'-misses, \& `d'- correct  negatives

\item[{Inputs: obs- the observed values has to be a numpy array(or whatever}] \leavevmode\begin{quote}

you decide)
\end{quote}

fcst - the forecast values
th  - the threshold value for which the contingency table needs
\begin{quote}

to be created (floating point value please!!)
\end{quote}

By default obs, fcst, th are None. Instead of passing obs, fcst,
and th values, you can pass `ctg\_table' kwarg as 2x2 matrix value.

\end{description}

Outputs:
\begin{quote}

Range: -infinity to 1, 0 indicate no skill.

Perfect Score: 1
\end{quote}
\begin{description}
\item[{Reference: ``Statistical Methods in the Atmospheric Sciences'',}] \leavevmode
Daniel S Wilks, ACADEMIC PRESS(Page No: 248-249)

\end{description}

Links: \href{http://www.cawcr.gov.au/projects/verification/}{http://www.cawcr.gov.au/projects/verification/}
\begin{description}
\item[{Written by: Dileepkumar R,}] \leavevmode
JRF, IIT Delhi

\end{description}

Date: 24/02/2011

\end{fulllineitems}

\index{kss() (in module ctgfunction)}

\begin{fulllineitems}
\phantomsection\label{diagnosis:ctgfunction.kss}\pysiglinewithargsret{\code{ctgfunction.}\bfcode{kss}}{\emph{obs=None}, \emph{fcst=None}, \emph{th=None}, \emph{**ctg}}{}~\begin{description}
\item[{{\hyperref[diagnosis:ctgfunction.kss]{\code{kss()}}}:Hanssen and Kuipers discriminant(Kuipers Skill Score), the}] \leavevmode
contribution made to the Kuipers score by a correct ``no'' or
``yes'' forecast increases as the event is more or less likely,
respectively.A drawback of this score is that it tends to converge
to the POD for rare events, because the value of ``d'' becomes very
large.
\begin{quote}

HK= (ad-bc)/{[}(a + c)(b + d){]}; `a' -hits, `b'-false alarm,
`c'-misses, \& `d'- correct negatives
\end{quote}

\item[{Inputs: obs- the observed values has to be a numpy array(or whatever}] \leavevmode\begin{quote}

you decide)
\end{quote}

fcst - the forecast values
th  - the threshold value for which the contingency table needs
\begin{quote}

to be created (floating point value please!!)
\end{quote}

By default obs, fcst, th are None. Instead of passing obs, fcst,
and th values, you can pass `ctg\_table' kwarg as 2x2 matrix value.

\end{description}

Outputs:
\begin{quote}

Range:-1 to 1
Perfect Score: 1
\end{quote}
\begin{description}
\item[{Reference: ``Statistical Methods in the Atmospheric Sciences'',}] \leavevmode
Daniel S Wilks, ACADEMIC PRESS(Page No: 249-250)

\end{description}

Links: \href{http://www.cawcr.gov.au/projects/verification/}{http://www.cawcr.gov.au/projects/verification/}
\begin{description}
\item[{Written by: Dileepkumar R,}] \leavevmode
JRF, IIT Delhi

\end{description}

Date: 24/02/2011

\end{fulllineitems}

\index{logodr() (in module ctgfunction)}

\begin{fulllineitems}
\phantomsection\label{diagnosis:ctgfunction.logodr}\pysiglinewithargsret{\code{ctgfunction.}\bfcode{logodr}}{\emph{obs=None}, \emph{fcst=None}, \emph{th=None}, \emph{**ctg}}{}~\begin{description}
\item[{{\hyperref[diagnosis:ctgfunction.logodr]{\code{logodr()}}}: Log Odds ratio, LOR is the logaritham of odds ratio.}] \leavevmode
When the sample is small/moderate it is better to use Log
Odds Ratio. It is a good tool for finding associations
between variables.
\begin{quote}
\begin{description}
\item[{LOR=Log(Odds Ratio); Odds Ratio=ad/bc; `a' -hits,}] \leavevmode
`b'-false alarm, `c'-misses, \& `d'- correct negatives

\end{description}
\end{quote}

\item[{Inputs: obs- the observed values has to be a numpy array(or whatever}] \leavevmode\begin{quote}

you decide)
\end{quote}

fcst - the forecast values
th  - the threshold value for which the contingency table needs
\begin{quote}

to be created (floating point value please!!)
\end{quote}

By default obs, fcst, th are None. Instead of passing obs, fcst,
and th values, you can pass `ctg\_table' kwarg as 2x2 matrix value.

\end{description}

Outputs:
\begin{quote}

Range: -infinity to infinity, 0 indicate no skill.

Perfect Score: infinity
\end{quote}
\begin{description}
\item[{Written by: Dileepkumar R,}] \leavevmode
JRF, IIT Delhi

\end{description}

Date: 24/02/2011

\end{fulllineitems}

\index{odr() (in module ctgfunction)}

\begin{fulllineitems}
\phantomsection\label{diagnosis:ctgfunction.odr}\pysiglinewithargsret{\code{ctgfunction.}\bfcode{odr}}{\emph{obs=None}, \emph{fcst=None}, \emph{th=None}, \emph{**ctg}}{}~\begin{description}
\item[{{\hyperref[diagnosis:ctgfunction.odr]{\code{odr()}}}:Odds ratio, the odds ratio is the ratio of the odds of an}] \leavevmode
event occurring in one group to the odds of it occurring in
another group.cThe term is also used to refer to sample-based
estimates of this ratio.Do not use if any of the cells in the
contingency table are equal to 0. The logarithm of the odds
ratio is often used instead of the original value.Used widely
in medicine but not yet in meteorology.
\begin{quote}
\begin{description}
\item[{OD= ad/bc; `a' -hits, `b'-false alarm, `c'-misses, \&}] \leavevmode
`d'- correct negatives

\end{description}
\end{quote}

\item[{Inputs: obs- the observed values has to be a numpy array(or whatever}] \leavevmode\begin{quote}

you decide)
\end{quote}

fcst - the forecast values
th  - the threshold value for which the contingency table needs
\begin{quote}

to be created (floating point value please!!)
\end{quote}

By default obs, fcst, th are None. Instead of passing obs, fcst,
and th values, you can pass `ctg\_table' kwarg as 2x2 matrix value.

\end{description}

Outputs:
\begin{quote}

Range: 0 to infinity, 1 indicate no skill

Perfect Score: infinity
\end{quote}

Links: \href{http://www.cawcr.gov.au/projects/verification/}{http://www.cawcr.gov.au/projects/verification/}
\begin{description}
\item[{Written by: Dileepkumar R,}] \leavevmode
JRF, IIT Delhi

\end{description}

Date: 24/02/2011

\end{fulllineitems}

\index{orss() (in module ctgfunction)}

\begin{fulllineitems}
\phantomsection\label{diagnosis:ctgfunction.orss}\pysiglinewithargsret{\code{ctgfunction.}\bfcode{orss}}{\emph{obs=None}, \emph{fcst=None}, \emph{th=None}, \emph{**ctg}}{}~\begin{description}
\item[{{\hyperref[diagnosis:ctgfunction.orss]{\code{orss()}}}: Odds ratio skill score (Yule's Q), this score was proposed}] \leavevmode
long ago as a `measure of association' by the statistician
G. U. Yule (Yule 1900) and is referred to as Yule's Q. It is
based entirely on the joint conditional probabilities,
and so is not influenced in any way by the marginal totals.
\begin{description}
\item[{ORSS= (ad-bc)/(ad+bc); `a' -hits, `b'-false alarm, `c'-misses,}] \leavevmode
\& `d'- correct negatives

\end{description}

\item[{Inputs: obs- the observed values has to be a numpy array(or whatever}] \leavevmode\begin{quote}

you decide)
\end{quote}

fcst - the forecast values
th  - the threshold value for which the contingency table needs
\begin{quote}

to be created (floating point value please!!)
\end{quote}

By default obs, fcst, th are None. Instead of passing obs, fcst,
and th values, you can pass `ctg\_table' kwarg as 2x2 matrix value.

\end{description}

Outputs:
\begin{quote}

Range: -1 to 1, 0 indicates no skill

Perfect Score: 1
\end{quote}
\begin{description}
\item[{Reference: Stephenson, D.B., 2000: Use of the ``odds ratio'' for diagnosing}] \leavevmode
forecast skill. Wea. Forecasting, 15, 221-232.

\end{description}

Links: \href{http://www.cawcr.gov.au/projects/verification/}{http://www.cawcr.gov.au/projects/verification/}
\begin{description}
\item[{Written by: Dileepkumar R,}] \leavevmode
JRF, IIT Delhi

\end{description}

Date: 24/02/2011

\end{fulllineitems}

\index{pod() (in module ctgfunction)}

\begin{fulllineitems}
\phantomsection\label{diagnosis:ctgfunction.pod}\pysiglinewithargsret{\code{ctgfunction.}\bfcode{pod}}{\emph{obs=None}, \emph{fcst=None}, \emph{th=None}, \emph{**ctg}}{}~\begin{description}
\item[{{\hyperref[diagnosis:ctgfunction.pod]{\code{pod()}}}:Probability of detection(POD), simply the fraction of those}] \leavevmode
occasions when the forecast event occurred on which it was also
forecast.
\begin{quote}

POD= a/(a+c); `a' -hits, \& `c'-misses
\end{quote}

\item[{Inputs: obs- the observed values has to be a numpy array(or whatever}] \leavevmode\begin{quote}

you decide)
\end{quote}

fcst - the forecast values
th  - the threshold value for which the contingency table needs
\begin{quote}

to be created (floating point value please!!)
\end{quote}

By default obs, fcst, th are None. Instead of passing obs, fcst,
and th values, you can pass `ctg\_table' kwarg as 2x2 matrix value.

\item[{Outputs:}] \leavevmode
Range: 0 to 1

Perfect Score: 1

\item[{Reference: ``Statistical Methods in the Atmospheric Sciences'',}] \leavevmode
Daniel S Wilks, ACADEMIC PRESS(Page No: 240)

\end{description}

Links: \href{http://www.cawcr.gov.au/projects/verification/}{http://www.cawcr.gov.au/projects/verification/}
\begin{description}
\item[{Written by: Dileepkumar R,}] \leavevmode
JRF, IIT Delhi

\end{description}

Date: 24/02/2011

\end{fulllineitems}

\index{pofd() (in module ctgfunction)}

\begin{fulllineitems}
\phantomsection\label{diagnosis:ctgfunction.pofd}\pysiglinewithargsret{\code{ctgfunction.}\bfcode{pofd}}{\emph{obs=None}, \emph{fcst=None}, \emph{th=None}, \emph{**ctg}}{}~\begin{description}
\item[{\code{podf()}:Probability of false detection (false alarm rate), measures}] \leavevmode
the fraction of false alarms given the event did not occur.
\begin{quote}

POFD=b/(d+b); `b'-false alarm \& `d'- correct negatives
\end{quote}

\item[{Inputs: obs - the observed values has to be a numpy array(or whatever}] \leavevmode\begin{quote}

you decide)
\end{quote}

fcst - the forecast values
th  - the threshold value for which the contingency table needs
\begin{quote}

to be created (floating point value please!!)
\end{quote}

By default obs, fcst, th are None. Instead of passing obs, fcst,
and th values, you can pass `ctg\_table' kwarg as 2x2 matrix value.

\end{description}

Outputs:
\begin{quote}

Range: 0 to 1

Perfect Score: 0
\end{quote}

Links: \href{http://www.cawcr.gov.au/projects/verification/}{http://www.cawcr.gov.au/projects/verification/}

\end{fulllineitems}

\index{ts() (in module ctgfunction)}

\begin{fulllineitems}
\phantomsection\label{diagnosis:ctgfunction.ts}\pysiglinewithargsret{\code{ctgfunction.}\bfcode{ts}}{\emph{obs=None}, \emph{fcst=None}, \emph{th=None}, \emph{**ctg}}{}~\begin{description}
\item[{:func:'ts':    Threat Score (Critical Success Index), a frequently used}] \leavevmode
alternative to the hit rate, particularly when the event to be
forecast (as the ``yes'' event) occurs substantially less frequently
than the non-occurrence (``no'').
\begin{quote}

TS=a/(a+b+c); `a' -hits, `b'-false alarm, \&'c'-misses
\end{quote}

\item[{Inputs: obs- the observed values has to be a numpy array(or whatever}] \leavevmode\begin{quote}

you decide)
\end{quote}

fcst - the forecast values
th  - the threshold value for which the contingency table needs
\begin{quote}

to be created (floating point value please!!)
\end{quote}

By default obs, fcst, th are None. Instead of passing obs, fcst,
and th values, you can pass `ctg\_table' kwarg as 2x2 matrix value.

\item[{Outputs:}] \leavevmode
Range: 0 to 1

Perfect Score: 1

\item[{Reference: ``Statistical Methods in the Atmospheric Sciences'',}] \leavevmode
Daniel S Wilks, ACADEMIC PRESS(Page No: 240)

\end{description}

Links: \href{http://www.cawcr.gov.au/projects/verification/}{http://www.cawcr.gov.au/projects/verification/}
\begin{description}
\item[{Written by: Dileepkumar R,}] \leavevmode
JRF, IIT Delhi

\end{description}

Date: 24/02/2011

\end{fulllineitems}



\section{More}
\label{diagnosis:more}
More utilities will be added and optimized in near future.


\chapter{Indices and tables}
\label{index:indices-and-tables}\begin{itemize}
\item {} 
\emph{genindex}

\item {} 
\emph{modindex}

\item {} 
\emph{search}

\end{itemize}


\renewcommand{\indexname}{Python Module Index}
\begin{theindex}
\def\bigletter#1{{\Large\sffamily#1}\nopagebreak\vspace{1mm}}
\bigletter{c}
\item {\texttt{collect\_season\_fcst\_rainfall}}, \pageref{diagnosis:module-collect_season_fcst_rainfall}
\item {\texttt{collect\_season\_fcst\_rainfall.py}}, \pageref{diagnosis:module-collect_season_fcst_rainfall.py}
\item {\texttt{compute\_month\_anomaly}}, \pageref{diagnosis:module-compute_month_anomaly}
\item {\texttt{compute\_month\_anomaly.py}}, \pageref{diagnosis:module-compute_month_anomaly.py}
\item {\texttt{compute\_month\_fcst\_sys\_error}}, \pageref{diagnosis:module-compute_month_fcst_sys_error}
\item {\texttt{compute\_month\_fcst\_sys\_error.py}}, \pageref{diagnosis:module-compute_month_fcst_sys_error.py}
\item {\texttt{compute\_month\_mean}}, \pageref{diagnosis:module-compute_month_mean}
\item {\texttt{compute\_month\_mean.py}}, \pageref{diagnosis:module-compute_month_mean.py}
\item {\texttt{compute\_region\_statistical\_score}}, \pageref{diagnosis:module-compute_region_statistical_score}
\item {\texttt{compute\_region\_statistical\_score.py}}, \pageref{diagnosis:module-compute_region_statistical_score.py}
\item {\texttt{compute\_season\_mean}}, \pageref{diagnosis:module-compute_season_mean}
\item {\texttt{compute\_season\_mean.py}}, \pageref{diagnosis:module-compute_season_mean.py}
\item {\texttt{compute\_season\_stati\_score\_spatial\_distribution}}, \pageref{diagnosis:module-compute_season_stati_score_spatial_distribution}
\item {\texttt{compute\_season\_stati\_score\_spatial\_distribution.py}}, \pageref{diagnosis:module-compute_season_stati_score_spatial_distribution.py}
\item {\texttt{ctgfunction}}, \pageref{diagnosis:module-ctgfunction}
\indexspace
\bigletter{g}
\item {\texttt{generate\_iso\_plots}}, \pageref{diagnosis:module-generate_iso_plots}
\item {\texttt{generate\_stati\_score\_spatial\_distribution\_plots}}, \pageref{diagnosis:module-generate_stati_score_spatial_distribution_plots}
\item {\texttt{generate\_statistical\_score\_bars}}, \pageref{diagnosis:module-generate_statistical_score_bars}
\item {\texttt{generate\_statistical\_score\_bars.py}}, \pageref{diagnosis:module-generate_statistical_score_bars.py}
\item {\texttt{generate\_winds\_plots}}, \pageref{diagnosis:module-generate_winds_plots}
\indexspace
\bigletter{p}
\item {\texttt{plot}}, \pageref{diagnosisutils:module-plot}
\indexspace
\bigletter{t}
\item {\texttt{timeutils}}, \pageref{diagnosisutils:module-timeutils}
\indexspace
\bigletter{x}
\item {\texttt{xml\_data\_access}}, \pageref{diagnosisutils:module-xml_data_access}
\end{theindex}

\renewcommand{\indexname}{Index}
\printindex
\end{document}
