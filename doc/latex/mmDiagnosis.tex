% Generated by Sphinx.
\def\sphinxdocclass{report}
\documentclass[letterpaper,10pt,english]{sphinxmanual}
\usepackage[utf8]{inputenc}
\DeclareUnicodeCharacter{00A0}{\nobreakspace}
\usepackage{cmap}
\usepackage[T1]{fontenc}
\usepackage{babel}
\usepackage{times}
\usepackage[Bjarne]{fncychap}
\usepackage{longtable}
\usepackage{sphinx}
\usepackage{multirow}


\title{mmDiagnosis Documentation}
\date{April 13, 2015}
\release{1a}
\author{Arulalan.T, Dr.Krishna AchutaRao, Dileepkumar.R}
\newcommand{\sphinxlogo}{}
\renewcommand{\releasename}{Release}
\makeindex

\makeatletter
\def\PYG@reset{\let\PYG@it=\relax \let\PYG@bf=\relax%
    \let\PYG@ul=\relax \let\PYG@tc=\relax%
    \let\PYG@bc=\relax \let\PYG@ff=\relax}
\def\PYG@tok#1{\csname PYG@tok@#1\endcsname}
\def\PYG@toks#1+{\ifx\relax#1\empty\else%
    \PYG@tok{#1}\expandafter\PYG@toks\fi}
\def\PYG@do#1{\PYG@bc{\PYG@tc{\PYG@ul{%
    \PYG@it{\PYG@bf{\PYG@ff{#1}}}}}}}
\def\PYG#1#2{\PYG@reset\PYG@toks#1+\relax+\PYG@do{#2}}

\expandafter\def\csname PYG@tok@gd\endcsname{\def\PYG@tc##1{\textcolor[rgb]{0.63,0.00,0.00}{##1}}}
\expandafter\def\csname PYG@tok@gu\endcsname{\let\PYG@bf=\textbf\def\PYG@tc##1{\textcolor[rgb]{0.50,0.00,0.50}{##1}}}
\expandafter\def\csname PYG@tok@gt\endcsname{\def\PYG@tc##1{\textcolor[rgb]{0.00,0.27,0.87}{##1}}}
\expandafter\def\csname PYG@tok@gs\endcsname{\let\PYG@bf=\textbf}
\expandafter\def\csname PYG@tok@gr\endcsname{\def\PYG@tc##1{\textcolor[rgb]{1.00,0.00,0.00}{##1}}}
\expandafter\def\csname PYG@tok@cm\endcsname{\let\PYG@it=\textit\def\PYG@tc##1{\textcolor[rgb]{0.25,0.50,0.56}{##1}}}
\expandafter\def\csname PYG@tok@vg\endcsname{\def\PYG@tc##1{\textcolor[rgb]{0.73,0.38,0.84}{##1}}}
\expandafter\def\csname PYG@tok@m\endcsname{\def\PYG@tc##1{\textcolor[rgb]{0.13,0.50,0.31}{##1}}}
\expandafter\def\csname PYG@tok@mh\endcsname{\def\PYG@tc##1{\textcolor[rgb]{0.13,0.50,0.31}{##1}}}
\expandafter\def\csname PYG@tok@cs\endcsname{\def\PYG@tc##1{\textcolor[rgb]{0.25,0.50,0.56}{##1}}\def\PYG@bc##1{\setlength{\fboxsep}{0pt}\colorbox[rgb]{1.00,0.94,0.94}{\strut ##1}}}
\expandafter\def\csname PYG@tok@ge\endcsname{\let\PYG@it=\textit}
\expandafter\def\csname PYG@tok@vc\endcsname{\def\PYG@tc##1{\textcolor[rgb]{0.73,0.38,0.84}{##1}}}
\expandafter\def\csname PYG@tok@il\endcsname{\def\PYG@tc##1{\textcolor[rgb]{0.13,0.50,0.31}{##1}}}
\expandafter\def\csname PYG@tok@go\endcsname{\def\PYG@tc##1{\textcolor[rgb]{0.20,0.20,0.20}{##1}}}
\expandafter\def\csname PYG@tok@cp\endcsname{\def\PYG@tc##1{\textcolor[rgb]{0.00,0.44,0.13}{##1}}}
\expandafter\def\csname PYG@tok@gi\endcsname{\def\PYG@tc##1{\textcolor[rgb]{0.00,0.63,0.00}{##1}}}
\expandafter\def\csname PYG@tok@gh\endcsname{\let\PYG@bf=\textbf\def\PYG@tc##1{\textcolor[rgb]{0.00,0.00,0.50}{##1}}}
\expandafter\def\csname PYG@tok@ni\endcsname{\let\PYG@bf=\textbf\def\PYG@tc##1{\textcolor[rgb]{0.84,0.33,0.22}{##1}}}
\expandafter\def\csname PYG@tok@nl\endcsname{\let\PYG@bf=\textbf\def\PYG@tc##1{\textcolor[rgb]{0.00,0.13,0.44}{##1}}}
\expandafter\def\csname PYG@tok@nn\endcsname{\let\PYG@bf=\textbf\def\PYG@tc##1{\textcolor[rgb]{0.05,0.52,0.71}{##1}}}
\expandafter\def\csname PYG@tok@no\endcsname{\def\PYG@tc##1{\textcolor[rgb]{0.38,0.68,0.84}{##1}}}
\expandafter\def\csname PYG@tok@na\endcsname{\def\PYG@tc##1{\textcolor[rgb]{0.25,0.44,0.63}{##1}}}
\expandafter\def\csname PYG@tok@nb\endcsname{\def\PYG@tc##1{\textcolor[rgb]{0.00,0.44,0.13}{##1}}}
\expandafter\def\csname PYG@tok@nc\endcsname{\let\PYG@bf=\textbf\def\PYG@tc##1{\textcolor[rgb]{0.05,0.52,0.71}{##1}}}
\expandafter\def\csname PYG@tok@nd\endcsname{\let\PYG@bf=\textbf\def\PYG@tc##1{\textcolor[rgb]{0.33,0.33,0.33}{##1}}}
\expandafter\def\csname PYG@tok@ne\endcsname{\def\PYG@tc##1{\textcolor[rgb]{0.00,0.44,0.13}{##1}}}
\expandafter\def\csname PYG@tok@nf\endcsname{\def\PYG@tc##1{\textcolor[rgb]{0.02,0.16,0.49}{##1}}}
\expandafter\def\csname PYG@tok@si\endcsname{\let\PYG@it=\textit\def\PYG@tc##1{\textcolor[rgb]{0.44,0.63,0.82}{##1}}}
\expandafter\def\csname PYG@tok@s2\endcsname{\def\PYG@tc##1{\textcolor[rgb]{0.25,0.44,0.63}{##1}}}
\expandafter\def\csname PYG@tok@vi\endcsname{\def\PYG@tc##1{\textcolor[rgb]{0.73,0.38,0.84}{##1}}}
\expandafter\def\csname PYG@tok@nt\endcsname{\let\PYG@bf=\textbf\def\PYG@tc##1{\textcolor[rgb]{0.02,0.16,0.45}{##1}}}
\expandafter\def\csname PYG@tok@nv\endcsname{\def\PYG@tc##1{\textcolor[rgb]{0.73,0.38,0.84}{##1}}}
\expandafter\def\csname PYG@tok@s1\endcsname{\def\PYG@tc##1{\textcolor[rgb]{0.25,0.44,0.63}{##1}}}
\expandafter\def\csname PYG@tok@gp\endcsname{\let\PYG@bf=\textbf\def\PYG@tc##1{\textcolor[rgb]{0.78,0.36,0.04}{##1}}}
\expandafter\def\csname PYG@tok@sh\endcsname{\def\PYG@tc##1{\textcolor[rgb]{0.25,0.44,0.63}{##1}}}
\expandafter\def\csname PYG@tok@ow\endcsname{\let\PYG@bf=\textbf\def\PYG@tc##1{\textcolor[rgb]{0.00,0.44,0.13}{##1}}}
\expandafter\def\csname PYG@tok@sx\endcsname{\def\PYG@tc##1{\textcolor[rgb]{0.78,0.36,0.04}{##1}}}
\expandafter\def\csname PYG@tok@bp\endcsname{\def\PYG@tc##1{\textcolor[rgb]{0.00,0.44,0.13}{##1}}}
\expandafter\def\csname PYG@tok@c1\endcsname{\let\PYG@it=\textit\def\PYG@tc##1{\textcolor[rgb]{0.25,0.50,0.56}{##1}}}
\expandafter\def\csname PYG@tok@kc\endcsname{\let\PYG@bf=\textbf\def\PYG@tc##1{\textcolor[rgb]{0.00,0.44,0.13}{##1}}}
\expandafter\def\csname PYG@tok@c\endcsname{\let\PYG@it=\textit\def\PYG@tc##1{\textcolor[rgb]{0.25,0.50,0.56}{##1}}}
\expandafter\def\csname PYG@tok@mf\endcsname{\def\PYG@tc##1{\textcolor[rgb]{0.13,0.50,0.31}{##1}}}
\expandafter\def\csname PYG@tok@err\endcsname{\def\PYG@bc##1{\setlength{\fboxsep}{0pt}\fcolorbox[rgb]{1.00,0.00,0.00}{1,1,1}{\strut ##1}}}
\expandafter\def\csname PYG@tok@kd\endcsname{\let\PYG@bf=\textbf\def\PYG@tc##1{\textcolor[rgb]{0.00,0.44,0.13}{##1}}}
\expandafter\def\csname PYG@tok@ss\endcsname{\def\PYG@tc##1{\textcolor[rgb]{0.32,0.47,0.09}{##1}}}
\expandafter\def\csname PYG@tok@sr\endcsname{\def\PYG@tc##1{\textcolor[rgb]{0.14,0.33,0.53}{##1}}}
\expandafter\def\csname PYG@tok@mo\endcsname{\def\PYG@tc##1{\textcolor[rgb]{0.13,0.50,0.31}{##1}}}
\expandafter\def\csname PYG@tok@mi\endcsname{\def\PYG@tc##1{\textcolor[rgb]{0.13,0.50,0.31}{##1}}}
\expandafter\def\csname PYG@tok@kn\endcsname{\let\PYG@bf=\textbf\def\PYG@tc##1{\textcolor[rgb]{0.00,0.44,0.13}{##1}}}
\expandafter\def\csname PYG@tok@o\endcsname{\def\PYG@tc##1{\textcolor[rgb]{0.40,0.40,0.40}{##1}}}
\expandafter\def\csname PYG@tok@kr\endcsname{\let\PYG@bf=\textbf\def\PYG@tc##1{\textcolor[rgb]{0.00,0.44,0.13}{##1}}}
\expandafter\def\csname PYG@tok@s\endcsname{\def\PYG@tc##1{\textcolor[rgb]{0.25,0.44,0.63}{##1}}}
\expandafter\def\csname PYG@tok@kp\endcsname{\def\PYG@tc##1{\textcolor[rgb]{0.00,0.44,0.13}{##1}}}
\expandafter\def\csname PYG@tok@w\endcsname{\def\PYG@tc##1{\textcolor[rgb]{0.73,0.73,0.73}{##1}}}
\expandafter\def\csname PYG@tok@kt\endcsname{\def\PYG@tc##1{\textcolor[rgb]{0.56,0.13,0.00}{##1}}}
\expandafter\def\csname PYG@tok@sc\endcsname{\def\PYG@tc##1{\textcolor[rgb]{0.25,0.44,0.63}{##1}}}
\expandafter\def\csname PYG@tok@sb\endcsname{\def\PYG@tc##1{\textcolor[rgb]{0.25,0.44,0.63}{##1}}}
\expandafter\def\csname PYG@tok@k\endcsname{\let\PYG@bf=\textbf\def\PYG@tc##1{\textcolor[rgb]{0.00,0.44,0.13}{##1}}}
\expandafter\def\csname PYG@tok@se\endcsname{\let\PYG@bf=\textbf\def\PYG@tc##1{\textcolor[rgb]{0.25,0.44,0.63}{##1}}}
\expandafter\def\csname PYG@tok@sd\endcsname{\let\PYG@it=\textit\def\PYG@tc##1{\textcolor[rgb]{0.25,0.44,0.63}{##1}}}

\def\PYGZbs{\char`\\}
\def\PYGZus{\char`\_}
\def\PYGZob{\char`\{}
\def\PYGZcb{\char`\}}
\def\PYGZca{\char`\^}
\def\PYGZam{\char`\&}
\def\PYGZlt{\char`\<}
\def\PYGZgt{\char`\>}
\def\PYGZsh{\char`\#}
\def\PYGZpc{\char`\%}
\def\PYGZdl{\char`\$}
\def\PYGZhy{\char`\-}
\def\PYGZsq{\char`\'}
\def\PYGZdq{\char`\"}
\def\PYGZti{\char`\~}
% for compatibility with earlier versions
\def\PYGZat{@}
\def\PYGZlb{[}
\def\PYGZrb{]}
\makeatother

\begin{document}

\maketitle
\tableofcontents
\phantomsection\label{index::doc}


Contents:


\chapter{Getting started}
\label{getting_started:getting-started}\label{getting_started::doc}\label{getting_started:id1}\label{getting_started:welcome-to-mmdiagnosis-s-documentation}
UV-CDAT (Ultrascale Visualization Climate Data Analysis Tools) is a powerful and complete front-end to a rich set of visual-data exploration and analysis capabilities well suited for climate-data analysis problems. For more details visit \href{http://uv-cdat.org/}{http://uv-cdat.org/}


\section{UV-CDAT Installation Guide}
\label{getting_started:uv-cdat-installation-guide}

\subsection{Dependencies on Linux}
\label{getting_started:dependencies-on-linux}
The following are dependencies need to be installed before installing UV-CDAT in all   Linux distributions
\begin{itemize}
\item {} 
git

\item {} 
libqt4-dev (under RedHAT the system qt isn't working, user will need to get the binaries from the website)

\item {} 
libpng-dev

\item {} 
libxml2-dev

\item {} 
libxslt-dev

\item {} 
xorg-dev

\item {} 
sqlite3

\item {} 
libsqlite3-dev

\item {} 
libbz2-dev

\item {} 
libgdbm-dev

\item {} 
libxt-dev

\item {} 
openssl-dev (libssl-dev)

\item {} 
gfortran

\item {} 
g++

\item {} 
qt4-dev-tools / qtcreator

\item {} 
tcl-dev

\item {} 
tk-dev

\item {} 
libgdbm-dev

\item {} 
libdb4.8-dev

\item {} 
yasm

\item {} 
grace

\item {} 
grads

\end{itemize}


\subsection{Dependencies on Ubuntu (extra dependencies)}
\label{getting_started:dependencies-on-ubuntu-extra-dependencies}\begin{itemize}
\item {} 
libicu48

\item {} 
libxi-dev

\item {} 
libglu1-mesa-dev

\item {} 
libgl1-mesa-dev

\item {} 
libqt4-opengl-dev

\end{itemize}


\subsection{Dependencies on CentOS 6.3 (extra dependencies)}
\label{getting_started:dependencies-on-centos-6-3-extra-dependencies}\begin{itemize}
\item {} 
bzip2-devel

\item {} 
dbus-devel

\item {} 
dbus-c++-devel

\item {} 
dbus-glib-devel

\item {} 
gtkglext-devel

\item {} 
mesalibGL

\item {} 
mesalibGLU

\item {} 
openGL

\item {} 
gstreamer-devel

\item {} 
gstreamer-plugins-base*devel

\item {} 
gstreamer-plugins-bad*devel

\item {} 
gstreamer-plugins-good*devel

\item {} 
libcurl4

\item {} 
openssl-devel

\end{itemize}


\section{Download UV-CDAT Binary}
\label{getting_started:download-uv-cdat-binary}\begin{itemize}
\item {} 
More detailed system requirements can be found \href{https://github.com/UV-CDAT/uvcdat/wiki/System-Requirements}{here} (https://github.com/UV-CDAT/uvcdat/wiki/System-Requirements)

\item {} 
UV-CDAT latest releases binary tar ball can be found \href{http://sourceforge.net/projects/cdat/files/Releases/UV-CDAT/}{here} (http://sourceforge.net/projects/cdat/files/Releases/UV-CDAT/)

\item {} 
Download UV-CDAT latest binary version (2.0.0 as on Nov-2014) from \href{http://sourceforge.net/projects/cdat/files/Releases/UV-CDAT/2.0.0/}{here} (http://sourceforge.net/projects/cdat/files/Releases/UV-CDAT/2.0.0/)  with respect to your linux distribution.

\item {} 
You will need admin privileges to install UV-CDAT in your system.

\end{itemize}


\section{Installation on Linux  (Pre-Built Binary)}
\label{getting_started:installation-on-linux-pre-built-binary}
\textbf{Step 1:} Download and untar the binaries matching best your OS.
\begin{quote}

\$ cd /
\$ sudo tar xjvf UV-CDAT-{[}version-no{]}-{[}your OS here{]}.tar.bz2
\end{quote}

\textbf{Step 2:} For Mac and RH6 ONLY
\begin{itemize}
\item {} \begin{description}
\item[{Download the tarsal containing the version of QT UVCDAT has been compiled against RedHAT6}] \leavevmode
\$ cd /
\$ sudo tar xjvf qt-RH6-64bit-4.8.0.tar.bz2

\end{description}

\item {} \begin{description}
\item[{Mac}] \leavevmode
double click the .dmg file and follow instructions

\end{description}

\end{itemize}

\textbf{Step 3:} Set Environment Variables
\begin{itemize}
\item {} \begin{description}
\item[{Bash users:}] \leavevmode
\$ source /usr/local/uvcdat/{[}version-no{]}/bin/setup\_cdat.sh

\end{description}

\item {} \begin{description}
\item[{t/csh users}] \leavevmode
\$ source /usr/local/uvcdat/{[}version-no{]}/bin/setup\_cdat.csh

\end{description}

\end{itemize}

\textbf{Step 4:} Set Alias Paths
\begin{quote}

In your system \textasciitilde{}/.bash\_aliases or \textasciitilde{}/.bashrc  add the following 4 lines.
\begin{quote}

source /usr/local/uvcdat/2.0.0/bin/setup\_cdat.sh

alias  uvcdat = `/usr/local/uvcdat/2.0.0/bin/python'

alias  uvcdat-gui = `/usr/local/uvcdat/2.0.0/bin/uvcdat'

alias  uvcdscan = `/usr/local/uvcdat/2.0.0/bin/cdscan'
\end{quote}

Here shown version 2.0.0 for example only. User need to set proper version no whichever they installed in their system.
\end{quote}

\textbf{Step 5:} Start enjoying UVCDAT
\begin{itemize}
\item {} \begin{description}
\item[{GUI}] \leavevmode\begin{itemize}
\item {} 
Type  \$ uvcdat-gui  in your linux terminal to use gui

\end{itemize}

\end{description}

\item {} \begin{description}
\item[{Command Line Python}] \leavevmode\begin{itemize}
\item {} 
Type \$ uvcdat  in your linux terminal to use python shell

\item {} 
Type \$ uvcdat  sample\_program.py to execute any uvcdat support python programs.

\end{itemize}

\end{description}

\item {} \begin{description}
\item[{Cdscan}] \leavevmode\begin{itemize}
\item {} 
Type \$ uvcdsan  –x out.xml {\color{red}\bfseries{}*}.nc  to scan all available nc files and make links into small xml file which can be used to load all nc files data inside uvcdat scripts from single xml file.

\end{itemize}

\end{description}

\end{itemize}


\section{Installation on Linux  (Build From Source)}
\label{getting_started:installation-on-linux-build-from-source}\begin{itemize}
\item {} 
User can visit the following links to install from source. For AIX, UNIX kind of systems, its better start to build and install from UV-CDAT latest source itself.

\item {} \begin{description}
\item[{Download UV-CDAT latest Source Code from the following link}] \leavevmode\begin{itemize}
\item {} 
\href{https://github.com/UV-CDAT/uvcdat/releases}{https://github.com/UV-CDAT/uvcdat/releases}

\end{itemize}

\end{description}

\item {} \begin{description}
\item[{Guide to Install from Source}] \leavevmode\begin{itemize}
\item {} 
\href{http://uv-cdat.org/installing.html}{http://uv-cdat.org/installing.html}

\item {} 
\href{https://github.com/UV-CDAT/uvcdat/wiki/zold-Building-UVCDAT}{https://github.com/UV-CDAT/uvcdat/wiki/zold-Building-UVCDAT}

\end{itemize}

\end{description}

\end{itemize}


\section{Installation on Ubuntu}
\label{getting_started:installation-on-ubuntu}
Current Stable Release 2.0 Supporting for \textbf{Ubuntu 13.x} and \textbf{14.x}

\textbf{System Requirements}

\$ sudo apt-get install cmake cmake-curses-gui wget libqt4-dev libpng12-dev libxml2-dev libxslt1-dev xorg-dev sqlite3 libsqlite3-dev libbz2-dev libgdbm-dev libxt-dev libssl-dev gfortran g++ qt4-dev-tools tcl-dev tk-dev libgdbm-dev libdb-dev libicu-dev libxi-dev libglu1-mesa-dev libgl1-mesa-dev libqt4-opengl-dev libbz2-dev grads grace


\subsection{Installing the Binaries (Strongly Suggested)}
\label{getting_started:installing-the-binaries-strongly-suggested}
You must be Root

\$ sudo -s

\$ cd /

\$ wget \href{http://sourceforge.net/projects/cdat/files/Releases/UV-CDAT/2.0.0/UV-CDAT-2.0.0-Ubuntu-14.04-64bit.tar.gz}{http://sourceforge.net/projects/cdat/files/Releases/UV-CDAT/2.0.0/UV-CDAT-2.0.0-Ubuntu-14.04-64bit.tar.gz}

\$ tar xvjf UV-CDAT-2.0.0-Ubuntu-14.04-64bit.tar.gz

\$ source /usr/local/uvcdat/2.0.0/bin/setup\_runtime.sh

Running UV-CDAT GUI

(Don't forget to source setup\_runtime.sh)

\$ uvcdat

Bash users add source /usr/local/uvcdat/2.0.0/bin/setup\_runtime.sh to your \textasciitilde{}/.bashrc file

Csh users add source /usr/local/uvcdat/2.0.0/bin/setup\_runtime.csh to your \textasciitilde{}/.cshrc file

Set aliases in your bashrc file as mentioned in the previous section.

\href{https://github.com/UV-CDAT/uvcdat/wiki/Installation-on-Ubuntu}{https://github.com/UV-CDAT/uvcdat/wiki/Installation-on-Ubuntu}


\subsection{Installation on RedHat, Fedora \& CentOS}
\label{getting_started:installation-on-redhat-fedora-centos}
Current Stable Release 2.0 Supporting for \textbf{RedHat6} / \textbf{CentOS6}

System Requirements (Must be Root)

You must have access to the EPEL Repos \href{http://www.thegeekstuff.com/2012/06/enable-epel-repository}{help link} (http://www.thegeekstuff.com/2012/06/enable-epel-repository)

\$ sudo yum install cmake cmake-gui wget libpng-devel libxml2-devel libxslt-devel xorg* sqlite-devel bzip2 gdbm-devel libXt-devel openssl-devel gcc-gfortran libgfortran tcl-devel tk-devel libdbi-devel libicu-devel libXi-devel mesa-libGLU-devel mesa-libGL-devel PyQt4-devel gcc-c++ patch grace grads
Installation Qt Binary :

\$ cd /

\$ wget \href{http://sourceforge.net/projects/cdat/files/Releases/UV-CDAT/2.0.0/qt-CentOS-6.5-RedHat6-64bit-4.8.4.tar.bz2}{http://sourceforge.net/projects/cdat/files/Releases/UV-CDAT/2.0.0/qt-CentOS-6.5-RedHat6-64bit-4.8.4.tar.bz2}

\$ tar xvjf qt-CentOS-6.5-RedHat6-64bit-4.8.4.tar.bz2

please add Qt to your path (example in bash) add this to your .bashrc
\begin{quote}

export Qt=/usr/local/uvcdat/Qt/4.8.4/bin/

export PATH=\$PATH:\$Qt
\end{quote}

Installing the Binaries (Strongly Suggested)

You must be Root

\$ sudo -s

\$ cd /

\$ wget \href{http://sourceforge.net/projects/cdat/files/Releases/UV-CDAT/2.0.0/UV-CDAT-2.0.0-CentOS6.5-RedHat6-64bit.tar.gz}{http://sourceforge.net/projects/cdat/files/Releases/UV-CDAT/2.0.0/UV-CDAT-2.0.0-CentOS6.5-RedHat6-64bit.tar.gz}

\$ tar xvjf  UV-CDAT-2.0.0-CentOS6.5-RedHat6-64bit.tar.gz

\$ source /usr/local/uvcdat/2.0.0/bin/setup\_runtime.sh

Set aliases in your bashrc file as mentioned in the previous section.

Bash users add source /usr/local/uvcdat/2.0.0/bin/setup\_runtime.sh to your \textasciitilde{}/.bashrc file

Csh users add source /usr/local/uvcdat/2.0.0/bin/setup\_runtime.csh to your \textasciitilde{}/.cshrc file

\href{https://github.com/UV-CDAT/uvcdat/wiki/installation-on-RedHat---CentOS}{https://github.com/UV-CDAT/uvcdat/wiki/installation-on-RedHat---CentOS}


\section{UV-CDAT Documentation}
\label{getting_started:uv-cdat-documentation}

\subsection{Official Documentation}
\label{getting_started:official-documentation}\begin{itemize}
\item {} 
\href{http://uv-cdat.org/documentation/cdms/cdms.html}{CDMS} (http://uv-cdat.org/documentation/cdms/cdms.html) Manual

\item {} 
UV-CDAT \href{http://uv-cdat.org/documentation/utilities/utilities.html}{Utilities} (http://uv-cdat.org/documentation/utilities/utilities.html) Manual

\item {} 
\href{http://uv-cdat.org/documentation/vcs/vcs.html}{VCS} (http://uv-cdat.org/documentation/vcs/vcs.html) Manual

\end{itemize}


\subsection{Useful Tips \& Tricks}
\label{getting_started:useful-tips-tricks}\begin{itemize}
\item {} 
\href{http://www.johnny-lin.com/cdat\_tips/}{http://www.johnny-lin.com/cdat\_tips/}

\item {} 
\href{http://drclimate.wordpress.com/2014/01/02/a-beginners-guide-to-scripting-with-uv-cdat/}{http://drclimate.wordpress.com/2014/01/02/a-beginners-guide-to-scripting-with-uv-cdat/}

\item {} 
\href{http://tuxcoder.wordpress.com/category/python/cdat/}{http://tuxcoder.wordpress.com/category/python/cdat/}

\item {} 
\href{http://tuxcoder.wordpress.com/category/uvcdat-2/}{http://tuxcoder.wordpress.com/category/uvcdat-2/}

\end{itemize}


\subsection{Slides}
\label{getting_started:slides}\begin{itemize}
\item {} 
Lesson-1 \href{https://www.scribd.com/doc/56253490/Lesson1-Python-An-Introduction}{Python An Introduction} (https://www.scribd.com/doc/56253490/Lesson1-Python-An-Introduction)

\item {} 
Lesson-2 \href{https://www.scribd.com/doc/56254041/Lesson2-Numpy-Arrays}{Numpy Arrays} (https://www.scribd.com/doc/56254041/Lesson2-Numpy-Arrays)

\item {} 
Lesson-3 \href{https://www.scribd.com/doc/56254387/Lesson3-cdutil-genutil}{cdutil\_genutil} (https://www.scribd.com/doc/56254387/Lesson3-cdutil-genutil)

\item {} 
Lesson-4 \href{https://www.scribd.com/doc/56254572/Lesson4-VCS-XmGrace-CDAT-Graphics}{VCS\_XmGrace\_Graphics} (https://www.scribd.com/doc/56254572/Lesson4-VCS-XmGrace-CDAT-Graphics)

\end{itemize}


\subsection{IPython With Interactive Live Execution Examples Outputs}
\label{getting_started:ipython-with-interactive-live-execution-examples-outputs}\begin{itemize}
\item {} 
Introduction to \href{http://nbviewer.ipython.org/github/arulalant/UV-CDAT-IPython-Notebooks/blob/outputs/1.Introduction\_to\_NumPy\_Arrays.ipynb}{NumPy Arrays} (http://nbviewer.ipython.org/github/arulalant/UV-CDAT-IPython-Notebooks/blob/outputs/1.Introduction\_to\_NumPy\_Arrays.ipynb)

\item {} 
UV-CDAT-\href{http://nbviewer.ipython.org/github/arulalant/UV-CDAT-IPython-Notebooks/blob/outputs/2.UV-CDAT-cdms.ipynb}{cdms} (http://nbviewer.ipython.org/github/arulalant/UV-CDAT-IPython-Notebooks/blob/outputs/2.UV-CDAT-cdms.ipynb)

\item {} 
UV-CDAT-\href{http://nbviewer.ipython.org/github/arulalant/UV-CDAT-IPython-Notebooks/blob/outputs/3.UV-CDAT-cdutil\_gentuil.ipynb}{cdutil\_gentuil} (http://nbviewer.ipython.org/github/arulalant/UV-CDAT-IPython-Notebooks/blob/outputs/3.UV-CDAT-cdutil\_gentuil.ipynb)

\item {} 
UV-CDAT-\href{http://nbviewer.ipython.org/github/arulalant/UV-CDAT-IPython-Notebooks/blob/outputs/4.UV-CDAT-Graphics-vcs-xmgrace.ipynb}{Graphics-vcs-xmgrace} (http://nbviewer.ipython.org/github/arulalant/UV-CDAT-IPython-Notebooks/blob/outputs/4.UV-CDAT-Graphics-vcs-xmgrace.ipynb)

\end{itemize}
\begin{description}
\item[{UV-CDAT IPython Notebooks Source}] \leavevmode
\href{https://github.com/arulalant/UV-CDAT-IPython-Notebooks}{https://github.com/arulalant/UV-CDAT-IPython-Notebooks}

\end{description}


\section{Support}
\label{getting_started:support}

\subsection{Mailing List}
\label{getting_started:mailing-list}
\href{http://uv-cdat.org/mailing-list.html}{http://uv-cdat.org/mailing-list.html}  and  \href{https://uvcdat.llnl.gov/mailing-list/}{https://uvcdat.llnl.gov/mailing-list/}


\section{Gallery}
\label{getting_started:gallery}
The UV-CDAT gallery images contains all different type of plots, projection, 2D \&
3D visualization can be found from this link  \href{http://uvcdat.llnl.gov/gallery.php}{http://uvcdat.llnl.gov/gallery.php}


\section{License}
\label{getting_started:license}
UVCDAT comes under Free Open Source GNU GENERAL PUBLIC LICENSE V3+
Read about GPL License here \href{https://www.gnu.org/licenses/gpl.html}{https://www.gnu.org/licenses/gpl.html}


\chapter{Installation and Setup of mmDiagnosis Framework}
\label{diag_install:installation-and-setup-of-mmdiagnosis-framework}\label{diag_install::doc}\label{diag_install:mmdiagnosis-installation}

\section{Source Code}
\label{diag_install:source-code}\begin{quote}

The complete source code of this project can be found from online github repository

\textbf{Repository Link :} \href{https://github.com/arulalant/mmDiagnosis}{https://github.com/arulalant/mmDiagnosis}
\end{quote}


\section{Installing the mmDiagnosis}
\label{diag_install:installing-the-mmdiagnosis}
Download the latest version of mmDiagnosis framework source code from \href{https://github.com/arulalant/mmDiagnosis}{https://github.com/arulalant/mmDiagnosis}
and extract the zip file.
\begin{quote}

\$ cd mmDiagnosis

\$ sudo /usr/local/uvcdat/2.0.0/bin/python setup.py build install

Replace your uvcdat version number in above statement.
\end{quote}


\section{Setup and Configuration}
\label{diag_install:setup-and-configuration}

\section{License}
\label{diag_install:license}
Free Open Source GNU GENERAL PUBLIC LICENSE V3
\href{http://www.gnu.org/licenses/quick-guide-gplv3.html}{http://www.gnu.org/licenses/quick-guide-gplv3.html}


\chapter{Documentation of \textbf{diagnosisutils} source code}
\label{diagnosisutils:documentation-of-diagnosisutils-source-code}\label{diagnosisutils:diagnosisutils}\label{diagnosisutils::doc}
The diagnosisutils package contains the {\hyperref[diagnosisutils:data-access-utils]{Data Access Utils}} (\autopageref*{diagnosisutils:data-access-utils}), {\hyperref[diagnosisutils:time-axis-utils]{Time Axis Utils}} (\autopageref*{diagnosisutils:time-axis-utils}), and {\hyperref[diagnosisutils:plot-utils]{Plot Utils}} (\autopageref*{diagnosisutils:plot-utils}) modules.


\section{Data Access Utils}
\label{diagnosisutils:data-access-utils}
This data access utils uses the {\hyperref[diagnosisutils:time-axis-utils]{Time Axis Utils}} (\autopageref*{diagnosisutils:time-axis-utils}) .

The {\hyperref[diagnosisutils:xml-data-access]{xml\_data\_access}} (\autopageref*{diagnosisutils:xml-data-access}) module help us to access the all the grib files through single object.

Basically all the grib files are pointed into xml dom by cdscan command.

Then we can the xml files through uv-cdat.

Right now we are generating 8 xml files to access analysis grib files and 7 forecasts grib files.

In the {\hyperref[diagnosisutils:xml-data-access]{xml\_data\_access}} (\autopageref*{diagnosisutils:xml-data-access}) module, we are finding which xml needs to access depends upon the user inputs (Type, hour) in the functions of this module.

And once one xml object has initiated then through out that program execute session, it will remains exists and use when ever user needs it.


\subsection{xml\_data\_access}
\label{diagnosisutils:xml-data-access}\label{diagnosisutils:id1}\phantomsection\label{diagnosisutils:module-xml_data_access}\index{xml\_data\_access (module)}\index{GribXmlAccess (class in xml\_data\_access)}

\begin{fulllineitems}
\phantomsection\label{diagnosisutils:xml_data_access.GribXmlAccess}\pysiglinewithargsret{\strong{class }\code{xml\_data\_access.}\bfcode{GribXmlAccess}}{\emph{XmlDir}}{}
xml access methods
\index{closeXmlObjs() (xml\_data\_access.GribXmlAccess method)}

\begin{fulllineitems}
\phantomsection\label{diagnosisutils:xml_data_access.GribXmlAccess.closeXmlObjs}\pysiglinewithargsret{\bfcode{closeXmlObjs}}{}{}
{\hyperref[diagnosisutils:xml_data_access.GribXmlAccess.closeXmlObjs]{\code{closeXmlObjs()}}} (\autopageref*{diagnosisutils:xml_data_access.GribXmlAccess.closeXmlObjs}): close all the opened xml file objects by
cdms2. If we called this method, it will check all the 8 xml objects
are either opened or not. If that is opened by cdms2 means, it will
close that file object properly. We must call this method for the
safety purpose.
\begin{description}
\item[{..note:: If we called this method, at the end of the program then}] \leavevmode
it should be optimized one. If this method called at any inter
mediate level means, then again it need to create the xml object.

\end{description}

Written By: Arulalan.T

Date : 10.09.2011

\end{fulllineitems}

\index{findPartners() (xml\_data\_access.GribXmlAccess method)}

\begin{fulllineitems}
\phantomsection\label{diagnosisutils:xml_data_access.GribXmlAccess.findPartners}\pysiglinewithargsret{\bfcode{findPartners}}{\emph{Type}, \emph{date}, \emph{hour=None}, \emph{returnType='c'}}{}
{\hyperref[diagnosisutils:xml_data_access.GribXmlAccess.findPartners]{\code{findPartners()}}} (\autopageref*{diagnosisutils:xml_data_access.GribXmlAccess.findPartners}): To find the partners of the any particular
day anl or any particular day and hour of the fcst.
Each fcst file(day) has its truth anl file(day).
i.e. today 24 hour fcst file's partner is tomorrow's truth anl file.
today 48 hour fcst file's partner is the day after tomorrow's
truth anl file. Keep going on the fcst vc anl files.

Same concept for anl files partner but in reverse concept.
Today's truth anl file's partners are yesterdays' 24 hour fcst file,
day before yesterday's 48 hour fcst file and keep going backward ...

This what we are calling as the partners of anl and fcst files.
For present fcst hours partner is future anl file and for present anl
partners are the past fcst hours files.

Condition :
\begin{quote}

if `f' as passed then hour is mandatory one
else `a' as passed then hour is optional one.
returnType either `c' or `s'
\end{quote}

Inputs :
\begin{quote}

Type = `f' or `a' or `o' i.e fcst or anl or obs file
date must be cdtime.comptime object or its string formate
hour is like 24 multiples in case availability of the fcst files
\end{quote}

Outputs :
\begin{quote}
\begin{quote}

If `f' has passed this method returns a corresponding partner
of the anlysis date in cdtime.comptime object
If `a' or `o' has passed this method returns a dictionary.
It contains the availability of the fcst hours as key and its
corresponding fcst date in cdtime.comptime object as value of
the dict.

we can get the return date as yyyymmdd string formate by
passing returnType = `s'
\end{quote}

Usage :
\begin{quote}
\begin{description}
\item[{example 1 :}] \leavevmode
\begin{Verbatim}[commandchars=\\\{\}]
\PYG{g+gp}{\PYGZgt{}\PYGZgt{}\PYGZgt{} }\PYG{n}{findPartners}\PYG{p}{(}\PYG{l+s}{\PYGZsq{}}\PYG{l+s}{f}\PYG{l+s}{\PYGZsq{}}\PYG{p}{,}\PYG{l+s}{\PYGZsq{}}\PYG{l+s}{2010\PYGZhy{}5\PYGZhy{}25}\PYG{l+s}{\PYGZsq{}}\PYG{p}{,}\PYG{l+m+mi}{24}\PYG{p}{)}
\PYG{g+go}{    2010\PYGZhy{}5\PYGZhy{}26 0:0:0.0}
\end{Verbatim}
\begin{quote}

\begin{notice}{note}{Note:}
The passed date in comptime in string type.
\end{notice}
\end{quote}

\begin{Verbatim}[commandchars=\\\{\}]
\PYG{g+gp}{\PYGZgt{}\PYGZgt{}\PYGZgt{} }\PYG{n}{findPartners}\PYG{p}{(}\PYG{l+s}{\PYGZsq{}}\PYG{l+s}{f}\PYG{l+s}{\PYGZsq{}}\PYG{p}{,}\PYG{n}{cdtime}\PYG{o}{.}\PYG{n}{comptime}\PYG{p}{(}\PYG{l+m+mi}{2010}\PYG{p}{,}\PYG{l+m+mi}{5}\PYG{p}{,}\PYG{l+m+mi}{25}\PYG{p}{)}\PYG{p}{,}\PYG{l+m+mi}{24}\PYG{p}{)}
\PYG{g+go}{    2010\PYGZhy{}5\PYGZhy{}26 0:0:0.0}
\end{Verbatim}
\begin{quote}

\begin{notice}{note}{Note:}
The passed date in comptime object itself.
\end{notice}
\end{quote}

\begin{Verbatim}[commandchars=\\\{\}]
\PYG{g+gp}{\PYGZgt{}\PYGZgt{}\PYGZgt{} }\PYG{n}{findPartners}\PYG{p}{(}\PYG{l+s}{\PYGZsq{}}\PYG{l+s}{a}\PYG{l+s}{\PYGZsq{}}\PYG{p}{,}\PYG{l+s}{\PYGZsq{}}\PYG{l+s}{2010\PYGZhy{}5\PYGZhy{}26}\PYG{l+s}{\PYGZsq{}}\PYG{p}{)}
\PYG{g+go}{    \PYGZob{}24: 2010\PYGZhy{}5\PYGZhy{}25 0:0:0.0\PYGZcb{}}
\end{Verbatim}
\begin{quote}

\begin{notice}{note}{Note:}
Returns dictionary which contains key as hour and its
corresponding date
\end{notice}
\end{quote}

\item[{example 2 :}] \leavevmode
\begin{Verbatim}[commandchars=\\\{\}]
\PYG{g+gp}{\PYGZgt{}\PYGZgt{}\PYGZgt{} }\PYG{n}{findPartners}\PYG{p}{(}\PYG{l+s}{\PYGZsq{}}\PYG{l+s}{f}\PYG{l+s}{\PYGZsq{}}\PYG{p}{,}\PYG{l+s}{\PYGZsq{}}\PYG{l+s}{2010\PYGZhy{}5\PYGZhy{}25}\PYG{l+s}{\PYGZsq{}}\PYG{p}{,}\PYG{l+m+mi}{72}\PYG{p}{)}
\PYG{g+go}{    2010\PYGZhy{}5\PYGZhy{}28 0:0:0.0}
\end{Verbatim}

\begin{Verbatim}[commandchars=\\\{\}]
\PYG{g+gp}{\PYGZgt{}\PYGZgt{}\PYGZgt{} }\PYG{n}{findPartners}\PYG{p}{(}\PYG{l+s}{\PYGZsq{}}\PYG{l+s}{a}\PYG{l+s}{\PYGZsq{}}\PYG{p}{,}\PYG{l+s}{\PYGZsq{}}\PYG{l+s}{2010\PYGZhy{}6\PYGZhy{}1}\PYG{l+s}{\PYGZsq{}}\PYG{p}{,} \PYG{n}{returnType} \PYG{o}{=}\PYG{l+s}{\PYGZsq{}}\PYG{l+s}{s}\PYG{l+s}{\PYGZsq{}}\PYG{p}{)}
\PYG{g+go}{     \PYGZob{}24: \PYGZsq{}20100531\PYGZsq{},}
\PYG{g+go}{      48: \PYGZsq{}20100530\PYGZsq{},}
\PYG{g+go}{      72: \PYGZsq{}20100529\PYGZsq{},}
\PYG{g+go}{      96: \PYGZsq{}20100528\PYGZsq{},}
\PYG{g+go}{     120: \PYGZsq{}20100527\PYGZsq{},}
\PYG{g+go}{     144: \PYGZsq{}20100526\PYGZsq{},}
\PYG{g+go}{     168: \PYGZsq{}20100525\PYGZsq{}\PYGZcb{}}
\end{Verbatim}

\begin{notice}{note}{Note:}
Depends upon the availability of the fcst and anl
files, it should return partner date
\end{notice}

\item[{example 3 :}] \leavevmode
\begin{Verbatim}[commandchars=\\\{\}]
\PYG{g+gp}{\PYGZgt{}\PYGZgt{}\PYGZgt{} }\PYG{n}{findPartners}\PYG{p}{(}\PYG{l+s}{\PYGZsq{}}\PYG{l+s}{a}\PYG{l+s}{\PYGZsq{}}\PYG{p}{,}\PYG{l+s}{\PYGZsq{}}\PYG{l+s}{20100601}\PYG{l+s}{\PYGZsq{}}\PYG{p}{,}\PYG{l+m+mi}{144}\PYG{p}{)}
\PYG{g+go}{    2010\PYGZhy{}5\PYGZhy{}26 0:0:0.0}
\end{Verbatim}


\strong{See also:}


If not available for the passed hour means
it should return None



\end{description}
\end{quote}
\end{quote}

Written by : Arulalan.T

Date : 03.04.2011

\end{fulllineitems}

\index{getData() (xml\_data\_access.GribXmlAccess method)}

\begin{fulllineitems}
\phantomsection\label{diagnosisutils:xml_data_access.GribXmlAccess.getData}\pysiglinewithargsret{\bfcode{getData}}{\emph{var}, \emph{Type}, \emph{date}, \emph{hour=None}, \emph{level='all'}, \emph{**latlonregion}}{}~\begin{description}
\item[{{\hyperref[diagnosisutils:xml_data_access.GribXmlAccess.getData]{\code{getData()}}} (\autopageref*{diagnosisutils:xml_data_access.GribXmlAccess.getData}): It can extract either the data of a single date or}] \leavevmode
range of dates. It depends up on the input of the date argument.
Finally it should return MV2 variable.

\item[{Condition :}] \leavevmode
date is either tuple or string.
level is optional. level takes default `all'.
if level passed, it must be belongs to the data variable
hour is must when Type arg should be `f' (fcst) to choose
xml object.
Pass either (lat,lon) or region.

\item[{Inputs :}] \leavevmode\begin{quote}

Type - either `a'{[}analysis{]} or `f'{[}forecast{]} or `o'{[}observation{]}
var,level must be belongs to the data file
key word arg lat,lon or region should be passed
date formate must one of the followings
\begin{description}
\item[{date formate 1:}] \leavevmode
date = (startdate, enddate)
here startdate and enddate must be like cdtime.comptime formate.

\item[{date formate 2:}] \leavevmode
date = (startdate)

\item[{date formate 3:}] \leavevmode
date = `startdate' or date = `date'

\item[{eg for date input :}] \leavevmode
date = (`2010-5-1',`2010-6-30')
date = (`2010-5-30')
date = `2010-5-30'

\end{description}
\end{quote}

Outputs :
\begin{quote}

If user passed single date in the date argument, then it should
return the data of that particular date as single MV2 variable.

If user passed start and enddate in the date argument,
then it should return the data for the range of dates as
single MV2 variable with time axis.
\end{quote}

\end{description}

Written by: Arulalan.T

Date: 10.05.2011

\end{fulllineitems}

\index{getDataPartners() (xml\_data\_access.GribXmlAccess method)}

\begin{fulllineitems}
\phantomsection\label{diagnosisutils:xml_data_access.GribXmlAccess.getDataPartners}\pysiglinewithargsret{\bfcode{getDataPartners}}{\emph{var}, \emph{Type}, \emph{date}, \emph{hour=None}, \emph{level='all'}, \emph{orginData=0}, \emph{datePriority='o'}, \emph{**latlonregion}}{}
{\hyperref[diagnosisutils:xml_data_access.GribXmlAccess.getDataPartners]{\code{getDataPartners()}}} (\autopageref*{diagnosisutils:xml_data_access.GribXmlAccess.getDataPartners}): It can extract either the orginDate with its
partnersData or it can extract only the partnersData without its
orginData for a single date or range of dates.

It depends up on the input of the orginData, datePriority,date args.
Finally it should return partnersData and/or orginData as MV2 variable

Condition :
\begin{quote}

date is either tuple or string.
level is optional. level takes default `all'.
if level passed, it must be belongs to the data variable
hour is must when Type arg should be `f' (fcst) to select xml object.
hour is must when range of date passed, even thogut Type arg should
be `a'(anl) or `o'(obs),to choose one fcst xml object along with hour.
Pass either (lat,lon) or regionself.
\end{quote}

Inputs:
\begin{quote}

Type - either `a'{[}analysis{]} or `f'{[}forecast{]} or `o'{[}observation{]}
var,level must be belongs to the data file
orginData - either 0 or 1. 0 means it shouldnot return the
\begin{quote}
\begin{description}
\item[{orginData as single MV2 var.}] \leavevmode
1 means it should return both the orginData and its
partnersData as two seperate MV2 vars.

\end{description}
\end{quote}
\begin{description}
\item[{datePriority - either `o' or `p'. `o' means passed date is with}] \leavevmode\begin{quote}

respect to orginData. According to this
orginData's date, it should return its partnersData.
\end{quote}
\begin{description}
\item[{`p' means passed date is with respect to partnersData.}] \leavevmode
According to this partnersData's date, it should
return its orginData.

\end{description}

\end{description}

key word arg lat,lon or region should be passed
\begin{description}
\item[{date formate 1:}] \leavevmode
date = (startdate,enddate)
here startdate and enddate must be like cdtime.comptime formate.

\item[{date formate 2:}] \leavevmode
date = (startdate)

\item[{date formate 3:}] \leavevmode
date = `startdate' or date = `date'

\item[{eg for date input :}] \leavevmode\begin{description}
\item[{date = (`2010-5-1',`2010-6-30')}] \leavevmode
date = (`2010-5-30')
date = `2010-5-30'

\end{description}

\end{description}
\end{quote}

Outputs:
\begin{quote}

If user passed single date in the date argument, then it should
return the data of that particular date
(both orginData \& partnersData) as a single MV2 variable.

If user passed start and enddate in the date argument, then it
should return the data (both orginData \& partnersData)
for the range of dates as a single MV2 variable with time axis.
\end{quote}

Usage:
\begin{quote}

\begin{notice}{note}{Note:}
if `a'(anl) file is orginData means `f'(fcst) files are
its partnersData and vice versa.
\end{notice}
\begin{description}
\item[{example1:}] \leavevmode
\begin{Verbatim}[commandchars=\\\{\}]
\PYG{g+gp}{\PYGZgt{}\PYGZgt{}\PYGZgt{} }\PYG{n}{a}\PYG{p}{,}\PYG{n}{b} \PYG{o}{=} \PYG{n}{getDataPartners}\PYG{p}{(}\PYG{n}{var} \PYG{o}{=} \PYG{l+s}{\PYGZsq{}}\PYG{l+s}{U component of wind}\PYG{l+s}{\PYGZsq{}}\PYG{p}{,}\PYG{n}{Type} \PYG{o}{=} \PYG{l+s}{\PYGZsq{}}\PYG{l+s}{a}\PYG{l+s}{\PYGZsq{}}\PYG{p}{,}
\PYG{g+go}{       date = \PYGZsq{}2010\PYGZhy{}6\PYGZhy{}5\PYGZsq{},hour = None,level = \PYGZsq{}all\PYGZsq{},orginData = 1,}
\PYG{g+go}{       datePriority = \PYGZsq{}o\PYGZsq{}, lat=(\PYGZhy{}90,90),lon=(0,359.5))}
\end{Verbatim}

a is orginData. i.e. anl. its timeAxis date is `2010-6-5'.
b is partnersData. i.e. fcst. its 24 hour fcst date w.r.t
orginData is `2010-6-4'. 48 hour is `2010-6-3'.

Depends upon the availability of date of fcst files,it should
return the data.
In NCMRWF2010 model, it should return maximum of 7 days fcst.

If we will specify any hour in the same eg, that should return
only that hour fcst file data instead of returning all the
available fcst hours data.

\item[{example2:}] \leavevmode
\begin{Verbatim}[commandchars=\\\{\}]
\PYG{g+gp}{\PYGZgt{}\PYGZgt{}\PYGZgt{} }\PYG{n}{a}\PYG{p}{,}\PYG{n}{b} \PYG{o}{=} \PYG{n}{getDataPartners}\PYG{p}{(}\PYG{n}{var} \PYG{o}{=} \PYG{l+s}{\PYGZsq{}}\PYG{l+s}{U component of wind}\PYG{l+s}{\PYGZsq{}}\PYG{p}{,}\PYG{n}{Type} \PYG{o}{=} \PYG{l+s}{\PYGZsq{}}\PYG{l+s}{f}\PYG{l+s}{\PYGZsq{}}\PYG{p}{,}
\PYG{g+go}{         date = \PYGZsq{}2010\PYGZhy{}6\PYGZhy{}5\PYGZsq{},hour = 24,level = \PYGZsq{}all\PYGZsq{},orginData = 1,}
\PYG{g+go}{         datePriority = \PYGZsq{}o\PYGZsq{}, lat=(\PYGZhy{}90,90),lon=(0,359.5))}
\end{Verbatim}

a is orginData. i.e.fcst 24 hour.its timeAxis date is `2010-6-5'.
b is partnersData. i.e. anl. its anl date w.r.t orginData is
`2010-6-6'.

\item[{example3:}] \leavevmode
\begin{Verbatim}[commandchars=\\\{\}]
\PYG{g+gp}{\PYGZgt{}\PYGZgt{}\PYGZgt{} }\PYG{n}{b} \PYG{o}{=} \PYG{n}{getDataPartners}\PYG{p}{(}\PYG{n}{var} \PYG{o}{=} \PYG{l+s}{\PYGZsq{}}\PYG{l+s}{U component of wind}\PYG{l+s}{\PYGZsq{}}\PYG{p}{,}\PYG{n}{Type} \PYG{o}{=} \PYG{l+s}{\PYGZsq{}}\PYG{l+s}{f}\PYG{l+s}{\PYGZsq{}}\PYG{p}{,}
\PYG{g+go}{        date = \PYGZsq{}2010\PYGZhy{}6\PYGZhy{}5\PYGZsq{},hour = 24,level = \PYGZsq{}all\PYGZsq{}, orginData = 0,}
\PYG{g+go}{        datePriority = \PYGZsq{}o\PYGZsq{}, lat=(\PYGZhy{}90,90),lon=(0,359.5))}
\end{Verbatim}

b is partnersData. i.e. anl. its anl date w.r.t orginData is
`2010-6-6'. No orginData. Because we passed orginData as 0.

\item[{example4:}] \leavevmode
\begin{Verbatim}[commandchars=\\\{\}]
\PYG{g+gp}{\PYGZgt{}\PYGZgt{}\PYGZgt{} }\PYG{n}{a}\PYG{p}{,}\PYG{n}{b} \PYG{o}{=} \PYG{n}{getDataPartners}\PYG{p}{(}\PYG{n}{var} \PYG{o}{=} \PYG{l+s}{\PYGZsq{}}\PYG{l+s}{U component of wind}\PYG{l+s}{\PYGZsq{}}\PYG{p}{,}\PYG{n}{Type} \PYG{o}{=} \PYG{l+s}{\PYGZsq{}}\PYG{l+s}{f}\PYG{l+s}{\PYGZsq{}}\PYG{p}{,}
\PYG{g+go}{         date = \PYGZsq{}2010\PYGZhy{}6\PYGZhy{}5\PYGZsq{},hour = 24,level = \PYGZsq{}all\PYGZsq{},orginData = 1,}
\PYG{g+go}{         datePriority = \PYGZsq{}p\PYGZsq{}, lat=(\PYGZhy{}90,90),lon=(0,359.5))}
\end{Verbatim}

a is orginData. i.e. fcst 24 hour.its timeAxis date is `2010-6-6'.
b is partnersData. i.e. anl. its anl date w.r.t orginData is
`2010-6-5'.
we can compare this eg4 with eg2. In this we passed datePriority
as `p'. So the passed date as set to the partnersData and
orginData's date has shifted to the next day.

\item[{example5:}] \leavevmode
\begin{Verbatim}[commandchars=\\\{\}]
\PYG{g+gp}{\PYGZgt{}\PYGZgt{}\PYGZgt{} }\PYG{n}{a}\PYG{p}{,}\PYG{n}{b} \PYG{o}{=} \PYG{n}{getDataPartners}\PYG{p}{(}\PYG{n}{var} \PYG{o}{=} \PYG{l+s}{\PYGZsq{}}\PYG{l+s}{U component of wind}\PYG{l+s}{\PYGZsq{}}\PYG{p}{,}\PYG{n}{Type} \PYG{o}{=} \PYG{l+s}{\PYGZsq{}}\PYG{l+s}{a}\PYG{l+s}{\PYGZsq{}}\PYG{p}{,}
\PYG{g+go}{         date = (\PYGZsq{}2010\PYGZhy{}6\PYGZhy{}5\PYGZsq{},\PYGZsq{}2010\PYGZhy{}6\PYGZhy{}6\PYGZsq{}),hour = 24,level = \PYGZsq{}all\PYGZsq{},}
\PYG{g+go}{    orginData = 1,datePriority = \PYGZsq{}o\PYGZsq{}, lat=(\PYGZhy{}90,90),lon=(0,359.5))}
\end{Verbatim}
\begin{quote}

\begin{notice}{note}{Note:}
Even though we passed `a' Type, we must choose the
hour option to select the fcst file, since we are
passing the range of dates.
\end{notice}
\end{quote}
\begin{description}
\item[{a is orginData. i.e. anl. its timeAxis size is 2.}] \leavevmode
date are `2010-6-5' and `2010-6-6'.

\end{description}

b is partnersData. i.e. fcst 24 hour data.
its timeAxis size is 2. date w.r.t orginData are `2010-6-4' and
`2010-6-5'.

a's `2010-6-5' has partner is b's `2010-6-4'.i.e.orginData(anl)
partners is partnersData's (fcst).

same concept for the remains day.
a's `2010-6-6' has partner is b's `2010-6-5'.

\item[{example6:}] \leavevmode
\begin{Verbatim}[commandchars=\\\{\}]
\PYG{g+gp}{\PYGZgt{}\PYGZgt{}\PYGZgt{} }\PYG{n}{a}\PYG{p}{,}\PYG{n}{b} \PYG{o}{=} \PYG{n}{getDataPartners}\PYG{p}{(}\PYG{n}{var} \PYG{o}{=} \PYG{l+s}{\PYGZsq{}}\PYG{l+s}{U component of wind}\PYG{l+s}{\PYGZsq{}}\PYG{p}{,}\PYG{n}{Type} \PYG{o}{=} \PYG{l+s}{\PYGZsq{}}\PYG{l+s}{a}\PYG{l+s}{\PYGZsq{}}\PYG{p}{,}
\PYG{g+go}{          date = (\PYGZsq{}2010\PYGZhy{}6\PYGZhy{}5\PYGZsq{},\PYGZsq{}2010\PYGZhy{}6\PYGZhy{}6\PYGZsq{}),hour = 24,level = \PYGZsq{}all\PYGZsq{},}
\PYG{g+go}{    orginData = 1,datePriority = \PYGZsq{}p\PYGZsq{}, lat=(\PYGZhy{}90,90),lon=(0,359.5))}
\end{Verbatim}
\begin{quote}

\begin{notice}{note}{Note:}
Even though we passed `a' Type, we must choose the
hour option to select the fcst file, since we are
passing the range of dates.
\end{notice}
\end{quote}

a is orginData. i.e. anl. its timeAxis size is 2.
date are `2010-6-6' and `2010-6-7'.

b is partnersData. i.e. fcst 24 hour data.
its timeAxis size is 2. date w.r.t orginData are `2010-6-5' and
`2010-6-6'.

a's `2010-6-6' has partner is b's `2010-6-5'.i.e.orginData(anl)
partners is partnersData's (fcst).

same concept for the remains day.
a's `2010-6-7' has partner is b's `2010-6-6'.
we can compare this eg6 with eg5.In this we passed datePriority
as `p'. So the passed date as set to the partnersData and
orginData's date has shifted towards the next days.

\end{description}
\end{quote}

Written by: Arulalan.T

Date: 27.05.2011

\end{fulllineitems}

\index{getMonthAvgData() (xml\_data\_access.GribXmlAccess method)}

\begin{fulllineitems}
\phantomsection\label{diagnosisutils:xml_data_access.GribXmlAccess.getMonthAvgData}\pysiglinewithargsret{\bfcode{getMonthAvgData}}{\emph{var}, \emph{Type}, \emph{level}, \emph{month}, \emph{year}, \emph{calendarName=None}, \emph{hour=None}, \emph{**latlonregion}}{}~\begin{description}
\item[{{\hyperref[diagnosisutils:xml_data_access.GribXmlAccess.getMonthAvgData]{\code{getMonthAvgData()}}} (\autopageref*{diagnosisutils:xml_data_access.GribXmlAccess.getMonthAvgData}): It returns the average of the given month}] \leavevmode
data (all days in the month) for the passed variable options

\item[{Condition :}] \leavevmode
calendarName,level are optional.
level takes default `all' if not pass arg for it.
hour is must when Type arg should be `f' (fcst) to select xml object.
Pass either (lat,lon) or region.

\item[{Inputs :}] \leavevmode
Type - either `a' or `f' or `o'
var,level must be belongs to the data file
month may be even in 3 char like `apr' or `April' or `aPRiL' or
like any month
year must be passed as integer
calendarName default None, it takes cdtime.DefaultCalendar
key word arg lat,lon or region should be passed

\item[{Outputs :}] \leavevmode
It should return the average of the whole month data for the given
vars as MV2 variable

\end{description}

Usage :
\begin{quote}
\begin{description}
\item[{example :}] \leavevmode
\begin{Verbatim}[commandchars=\\\{\}]
\PYG{g+gp}{\PYGZgt{}\PYGZgt{}\PYGZgt{} }\PYG{n}{getMonthAvgData}\PYG{p}{(}\PYG{n}{var} \PYG{o}{=} \PYG{l+s}{\PYGZdq{}}\PYG{l+s}{Geopotential Height}\PYG{l+s}{\PYGZdq{}}\PYG{p}{,}\PYG{n}{Type} \PYG{o}{=} \PYG{l+s}{\PYGZsq{}}\PYG{l+s}{f}\PYG{l+s}{\PYGZsq{}}\PYG{p}{,}
\PYG{g+go}{            level = \PYGZsq{}all\PYGZsq{}, month = \PYGZsq{}july\PYGZsq{},year=2010 , hour = 24,}
\PYG{g+go}{             region = AIR ) \PYGZsh{}lat=(\PYGZhy{}90,90),lon=(0,359.5)}
\end{Verbatim}

returns dataAvg of one month data as single MV2 variable

\end{description}
\end{quote}

Written by: Arulalan.T

Date: 29.04.2011

\end{fulllineitems}

\index{getRainfallData() (xml\_data\_access.GribXmlAccess method)}

\begin{fulllineitems}
\phantomsection\label{diagnosisutils:xml_data_access.GribXmlAccess.getRainfallData}\pysiglinewithargsret{\bfcode{getRainfallData}}{\emph{date=None}, \emph{level='all'}, \emph{**latlonregion}}{}~\begin{description}
\item[{{\hyperref[diagnosisutils:xml_data_access.GribXmlAccess.getRainfallData]{\code{getRainfallData()}}} (\autopageref*{diagnosisutils:xml_data_access.GribXmlAccess.getRainfallData}): Extract the rainfall data from the xml file}] \leavevmode
which has created by the cdscan command.

\item[{Inputs}] \leavevmode{[}var takes from the instance member of GribAccess class{]}\begin{description}
\item[{if we passed date,level then it should return data accordingly}] \leavevmode
By default level takes `all' levels.
Pass either (lat,lon) or region keyword arg

\end{description}

\item[{Condition}] \leavevmode{[}we must set the member variable called rainfallXmlPath{]}
and (rainfallXmlVar or rainfallModel), then only we can
access this method. rainfallXmlVar is the obeservation
rainfall variable name to access the data. OR
rainfallModel is the model name, which has set in the
global variable names settings. By Using it, this method
should get the observation rainfall variable name.

\end{description}

Written by : Arulalan.T

Date : 29.05.2011

\end{fulllineitems}

\index{getRainfallDataPartners() (xml\_data\_access.GribXmlAccess method)}

\begin{fulllineitems}
\phantomsection\label{diagnosisutils:xml_data_access.GribXmlAccess.getRainfallDataPartners}\pysiglinewithargsret{\bfcode{getRainfallDataPartners}}{\emph{date}, \emph{hour=None}, \emph{level='all'}, \emph{orginData=1}, \emph{datePriority='o'}, \emph{**latlonregion}}{}~\begin{description}
\item[{{\hyperref[diagnosisutils:xml_data_access.GribXmlAccess.getRainfallDataPartners]{\code{getRainfallDataPartners()}}} (\autopageref*{diagnosisutils:xml_data_access.GribXmlAccess.getRainfallDataPartners}): It returns the rainfall data \& its}] \leavevmode
partners data(fcst is the partner of the observation.
i.e. rainfall) as MV2 vars

\item[{Condition:}] \leavevmode
startdate is must. enddata is optional one.
If both startdate and enddate has passed means, it should
return the rainfall data and partnersData within that range.

\end{description}

Inputs:
\begin{quote}
\begin{description}
\item[{orginData - either 0 or 1. 0 means it shouldnot return the}] \leavevmode\begin{description}
\item[{orginData as single MV2 var.}] \leavevmode
1 means it should return both the orginData and its
partnersData as two seperate MV2 vars.

\end{description}

\item[{datePriority - either `o' or `p'. `o' means passed date is with}] \leavevmode
respect to orginData. According to this orginData's
date, it should return its partnersData.
`p' means passed date is with respect to partnersData.
\begin{quote}

According to this partnersData's date, it should
return its orginData.
\end{quote}

\end{description}

key word arg lat,lon or region should be passed
By default hour is None and level is `all'.

self.rainfallXmlPath is mandatory one when you choosed orginData
is 1. you must set the rainfall xml path to the rainfallXmlPath.

self.rainfallModel is the model name, which has set in the global
variables settings, to get the model fcst and its obeservation
variable name to access the data. It is mandatory one.
\begin{description}
\item[{date formate 1:}] \leavevmode
date = (startdate,enddate)
here startdate and enddate must be like cdtime.comptime formate.

\item[{date formate 2:}] \leavevmode
date = (startdate)

\item[{date formate 3:}] \leavevmode
date = `startdate' or date = `date'

\item[{eg for date input :}] \leavevmode\begin{quote}

date = (`2010-5-1',`2010-6-30')
date = (`2010-5-30')
date = `2010-5-30'
\end{quote}

By default skipdays as 1 takes place. User cant override till now.

\end{description}
\end{quote}

Outputs :
\begin{quote}

If user passed single date in the date argument, then it should
return the data of that particular date
(both orginData \& partnersData) as MV2 variable.

If user passed start and enddate in the date argument, then it
should return the data (both orginData \& partnersData) for the
range of dates as MV2 variable with time axis.
\end{quote}

Usage :
\begin{quote}
\begin{quote}

\begin{notice}{note}{Note:}
if `r'(observation) file is orginData means `f'(fcst)
files are its partnersData.
\end{notice}
\end{quote}
\begin{description}
\item[{example1:}] \leavevmode
\begin{Verbatim}[commandchars=\\\{\}]
\PYG{g+gp}{\PYGZgt{}\PYGZgt{}\PYGZgt{} }\PYG{n}{a}\PYG{p}{,}\PYG{n}{b} \PYG{o}{=} \PYG{n}{getRainfallDataPartners}\PYG{p}{(}\PYG{n}{date} \PYG{o}{=} \PYG{l+s}{\PYGZsq{}}\PYG{l+s}{2010\PYGZhy{}6\PYGZhy{}5}\PYG{l+s}{\PYGZsq{}}\PYG{p}{,}\PYG{n}{hour} \PYG{o}{=} \PYG{n+nb+bp}{None}\PYG{p}{,}
\PYG{g+go}{                 level = \PYGZsq{}all\PYGZsq{},orginData = 1,datePriority = \PYGZsq{}o\PYGZsq{},}
\PYG{g+go}{                                     lat=(\PYGZhy{}90,90),lon=(0,359.5))}
\end{Verbatim}

a is orginData. i.e. rainfall observation.
its timeAxis date is `2010-6-5'.

b is partnersData. i.e. fcst. its 24 hour fcst date w.r.t
orginData is `2010-6-4'. 48 hour is `2010-6-3'.

Depends upon the availability of date of fcst files,it should
return the data.
In NCMRWF2010 model, it should return maximum of 7 days fcst.

If we will specify any hour in the same eg, that should return
only that hour fcst file data instead of returning all the
available fcst hours data.

\item[{example2:}] \leavevmode
\begin{Verbatim}[commandchars=\\\{\}]
\PYG{g+gp}{\PYGZgt{}\PYGZgt{}\PYGZgt{} }\PYG{n}{a}\PYG{p}{,}\PYG{n}{b} \PYG{o}{=} \PYG{n}{getRainfallDataPartners}\PYG{p}{(}\PYG{n}{date} \PYG{o}{=} \PYG{l+s}{\PYGZsq{}}\PYG{l+s}{2010\PYGZhy{}6\PYGZhy{}5}\PYG{l+s}{\PYGZsq{}}\PYG{p}{,}\PYG{n}{hour} \PYG{o}{=} \PYG{l+m+mi}{24}\PYG{p}{,}
\PYG{g+go}{                 level = \PYGZsq{}all\PYGZsq{},orginData = 1,datePriority = \PYGZsq{}o\PYGZsq{},}
\PYG{g+go}{                                     lat=(\PYGZhy{}90,90),lon=(0,359.5))}
\end{Verbatim}

a is orginData. i.e. rainfall observation.
its timeAxis date is `2010-6-5'.
b is partnersData. i.e. fcst 24 hour. its fcst date w.r.t
orginData is `2010-6-6'.

\item[{example3:}] \leavevmode
\begin{Verbatim}[commandchars=\\\{\}]
\PYG{g+gp}{\PYGZgt{}\PYGZgt{}\PYGZgt{} }\PYG{n}{b} \PYG{o}{=} \PYG{n}{getRainfallDataPartners}\PYG{p}{(}\PYG{n}{date} \PYG{o}{=} \PYG{l+s}{\PYGZsq{}}\PYG{l+s}{2010\PYGZhy{}6\PYGZhy{}5}\PYG{l+s}{\PYGZsq{}}\PYG{p}{,}\PYG{n}{hour} \PYG{o}{=} \PYG{l+m+mi}{24}\PYG{p}{,}
\PYG{g+go}{                 level = \PYGZsq{}all\PYGZsq{},orginData = 0,datePriority = \PYGZsq{}o\PYGZsq{},}
\PYG{g+go}{                                     lat=(\PYGZhy{}90,90),lon=(0,359.5))}
\end{Verbatim}

b is partnersData. i.e. fcst. its fcst date w.r.t orginData
is `2010-6-6'.  No orginData. Because we passed orginData as 0.

\item[{example4:}] \leavevmode
\begin{Verbatim}[commandchars=\\\{\}]
\PYG{g+gp}{\PYGZgt{}\PYGZgt{}\PYGZgt{} }\PYG{n}{a}\PYG{p}{,}\PYG{n}{b} \PYG{o}{=} \PYG{n}{getRainfallDataPartnerss}\PYG{p}{(}\PYG{n}{date} \PYG{o}{=} \PYG{l+s}{\PYGZsq{}}\PYG{l+s}{2010\PYGZhy{}6\PYGZhy{}5}\PYG{l+s}{\PYGZsq{}}\PYG{p}{,}\PYG{n}{hour} \PYG{o}{=} \PYG{l+m+mi}{24}\PYG{p}{,}
\PYG{g+go}{                 level = \PYGZsq{}all\PYGZsq{},orginData = 1,datePriority = \PYGZsq{}p\PYGZsq{},}
\PYG{g+go}{                                     lat=(\PYGZhy{}90,90),lon=(0,359.5))}
\end{Verbatim}

a is orginData. i.e. rainfall observation.
its timeAxis date is `2010-6-6'.

b is partnersData. i.e. fcst 24 hour. its fcst date w.r.t
orginData is `2010-6-5'.  we can compare this eg4 with eg2.

In this we passed datePriority as `p'. So the passed date as
set to the partnersData and orginData's date has shifted to
the next day.

\item[{example5:}] \leavevmode
\begin{Verbatim}[commandchars=\\\{\}]
\PYG{g+gp}{\PYGZgt{}\PYGZgt{}\PYGZgt{} }\PYG{n}{a}\PYG{p}{,}\PYG{n}{b} \PYG{o}{=} \PYG{n}{getRainfallDataPartners}\PYG{p}{(}\PYG{n}{date} \PYG{o}{=} \PYG{p}{(}\PYG{l+s}{\PYGZsq{}}\PYG{l+s}{2010\PYGZhy{}6\PYGZhy{}5}\PYG{l+s}{\PYGZsq{}}\PYG{p}{,}\PYG{l+s}{\PYGZsq{}}\PYG{l+s}{2010\PYGZhy{}6\PYGZhy{}6}\PYG{l+s}{\PYGZsq{}}\PYG{p}{)}\PYG{p}{,}
\PYG{g+go}{                          hour = 24,level = \PYGZsq{}all\PYGZsq{},orginData = 1,}
\PYG{g+go}{                 datePriority = \PYGZsq{}o\PYGZsq{}, lat=(\PYGZhy{}90,90),lon=(0,359.5))}
\end{Verbatim}
\begin{quote}

\begin{notice}{note}{Note:}
We must choose the hour option to select the fcst
file, since we are passing the range of dates.
\end{notice}
\end{quote}

a is orginData.i.e.rainfall observation.its timeAxis size is 2.
date are `2010-6-5' and `2010-6-6'.

b is partnersData.i.e.fcst 24 hour data.its timeAxis size is 2.
date w.r.t orginData are `2010-6-4' and `2010-6-5'.

a's `2010-6-5' has partner is b's `2010-6-4'. i.e.
orginData(rainfall observation) partners is partnersData(fcst)

same concept for the remains day.
a's `2010-6-6' has partner is b's `2010-6-5'.

\item[{example6:}] \leavevmode
\begin{Verbatim}[commandchars=\\\{\}]
\PYG{g+gp}{\PYGZgt{}\PYGZgt{}\PYGZgt{} }\PYG{n}{a}\PYG{p}{,}\PYG{n}{b} \PYG{o}{=} \PYG{n}{getRainfallDataPartners}\PYG{p}{(}\PYG{n}{date} \PYG{o}{=} \PYG{p}{(}\PYG{l+s}{\PYGZsq{}}\PYG{l+s}{2010\PYGZhy{}6\PYGZhy{}5}\PYG{l+s}{\PYGZsq{}}\PYG{p}{,}\PYG{l+s}{\PYGZsq{}}\PYG{l+s}{2010\PYGZhy{}6\PYGZhy{}6}\PYG{l+s}{\PYGZsq{}}\PYG{p}{)}\PYG{p}{,}
\PYG{g+go}{                         hour = 24,level = \PYGZsq{}all\PYGZsq{},orginData = 1,}
\PYG{g+go}{                 datePriority = \PYGZsq{}p\PYGZsq{}, lat=(\PYGZhy{}90,90),lon=(0,359.5))}
\end{Verbatim}
\begin{quote}
\end{quote}

a is orginData. i.e. rainfall observation.
its timeAxis size is 2. date are `2010-6-6' and `2010-6-7'.

b is partnersData. i.e. fcst 24 hour data.
its timeAxis size is 2. date w.r.t orginData are `2010-6-5'
and `2010-6-6'.

a's `2010-6-6' has partner is b's `2010-6-5'. i.e.
orginData(rainfall observation) partners is partnersData(fcst)

same concept for the remains day.
a's `2010-6-7' has partner is b's `2010-6-6'. we can compare
this eg6 with eg5. In this we passed datePriority as `p'.
So the passed date as set to the partnersData and orginData's
date has shifted towards the next days.

\end{description}
\end{quote}

Written by: Arulalan.T

Date: 29.05.2011

\end{fulllineitems}

\index{getXmlPath() (xml\_data\_access.GribXmlAccess method)}

\begin{fulllineitems}
\phantomsection\label{diagnosisutils:xml_data_access.GribXmlAccess.getXmlPath}\pysiglinewithargsret{\bfcode{getXmlPath}}{\emph{Type}, \emph{hour=None}}{}
{\hyperref[diagnosisutils:xml_data_access.GribXmlAccess.getXmlPath]{\code{getXmlPath()}}} (\autopageref*{diagnosisutils:xml_data_access.GribXmlAccess.getXmlPath}): To get the xml's absolute path.
\begin{description}
\item[{Inputs}] \leavevmode{[}Type is either `a' or `o' or `f' or `r'{]}
hour is mandatory when you pass `f' or `r'.
`a' - analysis, `f' - forecast,
`o' - observation, `r' - reference.

\end{description}

Written by : Arulalan.T

Date : 21.08.2011

\end{fulllineitems}

\index{listvariable() (xml\_data\_access.GribXmlAccess method)}

\begin{fulllineitems}
\phantomsection\label{diagnosisutils:xml_data_access.GribXmlAccess.listvariable}\pysiglinewithargsret{\bfcode{listvariable}}{\emph{Type}, \emph{hour=None}}{}
:func:'listvariable': By passing Type and/or hour args to this method,
it will return the listvariable method of the appropriate xml file.

Returns the listvariable of cdms2 open object method result

\end{fulllineitems}


\end{fulllineitems}



\section{Time Axis Utils}
\label{diagnosisutils:time-axis-utils}
This {\hyperref[diagnosisutils:timeutils]{timeutils}} (\autopageref*{diagnosisutils:timeutils}) module helps us to generate our own time axis, correct existing time axis bounds and generate bounds.

Here we used inbuilt methods of cdtime and cdutil module of uv-cdat.


\subsection{timeutils}
\label{diagnosisutils:id2}\label{diagnosisutils:timeutils}\index{TimeUtility (class in xml\_data\_access)}

\begin{fulllineitems}
\phantomsection\label{diagnosisutils:xml_data_access.TimeUtility}\pysigline{\strong{class }\code{xml\_data\_access.}\bfcode{TimeUtility}}~\index{comp2timestr() (xml\_data\_access.TimeUtility method)}

\begin{fulllineitems}
\phantomsection\label{diagnosisutils:xml_data_access.TimeUtility.comp2timestr}\pysiglinewithargsret{\bfcode{comp2timestr}}{\emph{comptime}, \emph{returnHour='y'}}{}~\begin{description}
\item[{{\hyperref[diagnosisutils:xml_data_access.TimeUtility.comp2timestr]{\code{comp2timestr()}}} (\autopageref*{diagnosisutils:xml_data_access.TimeUtility.comp2timestr}): To convert date from cdtime.comptime into}] \leavevmode
`yyyymmdd' formate or `yyyymmddhh' formate
as string

\end{description}

Condition :   passing date must be comptime formate
\begin{description}
\item[{Inputs}] \leavevmode{[}date in comptime.{]}
returnHour takes `y' or `yes' or `n' or `no'.
Default it takes `y'.

\item[{Outputs}] \leavevmode{[}It should return the date in `yyyymmddhh' string formate, if{]}
returnHour passed `y' or `yes'.
It should return the date in `yyyymmdd' string formate, if
returnHour passed `n' or `no'.

\item[{Usage :}] \leavevmode\begin{description}
\item[{example1 :}] \leavevmode
\begin{Verbatim}[commandchars=\\\{\}]
\PYG{g+gp}{\PYGZgt{}\PYGZgt{}\PYGZgt{} }\PYG{n}{compobj} \PYG{o}{=} \PYG{n}{cdtime}\PYG{o}{.}\PYG{n}{comptime}\PYG{p}{(}\PYG{l+m+mi}{2010}\PYG{p}{,}\PYG{l+m+mi}{4}\PYG{p}{,}\PYG{l+m+mi}{29}\PYG{p}{)} \PYG{o}{\PYGZhy{}}\PYG{o}{\PYGZgt{}} \PYG{l+m+mi}{2010}\PYG{o}{\PYGZhy{}}\PYG{l+m+mi}{4}\PYG{o}{\PYGZhy{}}\PYG{l+m+mi}{29} \PYG{l+m+mi}{0}\PYG{p}{:}\PYG{l+m+mi}{0}\PYG{p}{:}\PYG{l+m+mf}{0.0}
\PYG{g+gp}{\PYGZgt{}\PYGZgt{}\PYGZgt{} }\PYG{n}{comp2timestr}\PYG{p}{(}\PYG{n}{compobj}\PYG{p}{)}
\PYG{g+gp}{\PYGZgt{}\PYGZgt{}\PYGZgt{} }\PYG{l+s}{\PYGZsq{}}\PYG{l+s}{2010042900}\PYG{l+s}{\PYGZsq{}}
\end{Verbatim}

\begin{notice}{note}{Note:}
It should return in yyyymmddhh string formate by default.
Hour is 00.
\end{notice}

\item[{example2 :}] \leavevmode
\begin{Verbatim}[commandchars=\\\{\}]
\PYG{g+gp}{\PYGZgt{}\PYGZgt{}\PYGZgt{} }\PYG{n}{compobj} \PYG{o}{=} \PYG{n}{cdtime}\PYG{o}{.}\PYG{n}{comptime}\PYG{p}{(}\PYG{l+m+mi}{2010}\PYG{p}{,}\PYG{l+m+mi}{4}\PYG{p}{,}\PYG{l+m+mi}{29}\PYG{p}{,}\PYG{l+m+mi}{10}\PYG{p}{)} \PYG{o}{\PYGZhy{}}\PYG{o}{\PYGZgt{}} \PYG{l+m+mi}{2010}\PYG{o}{\PYGZhy{}}\PYG{l+m+mi}{4}\PYG{o}{\PYGZhy{}}\PYG{l+m+mi}{29} \PYG{l+m+mi}{10}\PYG{p}{:}\PYG{l+m+mi}{0}\PYG{p}{:}\PYG{l+m+mf}{0.0}
\PYG{g+gp}{\PYGZgt{}\PYGZgt{}\PYGZgt{} }\PYG{n}{comp2timestr}\PYG{p}{(}\PYG{n}{compobj}\PYG{p}{)}
\PYG{g+gp}{\PYGZgt{}\PYGZgt{}\PYGZgt{} }\PYG{l+s}{\PYGZsq{}}\PYG{l+s}{2010042910}\PYG{l+s}{\PYGZsq{}}
\end{Verbatim}

\begin{notice}{note}{Note:}
It should return in yyyymmddhh string formate by default.
Hour is 10.
\end{notice}

\item[{example2 :}] \leavevmode
\begin{Verbatim}[commandchars=\\\{\}]
\PYG{g+gp}{\PYGZgt{}\PYGZgt{}\PYGZgt{} }\PYG{n}{compobj} \PYG{o}{=} \PYG{n}{cdtime}\PYG{o}{.}\PYG{n}{comptime}\PYG{p}{(}\PYG{l+m+mi}{2010}\PYG{p}{,}\PYG{l+m+mi}{4}\PYG{p}{,}\PYG{l+m+mi}{29}\PYG{p}{,}\PYG{l+m+mi}{10}\PYG{p}{)} \PYG{o}{\PYGZhy{}}\PYG{o}{\PYGZgt{}} \PYG{l+m+mi}{2010}\PYG{o}{\PYGZhy{}}\PYG{l+m+mi}{4}\PYG{o}{\PYGZhy{}}\PYG{l+m+mi}{29} \PYG{l+m+mi}{10}\PYG{p}{:}\PYG{l+m+mi}{0}\PYG{p}{:}\PYG{l+m+mf}{0.0}
\PYG{g+gp}{\PYGZgt{}\PYGZgt{}\PYGZgt{} }\PYG{n}{comp2timestr}\PYG{p}{(}\PYG{n}{compobj}\PYG{p}{,} \PYG{n}{returnHour} \PYG{o}{=} \PYG{l+s}{\PYGZsq{}}\PYG{l+s}{n}\PYG{l+s}{\PYGZsq{}}\PYG{p}{)}
\PYG{g+gp}{\PYGZgt{}\PYGZgt{}\PYGZgt{} }\PYG{l+s}{\PYGZsq{}}\PYG{l+s}{20100429}\PYG{l+s}{\PYGZsq{}}
\end{Verbatim}

\begin{notice}{note}{Note:}
It should return in yyyymmdd string formate only even
though hour passed in the component object. Because we
passed returnHour as `n'.
\end{notice}

\end{description}

\end{description}

Written by : Arulalan.T

Date : 29.04.2011
Updated : 21.08.2011

\end{fulllineitems}

\index{getSameDayData() (xml\_data\_access.TimeUtility method)}

\begin{fulllineitems}
\phantomsection\label{diagnosisutils:xml_data_access.TimeUtility.getSameDayData}\pysiglinewithargsret{\bfcode{getSameDayData}}{\emph{varName}, \emph{fpath}, \emph{day}, \emph{mon}, \emph{hr=0}, \emph{**kwarg}}{}~\begin{description}
\item[{getSameDayData}] \leavevmode{[}get same day data from all the available years or{]}
needed year/s.

\item[{Inputs :}] \leavevmode
varName - variable name
fpath - file path
day : day of the needed  {[}of all the years of the data{]}.
mon : month of the needed {[}of all the years of the data{]}.
hr : hour of the needed
KWargs : (latitude and/or longitude) or (region) and/or level
\begin{quote}

and/or year
\end{quote}

\item[{..note:: It will extract the particular day from all the available years.}] \leavevmode
Ofcourse you can control the needed year/s also by using year kwarg.
Eg1: To get leap day data from all the available years,
\begin{quote}

getSameDayData(varName, path, day=29, mon=2)
It will return all years the leapday data (Feb 29) along with
its timeAxis and missing values.
\end{quote}

\end{description}

Refer: getSeasonData

Written By : Arulalan.T

Date : 13.08.2013

\end{fulllineitems}

\index{getSeasonName() (xml\_data\_access.TimeUtility method)}

\begin{fulllineitems}
\phantomsection\label{diagnosisutils:xml_data_access.TimeUtility.getSeasonName}\pysiglinewithargsret{\bfcode{getSeasonName}}{\emph{startBound}, \emph{endBound}, \emph{year=`1'}, \emph{units='days'}}{}
Inputs : start bound and end bound of the season.
Returns : Season name
Date : 03.06.2012

\end{fulllineitems}

\index{getSeasonalData() (xml\_data\_access.TimeUtility method)}

\begin{fulllineitems}
\phantomsection\label{diagnosisutils:xml_data_access.TimeUtility.getSeasonalData}\pysiglinewithargsret{\bfcode{getSeasonalData}}{\emph{varName}, \emph{fpath}, \emph{sday}, \emph{smon}, \emph{eday}, \emph{emon}, \emph{hr=0}, \emph{**kwarg}}{}~\begin{description}
\item[{getSeasonalData}] \leavevmode{[}user defined season data extraction from the filepath.{]}
It will extract the specified season (passed in the args)
from all the available years or user can pass the needed
year/s.
User can extract same day for all the available years/
needed year/s. For eg : One can get all the leapday data
alone from all the years.

\item[{Inputs:}] \leavevmode
varName : variable name
fpath : data (nc/ctl/grib/xml/cdml) file path
sday : starting day of the season {[}of all the years of the data{]}
smon : starting month of the season {[}of all the years of the data{]}
eday : ending day of the season {[}of all the years of the data{]}
emon : ending month of the season {[}of all the years of the data{]}
hr : hour for both start and end date
KWargs : (latitude and/or longitude) or (region) and/or level
\begin{quote}
\begin{quote}

and/or cyclic.
\end{quote}
\begin{description}
\item[{cyclic}] \leavevmode{[}If cyclic is true, it extract the cyclic year data also.{]}\begin{quote}

i.e. In winter season (Nov to Apr), then it will extract the
first year (Jan to Apr) and last year (Nov to Dec) also.
So it will be look like cyclic year of seasonData.
\begin{quote}

If cyclic is false means, it will not extract the last
\end{quote}

year partial months data. i.e. it will not extract last year
(Nov to Dec) for winter season.
\end{quote}
\begin{description}
\item[{year}] \leavevmode{[}Its optional only. By default it is None. i.e. It will{]}
extract all the available years seasonalData.
If one interger year has passed means, then it will do
extract of that particular year seasonData alone.
If two years has passed in tuple, then it will extract
the range of years seasonData from year{[}0{]} to year{[}1{]}.
eg1 : year=2005 it will extract seasonData of 2005 alone.
eg2 : year=(1971, 2013) it will extract seasonData from
\begin{quote}

1971 to 2013 years.
\end{quote}

\end{description}

\end{description}
\end{quote}

\item[{..note:: If end day and end month is lower than the start day \& start}] \leavevmode\begin{quote}

month, then we need to extract the both current and next year
data. For eg : Winter Season (November to April).
It can not be reversed for this winter season. We need to
extract data from current year november month upto next year
march month.
\begin{quote}

If you pass one year data and passed the above
\end{quote}

winter season, then it will be extracted november and december
months data and will be calculated variance for that alone.
\begin{quote}

If you will pass two year data then it will extract the data
\end{quote}

from november \& december of first year and january, feburary \&
march of next year will be extracted and calculated variance
for that.
\end{quote}
\begin{description}
\item[{..note:: If both sday == eday and smon == emon then it will extract}] \leavevmode
the this particular day from all the available years.
Ofcourse you can control the needed year/s also.
Eg1: To get leap day data from all the available years,
\begin{quote}

getSeasonalData(varName, path, sday=29, smon=2, emon=29, emon=2)
It will return all years the leapday data (Feb 29) along with
its timeAxis and missing values.
\end{quote}

\end{description}

\end{description}

Written By : Arulalan.T

Date : 26.07.2012
Updated : 13.08.2013

\end{fulllineitems}

\index{getSummerData() (xml\_data\_access.TimeUtility method)}

\begin{fulllineitems}
\phantomsection\label{diagnosisutils:xml_data_access.TimeUtility.getSummerData}\pysiglinewithargsret{\bfcode{getSummerData}}{\emph{varName}, \emph{fpath}, \emph{sday=1}, \emph{smon=5}, \emph{eday=31}, \emph{emon=10}, \emph{hr=0}, \emph{**kwarg}}{}
getSummerData : summer season (May to October) data
\begin{description}
\item[{Inputs :}] \leavevmode
varName - variable name
fpath - file path
sday : starting day of the summer season {[}of all the years of the data{]}.
smon : starting month of the summer season {[}of all the years of the data{]}.
eday : ending day of the summer season {[}of all the years of the data{]}.
emon : ending month of the summer season {[}of all the years of the data{]}.
hr : hour for both start and end date
KWargs : (latitude and/or longitude) or (region) and/or level

\end{description}

Refer: getSeasonData

Written By : Arulalan.T

Date : 26.07.2012

\end{fulllineitems}

\index{getTimeAxisFullMonths() (xml\_data\_access.TimeUtility method)}

\begin{fulllineitems}
\phantomsection\label{diagnosisutils:xml_data_access.TimeUtility.getTimeAxisFullMonths}\pysiglinewithargsret{\bfcode{getTimeAxisFullMonths}}{\emph{timeAxis}, \emph{returnType='c'}, \emph{returnHour='y'}}{}~\begin{description}
\item[{{\hyperref[diagnosisutils:xml_data_access.TimeUtility.getTimeAxisFullMonths]{\code{getTimeAxisFullMonths()}}} (\autopageref*{diagnosisutils:xml_data_access.TimeUtility.getTimeAxisFullMonths}): Get the fully available months name and}] \leavevmode
its firstday \& lastday from the passed timeAxis.

\end{description}

Condition : timeAxis must be an instance of cdms2.axis.TransientAxis
\begin{description}
\item[{Inputs}] \leavevmode{[}Pass the any range of timeAxis object.{]}
returnType is either `c' or `s'. If `c' means the dates are
cdtime.comptime object itself. if `s' means the dates are
yyyymmddhh string (by default) or yyyymmdd w.r.t returnHour.
returnHour takes either `y/yes' or `n/no'.

\item[{Outputs}] \leavevmode{[}It should return a dictionary which has key as the year.{]}
This year key has the value as dictionary type itself.
The nested dictionary has month as key and
value as tuple, which contains the stardate and enddate of
that month.
It should return month only for fully available month in the
passed timeAxis.
i.e. If timeAxis has some incomplete months of particular
year means, it should not return that month and its dates.

\item[{Usage :}] \leavevmode\begin{description}
\item[{..seealso:: we can pass any range of timeAxis. Even its hourly}] \leavevmode
series, it should works. But it should return month \&
dates for fully available dates of months in timeAxis.

\item[{example 1:}] \leavevmode
\begin{Verbatim}[commandchars=\\\{\}]
\PYG{g+gp}{\PYGZgt{}\PYGZgt{}\PYGZgt{} }\PYG{n}{tim} \PYG{o}{=} \PYG{n}{\PYGZus{}generateTimeAxis}\PYG{p}{(}\PYG{l+m+mi}{70}\PYG{p}{,} \PYG{l+s}{\PYGZsq{}}\PYG{l+s}{2011\PYGZhy{}5\PYGZhy{}25}\PYG{l+s}{\PYGZsq{}}\PYG{p}{)}
\PYG{g+gp}{\PYGZgt{}\PYGZgt{}\PYGZgt{} }\PYG{n}{tim}
\PYG{g+go}{   id: time}
\PYG{g+go}{   Designated a time axis.}
\PYG{g+go}{   units:  days since 2011\PYGZhy{}5\PYGZhy{}25}
\PYG{g+go}{   Length: 70}
\PYG{g+go}{   First:  0}
\PYG{g+go}{   Last:   69}
\PYG{g+go}{   Other axis attributes:}
\PYG{g+go}{      calendar: gregorian}
\PYG{g+go}{      axis: T}
\PYG{g+go}{   Python id:  0x8aea74c}
\end{Verbatim}

\begin{Verbatim}[commandchars=\\\{\}]
\PYG{g+gp}{\PYGZgt{}\PYGZgt{}\PYGZgt{} }\PYG{n}{getTimeAxisFullMonths}\PYG{p}{(}\PYG{n}{tim}\PYG{p}{)}
\PYG{g+go}{\PYGZob{}2011: \PYGZob{}}
\PYG{g+go}{\PYGZsq{}JULY\PYGZsq{}: (2011\PYGZhy{}7\PYGZhy{}1 0:0:0.0, 2011\PYGZhy{}7\PYGZhy{}31 0:0:0.0),}
\PYG{g+go}{\PYGZsq{}JUNE\PYGZsq{}: (2011\PYGZhy{}6\PYGZhy{}1 0:0:0.0, 2011\PYGZhy{}6\PYGZhy{}30 0:0:0.0)\PYGZcb{}}
\PYG{g+go}{\PYGZcb{}}
\end{Verbatim}
\begin{description}
\item[{..note::  Here 2011 as the key for the primary dictionary.}] \leavevmode\begin{quote}

And JUNE, JULY are the keys for the secondary(inner)
dictionary, which contains the stardate and enddate
of that month and year.
\end{quote}

\begin{Verbatim}[commandchars=\\\{\}]
\PYG{g+gp}{\PYGZgt{}\PYGZgt{}\PYGZgt{} }\PYG{n}{tim}\PYG{o}{.}\PYG{n}{asComponentTime}\PYG{p}{(}\PYG{p}{)}\PYG{p}{[}\PYG{l+m+mi}{0}\PYG{p}{]}
\PYG{g+go}{2011\PYGZhy{}5\PYGZhy{}25 0:0:0.0}
\end{Verbatim}

\item[{..note:: The actual firstdate of the timeAxis is 25th may 2011.}] \leavevmode\begin{quote}

But its not complete month. So this may month is not
returned in the above.
\end{quote}

\begin{Verbatim}[commandchars=\\\{\}]
\PYG{g+gp}{\PYGZgt{}\PYGZgt{}\PYGZgt{} }\PYG{n}{tim}\PYG{o}{.}\PYG{n}{asComponentTime}\PYG{p}{(}\PYG{p}{)}\PYG{p}{[}\PYG{o}{\PYGZhy{}}\PYG{l+m+mi}{1}\PYG{p}{]}
\PYG{g+go}{2011\PYGZhy{}8\PYGZhy{}2 0:0:0.0}
\end{Verbatim}

\item[{..note:: The actual lastdate of the timeAxis is 2nd aug 2011.}] \leavevmode
But its not complete month. So this aug month is not
returned in the above.

\end{description}

\item[{example 2:}] \leavevmode
\begin{Verbatim}[commandchars=\\\{\}]
\PYG{g+gp}{\PYGZgt{}\PYGZgt{}\PYGZgt{} }\PYG{n}{tim1} \PYG{o}{=} \PYG{n}{\PYGZus{}generateTimeAxis}\PYG{p}{(}\PYG{l+m+mi}{70}\PYG{p}{,} \PYG{l+s}{\PYGZsq{}}\PYG{l+s}{2011\PYGZhy{}12\PYGZhy{}1}\PYG{l+s}{\PYGZsq{}}\PYG{p}{)}
\PYG{g+gp}{\PYGZgt{}\PYGZgt{}\PYGZgt{} }\PYG{n}{tim1}
\PYG{g+go}{   id: time}
\PYG{g+go}{   Designated a time axis.}
\PYG{g+go}{   units:  days since 2011\PYGZhy{}12\PYGZhy{}1}
\PYG{g+go}{   Length: 70}
\PYG{g+go}{   First:  0}
\PYG{g+go}{   Last:   69}
\PYG{g+go}{   Other axis attributes:}
\PYG{g+go}{      calendar: gregorian}
\PYG{g+go}{      axis: T}
\PYG{g+go}{   Python id:  0xa2c10ac}
\end{Verbatim}

\begin{Verbatim}[commandchars=\\\{\}]
\PYG{g+gp}{\PYGZgt{}\PYGZgt{}\PYGZgt{} }\PYG{n}{getTimeAxisFullMonths}\PYG{p}{(}\PYG{n}{tim1}\PYG{p}{,} \PYG{n}{returnType} \PYG{o}{=} \PYG{l+s}{\PYGZsq{}}\PYG{l+s}{s}\PYG{l+s}{\PYGZsq{}}\PYG{p}{,}
\PYG{g+go}{                                            returnHour = \PYGZsq{}n\PYGZsq{})}
\PYG{g+go}{\PYGZob{}2011: \PYGZob{}\PYGZsq{}DECEMBER\PYGZsq{}: (\PYGZsq{}20111201\PYGZsq{}, \PYGZsq{}20111231\PYGZsq{})\PYGZcb{},}
\PYG{g+go}{ 2012: \PYGZob{}\PYGZsq{}JANUARY\PYGZsq{}: (\PYGZsq{}20120101\PYGZsq{}, \PYGZsq{}20120131\PYGZsq{})\PYGZcb{}\PYGZcb{}}
\end{Verbatim}
\begin{description}
\item[{..note:: In this example, we will get 2011 and 2012 are the}] \leavevmode
keys of the primary dictionary. And 2011 has `DECEMBER'
month and its startdate \& enddate as key and value.
Same as for 2012 has `JANUARY' month and its startdate
\& enddate as key and value. Here dates are in yyyymmdd
string formate. Here we passed returnHour as `n'.

\end{description}

\end{description}

\end{description}

Written By : Arulalan.T

Date : 24.08.2011

\end{fulllineitems}

\index{getTimeAxisMonths() (xml\_data\_access.TimeUtility method)}

\begin{fulllineitems}
\phantomsection\label{diagnosisutils:xml_data_access.TimeUtility.getTimeAxisMonths}\pysiglinewithargsret{\bfcode{getTimeAxisMonths}}{\emph{timeAxis}, \emph{returnType='c'}, \emph{returnHour='y'}}{}~\begin{description}
\item[{{\hyperref[diagnosisutils:xml_data_access.TimeUtility.getTimeAxisMonths]{\code{getTimeAxisMonths()}}} (\autopageref*{diagnosisutils:xml_data_access.TimeUtility.getTimeAxisMonths}): Get the available months name and its}] \leavevmode
firstday \& lastday from the passed timeAxis.

\end{description}

Condition : timeAxis must be an instance of cdms2.axis.TransientAxis
\begin{description}
\item[{Inputs}] \leavevmode{[}Pass the any range of timeAxis object.{]}
returnType is either `c' or `s'. If `c' means the dates are
cdtime.comptime object itself. if `s' means the dates are
yyyymmddhh string (by default) or yyyymmdd w.r.t returnHour.
returnHour takes either `y/yes' or `n/no'.

\item[{Outputs}] \leavevmode{[}It should return a dictionary which has key as the year.{]}
This year key has the value as dictionary type itself.
The nested dictionary has month as key and
value as tuple, which contains the stardate and enddate of
that month.
It should return month only for the available months in the
passed timeAxis.

\item[{Usage :}] \leavevmode\begin{description}
\item[{..seealso:: we can pass any range of timeAxis. Even its hourly}] \leavevmode
series, it should works.

\end{description}

\end{description}

..seealso:: getTimeAxisFullMonths(), getTimeAxisPartialMonths()
\begin{quote}
\begin{description}
\item[{example 1:}] \leavevmode
\begin{Verbatim}[commandchars=\\\{\}]
\PYG{g+gp}{\PYGZgt{}\PYGZgt{}\PYGZgt{} }\PYG{n}{tim} \PYG{o}{=} \PYG{n}{\PYGZus{}generateTimeAxis}\PYG{p}{(}\PYG{l+m+mi}{70}\PYG{p}{,} \PYG{l+s}{\PYGZsq{}}\PYG{l+s}{2011\PYGZhy{}5\PYGZhy{}25}\PYG{l+s}{\PYGZsq{}}\PYG{p}{)}
\PYG{g+gp}{\PYGZgt{}\PYGZgt{}\PYGZgt{} }\PYG{n}{tim}
\PYG{g+go}{   id: time}
\PYG{g+go}{   Designated a time axis.}
\PYG{g+go}{   units:  days since 2011\PYGZhy{}5\PYGZhy{}25}
\PYG{g+go}{   Length: 70}
\PYG{g+go}{   First:  0}
\PYG{g+go}{   Last:   69}
\PYG{g+go}{   Other axis attributes:}
\PYG{g+go}{      calendar: gregorian}
\PYG{g+go}{      axis: T}
\PYG{g+go}{   Python id:  0x8aea74c}
\end{Verbatim}

\begin{Verbatim}[commandchars=\\\{\}]
\PYG{g+gp}{\PYGZgt{}\PYGZgt{}\PYGZgt{} }\PYG{n}{getTimeAxisMonths}\PYG{p}{(}\PYG{n}{tim}\PYG{p}{)}
\PYG{g+go}{\PYGZob{}2011: \PYGZob{}}
\PYG{g+go}{\PYGZsq{}MAY\PYGZsq{}: (2011\PYGZhy{}5\PYGZhy{}25 0:0:0.0, 2011\PYGZhy{}5\PYGZhy{}31 0:0:0.0),}
\PYG{g+go}{\PYGZsq{}JUNE\PYGZsq{}: (2011\PYGZhy{}6\PYGZhy{}1 0:0:0.0, 2011\PYGZhy{}6\PYGZhy{}30 0:0:0.0),}
\PYG{g+go}{\PYGZsq{}JULY\PYGZsq{}: (2011\PYGZhy{}7\PYGZhy{}1 0:0:0.0, 2011\PYGZhy{}7\PYGZhy{}31 0:0:0.0),}
\PYG{g+go}{\PYGZsq{}AUGUST\PYGZsq{}: (2011\PYGZhy{}8\PYGZhy{}1 :0:0.0, 2011\PYGZhy{}8\PYGZhy{}2 0:0:0.0)\PYGZcb{}}
\PYG{g+go}{\PYGZcb{}}
\end{Verbatim}
\begin{description}
\item[{..note::  Here 2011 as the key for the primary dictionary.}] \leavevmode\begin{quote}

And MAY, JUNE, JULY and AUGUST are the keys for the
secondary(inner) dictionary, which contains the
stardate and enddate of that month and year.
\end{quote}

\begin{Verbatim}[commandchars=\\\{\}]
\PYG{g+gp}{\PYGZgt{}\PYGZgt{}\PYGZgt{} }\PYG{n}{tim}\PYG{o}{.}\PYG{n}{asComponentTime}\PYG{p}{(}\PYG{p}{)}\PYG{p}{[}\PYG{l+m+mi}{0}\PYG{p}{]}
\PYG{g+go}{2011\PYGZhy{}5\PYGZhy{}25 0:0:0.0}
\end{Verbatim}

\item[{..note:: The actual firstdate of the timeAxis is 25th may 2011.}] \leavevmode\begin{quote}

But its not complete month. Even though it is returning
that available may month startdate and its enddate.
\end{quote}

\begin{Verbatim}[commandchars=\\\{\}]
\PYG{g+gp}{\PYGZgt{}\PYGZgt{}\PYGZgt{} }\PYG{n}{tim}\PYG{o}{.}\PYG{n}{asComponentTime}\PYG{p}{(}\PYG{p}{)}\PYG{p}{[}\PYG{o}{\PYGZhy{}}\PYG{l+m+mi}{1}\PYG{p}{]}
\PYG{g+go}{2011\PYGZhy{}8\PYGZhy{}2 0:0:0.0}
\end{Verbatim}

\item[{..note:: The actual lastdate of the timeAxis is 2nd aug 2011.}] \leavevmode
But its not complete month. Even though it is returning
that available may month startdate and its enddate.

\item[{..note:: It also returnning the fully available months, in this}] \leavevmode
example June \& July.

\end{description}

\item[{example 2:}] \leavevmode
\begin{Verbatim}[commandchars=\\\{\}]
\PYG{g+gp}{\PYGZgt{}\PYGZgt{}\PYGZgt{} }\PYG{n}{tim1} \PYG{o}{=} \PYG{n}{\PYGZus{}generateTimeAxis}\PYG{p}{(}\PYG{l+m+mi}{70}\PYG{p}{,} \PYG{l+s}{\PYGZsq{}}\PYG{l+s}{2011\PYGZhy{}12\PYGZhy{}1}\PYG{l+s}{\PYGZsq{}}\PYG{p}{)}
\PYG{g+gp}{\PYGZgt{}\PYGZgt{}\PYGZgt{} }\PYG{n}{tim1}
\PYG{g+go}{   id: time}
\PYG{g+go}{   Designated a time axis.}
\PYG{g+go}{   units:  days since 2011\PYGZhy{}12\PYGZhy{}1}
\PYG{g+go}{   Length: 70}
\PYG{g+go}{   First:  0}
\PYG{g+go}{   Last:   69}
\PYG{g+go}{   Other axis attributes:}
\PYG{g+go}{      calendar: gregorian}
\PYG{g+go}{      axis: T}
\PYG{g+go}{   Python id:  0xa2c10ac}
\end{Verbatim}

\begin{Verbatim}[commandchars=\\\{\}]
\PYG{g+gp}{\PYGZgt{}\PYGZgt{}\PYGZgt{} }\PYG{n}{getTimeAxisMonths}\PYG{p}{(}\PYG{n}{tim1}\PYG{p}{,} \PYG{n}{returnType} \PYG{o}{=} \PYG{l+s}{\PYGZsq{}}\PYG{l+s}{s}\PYG{l+s}{\PYGZsq{}}\PYG{p}{,}
\PYG{g+go}{                             returnHour = \PYGZsq{}n\PYGZsq{})}
\PYG{g+go}{\PYGZob{}2011: \PYGZob{}\PYGZsq{}DECEMBER\PYGZsq{}: (\PYGZsq{}20111201\PYGZsq{}, \PYGZsq{}20111231\PYGZsq{})\PYGZcb{},}
\PYG{g+go}{ 2012: \PYGZob{}\PYGZsq{}JANUARY\PYGZsq{}: (\PYGZsq{}20120101\PYGZsq{}, \PYGZsq{}20120131\PYGZsq{})\PYGZcb{}\PYGZcb{}}
\end{Verbatim}
\begin{description}
\item[{..note:: In this example, we will get 2011 and 2012 are the}] \leavevmode
keys of the primary dictionary. And 2011 has `DECEMBER'
month and its startdate \& enddate as key and value.
Same as for 2012 has `JANUARY' month and its startdate
\& enddate as key and value. Here dates are in yyyymmdd
string formate. Here we passed returnHour as `n'.

\end{description}

\end{description}
\end{quote}

Written By : Arulalan.T

Date : 06.03.2013

\end{fulllineitems}

\index{getTimeAxisPartialMonths() (xml\_data\_access.TimeUtility method)}

\begin{fulllineitems}
\phantomsection\label{diagnosisutils:xml_data_access.TimeUtility.getTimeAxisPartialMonths}\pysiglinewithargsret{\bfcode{getTimeAxisPartialMonths}}{\emph{timeAxis}, \emph{fullMonths='auto'}, \emph{returnType='c'}, \emph{returnHour='y'}}{}~\begin{description}
\item[{{\hyperref[diagnosisutils:xml_data_access.TimeUtility.getTimeAxisPartialMonths]{\code{getTimeAxisPartialMonths()}}} (\autopageref*{diagnosisutils:xml_data_access.TimeUtility.getTimeAxisPartialMonths}): Get the partially available months}] \leavevmode
name and  its firstday \& lastday from the passed timeAxis.
Its not fully available months.

\end{description}

Condition : timeAxis must be an instance of cdms2.axis.TransientAxis

Inputs : Pass the any range of timeAxis object.
\begin{quote}

fullMonths is which contains the year and its full months
name in dictionary. i.e. fullMonths is the output of the
\emph{getTimeAxisFullMonths{}`(...) method for the same timeAxis.
By default it takes `auto' string. It means, it should call
the {}`getTimeAxisFullMonths} method. Otherwise user can pass
the same kind of input.

returnType is either `c' or `s'. If `c' means the dates are
cdtime.comptime object itself. if `s' means the dates are
yyyymmddhh string (by default) or yyyymmdd w.r.t returnHour.

returnHour takes either `y/yes' or `n/no'.
\end{quote}
\begin{description}
\item[{Outputs}] \leavevmode{[}It should return a dictionary which has key as the year.{]}
This year key has the value as dictionary type itself.
The nested dictionary has month as key and
value as tuple, which contains the stardate, enddate and
no of days of that partial month.
It should return month only for partially available month
in the passed timeAxis.
i.e. If timeAxis has some incomplete months of particular
year means, those months start,enddate and its no of days
should be retuned.

\item[{Usage :}] \leavevmode\begin{description}
\item[{..seealso:: we can pass any range of timeAxis. Even its hourly}] \leavevmode
series, it should works. But it should return month,
dates, total days for partially available dates of
months in timeAxis.

\item[{example 1:}] \leavevmode
\begin{Verbatim}[commandchars=\\\{\}]
\PYG{g+gp}{\PYGZgt{}\PYGZgt{}\PYGZgt{} }\PYG{n}{tim} \PYG{o}{=} \PYG{n}{\PYGZus{}generateTimeAxis}\PYG{p}{(}\PYG{l+m+mi}{37}\PYG{p}{,} \PYG{l+s}{\PYGZsq{}}\PYG{l+s}{2011\PYGZhy{}5\PYGZhy{}25}\PYG{l+s}{\PYGZsq{}}\PYG{p}{)}
\PYG{g+go}{\PYGZsh{} here we genearate the time axis with partial May month and}
\PYG{g+go}{\PYGZsh{} complete June month.}
\PYG{g+go}{\PYGZgt{}\PYGZgt{}\PYGZgt{}}
\PYG{g+gp}{\PYGZgt{}\PYGZgt{}\PYGZgt{} }\PYG{n}{tim}
\PYG{g+go}{   id: time}
\PYG{g+go}{   Designated a time axis.}
\PYG{g+go}{   units:  days since 2011\PYGZhy{}5\PYGZhy{}25}
\PYG{g+go}{   Length: 37}
\PYG{g+go}{   First:  0}
\PYG{g+go}{   Last:   36}
\PYG{g+go}{   Other axis attributes:}
\PYG{g+go}{      calendar: gregorian}
\PYG{g+go}{      axis: T}
\PYG{g+go}{   Python id:  0xa5604ec}
\PYG{g+go}{\PYGZgt{}\PYGZgt{}\PYGZgt{}}
\PYG{g+gp}{\PYGZgt{}\PYGZgt{}\PYGZgt{} }\PYG{n}{getTimeAxisPartialMonths}\PYG{p}{(}\PYG{n}{tim}\PYG{p}{)}
\PYG{g+go}{\PYGZob{}2011: \PYGZob{}\PYGZsq{}MAY\PYGZsq{}: (2011\PYGZhy{}5\PYGZhy{}25 0:0:0.0, 2011\PYGZhy{}5\PYGZhy{}31 0:0:0.0, 7)\PYGZcb{}\PYGZcb{}}
\end{Verbatim}
\begin{description}
\item[{..note::  Here 2011 as the key for the primary dictionary.}] \leavevmode\begin{quote}
\begin{quote}

And May is the key for the secondary(inner)
dictionary, which contains the stardate, enddate and
total days of the May month.
of that month and year.
\end{quote}

\begin{Verbatim}[commandchars=\\\{\}]
\PYG{g+gp}{\PYGZgt{}\PYGZgt{}\PYGZgt{} }\PYG{n}{getTimeAxisFullMonths}\PYG{p}{(}\PYG{n}{tim}\PYG{p}{)}
\PYG{g+go}{\PYGZob{}2011: \PYGZob{}\PYGZsq{}JUNE\PYGZsq{}: (2011\PYGZhy{}6\PYGZhy{}1 0:0:0.0, 2011\PYGZhy{}6\PYGZhy{}30 0:0:0.0)\PYGZcb{}\PYGZcb{}}
\end{Verbatim}
\end{quote}
\begin{description}
\item[{..note:: Here we called the fullMonths. So the June month has}] \leavevmode\begin{quote}

returned with its stardate and enddate.
\end{quote}

\begin{Verbatim}[commandchars=\\\{\}]
\PYG{g+gp}{\PYGZgt{}\PYGZgt{}\PYGZgt{} }\PYG{n}{tim}\PYG{o}{.}\PYG{n}{asComponentTime}\PYG{p}{(}\PYG{p}{)}\PYG{p}{[}\PYG{l+m+mi}{0}\PYG{p}{]}
\PYG{g+go}{2011\PYGZhy{}5\PYGZhy{}25 0:0:0.0}
\end{Verbatim}

\end{description}

\item[{..note:: The actual firstdate of the timeAxis is 25th may 2011.}] \leavevmode\begin{quote}

But its not complete month. So this may month is
returned in the above getTimeAxisPartialMonths() method
example.
\end{quote}

\begin{Verbatim}[commandchars=\\\{\}]
\PYG{g+gp}{\PYGZgt{}\PYGZgt{}\PYGZgt{} }\PYG{n}{tim}\PYG{o}{.}\PYG{n}{asComponentTime}\PYG{p}{(}\PYG{p}{)}\PYG{p}{[}\PYG{o}{\PYGZhy{}}\PYG{l+m+mi}{1}\PYG{p}{]}
\PYG{g+go}{2011\PYGZhy{}6\PYGZhy{}30 0:0:0.0}
\end{Verbatim}

\item[{..note:: The June month is complete month. So it should return}] \leavevmode
only when we call the getTimeAxisFullMonths() method,
not in getTimeAxisPartialMonths() method.

\end{description}

\end{description}

\end{description}

Written By : Arulalan.T

Date : 28.11.2011

\end{fulllineitems}

\index{getWinterData() (xml\_data\_access.TimeUtility method)}

\begin{fulllineitems}
\phantomsection\label{diagnosisutils:xml_data_access.TimeUtility.getWinterData}\pysiglinewithargsret{\bfcode{getWinterData}}{\emph{varName}, \emph{fpath}, \emph{sday=1}, \emph{smon=11}, \emph{eday=30}, \emph{emon=4}, \emph{hr=0}, \emph{**kwarg}}{}
getWinterData : winter season (November to April) data
\begin{description}
\item[{Inputs :}] \leavevmode
varName - variable name
fpath - file path
sday : starting day of the winter season {[}of all the years of the data{]}.
smon : starting month of the winter season {[}of all the years of the data{]}.
eday : ending day of the winter season {[}of all the years of the data{]}.
emon : ending month of the winter season {[}of all the years of the data{]}.
hr : hour for both start and end date
KWargs : (latitude and/or longitude) or (region) and/or level

\end{description}

Refer: getSeasonData

Written By : Arulalan.T

Date : 26.07.2012

\end{fulllineitems}

\index{has\_all() (xml\_data\_access.TimeUtility method)}

\begin{fulllineitems}
\phantomsection\label{diagnosisutils:xml_data_access.TimeUtility.has_all}\pysiglinewithargsret{\bfcode{has\_all}}{\emph{timeAxis}, \emph{deepsearch=False}, \emph{missingYears=0}, \emph{missingMonths=0}, \emph{missingHours=0}}{}
has\_all : either the passed time axis has all the time series or not !
\begin{description}
\item[{Returns}] \leavevmode{[}True it timeAxis has no missing value and no duplicates.{]}
Otherwise returns False.

\item[{deepsearch - 1 \textbar{} 0.}] \leavevmode
If deepsearch enabled return if it has missing time series
in between, then it should return those missing time series
as component time string in list along with False.

deepsearch 0 will return the boolean value in rapid speed.

\item[{missingYears - 1 \textbar{} 0.}] \leavevmode
If deepsearch is 1 and missingYears is 1, then it should find
only the missing years from the passed timeAxis. It will not
worry about the months \& hours until missingMonths \&
missingHours are enabled.

\item[{missingMonths - 1 \textbar{} 0.}] \leavevmode
If deepsearch is 1, missingYears is 1 and missingHours is 1,
then it will find the missing months also in the timeAxis.
It will not worry about the missing hours.

\item[{missingHours - 1 \textbar{} 0.}] \leavevmode
If deepsearch is 1, missingYears is 1, missingMonths is 1
and missingHours is 1, it will do just find all missing
sequance time slice from the timeAxis.

User can enable deepsearch alone instead of enabling all the
missingYears, missingMonths and missingHours.

\item[{..note:: By default deepsearch flag, will try to find out the missing}] \leavevmode
time sequance from the user passed timeAxis itself.
This function should works correctly for the timeAxis which
contains same delta(diff b/w first and second timeAxisIndex)
through out the timeAxis. Otherwise user can use missingYears
and/or missingMonths for complex timeAxis.

\item[{..note:: return true if user passed length of 1 timeAxis or if user}] \leavevmode
enabled missingYears, then if that retunes only one year,
then we can just return True ! (In that case it could be
daily or monthly or hourly) it may be have missing time
slice also, if user passed missingYears. so user has take
care of this !

\end{description}

Refer : \_getYears()

Written By : Arulalan.T

Date : 09.10.2012

\end{fulllineitems}

\index{has\_missing() (xml\_data\_access.TimeUtility method)}

\begin{fulllineitems}
\phantomsection\label{diagnosisutils:xml_data_access.TimeUtility.has_missing}\pysiglinewithargsret{\bfcode{has\_missing}}{\emph{timeAxis}, \emph{deepsearch=False}, \emph{missingYears=0}, \emph{missingMonths=0}, \emph{missingHours=0}}{}~\begin{description}
\item[{has\_missing}] \leavevmode{[}either the passed time axis has missing time series{]}
or not !

\end{description}

It just return the opposite boolean flag of the function
self.has\_all(timeAxis, ...)

For more doc, see the has\_all.\_\_doc\_\_

Refer : has\_all(), \_getYears()

Date : 11.10.2012

\end{fulllineitems}

\index{monthFirstLast() (xml\_data\_access.TimeUtility method)}

\begin{fulllineitems}
\phantomsection\label{diagnosisutils:xml_data_access.TimeUtility.monthFirstLast}\pysiglinewithargsret{\bfcode{monthFirstLast}}{\emph{month}, \emph{year}, \emph{calendarName=None}, \emph{returnType='c'}, \emph{returnHour='n'}}{}
{\hyperref[diagnosisutils:xml_data_access.TimeUtility.monthFirstLast]{\code{monthFirstLast()}}} (\autopageref*{diagnosisutils:xml_data_access.TimeUtility.monthFirstLast}): To find and return the first date and
last date of the given month of the year, with cdtime.calendar option.
\begin{description}
\item[{Condition :}] \leavevmode
passing month should be either integer of month or name of
the month in string.
year should be an integer or string
calendar is optional. It takes default calendar

\item[{Inputs :}] \leavevmode
month may be even in 3 char like `apr' or `April' or `aPRiL' or
like any month
year must be passed as integer.
returnType is either `c' or `s'. If `c' means it should return
as cdtime.comptime object and if `s' means it should return date as
yyymmddhh or yyyymmdd string formate.
returnHour is either `y' or `n'. If yes means, and returnType is
`s' means, it should return hour also.(i.e yyymmddhh), otherwise
yyyymmdd only (by default).

\item[{Outputs :}] \leavevmode
It should return the first date and last date of the given month
\& year in yyyymmdd string formate inside tuple

\item[{Usage :}] \leavevmode\begin{description}
\item[{example1 :}] \leavevmode
\begin{Verbatim}[commandchars=\\\{\}]
\PYG{g+gp}{\PYGZgt{}\PYGZgt{}\PYGZgt{} }\PYG{n}{monthFirstLast}\PYG{p}{(}\PYG{l+m+mi}{4}\PYG{p}{,}\PYG{l+m+mi}{2010}\PYG{p}{)}
\PYG{g+gp}{\PYGZgt{}\PYGZgt{}\PYGZgt{} }\PYG{p}{(}\PYG{l+m+mi}{2010}\PYG{o}{\PYGZhy{}}\PYG{l+m+mi}{4}\PYG{o}{\PYGZhy{}}\PYG{l+m+mi}{1} \PYG{l+m+mi}{0}\PYG{p}{:}\PYG{l+m+mi}{0}\PYG{p}{:}\PYG{l+m+mf}{0.0}\PYG{p}{,} \PYG{l+m+mi}{2010}\PYG{o}{\PYGZhy{}}\PYG{l+m+mi}{4}\PYG{o}{\PYGZhy{}}\PYG{l+m+mi}{30} \PYG{l+m+mi}{0}\PYG{p}{:}\PYG{l+m+mi}{0}\PYG{p}{:}\PYG{l+m+mf}{0.0}\PYG{p}{)}
\end{Verbatim}

\begin{notice}{note}{Note:}
It should return as cdtime.comptime object
\end{notice}

\item[{example2 :}] \leavevmode
\begin{Verbatim}[commandchars=\\\{\}]
\PYG{g+gp}{\PYGZgt{}\PYGZgt{}\PYGZgt{} }\PYG{n}{monthFirstLast}\PYG{p}{(}\PYG{l+s}{\PYGZsq{}}\PYG{l+s}{feb}\PYG{l+s}{\PYGZsq{}}\PYG{p}{,}\PYG{l+s}{\PYGZsq{}}\PYG{l+s}{2010}\PYG{l+s}{\PYGZsq{}}\PYG{p}{,}\PYG{n}{returnType} \PYG{o}{=} \PYG{l+s}{\PYGZsq{}}\PYG{l+s}{s}\PYG{l+s}{\PYGZsq{}}\PYG{p}{)}
\PYG{g+gp}{\PYGZgt{}\PYGZgt{}\PYGZgt{} }\PYG{p}{(}\PYG{l+s}{\PYGZsq{}}\PYG{l+s}{20100228}\PYG{l+s}{\PYGZsq{}}\PYG{p}{,} \PYG{l+s}{\PYGZsq{}}\PYG{l+s}{20100228}\PYG{l+s}{\PYGZsq{}}\PYG{p}{)}
\end{Verbatim}

\begin{notice}{note}{Note:}
It should return in yyyymmdd string formate
\end{notice}

\end{description}

\end{description}

Written by : Arulalan.T

Date : 29.04.2011

\end{fulllineitems}

\index{moveTime() (xml\_data\_access.TimeUtility method)}

\begin{fulllineitems}
\phantomsection\label{diagnosisutils:xml_data_access.TimeUtility.moveTime}\pysiglinewithargsret{\bfcode{moveTime}}{\emph{year}, \emph{month}, \emph{day}, \emph{moveday=0}, \emph{movehour=0}, \emph{calendarName=None}, \emph{returnType='c'}}{}~\begin{description}
\item[{{\hyperref[diagnosisutils:xml_data_access.TimeUtility.moveTime]{\code{moveTime()}}} (\autopageref*{diagnosisutils:xml_data_access.TimeUtility.moveTime}): To move the day or/and hour in both direction and}] \leavevmode
get the moved date yyyymmdd/yyyymmddhh format

\item[{Condition}] \leavevmode{[}passing year,month,day,moveday,movehour should be integer{]}
type.

\item[{Inputs}] \leavevmode{[}moveday is an integer to move the date. If it is negative,{]}
then we should get the previous date with interval of the
no of days {[}i.e. moveday{]}

movehour is an integer to move the hours. If it is negative,
then we should get the previous day hours with interval of
the hours {[}i.e. movehour{]}

returnType is either `c' or `s'.
`c' means cdtime.comptime object
`s' means yyyymmdd string formate if movehour is 0 and
\begin{quote}

yyyymmddhh if movehour has passed some hour.
\end{quote}

\item[{Outputs}] \leavevmode{[}It should return the comptime date object by default.{]}
using returnType = `s', we can get yyyymmdd or yyymmddhh
string formate.

\item[{Usage :}] \leavevmode\begin{description}
\item[{example1 :}] \leavevmode
\begin{Verbatim}[commandchars=\\\{\}]
\PYG{g+gp}{\PYGZgt{}\PYGZgt{}\PYGZgt{} }\PYG{n}{moveTime}\PYG{p}{(}\PYG{l+m+mi}{2011}\PYG{p}{,} \PYG{l+m+mo}{04}\PYG{p}{,} \PYG{l+m+mo}{03}\PYG{p}{,} \PYG{n}{moveday}\PYG{o}{=}\PYG{l+m+mi}{200}\PYG{p}{)}
\PYG{g+go}{2011\PYGZhy{}10\PYGZhy{}20 0:0:0.0}
\PYG{g+go}{    ..note:: 200 days moved}
\PYG{g+gp}{\PYGZgt{}\PYGZgt{}\PYGZgt{} }\PYG{n}{moveTime}\PYG{p}{(}\PYG{l+m+mi}{2011}\PYG{p}{,} \PYG{l+m+mo}{04}\PYG{p}{,} \PYG{l+m+mo}{03}\PYG{p}{,} \PYG{n}{moveday} \PYG{o}{=} \PYG{o}{\PYGZhy{}}\PYG{l+m+mi}{200}\PYG{p}{,} \PYG{n}{returnType}\PYG{o}{=}\PYG{l+s}{\PYGZsq{}}\PYG{l+s}{s}\PYG{l+s}{\PYGZsq{}}\PYG{p}{)}
\PYG{g+go}{\PYGZsq{}2010\PYGZhy{}9\PYGZhy{}15\PYGZsq{}}
\PYG{g+go}{    ..note:: 200 days moved in backward and return as string}
\PYG{g+go}{             in yyyymmdd formate}
\end{Verbatim}

\item[{example2 :}] \leavevmode
\begin{Verbatim}[commandchars=\\\{\}]
\PYG{g+gp}{\PYGZgt{}\PYGZgt{}\PYGZgt{} }\PYG{n}{moveTime}\PYG{p}{(}\PYG{l+m+mi}{2011}\PYG{p}{,} \PYG{l+m+mi}{4}\PYG{p}{,} \PYG{l+m+mi}{3}\PYG{p}{,} \PYG{n}{moveday} \PYG{o}{=} \PYG{l+m+mi}{0}\PYG{p}{,} \PYG{n}{movehour} \PYG{o}{=} \PYG{l+m+mi}{10}\PYG{p}{)}
\PYG{g+go}{2011\PYGZhy{}4\PYGZhy{}3 10:0:0.0}
\PYG{g+go}{    ..note:: 0 days moved.But 10 hours moved.}
\end{Verbatim}

\begin{Verbatim}[commandchars=\\\{\}]
\PYG{g+gp}{\PYGZgt{}\PYGZgt{}\PYGZgt{} }\PYG{n}{moveTime}\PYG{p}{(}\PYG{l+m+mi}{2011}\PYG{p}{,}\PYG{l+m+mi}{4}\PYG{p}{,}\PYG{l+m+mi}{3}\PYG{p}{,} \PYG{n}{moveday} \PYG{o}{=} \PYG{l+m+mi}{2}\PYG{p}{,} \PYG{n}{movehour} \PYG{o}{=} \PYG{l+m+mi}{10}\PYG{p}{)}
\PYG{g+go}{2011\PYGZhy{}4\PYGZhy{}5 10:0:0.0}
\PYG{g+go}{    ..note:: Both days and hours are moved.}
\PYG{g+go}{             2 days, 10 hours moved.}
\end{Verbatim}

\item[{example3 :}] \leavevmode
\begin{Verbatim}[commandchars=\\\{\}]
\PYG{g+gp}{\PYGZgt{}\PYGZgt{}\PYGZgt{} }\PYG{n}{moveTime}\PYG{p}{(}\PYG{l+m+mi}{2011}\PYG{p}{,}\PYG{l+m+mi}{4}\PYG{p}{,}\PYG{l+m+mi}{3}\PYG{p}{,}\PYG{n}{moveday}\PYG{o}{=}\PYG{l+m+mi}{2}\PYG{p}{,}\PYG{n}{movehour}\PYG{o}{=}\PYG{l+m+mi}{10}\PYG{p}{,}\PYG{n}{returnType}\PYG{o}{=}\PYG{l+s}{\PYGZsq{}}\PYG{l+s}{s}\PYG{l+s}{\PYGZsq{}}\PYG{p}{)}
\PYG{g+go}{\PYGZsq{}2011040510\PYGZsq{}}
\PYG{g+go}{    ..note:: Here passed returnType as \PYGZsq{}s\PYGZsq{}. Also passed hour}
\PYG{g+go}{             as 10. So it should return as yyyymmddhh fromate}
\end{Verbatim}

\begin{Verbatim}[commandchars=\\\{\}]
\PYG{g+gp}{\PYGZgt{}\PYGZgt{}\PYGZgt{} }\PYG{n}{moveTime}\PYG{p}{(}\PYG{l+m+mi}{2011}\PYG{p}{,}\PYG{l+m+mi}{4}\PYG{p}{,}\PYG{l+m+mi}{3}\PYG{p}{,}\PYG{n}{moveday} \PYG{o}{=} \PYG{l+m+mi}{2}\PYG{p}{,}\PYG{n}{returnType}\PYG{o}{=}\PYG{l+s}{\PYGZsq{}}\PYG{l+s}{s}\PYG{l+s}{\PYGZsq{}}\PYG{p}{)}
\PYG{g+go}{\PYGZsq{}20110405\PYGZsq{}}
\PYG{g+go}{    ..note:: Here we didnt pass hour. So it should return}
\PYG{g+go}{             yyyymmdd formate only.}
\end{Verbatim}

\item[{example4 :}] \leavevmode
\begin{Verbatim}[commandchars=\\\{\}]
\PYG{g+gp}{\PYGZgt{}\PYGZgt{}\PYGZgt{} }\PYG{n}{moveTime}\PYG{p}{(}\PYG{l+m+mi}{2011}\PYG{p}{,}\PYG{l+m+mi}{4}\PYG{p}{,}\PYG{l+m+mi}{3}\PYG{p}{,}\PYG{n}{returnType}\PYG{o}{=}\PYG{l+s}{\PYGZsq{}}\PYG{l+s}{s}\PYG{l+s}{\PYGZsq{}}\PYG{p}{)}
\PYG{g+go}{\PYGZsq{}20110403\PYGZsq{}}
\PYG{g+go}{    ..note:: Here we didnt pass any moveday and any movehour.}
\PYG{g+go}{             So it should return only what we passed the date,}
\PYG{g+go}{             without any movements in days/hours.}
\end{Verbatim}

\item[{example4 :}] \leavevmode
\begin{Verbatim}[commandchars=\\\{\}]
\PYG{g+gp}{\PYGZgt{}\PYGZgt{}\PYGZgt{} }\PYG{n}{moveTime}\PYG{p}{(}\PYG{l+m+mi}{2011}\PYG{p}{,}\PYG{l+m+mi}{2}\PYG{p}{,}\PYG{l+m+mi}{28}\PYG{p}{,}\PYG{l+m+mi}{366}\PYG{p}{)}
\PYG{g+go}{2012\PYGZhy{}2\PYGZhy{}29 0:0:0.0}
\PYG{g+go}{    ..note:: 2012 is a leap year. By default it take}
\PYG{g+go}{             cdtime.DefaultCalendar.}
\end{Verbatim}

\begin{Verbatim}[commandchars=\\\{\}]
\PYG{g+gp}{\PYGZgt{}\PYGZgt{}\PYGZgt{} }\PYG{n}{moveTime}\PYG{p}{(}\PYG{l+m+mi}{2011}\PYG{p}{,}\PYG{l+m+mi}{2}\PYG{p}{,}\PYG{l+m+mi}{28}\PYG{p}{,}\PYG{l+m+mi}{366}\PYG{p}{,}\PYG{n}{calendarName}\PYG{o}{=}\PYG{n}{cdtime}\PYG{o}{.}\PYG{n}{NoLeapCalendar}\PYG{p}{)}
\PYG{g+go}{2012\PYGZhy{}3\PYGZhy{}1 0:0:0.0}
\PYG{g+go}{    ..note:: Eventhough 2012 is a leap year, it doesnt give}
\PYG{g+go}{             date like previous example, because we have}
\PYG{g+go}{             passed cdtime.cdtime.NoLeapCalendar.}
\end{Verbatim}

\end{description}

\end{description}

Written by : Arulalan.T

Date : 06.04.2011
Updated : 21.08.2011

\end{fulllineitems}

\index{tRange() (xml\_data\_access.TimeUtility method)}

\begin{fulllineitems}
\phantomsection\label{diagnosisutils:xml_data_access.TimeUtility.tRange}\pysiglinewithargsret{\bfcode{tRange}}{\emph{startdate}, \emph{enddate}, \emph{stepday=0}, \emph{stephour=0}, \emph{calendarName=None}, \emph{returnType='c'}, \emph{returnHour='y'}}{}~\begin{description}
\item[{{\hyperref[diagnosisutils:xml_data_access.TimeUtility.tRange]{\code{tRange()}}} (\autopageref*{diagnosisutils:xml_data_access.TimeUtility.tRange}): generate the dates in yyyymmdd formate or yyymmddhh}] \leavevmode
formate or cdtime.comptime object from startdate to enddate
with stepday or stephour.
we can set the cdtime.calendarName to generate the date(s)
in between the given range.

tRange means timeRange

\item[{Condition :}] \leavevmode
The startdate and enddate must be either yyyymmdd or
yyyymmddhh or cdtime.comptime object or cdtime.comptime
string formate.
We can use either stepday or stephour at a time. Can not use
both(stepday and stephour) at the same time.
if enddate is higher than the startdate, then stepday/
stephour must be +ve.
if enddate is lower than the startdate, then stepday/stephour
must be -ve.
By default stepday is 0 day and stephour is 0 hour.

\item[{Inputs :}] \leavevmode
startdate, enddate
stepday to skip the days.
stephour to skip the hours.
calendarName is one of the cdtime calendar type
returnType is either `s' or `c'. if `s' means the return date
should be in string type. if `c' means the return date should
be cdtime type itself.
Default returnType takes `c' as arg.
returnHour is either `y' or `yes' or `n' or `no'. If `y/yes'
means it should return the hour (yyymmddhh), if returnType
is `s'. If `n/no' means it shouldnt return hour (yyyymmdd),
if returnType is `s'.
Default returnHour takes `y' as arg.

\item[{Outputs :}] \leavevmode
It should return a list which contains the date(s) in between
the startdate and enddate including both the startdate and
enddate.

\item[{Usage :}] \leavevmode\begin{description}
\item[{example1 :}] \leavevmode
\begin{Verbatim}[commandchars=\\\{\}]
\PYG{g+gp}{\PYGZgt{}\PYGZgt{}\PYGZgt{} }\PYG{n}{tRange}\PYG{p}{(}\PYG{l+m+mi}{20110407}\PYG{p}{,} \PYG{l+m+mi}{20110410}\PYG{p}{,} \PYG{n}{stepday} \PYG{o}{=} \PYG{l+m+mi}{1}\PYG{p}{,} \PYG{n}{returnType} \PYG{o}{=} \PYG{l+s}{\PYGZsq{}}\PYG{l+s}{s}\PYG{l+s}{\PYGZsq{}}\PYG{p}{)}
\PYG{g+go}{[\PYGZsq{}2011040700\PYGZsq{}, \PYGZsq{}2011040800\PYGZsq{}, \PYGZsq{}2011040900\PYGZsq{}, \PYGZsq{}2011041000\PYGZsq{}]}
\end{Verbatim}
\begin{description}
\item[{..note::  Here returnType is `s' and returnHour is `yes' by}] \leavevmode\begin{quote}

default. So it should return with hour (yyymmddhh).
\end{quote}

\begin{Verbatim}[commandchars=\\\{\}]
\PYG{g+gp}{\PYGZgt{}\PYGZgt{}\PYGZgt{} }\PYG{n}{tRange}\PYG{p}{(}\PYG{l+m+mi}{20110407}\PYG{p}{,} \PYG{l+m+mi}{20110410}\PYG{p}{,} \PYG{n}{stepday} \PYG{o}{=} \PYG{l+m+mi}{1}\PYG{p}{,} \PYG{n}{returnType} \PYG{o}{=} \PYG{l+s}{\PYGZsq{}}\PYG{l+s}{s}\PYG{l+s}{\PYGZsq{}}\PYG{p}{,}
\PYG{g+go}{                                         returnHour = \PYGZsq{}no\PYGZsq{})}
\PYG{g+go}{[\PYGZsq{}20110407\PYGZsq{}, \PYGZsq{}20110408\PYGZsq{}, \PYGZsq{}20110409\PYGZsq{}, \PYGZsq{}20110410\PYGZsq{}]}
\end{Verbatim}

\item[{..note:: Here we passed returnHour is `no'. So it should not}] \leavevmode
return hour. (only yyyymmdd)

\end{description}

\item[{example2 :}] \leavevmode
\begin{Verbatim}[commandchars=\\\{\}]
\PYG{g+gp}{\PYGZgt{}\PYGZgt{}\PYGZgt{} }\PYG{n}{tRange}\PYG{p}{(}\PYG{l+m+mi}{20120227}\PYG{p}{,} \PYG{l+m+mi}{20120301} \PYG{p}{,} \PYG{n}{stepday} \PYG{o}{=} \PYG{l+m+mi}{1}\PYG{p}{,}
\PYG{g+go}{                  calendarName = cdtime.NoLeapCalendar,}
\PYG{g+go}{                  returnType = \PYGZsq{}s\PYGZsq{}, returnHour = \PYGZsq{}no\PYGZsq{})}
\PYG{g+go}{[\PYGZsq{}20120227\PYGZsq{}, \PYGZsq{}20120228\PYGZsq{}, \PYGZsq{}20120301\PYGZsq{}]}
\end{Verbatim}

\begin{notice}{note}{Note:}
In the example 2, 2012 is leap year, since we passed
cdtime.NoLeapCalendar it generated without 29th day in feb 2012.
we can use stepday as any +ve integer number.
The generator returns both startdate and enddate also.
\end{notice}

\item[{example3 :}] \leavevmode
\begin{Verbatim}[commandchars=\\\{\}]
\PYG{g+gp}{\PYGZgt{}\PYGZgt{}\PYGZgt{} }\PYG{n}{tRange}\PYG{p}{(}\PYG{n}{startdate} \PYG{o}{=} \PYG{l+m+mi}{20110407}\PYG{p}{,} \PYG{n}{enddate} \PYG{o}{=} \PYG{l+m+mi}{20110410}\PYG{p}{)}
\PYG{g+go}{[]}
\end{Verbatim}

\begin{notice}{note}{Note:}
In this example it should not generate any dates in
between the startdate and enddate, since we didnt pass
either stepday or stephour.
\end{notice}

\item[{example4 :}] \leavevmode
\begin{Verbatim}[commandchars=\\\{\}]
\PYG{g+gp}{\PYGZgt{}\PYGZgt{}\PYGZgt{} }\PYG{n}{tRange}\PYG{p}{(}\PYG{n}{startdate} \PYG{o}{=} \PYG{n}{cdtime}\PYG{o}{.}\PYG{n}{comptime}\PYG{p}{(}\PYG{l+m+mi}{2011}\PYG{p}{,}\PYG{l+m+mo}{04}\PYG{p}{,}\PYG{l+m+mo}{07}\PYG{p}{)}\PYG{p}{,}
\PYG{g+gp}{... }         \PYG{n}{enddate} \PYG{o}{=} \PYG{n}{cdtime}\PYG{o}{.}\PYG{n}{comptime}\PYG{p}{(}\PYG{l+m+mi}{2011}\PYG{p}{,}\PYG{l+m+mo}{04}\PYG{p}{,}\PYG{l+m+mi}{10}\PYG{p}{)}\PYG{p}{,} \PYG{n}{stepday} \PYG{o}{=} \PYG{l+m+mi}{1}\PYG{p}{,}
\PYG{g+gp}{... }                                           \PYG{n}{returnType} \PYG{o}{=} \PYG{l+s}{\PYGZsq{}}\PYG{l+s}{c}\PYG{l+s}{\PYGZsq{}}\PYG{p}{)}
\PYG{g+go}{[2011\PYGZhy{}4\PYGZhy{}7 0:0:0.0, 2011\PYGZhy{}4\PYGZhy{}8 0:0:0.0, 2011\PYGZhy{}4\PYGZhy{}9 0:0:0.0,}
\PYG{g+go}{                                    2011\PYGZhy{}4\PYGZhy{}10 0:0:0.0]}
\end{Verbatim}

\begin{notice}{note}{Note:}
Here the input dates are cdtime.comptime object itself.
\end{notice}

\item[{example5 :}] \leavevmode
\begin{Verbatim}[commandchars=\\\{\}]
\PYG{g+gp}{\PYGZgt{}\PYGZgt{}\PYGZgt{} }\PYG{n}{tRange}\PYG{p}{(}\PYG{n}{startdate} \PYG{o}{=} \PYG{n}{cdtime}\PYG{o}{.}\PYG{n}{comptime}\PYG{p}{(}\PYG{l+m+mi}{2011}\PYG{p}{,}\PYG{l+m+mo}{04}\PYG{p}{,}\PYG{l+m+mi}{11}\PYG{p}{)}\PYG{p}{,}
\PYG{g+gp}{... }         \PYG{n}{enddate} \PYG{o}{=} \PYG{n}{cdtime}\PYG{o}{.}\PYG{n}{comptime}\PYG{p}{(}\PYG{l+m+mi}{2011}\PYG{p}{,}\PYG{l+m+mo}{04}\PYG{p}{,}\PYG{l+m+mi}{7}\PYG{p}{)}\PYG{p}{,} \PYG{n}{stepday} \PYG{o}{=} \PYG{o}{\PYGZhy{}}\PYG{l+m+mi}{1}\PYG{p}{,}
\PYG{g+gp}{... }         \PYG{n}{returnType} \PYG{o}{=} \PYG{l+s}{\PYGZsq{}}\PYG{l+s}{c}\PYG{l+s}{\PYGZsq{}}\PYG{p}{)}
\PYG{g+go}{[2011\PYGZhy{}4\PYGZhy{}11 0:0:0.0, 2011\PYGZhy{}4\PYGZhy{}10 0:0:0.0, 2011\PYGZhy{}4\PYGZhy{}9 0:0:0.0,}
\PYG{g+go}{                      2011\PYGZhy{}4\PYGZhy{}8 0:0:0.0, 2011\PYGZhy{}4\PYGZhy{}7 0:0:0.0]}
\end{Verbatim}

\begin{notice}{note}{Note:}
In this example we have passed startdate is higher than
then enddate, So we must have to pass the stepdays in -ve sign.
\end{notice}

\item[{example 6:}] \leavevmode\begin{quote}

\begin{Verbatim}[commandchars=\\\{\}]
\PYG{g+gp}{\PYGZgt{}\PYGZgt{}\PYGZgt{} }\PYG{n}{tRange}\PYG{p}{(}\PYG{l+s}{\PYGZsq{}}\PYG{l+s}{2011\PYGZhy{}4\PYGZhy{}7}\PYG{l+s}{\PYGZsq{}}\PYG{p}{,} \PYG{l+s}{\PYGZsq{}}\PYG{l+s}{2011\PYGZhy{}4\PYGZhy{}10}\PYG{l+s}{\PYGZsq{}}\PYG{p}{,} \PYG{n}{stepday} \PYG{o}{=} \PYG{l+m+mi}{1}\PYG{p}{)}
\PYG{g+go}{[2011\PYGZhy{}4\PYGZhy{}7 0:0:0.0, 2011\PYGZhy{}4\PYGZhy{}8 0:0:0.0, 2011\PYGZhy{}4\PYGZhy{}9 0:0:0.0,}
\PYG{g+go}{                                     2011\PYGZhy{}4\PYGZhy{}10 0:0:0.0]}
\end{Verbatim}
\begin{description}
\item[{..note:: Here it genearates the cdtime.comptime object by default}] \leavevmode\begin{quote}

We passed the inputs are cdtime.comptime date string
formate (yyyymmdd)only.
\end{quote}

\begin{Verbatim}[commandchars=\\\{\}]
\PYG{g+gp}{\PYGZgt{}\PYGZgt{}\PYGZgt{} }\PYG{n}{tRange}\PYG{p}{(}\PYG{l+s}{\PYGZsq{}}\PYG{l+s}{2011\PYGZhy{}4\PYGZhy{}7 12:0:0.0}\PYG{l+s}{\PYGZsq{}}\PYG{p}{,} \PYG{l+s}{\PYGZsq{}}\PYG{l+s}{2011\PYGZhy{}4\PYGZhy{}10 0:0:0.0}\PYG{l+s}{\PYGZsq{}}\PYG{p}{,}
\PYG{g+gp}{... }                                           \PYG{n}{stephour} \PYG{o}{=} \PYG{l+m+mi}{12}\PYG{p}{)}
\PYG{g+go}{[2011\PYGZhy{}4\PYGZhy{}7 12:0:0.0, 2011\PYGZhy{}4\PYGZhy{}8 0:0:0.0, 2011\PYGZhy{}4\PYGZhy{}8 12:0:0.0,}
\PYG{g+go}{ 2011\PYGZhy{}4\PYGZhy{}9 0:0:0.0, 2011\PYGZhy{}4\PYGZhy{}9 12:0:0.0, 2011\PYGZhy{}4\PYGZhy{}10 0:0:0.0]}
\end{Verbatim}

\item[{..note:: Here we passed 12:0:0.0 hours in startdate, 0:0:0.0}] \leavevmode
hours in enddate, and stephour as 12. Note here the
input dates are not cdtime.comptime object. But those
are cdtime.comptime string formate (yyymmddhh).

\end{description}
\end{quote}
\begin{description}
\item[{example 7:}] \leavevmode
\begin{Verbatim}[commandchars=\\\{\}]
\PYG{g+gp}{\PYGZgt{}\PYGZgt{}\PYGZgt{} }\PYG{n}{tRange}\PYG{p}{(}\PYG{l+s}{\PYGZsq{}}\PYG{l+s}{2011040712}\PYG{l+s}{\PYGZsq{}}\PYG{p}{,} \PYG{l+s}{\PYGZsq{}}\PYG{l+s}{2011\PYGZhy{}4\PYGZhy{}10 10:0:0.0}\PYG{l+s}{\PYGZsq{}}\PYG{p}{,} \PYG{n}{stephour} \PYG{o}{=} \PYG{l+m+mi}{12}\PYG{p}{)}
\PYG{g+go}{[2011\PYGZhy{}4\PYGZhy{}7 12:0:0.0, 2011\PYGZhy{}4\PYGZhy{}8 0:0:0.0, 2011\PYGZhy{}4\PYGZhy{}8 12:0:0.0,}
\PYG{g+go}{ 2011\PYGZhy{}4\PYGZhy{}9 0:0:0.0, 2011\PYGZhy{}4\PYGZhy{}9 12:0:0.0, 2011\PYGZhy{}4\PYGZhy{}10 0:0:0.0]}
\end{Verbatim}

\item[{..note:: Here startdate as in `yyymmddhh' string formate and}] \leavevmode
enddate as in cdtime.comptime string formate. you can
play with combination of differnt inputs.

\end{description}

\end{description}

\end{description}

Written by : Arulalan.T

Date : 23.08.2011

\end{fulllineitems}

\index{timestr2comp() (xml\_data\_access.TimeUtility method)}

\begin{fulllineitems}
\phantomsection\label{diagnosisutils:xml_data_access.TimeUtility.timestr2comp}\pysiglinewithargsret{\bfcode{timestr2comp}}{\emph{date}}{}~\begin{description}
\item[{{\hyperref[diagnosisutils:xml_data_access.TimeUtility.timestr2comp]{\code{timestr2comp()}}} (\autopageref*{diagnosisutils:xml_data_access.TimeUtility.timestr2comp}): To convert date from yyyymmdd{[}hh{]} formate into}] \leavevmode
cdtime.comptime formate

\item[{Condition :}] \leavevmode
passing date must be yyyymmdd formate in either int or str

\item[{Inputs:}] \leavevmode
date in yyyymmdd formate or yyyymmddhh formate.
i.e. hour(hh) is optional.

\item[{Outputs:}] \leavevmode
It should return the date in cdtime.comptime object type

\item[{Usage:}] \leavevmode\begin{description}
\item[{example1:}] \leavevmode
\begin{Verbatim}[commandchars=\\\{\}]
\PYG{g+gp}{\PYGZgt{}\PYGZgt{}\PYGZgt{} }\PYG{n}{timestr2comp}\PYG{p}{(}\PYG{l+m+mi}{20110423}\PYG{p}{)}
\PYG{g+go}{2011\PYGZhy{}4\PYGZhy{}23 0:0:0.0}
\PYG{g+go}{  .. note:: It should return as cdtime.comptype. Here we didnt}
\PYG{g+go}{            pass the hour. i.e only yyyymmdd formate}
\end{Verbatim}

\item[{example2:}] \leavevmode
\begin{Verbatim}[commandchars=\\\{\}]
\PYG{g+gp}{\PYGZgt{}\PYGZgt{}\PYGZgt{} }\PYG{n}{timestr2comp}\PYG{p}{(}\PYG{l+m+mi}{2011082010}\PYG{p}{)}
\PYG{g+go}{2011\PYGZhy{}8\PYGZhy{}20 10:0:0.0}
\PYG{g+go}{  ..note:: Here it should return cdtime with hours also.}
\PYG{g+go}{           We passed yyyymmddhh formate. i.e include hh}
\end{Verbatim}

\item[{example3:}] \leavevmode
\begin{Verbatim}[commandchars=\\\{\}]
\PYG{g+gp}{\PYGZgt{}\PYGZgt{}\PYGZgt{} }\PYG{n}{timestr2comp}\PYG{p}{(}\PYG{l+m+mi}{2011082023}\PYG{p}{)}
\PYG{g+go}{2011\PYGZhy{}8\PYGZhy{}20 23:0:0.0}
\PYG{g+go}{  ..note:: we cannot pass 24 as hour here. Max 23 hours only.}
\end{Verbatim}

\end{description}

\end{description}

Written by: Arulalan.T

Date: 23.04.2011
Updated : 21.08.2011

\end{fulllineitems}

\index{xtRange() (xml\_data\_access.TimeUtility method)}

\begin{fulllineitems}
\phantomsection\label{diagnosisutils:xml_data_access.TimeUtility.xtRange}\pysiglinewithargsret{\bfcode{xtRange}}{\emph{startdate}, \emph{enddate}, \emph{stepday=0}, \emph{stephour=0}, \emph{calendarName=None}, \emph{returnType='c'}, \emph{returnHour='y'}}{}~\begin{description}
\item[{{\hyperref[diagnosisutils:xml_data_access.TimeUtility.xtRange]{\code{xtRange()}}} (\autopageref*{diagnosisutils:xml_data_access.TimeUtility.xtRange}): generate the dates in yyyymmdd formate or yyymmddhh}] \leavevmode
formate or cdtime.comptime object from startdate to enddate
with stepday or stephour.
we can set the cdtime.calendarName to generate the date(s)
in between the given range.

xtRange means xtimeRange

\item[{Condition :}] \leavevmode
The startdate and enddate must be either yyyymmdd or
yyyymmddhh or cdtime.comptime object or cdtime.comptime
string formate.
We can use either stepday or stephour at a time. Can not use
both(stepday and stephour) at the same time.
if enddate is higher than the startdate, then stepday/
stephour must be +ve.
if enddate is lower than the startdate, then stepday/stephour
must be -ve.
By default stepday is 0 day and stephour is 0 hour.

\item[{Inputs :}] \leavevmode
startdate, enddate
stepday to skip the days.
stephour to skip the hours.
calendarName is one of the cdtime calendar type
returnType is either `s' or `c'. if `s' means the return date
should be in string type. if `c' means the return date should
be cdtime type itself.
Default returnType takes `c' as arg.
returnHour is either `y' or `yes' or `n' or `no'. If `y/yes'
means it should return the hour (yyymmddhh), if returnType
is `s'. If `n/no' means it shouldnt return hour (yyyymmdd),
if returnType is `s'.
Default returnHour takes `y' as arg.

\item[{Outputs :}] \leavevmode
It should return a generator not as list.
Using this generator we can produce the date(s) in between
the startdate and enddate including both the startdate and
enddate.

\item[{Usage :}] \leavevmode\begin{description}
\item[{example1 :}] \leavevmode
\begin{Verbatim}[commandchars=\\\{\}]
\PYG{g+gp}{\PYGZgt{}\PYGZgt{}\PYGZgt{} }\PYG{n}{gen} \PYG{o}{=} \PYG{n}{xtRange}\PYG{p}{(}\PYG{l+m+mi}{20110407}\PYG{p}{,} \PYG{l+m+mi}{20110410}\PYG{p}{,} \PYG{n}{stepday} \PYG{o}{=} \PYG{l+m+mi}{1}\PYG{p}{,}
\PYG{g+gp}{... }                                        \PYG{n}{returnType} \PYG{o}{=} \PYG{l+s}{\PYGZsq{}}\PYG{l+s}{s}\PYG{l+s}{\PYGZsq{}}\PYG{p}{)}
\PYG{g+gp}{\PYGZgt{}\PYGZgt{}\PYGZgt{} }\PYG{k}{for} \PYG{n}{i} \PYG{o+ow}{in} \PYG{n}{gen}\PYG{p}{:}
\PYG{g+gp}{... }    \PYG{k}{print} \PYG{n}{i}
\PYG{g+gp}{...}
\PYG{g+go}{2011040700}
\PYG{g+go}{2011040800}
\PYG{g+go}{2011040900}
\PYG{g+go}{2011041000}
\end{Verbatim}
\begin{description}
\item[{..note::  Here returnType is `s' and returnHour is `yes' by}] \leavevmode\begin{quote}

default. So it should return with hour (yyymmddhh).
\end{quote}

\begin{Verbatim}[commandchars=\\\{\}]
\PYG{g+gp}{\PYGZgt{}\PYGZgt{}\PYGZgt{} }\PYG{n}{gen} \PYG{o}{=} \PYG{n}{xtRange}\PYG{p}{(}\PYG{l+m+mi}{20110407}\PYG{p}{,} \PYG{l+m+mi}{20110410}\PYG{p}{,} \PYG{n}{stepday} \PYG{o}{=} \PYG{l+m+mi}{1}\PYG{p}{,}
\PYG{g+gp}{... }                      \PYG{n}{returnType} \PYG{o}{=} \PYG{l+s}{\PYGZsq{}}\PYG{l+s}{s}\PYG{l+s}{\PYGZsq{}}\PYG{p}{,} \PYG{n}{returnHour} \PYG{o}{=} \PYG{l+s}{\PYGZsq{}}\PYG{l+s}{no}\PYG{l+s}{\PYGZsq{}}\PYG{p}{)}
\PYG{g+gp}{\PYGZgt{}\PYGZgt{}\PYGZgt{} }\PYG{k}{for} \PYG{n}{i} \PYG{o+ow}{in} \PYG{n}{gen}\PYG{p}{:}
\PYG{g+gp}{... }    \PYG{k}{print} \PYG{n}{i}
\PYG{g+gp}{...}
\PYG{g+go}{20110407}
\PYG{g+go}{20110408}
\PYG{g+go}{20110409}
\PYG{g+go}{20110410}
\end{Verbatim}

\item[{..note:: Here we passed returnHour is `no'. So it should not}] \leavevmode
return hour. (only yyyymmdd)

\end{description}

\item[{example2 :}] \leavevmode
\begin{Verbatim}[commandchars=\\\{\}]
\PYG{g+gp}{\PYGZgt{}\PYGZgt{}\PYGZgt{} }\PYG{n}{gen} \PYG{o}{=} \PYG{n}{xtRange}\PYG{p}{(}\PYG{l+m+mi}{20120227}\PYG{p}{,} \PYG{l+m+mi}{20120301} \PYG{p}{,} \PYG{n}{stepday} \PYG{o}{=} \PYG{l+m+mi}{1}\PYG{p}{,}
\PYG{g+gp}{... }                 \PYG{n}{calendarName} \PYG{o}{=} \PYG{n}{cdtime}\PYG{o}{.}\PYG{n}{NoLeapCalendar}\PYG{p}{,}
\PYG{g+gp}{... }                  \PYG{n}{returnType} \PYG{o}{=} \PYG{l+s}{\PYGZsq{}}\PYG{l+s}{s}\PYG{l+s}{\PYGZsq{}}\PYG{p}{,} \PYG{n}{returnHour} \PYG{o}{=} \PYG{l+s}{\PYGZsq{}}\PYG{l+s}{no}\PYG{l+s}{\PYGZsq{}}\PYG{p}{)}
\PYG{g+gp}{\PYGZgt{}\PYGZgt{}\PYGZgt{} }\PYG{k}{for} \PYG{n}{i} \PYG{o+ow}{in} \PYG{n}{gen}\PYG{p}{:}
\PYG{g+gp}{... }    \PYG{k}{print} \PYG{n}{i}
\PYG{g+gp}{...}
\PYG{g+go}{20120227}
\PYG{g+go}{20120228}
\PYG{g+go}{20120301}
\end{Verbatim}

\begin{notice}{note}{Note:}
In the example 2, 2012 is leap year, since we passed
cdtime.NoLeapCalendar it generated without 29th day in feb 2012.
we can use stepday as any +ve integer number.
The generator returns both startdate and enddate also.
\end{notice}

\item[{example3 :}] \leavevmode
\begin{Verbatim}[commandchars=\\\{\}]
\PYG{g+gp}{\PYGZgt{}\PYGZgt{}\PYGZgt{} }\PYG{n}{gen} \PYG{o}{=} \PYG{n}{xtRange}\PYG{p}{(}\PYG{n}{startdate} \PYG{o}{=} \PYG{l+m+mi}{20110407}\PYG{p}{,} \PYG{n}{enddate} \PYG{o}{=} \PYG{l+m+mi}{20110410}\PYG{p}{)}
\PYG{g+gp}{\PYGZgt{}\PYGZgt{}\PYGZgt{} }\PYG{k}{for} \PYG{n}{i} \PYG{o+ow}{in} \PYG{n}{gen}\PYG{p}{:}
\PYG{g+gp}{... }    \PYG{k}{print} \PYG{n}{i}
\PYG{g+gp}{...}
\PYG{g+go}{\PYGZgt{}\PYGZgt{}\PYGZgt{}}
\end{Verbatim}

\begin{notice}{note}{Note:}
In this example it should not generate any dates in
between the startdate and enddate, since we didnt pass
either stepday or stephour.
\end{notice}

\item[{example4 :}] \leavevmode
\begin{Verbatim}[commandchars=\\\{\}]
\PYG{g+gp}{\PYGZgt{}\PYGZgt{}\PYGZgt{} }\PYG{n}{gen} \PYG{o}{=} \PYG{n}{xtRange}\PYG{p}{(}\PYG{n}{startdate} \PYG{o}{=} \PYG{n}{cdtime}\PYG{o}{.}\PYG{n}{comptime}\PYG{p}{(}\PYG{l+m+mi}{2011}\PYG{p}{,}\PYG{l+m+mo}{04}\PYG{p}{,}\PYG{l+m+mo}{07}\PYG{p}{)}\PYG{p}{,}
\PYG{g+gp}{... }         \PYG{n}{enddate} \PYG{o}{=} \PYG{n}{cdtime}\PYG{o}{.}\PYG{n}{comptime}\PYG{p}{(}\PYG{l+m+mi}{2011}\PYG{p}{,}\PYG{l+m+mo}{04}\PYG{p}{,}\PYG{l+m+mi}{10}\PYG{p}{)}\PYG{p}{,} \PYG{n}{stepday} \PYG{o}{=} \PYG{l+m+mi}{1}\PYG{p}{,}
\PYG{g+gp}{... }                                           \PYG{n}{returnType} \PYG{o}{=} \PYG{l+s}{\PYGZsq{}}\PYG{l+s}{c}\PYG{l+s}{\PYGZsq{}}\PYG{p}{)}
\PYG{g+gp}{\PYGZgt{}\PYGZgt{}\PYGZgt{} }\PYG{k}{for} \PYG{n}{i} \PYG{o+ow}{in} \PYG{n}{gen}\PYG{p}{:}
\PYG{g+gp}{... }    \PYG{k}{print} \PYG{n}{i}
\PYG{g+gp}{...}
\PYG{g+go}{2011\PYGZhy{}4\PYGZhy{}7 0:0:0.0}
\PYG{g+go}{2011\PYGZhy{}4\PYGZhy{}8 0:0:0.0}
\PYG{g+go}{2011\PYGZhy{}4\PYGZhy{}9 0:0:0.0}
\PYG{g+go}{2011\PYGZhy{}4\PYGZhy{}10 0:0:0.0}
\end{Verbatim}

\begin{notice}{note}{Note:}
Here the input dates are cdtime.comptime object itself.
\end{notice}

\item[{example5 :}] \leavevmode
\begin{Verbatim}[commandchars=\\\{\}]
\PYG{g+gp}{\PYGZgt{}\PYGZgt{}\PYGZgt{} }\PYG{n}{gen} \PYG{o}{=} \PYG{n}{xtRange}\PYG{p}{(}\PYG{n}{startdate} \PYG{o}{=} \PYG{n}{cdtime}\PYG{o}{.}\PYG{n}{comptime}\PYG{p}{(}\PYG{l+m+mi}{2011}\PYG{p}{,}\PYG{l+m+mo}{04}\PYG{p}{,}\PYG{l+m+mi}{11}\PYG{p}{)}\PYG{p}{,}
\PYG{g+gp}{... }         \PYG{n}{enddate} \PYG{o}{=} \PYG{n}{cdtime}\PYG{o}{.}\PYG{n}{comptime}\PYG{p}{(}\PYG{l+m+mi}{2011}\PYG{p}{,}\PYG{l+m+mo}{04}\PYG{p}{,}\PYG{l+m+mi}{7}\PYG{p}{)}\PYG{p}{,} \PYG{n}{stepday} \PYG{o}{=} \PYG{o}{\PYGZhy{}}\PYG{l+m+mi}{1}\PYG{p}{,}
\PYG{g+gp}{... }         \PYG{n}{returnType} \PYG{o}{=} \PYG{l+s}{\PYGZsq{}}\PYG{l+s}{c}\PYG{l+s}{\PYGZsq{}}\PYG{p}{)}
\PYG{g+gp}{\PYGZgt{}\PYGZgt{}\PYGZgt{} }\PYG{k}{for} \PYG{n}{i} \PYG{o+ow}{in} \PYG{n}{gen}\PYG{p}{:}
\PYG{g+gp}{... }    \PYG{k}{print} \PYG{n}{i}
\PYG{g+gp}{...}
\PYG{g+go}{2011\PYGZhy{}4\PYGZhy{}10 0:0:0.0}
\PYG{g+go}{2011\PYGZhy{}4\PYGZhy{}9 0:0:0.0}
\PYG{g+go}{2011\PYGZhy{}4\PYGZhy{}8 0:0:0.0}
\PYG{g+go}{2011\PYGZhy{}4\PYGZhy{}7 0:0:0.0}
\end{Verbatim}

\begin{notice}{note}{Note:}
In this example we have passed startdate is higher than
then enddate, So we must have to pass the stepdays in -ve sign.
\end{notice}

\item[{example 6:}] \leavevmode
\begin{Verbatim}[commandchars=\\\{\}]
\PYG{g+gp}{\PYGZgt{}\PYGZgt{}\PYGZgt{} }\PYG{n}{gen} \PYG{o}{=} \PYG{n}{xtRange}\PYG{p}{(}\PYG{l+s}{\PYGZsq{}}\PYG{l+s}{2011\PYGZhy{}4\PYGZhy{}7}\PYG{l+s}{\PYGZsq{}}\PYG{p}{,} \PYG{l+s}{\PYGZsq{}}\PYG{l+s}{2011\PYGZhy{}4\PYGZhy{}10}\PYG{l+s}{\PYGZsq{}}\PYG{p}{,} \PYG{n}{stepday} \PYG{o}{=} \PYG{l+m+mi}{1}\PYG{p}{)}
\PYG{g+gp}{\PYGZgt{}\PYGZgt{}\PYGZgt{} }\PYG{k}{for} \PYG{n}{i} \PYG{o+ow}{in} \PYG{n}{gen}\PYG{p}{:}
\PYG{g+gp}{... }    \PYG{k}{print} \PYG{n}{i}
\PYG{g+gp}{...}
\PYG{g+go}{2011\PYGZhy{}4\PYGZhy{}7 0:0:0.0}
\PYG{g+go}{2011\PYGZhy{}4\PYGZhy{}8 0:0:0.0}
\PYG{g+go}{2011\PYGZhy{}4\PYGZhy{}9 0:0:0.0}
\PYG{g+go}{2011\PYGZhy{}4\PYGZhy{}10 0:0:0.0}
\end{Verbatim}
\begin{description}
\item[{..note:: Here it genearates the cdtime.comptime object by default}] \leavevmode\begin{quote}

We passed the inputs are cdtime.comptime date string
formate (yyyymmdd)only.
\end{quote}

\begin{Verbatim}[commandchars=\\\{\}]
\PYG{g+gp}{\PYGZgt{}\PYGZgt{}\PYGZgt{} }\PYG{n}{gen} \PYG{o}{=} \PYG{n}{xtRange}\PYG{p}{(}\PYG{l+s}{\PYGZsq{}}\PYG{l+s}{2011\PYGZhy{}4\PYGZhy{}7 12:0:0.0}\PYG{l+s}{\PYGZsq{}}\PYG{p}{,} \PYG{l+s}{\PYGZsq{}}\PYG{l+s}{2011\PYGZhy{}4\PYGZhy{}10 0:0:0.0}\PYG{l+s}{\PYGZsq{}}\PYG{p}{,}
\PYG{g+go}{                                                  stephour = 12)}
\PYG{g+gp}{\PYGZgt{}\PYGZgt{}\PYGZgt{} }\PYG{k}{for} \PYG{n}{i} \PYG{o+ow}{in} \PYG{n}{gen}\PYG{p}{:}
\PYG{g+gp}{... }    \PYG{k}{print} \PYG{n}{i}
\PYG{g+gp}{...}
\PYG{g+go}{2011\PYGZhy{}4\PYGZhy{}7 12:0:0.0}
\PYG{g+go}{2011\PYGZhy{}4\PYGZhy{}8 0:0:0.0}
\PYG{g+go}{2011\PYGZhy{}4\PYGZhy{}8 12:0:0.0}
\PYG{g+go}{2011\PYGZhy{}4\PYGZhy{}9 0:0:0.0}
\PYG{g+go}{2011\PYGZhy{}4\PYGZhy{}9 12:0:0.0}
\PYG{g+go}{2011\PYGZhy{}4\PYGZhy{}10 0:0:0.0}
\end{Verbatim}

\item[{..note:: Here we passed 12:0:0.0 hours in startdate, 0:0:0.0}] \leavevmode
hours in enddate, and stephour as 12. Note here the
input dates are not cdtime.comptime object. But those
are cdtime.comptime string formate (yyymmddhh).

\end{description}

\item[{example 7:}] \leavevmode
\begin{Verbatim}[commandchars=\\\{\}]
\PYG{g+gp}{\PYGZgt{}\PYGZgt{}\PYGZgt{} }\PYG{n}{gen} \PYG{o}{=} \PYG{n}{xtRange}\PYG{p}{(}\PYG{l+s}{\PYGZsq{}}\PYG{l+s}{2011040712}\PYG{l+s}{\PYGZsq{}}\PYG{p}{,} \PYG{l+s}{\PYGZsq{}}\PYG{l+s}{2011\PYGZhy{}4\PYGZhy{}10 10:0:0.0}\PYG{l+s}{\PYGZsq{}}\PYG{p}{,}
\PYG{g+gp}{... }                                            \PYG{n}{stephour} \PYG{o}{=} \PYG{l+m+mi}{12}\PYG{p}{)}
\PYG{g+gp}{\PYGZgt{}\PYGZgt{}\PYGZgt{} }\PYG{k}{for} \PYG{n}{i} \PYG{o+ow}{in} \PYG{n}{gen}\PYG{p}{:}
\PYG{g+gp}{... }    \PYG{k}{print} \PYG{n}{i}
\PYG{g+gp}{...}
\PYG{g+go}{2011\PYGZhy{}4\PYGZhy{}7 12:0:0.0}
\PYG{g+go}{2011\PYGZhy{}4\PYGZhy{}8 0:0:0.0}
\PYG{g+go}{2011\PYGZhy{}4\PYGZhy{}8 12:0:0.0}
\PYG{g+go}{2011\PYGZhy{}4\PYGZhy{}9 0:0:0.0}
\PYG{g+go}{2011\PYGZhy{}4\PYGZhy{}9 12:0:0.0}
\PYG{g+go}{2011\PYGZhy{}4\PYGZhy{}10 0:0:0.0}
\end{Verbatim}
\begin{description}
\item[{..note:: Here startdate as in `yyymmddhh' string formate and}] \leavevmode
enddate as in cdtime.comptime string formate. you can
play with combination of differnt inputs.

\end{description}

\end{description}

\end{description}

Written by : Arulalan.T

Date : 07.04.2011

Updated : 23.08.2011

\end{fulllineitems}


\end{fulllineitems}



\section{Plot Utils}
\label{diagnosisutils:plot-utils}
The {\hyperref[diagnosisutils:plot]{plot}} (\autopageref*{diagnosisutils:plot}) module has the properties to plot the vcs vector with some default template look out.

User can control the reference point, scale of arrow marks of the vector plot.


\subsection{plot}
\label{diagnosisutils:plot}\label{diagnosisutils:module-plot}\index{plot (module)}\index{reference\_std\_dev (class in plot)}

\begin{fulllineitems}
\phantomsection\label{diagnosisutils:plot.reference_std_dev}\pysigline{\strong{class }\code{plot.}\bfcode{reference\_std\_dev}}
This class implements the type: standard deviation of a reference variable.
It is just a float value with one method that will be used in the computation of
a test variable RMS.
\index{compute\_RMS\_function() (plot.reference\_std\_dev method)}

\begin{fulllineitems}
\phantomsection\label{diagnosisutils:plot.reference_std_dev.compute_RMS_function}\pysiglinewithargsret{\bfcode{compute\_RMS\_function}}{\emph{s}, \emph{R}}{}
Compute and return the centered-pattern RMS of a test variable
from its standard-deviation and correlation.
\begin{description}
\item[{Input:}] \leavevmode
self:   reference-variable standard deviation
s:      test variable standard deviation
R:      test-variable correlation with reference variable

\end{description}

\end{fulllineitems}


\end{fulllineitems}

\index{vectorPlot() (in module plot)}

\begin{fulllineitems}
\phantomsection\label{diagnosisutils:plot.vectorPlot}\pysiglinewithargsret{\code{plot.}\bfcode{vectorPlot}}{\emph{u}, \emph{v}, \emph{name}, \emph{path=None}, \emph{reference=20.0}, \emph{scale=1}, \emph{interval=1}, \emph{svg=1}, \emph{png=0}, \emph{latlabel='lat5'}, \emph{lonlabel='lon5'}, \emph{style='portrait'}}{}
{\hyperref[diagnosisutils:plot.vectorPlot]{\code{vectorPlot()}}} (\autopageref*{diagnosisutils:plot.vectorPlot}): Plotting the vector with some default preferences.
\begin{description}
\item[{Input}] \leavevmode{[}u - u variable{]}
v - v variable
name - name to plot on the top of the vcs
path - path to save as the image file.
reference - vector reference. Default it takes 20.0 (i.e 2 degree)
scale - scaling of the arrow mark in vector plot
interval - slicing the data to reduce the density (noise)
\begin{quote}

in the vector plot, with respect to the interval.
Default it takes 1. (i.e. doesnot affect the u \& v)
\end{quote}

svg - to save image as svg
png - to save image as png

\item[{Condition}] \leavevmode{[}u must be `u variable' and v must be `v variable'.{]}
name must pass to set the name on the vector vcs
path is not passed means, it takes current workig directory
reference must be float.
interval not be 0.

\end{description}

Usage : using this function, user can plot the vector.
\begin{quote}

user can control the reference point of the vector, and scale
length of the arrow marks in plot.

Also can control the u and v data shape by interval.

filename should be generated from the `name' passed by the user,
just replacing the space into underscore `\_'.

if svg and png passed 1, the image will be saved with these
extensions in the filename.
\end{quote}

Written By : Arulalan.T

Date : 26.07.2011

\end{fulllineitems}



\section{More}
\label{diagnosisutils:more}
More utilities will be added and optimized in near future.


\chapter{Documentation of \textbf{diagnosis} source code}
\label{diagnosis:documentation-of-diagnosis-source-code}\label{diagnosis::doc}\label{diagnosis:diagnosis}
The diagnosis package contains the following modules.
\begin{itemize}
\item {} 
{\hyperref[diagnosis:monthly-progress]{Monthly Progress}} (\autopageref*{diagnosis:monthly-progress})

\item {} 
{\hyperref[diagnosis:seasonly-progress]{Seasonly Progress}} (\autopageref*{diagnosis:seasonly-progress})

\item {} 
{\hyperref[diagnosis:statistical-scores]{Statistical Scores}} (\autopageref*{diagnosis:statistical-scores})

\item {} 
{\hyperref[diagnosis:more]{More}} (\autopageref*{diagnosis:more})

\end{itemize}


\section{Monthly Progress}
\label{diagnosis:monthly-progress}
The monthly progress of {\hyperref[diagnosis:diagnosis]{diagnosis}} (\autopageref*{diagnosis:diagnosis}) are listed below.
\begin{itemize}
\item {} 
{\hyperref[diagnosis:climatology]{Climatology}} (\autopageref*{diagnosis:climatology})

\item {} 
{\hyperref[diagnosis:month-mean]{Month Mean}} (\autopageref*{diagnosis:month-mean})

\item {} 
{\hyperref[diagnosis:month-anomaly]{Month Anomaly}} (\autopageref*{diagnosis:month-anomaly})

\item {} 
{\hyperref[diagnosis:month-fcst-sys-error]{Month Fcst Sys Error}} (\autopageref*{diagnosis:month-fcst-sys-error})

\end{itemize}

These monthly progress will be automated.


\subsection{Climatology}
\label{diagnosis:climatology}\label{diagnosis:module-climatology_utils}\index{climatology\_utils (module)}\index{dailyClimatology() (in module climatology\_utils)}

\begin{fulllineitems}
\phantomsection\label{diagnosis:climatology_utils.dailyClimatology}\pysiglinewithargsret{\code{climatology\_utils.}\bfcode{dailyClimatology}}{\emph{varName}, \emph{infile}, \emph{outfile}, \emph{leapday=False}, \emph{**kwarg}}{}~\begin{description}
\item[{dailyClimatology}] \leavevmode{[}It will create the daily climatolgy and stored{]}
in the outfile.

\item[{Inputs:}] \leavevmode
varName : variable name to extract from the input file
infile : Input file absolute path
outfile : outfile absolute path (will be created in write mode)
leapday : False \textbar{} True
\begin{quote}

If it is True, then it will create 366 days climatolgy
(include 29th feb)
If it is False, then it will create 365 days climatolgy
\end{quote}

\end{description}

KWargs:
\begin{quote}
\begin{description}
\item[{ovar}] \leavevmode{[}out varName. If it is passed then the climatology variable{]}
name will be set as ovar. Otherwise the input varName will
be set to it.

\end{description}

squeeze : 1 (it will squeeze single dimension in the climatolgy)
\end{quote}

todo : need to set year 1 for 366 days climatology.

Written By : Arulalan.T
Date : 13.08.2013

\end{fulllineitems}

\index{monthlyClimatology() (in module climatology\_utils)}

\begin{fulllineitems}
\phantomsection\label{diagnosis:climatology_utils.monthlyClimatology}\pysiglinewithargsret{\code{climatology\_utils.}\bfcode{monthlyClimatology}}{\emph{varName}, \emph{infile}, \emph{outfile}, \emph{memory='low'}, \emph{**kwarg}}{}~\begin{description}
\item[{monthlyClimatology}] \leavevmode{[}It will create the monthly climatolgy.{]}
Its timeaxis dimension length is 12.

\item[{memory}] \leavevmode{[}`low'/'high'.{]}
If it is low, then it compute climatology in optimized
manner by extracting full timeseries data of particular
latitude, longitude \& level points by loop throughing
each latitude, longitude \& level axis. It needs low RAM memory.

If it is `high', then it load the whole data from the input
file and compute climatology. It needs high RAM memory.

\end{description}

KWargs:
\begin{quote}
\begin{description}
\item[{ovar}] \leavevmode{[}out varName. If it is passed then the climatology variable{]}
name will be set as ovar. Otherwise the input varName will
be set to it.

\end{description}

squeeze : 1 (it will squeeze single dimension in the climatolgy)
\end{quote}

todo : need to give option to create 366 days climatology.

Written By : Arulalan.T
Date : 13.08.2013

\end{fulllineitems}



\subsection{Month Mean}
\label{diagnosis:month-mean}
The word \emph{Mean} means average of the data. The average will be taken over the month time axis is called month mean.

The below script \emph{compute\_month\_mean.py} should explain more how we are implementing monthly mean and generating the nc files.
\phantomsection\label{diagnosis:module-compute_month_mean}\index{compute\_month\_mean (module)}\phantomsection\label{diagnosis:module-compute_month_mean.py}\index{compute\_month\_mean.py (module)}\index{genMonthMeanDirs() (in module compute\_month\_mean)}

\begin{fulllineitems}
\phantomsection\label{diagnosis:compute_month_mean.genMonthMeanDirs}\pysiglinewithargsret{\code{compute\_month\_mean.}\bfcode{genMonthMeanDirs}}{\emph{modelname}, \emph{modelpath}, \emph{modelhour}}{}~\begin{description}
\item[{\code{genMonthMeanDirs()}}] \leavevmode{[}It should generate the directory structure{]}
whenever it needs. It reads the timeAxis information of the
model data xml file(which is updating it by cdscan), and once
the full months is completed, then it should check either that
month directory is empty or not.
\begin{description}
\item[{case 1: If that directory is empty means, it should call the}] \leavevmode
function called \emph{genMonthMeanFiles}, to calculate
the mean analysis and anomaly for that month and should
store the processed files in side that directory.

\item[{case 2: If that directory is non empty means,}] \leavevmode
\textbf{**have to update***}

\end{description}

\item[{Inputs}] \leavevmode{[}modelname is the model data name, which will become part of the{]}
directory structure.
modelpath is the absolute path of data where the model xml files
are located.
climatolgyyear is the year of climatolgy data.
climregridpath is the absolute path of the climatolgy regridded
path w.r.t to this model data resolution (both horizontal and
vertical)
climpfilename is the climatolgy Partial File Name to combine the
this passed name with (at the end) of the climatolgy var name to
open the climatolgy files.

\item[{Outputs}] \leavevmode{[}It should create the directory structure in the processfilesPath{]}
and create the processed nc files.

\end{description}

Written By : Arulalan.T

Date : 01.12.2011

\end{fulllineitems}

\index{genMonthMeanFiles() (in module compute\_month\_mean)}

\begin{fulllineitems}
\phantomsection\label{diagnosis:compute_month_mean.genMonthMeanFiles}\pysiglinewithargsret{\code{compute\_month\_mean.}\bfcode{genMonthMeanFiles}}{\emph{meanMonthPath}, \emph{monthdate}, \emph{year}, \emph{typehour}, \emph{**model}}{}~\begin{description}
\item[{\code{genMonthMeanFiles()}}] \leavevmode{[}It should calculate monthly mean analysis \&{]}
monthly mean forecast hours value for the month (of year). Finally
stores it as nc files in corresponding directory path which are
passed in this function args.

\item[{Inputs}] \leavevmode{[}meanMonthPath is the absolute path where the processed month mean{]}
analysis \& fcst hour nc files are going to store.
monthdate (which contains monthname, startdate \& enddate) and
year are the inputs to extract the monthly data.
typehour is tuple which has the type key character and fcst hour
to create sub directories inside mean directory.

\end{description}

KWargs: modelName, modelXmlPath, modelXmlObj
\begin{quote}

modelName is the model data name which will become part of the
process nc files name.
modelPath is the absolute path of data where the model xml files
are located.
modelXmlObj is an instance of the GribXmlAccess class instance.
If we are passing modelXmlObj means, it will be optimized one
when we calls this same function for same model for different
months.

We can pass either modelXmlPath or modelXmlObj KWarg is enough.
\end{quote}
\begin{description}
\item[{Outputs}] \leavevmode{[}It should create monthly mean analysis and monthly mean forecast{]}
hours for all the available variables in the vars.txt file \&
store it as nc file formate in the proper directories structure
(modelname, process name, year, month and then
{[}Analysis or hours{]} hierarchy).

\end{description}

Written By : Arulalan.T

Date : 08.09.2011
Updated : 06.12.2011

\end{fulllineitems}



\subsection{Month Anomaly}
\label{diagnosis:month-anomaly}
Anomaly means the difference between the model analysis and climatology.

Monthly Anomaly : Take the difference between the model analysis data of the particular month and the climatology data of the corresponding month.

Anomaly = Analysis - Climatology

The below script \emph{compute\_month\_anomaly.py} should explain more how we are implementing monthly anomaly and generating the nc files.
\phantomsection\label{diagnosis:module-compute_month_anomaly}\index{compute\_month\_anomaly (module)}\phantomsection\label{diagnosis:module-compute_month_anomaly.py}\index{compute\_month\_anomaly.py (module)}\index{genMonthAnomalyDirs() (in module compute\_month\_anomaly)}

\begin{fulllineitems}
\phantomsection\label{diagnosis:compute_month_anomaly.genMonthAnomalyDirs}\pysiglinewithargsret{\code{compute\_month\_anomaly.}\bfcode{genMonthAnomalyDirs}}{\emph{modelname}, \emph{modelpath}, \emph{climregridpath}, \emph{climpfilename}, \emph{climatologyyear}}{}~\begin{description}
\item[{\code{genMonthAnomalyDirs()}}] \leavevmode{[}It should generate the directory structure{]}
whenever it needs. It reads the timeAxis information of the
model data xml file(which is updating it by cdscan), and once
the full months is completed, then it should check either that
month directory is empty or not.
\begin{description}
\item[{case 1: If that directory is empty means, it should call the}] \leavevmode
function called \emph{genMonthAnomalyFiles}, to calculate
the mean analysis and anomaly for that month and should
store the processed files in side that directory.

\item[{case 2: If that directory is non empty means,}] \leavevmode
\textbf{**have to update***}

\end{description}

\item[{Inputs}] \leavevmode{[}modelname is the model data name, which will become part of the{]}
directory structure.
modelpath is the absolute path of data where the model xml files
are located.
climregridpath is the absolute path of the climatolgy regridded
path w.r.t to this model data resolution (both horizontal and
vertical)
climpfilename is the climatolgy Partial File Name to combine the
this passed name with (at the end) of the climatolgy var name to
open the climatolgy files.
climatolgyyear is the year of climatolgy data.

\item[{Outputs}] \leavevmode{[}It should create the directory structure in the processfilesPath{]}
and create the processed nc files.

\end{description}

Written By : Arulalan.T

Date : 01.12.2011

\end{fulllineitems}

\index{genMonthAnomalyFiles() (in module compute\_month\_anomaly)}

\begin{fulllineitems}
\phantomsection\label{diagnosis:compute_month_anomaly.genMonthAnomalyFiles}\pysiglinewithargsret{\code{compute\_month\_anomaly.}\bfcode{genMonthAnomalyFiles}}{\emph{meanAnomalyPath}, \emph{meanAnalysisPath}, \emph{climRegridPath}, \emph{climPFileName}, \emph{climatologyYear}, \emph{monthdate}, \emph{year}, \emph{**model}}{}~\begin{description}
\item[{\code{genMonthAnomalyFiles()}}] \leavevmode{[}It should calculate monthly mean anomaly{]}
from the monthly mean analysis and monthly mean climatolgy,
for the month (of year) and process it. Finally
stores it as nc files in corresponding directory path which are
passed in this function args.

\item[{Inputs}] \leavevmode{[}meanAnomalyPath is the absolute path where the processed mean{]}
anomaly nc files are going to store.
meanAnalysisPath is the absolute path where the processed mean
analysis nc files were already stored.
climRegridPath is the absolute path where the regridded monthly
mean climatologies (w.r.t the model vertical resolution)
nc files were already stored.
climPFileName is the partial nc filename of the climatolgy.
climatologyYear is the year of the climatolgy to access it.
monthdate (which contains monthname, startdate \& enddate) and
year are the inputs to extract the monthly data.

\end{description}

KWargs: modelName, modelXmlPath, modelXmlObj
\begin{quote}

modelName is the model data name which will become part of the
process nc files name.
modelPath is the absolute path of data where the model xml files
are located.
modelXmlObj is an instance of the GribXmlAccess class instance.
If we are passing modelXmlObj means, it will be optimized one
when we calls this same function for same model for different
months.

We can pass either modelXmlPath or modelXmlObj KWarg is enough.
\end{quote}
\begin{description}
\item[{Outputs}] \leavevmode{[}It should create mean anomaly for the particular variables which{]}
are all set the clim\_var option in the vars.txt file. Finally
store it as nc file formate in the proper directories structure
(modelname, process name, year and then month hierarchy).

\end{description}

Written By : Arulalan.T

Date : 08.09.2011
Updated : 07.12.2011

\end{fulllineitems}



\subsection{Month Fcst Sys Error}
\label{diagnosis:month-fcst-sys-error}
Forecast Systematic Error means the difference between the model forecast hour data and model analysis.

This also called as \emph{Fcst Sys Err}.

Month Fcst Sys Error : Take the difference between the model forecast hour data of the particular month and the model analysis data of the same month.

Fcst Sys Err = Model Fcst Hour Data - Model Analysis

The below script \emph{compute\_month\_fcst\_sys\_error.py} should explain more how we are implementing monthly fcst sys err and generating the nc files.
\phantomsection\label{diagnosis:module-compute_month_fcst_sys_error}\index{compute\_month\_fcst\_sys\_error (module)}\phantomsection\label{diagnosis:module-compute_month_fcst_sys_error.py}\index{compute\_month\_fcst\_sys\_error.py (module)}\index{genMonthFcstSysErrDirs() (in module compute\_month\_fcst\_sys\_error)}

\begin{fulllineitems}
\phantomsection\label{diagnosis:compute_month_fcst_sys_error.genMonthFcstSysErrDirs}\pysiglinewithargsret{\code{compute\_month\_fcst\_sys\_error.}\bfcode{genMonthFcstSysErrDirs}}{\emph{modelname}, \emph{modelpath}, \emph{modelhour}}{}~\begin{description}
\item[{\code{genMonthFcstSysErrDirs()}}] \leavevmode{[}It should generate the directory structure{]}
whenever it needs. It reads the timeAxis information of the
model data xml file(which is updating it by cdscan), and once
the full months is completed, then it should check either that
month directory is empty or not.
\begin{description}
\item[{case 1: If that directory is empty means, it should call the}] \leavevmode
function called \emph{genMonthFcstSysErrFiles}, to calculate
the mean analysis and fcstsyserr for that month and should
store the processed files in side that directory.

\item[{case 2: If that directory is non empty means,}] \leavevmode
\textbf{**have to update***}

\end{description}

\item[{Inputs}] \leavevmode{[}modelname is the model data name, which will become part of the{]}
directory structure.
modelpath is the absolute path of data where the model xml files
are located.
modelhour is the list of model data hours, which will become
part of the directory structure.

\item[{Outputs}] \leavevmode{[}It should create the directory structure in the processfilesPath{]}
and create the processed nc files.

\end{description}

Written By : Arulalan.T

Date : 08.12.2011

\end{fulllineitems}

\index{genMonthFcstSysErrFiles() (in module compute\_month\_fcst\_sys\_error)}

\begin{fulllineitems}
\phantomsection\label{diagnosis:compute_month_fcst_sys_error.genMonthFcstSysErrFiles}\pysiglinewithargsret{\code{compute\_month\_fcst\_sys\_error.}\bfcode{genMonthFcstSysErrFiles}}{\emph{meanFcstSysErrPath}, \emph{meanPath}, \emph{monthdate}, \emph{year}, \emph{modelhour}, \emph{**model}}{}~\begin{description}
\item[{\code{genMonthFcstSysErrFiles()}}] \leavevmode{[}It should calculate mean analysis \&{]}
fcstsyserr for the passed month (of year) and process it. Finally
stores it as nc files in corresponding directory path which are
passed in this function args.

\item[{Inputs}] \leavevmode{[}meanFcstSysErrPath is the absolute path where the processed mean{]}
fcstsyserr nc files are going to store.
meanPath is the absolute path (partial path) where the processed
monthly mean analysis and fcst hour nc files were stored already.
monthdate (which contains monthname, startdate \& enddate) and
year are the inputs to extract the monthly data.
modelhour is the list of model data hours, which will become
part of the directory structure.

\end{description}

KWargs: modelName, modelXmlPath, modelXmlObj
\begin{quote}

modelName is the model data name which will become part of the
process nc files name.
modelPath is the absolute path of data where the model xml files
are located.
modelXmlObj is an instance of the GribXmlAccess class instance.
If we are passing modelXmlObj means, it will be optimized one
when we calls this same function for same model for different
months.

We can pass either modelXmlPath or modelXmlObj KWarg is enough.
\end{quote}
\begin{description}
\item[{Outputs}] \leavevmode{[}It should create mean forecast systematic error for all the{]}
available variables in the vars.txt file \& store it as nc file
formate in the proper directories structure
(modelname, process name, year, month and then hours hierarchy).

\end{description}

Written By : Arulalan.T

Date : 08.09.2011
Updated : 08.12.2011

\end{fulllineitems}



\section{Seasonly Progress}
\label{diagnosis:seasonly-progress}
The seasonly progress of {\hyperref[diagnosis:diagnosis]{diagnosis}} (\autopageref*{diagnosis:diagnosis}) are listed below.
\begin{itemize}
\item {} 
{\hyperref[diagnosis:season-mean]{Season Mean}} (\autopageref*{diagnosis:season-mean})

\item {} 
{\hyperref[diagnosis:collect-season-fcst-rainfall]{Collect Season Fcst Rainfall}} (\autopageref*{diagnosis:collect-season-fcst-rainfall})

\item {} 
{\hyperref[diagnosis:region-statistical-score]{Region Statistical Score}} (\autopageref*{diagnosis:region-statistical-score})

\item {} 
{\hyperref[diagnosis:season-statistical-score-spatial-distribution]{Season Statistical Score Spatial Distribution}} (\autopageref*{diagnosis:season-statistical-score-spatial-distribution})

\end{itemize}

These season progress will be automated with respect to the given season as input in the configure.txt.

The word \emph{season} means the consecutive months.

For eg: JJAS contains June, July, August, September.


\subsection{Season Mean}
\label{diagnosis:season-mean}
The word \emph{Mean} means average of the data. The average will be taken over the season time axis is called season mean.

The below script \emph{compute\_season\_mean.py} should explain more how we are implementing seasonly mean and generating the nc files.
\phantomsection\label{diagnosis:module-compute_season_mean}\index{compute\_season\_mean (module)}\phantomsection\label{diagnosis:module-compute_season_mean.py}\index{compute\_season\_mean.py (module)}\index{genMeanAnlFcstErrDirs() (in module compute\_season\_mean)}

\begin{fulllineitems}
\phantomsection\label{diagnosis:compute_season_mean.genMeanAnlFcstErrDirs}\pysiglinewithargsret{\code{compute\_season\_mean.}\bfcode{genMeanAnlFcstErrDirs}}{\emph{modelname}, \emph{modelpath}, \emph{modelhour}}{}~\begin{description}
\item[{\code{genMeanAnlFcstErrDirs()}}] \leavevmode{[}It should create the directory structure{]}
whenever it needs. It reads the timeAxis information of the
model data xml file(which is updating it by cdscan), and once
the full seasonal months are completed, then it should check
either that season directory is empty or not.
\begin{description}
\item[{case 1: If that directory is empty means, it should call the}] \leavevmode
function called \emph{genSeasonMeanFiles}, to calculate
the mean analysis and fcstsyserr for that season and should
store the processed files in side that directory.

\item[{case 2: If that directory is non empty means,}] \leavevmode
\textbf{**have to update***}

\end{description}

\item[{Inputs}] \leavevmode{[}modelname is the model data name, which will become part of the{]}
directory structure.
modelpath is the absolute path of data where the model xml files
are located.
modelhour is the list of model data hours, which will become
part of the directory structure.

\item[{Outputs}] \leavevmode{[}It should create the directory structure in the processfilesPath{]}
and create the processed nc files.

\end{description}

Written By : Arulalan.T

Date : 08.12.2011

\end{fulllineitems}

\index{genSeasonMeanFiles() (in module compute\_season\_mean)}

\begin{fulllineitems}
\phantomsection\label{diagnosis:compute_season_mean.genSeasonMeanFiles}\pysiglinewithargsret{\code{compute\_season\_mean.}\bfcode{genSeasonMeanFiles}}{\emph{meanSeasonPath}, \emph{meanMonthPath}, \emph{seasonName}, \emph{seasonMonthDate}, \emph{year}, \emph{Type}, \emph{**model}}{}~\begin{description}
\item[{\code{genSeasonMeanFiles()}}] \leavevmode{[}It should calculate the seasonly mean for{]}
either analysis or forecast systematic error. It can be choosed by
the Type argment. Finally stores it as nc files in corresponding
directory path which are passed in this function args.

\item[{Inputs}] \leavevmode{[}meanSeasonPath is the absolute path where the processed season{]}
mean analysis or forecast systematic error nc files are going to
store. Inside the fcst hours directories will be created in this
path, if needed.

meanMonthPath is the absolute path where the processed monthly
mean analysis or monthly mean fcstsyserr nc files were stored,
already.
seasonName is the name of the season.
seasonMonthDate(list of months which contains monthname,
startdate \& enddate) for the season.
year is the part of the directory structure.
Type is either `a' for analysis or `f' for fcstsyserr.

\end{description}

KWargs: modelName, modelXmlPath, modelXmlObj
\begin{quote}

modelName is the model data name which will become part of the
process nc files name.
modelHour is the model hours as list, which will become part of
the directory structure.
modelPath is the absolute path of data where the model xml files
are located.
modelXmlObj is an instance of the GribXmlAccess class instance.
If we are passing modelXmlObj means, it will be optimized one
when we calls this same function for same model for different
months.

We can pass either modelXmlPath or modelXmlObj KWarg is enough.
\end{quote}
\begin{description}
\item[{Process}] \leavevmode{[}This function should compute the seasonly mean for analysis and{]}
fcstsyserr by just opening the monthly mena analysis/fcstsyserr
nc files (according to the season's months) and multiply the
monthly mean into its weights value. So that monthly mean data
should become monthly full data (not mean). Then add it together
for the season of months. Finally takes the average by just
divide the whole season data by sum of monthly mean weights.
\begin{quote}

So it should simplify our life, just extracting data which
\end{quote}

timeAxis length is 4, for 4 months in season. (eg JJAS).

\item[{Outputs}] \leavevmode{[}It should create seasonly mean analysis and forecast systematic{]}
error for all the available variables in the vars.txt file,
and store it as nc file in the proper directory structure
(modelname, process name, year, season, and/or hours hierarchy).

\end{description}

Written By : Arulalan.T

Date : 08.12.2011

\end{fulllineitems}



\subsection{Collect Season Fcst Rainfall}
\label{diagnosis:collect-season-fcst-rainfall}
This script \emph{collect\_season\_fcst\_rainfall.py} should collect the whole season forecast hourly rainfall with respect to the hours of season.
Finally it should create the nc files for every fcst hours that contains the fcst rainfall with needed time axis to compute the further process.
\phantomsection\label{diagnosis:module-collect_season_fcst_rainfall}\index{collect\_season\_fcst\_rainfall (module)}\phantomsection\label{diagnosis:module-collect_season_fcst_rainfall.py}\index{collect\_season\_fcst\_rainfall.py (module)}\begin{description}
\item[{Written by: Dileepkumar R}] \leavevmode
JRF- IIT DELHI

\end{description}

Date: 23.06.2011

Updated By : Arulalan.T
Date: 14.09.2011
Date: 19.10.2011
\index{createSeaonFcstRainfallData() (in module collect\_season\_fcst\_rainfall)}

\begin{fulllineitems}
\phantomsection\label{diagnosis:collect_season_fcst_rainfall.createSeaonFcstRainfallData}\pysiglinewithargsret{\code{collect\_season\_fcst\_rainfall.}\bfcode{createSeaonFcstRainfallData}}{\emph{modelname}, \emph{modelpath}, \emph{modelhour}, \emph{rainfallPath}, \emph{rainfallXmlName=None}}{}~\begin{description}
\item[{\code{createSeaonFcstRainfallData()}: It should create model hours forecast}] \leavevmode
rainfall data nc files, in side the `StatiScore' directory of
processfilesPath in hierarchy structure. The fcst rainfall
timeAxis are in partners timeAxis w.r.t observation rainfall and
fcst hours.

\end{description}

\end{fulllineitems}



\subsection{Region Statistical Score}
\label{diagnosis:region-statistical-score}
The below script \emph{compute\_region\_statistical\_score.py} should compute the various statistical scores regional wise and make the nc files.

Here regions are `Central India', `Peninsular India', etc. That is region name/variable defines the latitude and longitude range.
\phantomsection\label{diagnosis:module-compute_region_statistical_score}\index{compute\_region\_statistical\_score (module)}\phantomsection\label{diagnosis:module-compute_region_statistical_score.py}\index{compute\_region\_statistical\_score.py (module)}\begin{description}
\item[{Example:}] \leavevmode
Let `ts'(Threat Score) is a statistical score,
we are calculate this for different regions.

\item[{Written by: Dileepkumar R}] \leavevmode
JRF- IIT DELHI

\end{description}

Date: 02.08.2011;

Updated By : Arulalan.T
Date : 16.09.2011
\index{genStatisticalScore() (in module compute\_region\_statistical\_score)}

\begin{fulllineitems}
\phantomsection\label{diagnosis:compute_region_statistical_score.genStatisticalScore}\pysiglinewithargsret{\code{compute\_region\_statistical\_score.}\bfcode{genStatisticalScore}}{\emph{modelname}, \emph{modelhour}, \emph{seasonName}, \emph{year}, \emph{procStatiSeason}, \emph{procRegion}, \emph{plotCSV}, \emph{rainfallPath}, \emph{rainfallXmlName=None}}{}~\begin{description}
\item[{\code{genStatisticalScore()}}] \leavevmode{[}It should compute the statistical scores{]}
like '' Threat Score, Equitable Threat Score, Accuracy(Hit Rate),
Bias Score, Probability Of Detection, False Alarm Rate, Odds Ratio,
Probability Of False Detection, Kuipers Skill Score, Log Odd Ratio,
Heidke Skill Score, Odd Ratio Skill Score, \& Extreame Dependency Score''
by accessing the observation and forecast data.

It should compute the statistical scores for different regions.

Finally it should store the scores variable in both csv and nc files
in appropriate directory hierarchy structure.
\begin{description}
\item[{..note:: We are replacing the -ve values with zeros of both the}] \leavevmode
observation and fcst data to make correct statistical scores.

\end{description}

\item[{Inputs}] \leavevmode{[}modelname, modelhour, seasonName, year are helps to generate the{]}
path. procStatiSeason is the partial path of process statistical
score season path.
procRegion is an absolute path to store the nc files
plotCSV is an absolute path to store the csv files.
rainfallPath is the path of the observation rainfall.
rainfallXmlName is the name of the xml file name, it is an
optional one. By default it takes `rainfall\_regrided.xml'

\item[{Outputs}] \leavevmode{[}It should store the computed statistical scores for all the{]}
regions and store it as both ncfile and csv files in the
appropriate directory hierarchy structure.

\end{description}

\end{fulllineitems}

\index{genStatisticalScoreDirs() (in module compute\_region\_statistical\_score)}

\begin{fulllineitems}
\phantomsection\label{diagnosis:compute_region_statistical_score.genStatisticalScoreDirs}\pysiglinewithargsret{\code{compute\_region\_statistical\_score.}\bfcode{genStatisticalScoreDirs}}{\emph{modelname}, \emph{modelhour}, \emph{rainfallPath}, \emph{rainfallXmlName=None}}{}~\begin{quote}\begin{description}
\item[{Func }] \leavevmode
\emph{genStatisticalScoreDirs} : It should generate the appropriate
directory hierarchy structure for `StatiScore' in both the processfiles
path and plotgraph path. In plotgraph path, it should create `CSV'
directory to store the `statistical scores' in csv file formate.

This function should call the \emph{genStatisticalScore} function to
compute and statistical score.

\end{description}\end{quote}
\begin{description}
\item[{Inputs}] \leavevmode{[}modelname and modelhour are the part of the directory hierarchy{]}
structure.
rainfallPath is the path of the observation rainfall.
rainfallXmlName is the name of the xml file name.

\end{description}

\end{fulllineitems}



\subsection{Season Statistical Score Spatial Distribution}
\label{diagnosis:season-statistical-score-spatial-distribution}
The below script \emph{compute\_season\_stati\_score\_spatial\_distribution.py} should compute the various statistical scores w.r.t each \& every lat, lon
location of particular region.
\phantomsection\label{diagnosis:module-compute_season_stati_score_spatial_distribution}\index{compute\_season\_stati\_score\_spatial\_distribution (module)}\phantomsection\label{diagnosis:module-compute_season_stati_score_spatial_distribution.py}\index{compute\_season\_stati\_score\_spatial\_distribution.py (module)}\begin{description}
\item[{Example:}] \leavevmode
Let `TS'(Threat Score) is a statistical score,
we are calculate this spatially.

\item[{Written by: Dileepkumar R}] \leavevmode
JRF- IIT DELHI

\end{description}

Date: 02.09.2011;

Updated By : Arulalan.T
Date : 17.09.2011
Date : 06.10.2011
\index{genStatisticalScorePath() (in module compute\_season\_stati\_score\_spatial\_distribution)}

\begin{fulllineitems}
\phantomsection\label{diagnosis:compute_season_stati_score_spatial_distribution.genStatisticalScorePath}\pysiglinewithargsret{\code{compute\_season\_stati\_score\_spatial\_distribution.}\bfcode{genStatisticalScorePath}}{\emph{modelname}, \emph{modelhour}, \emph{rainfallPath}, \emph{rainfallXmlName=None}}{}
\code{genStatisticalScorePath()}: It should make the existing path of
process files statistical score. Also if that is correct path means, it
should calls the function \emph{genStatisticalScoreSpatialDistribution} to
compute the statistical score in spatially distributed way.
\begin{description}
\item[{Inputs}] \leavevmode{[}modelname and modelhour are the part of the directory hierarchy{]}
structure.

\end{description}

\end{fulllineitems}

\index{genStatisticalScoreSpatialDistribution() (in module compute\_season\_stati\_score\_spatial\_distribution)}

\begin{fulllineitems}
\phantomsection\label{diagnosis:compute_season_stati_score_spatial_distribution.genStatisticalScoreSpatialDistribution}\pysiglinewithargsret{\code{compute\_season\_stati\_score\_spatial\_distribution.}\bfcode{genStatisticalScoreSpatialDistribution}}{\emph{modelname}, \emph{modelhour}, \emph{season}, \emph{year}, \emph{statiSeasonPath}, \emph{rainfallPath}, \emph{rainfallXmlName=None}, \emph{lat=None}, \emph{lon=None}}{}~\begin{quote}\begin{description}
\item[{Func }] \leavevmode
\emph{genStatisticalScoreSpatialDistribution} : It should compute the
statistical scores like '' Threat Score, Equitable Threat Score,
Accuracy(Hit Rate), Bias Score, Probability Of Detection, Odds Ratio,
False Alarm Rate, Probability Of False Detection, Kuipers Skill Score,
Log Odd Ratio, Heidke Skill Score, Odd Ratio Skill Score, \&
Extreame Dependency Score'' in spatially distributed way (i.e compute
scores in each and every lat \& lon points) by accessing the
observation and forecast data.

\end{description}\end{quote}
\begin{description}
\item[{Inputs}] \leavevmode{[}modelname, modelhour, season, year are helps to generate the{]}
path. statiSeasonPath is the partial path of process statistical
score season path.
rainfallPath is the path of the observation rainfall.
rainfallXmlName is the name of the xml file name, it is an
optional one. By default it takes `rainfall\_regrided.xml'.
lat, lon takes tuple args. If we passed it, then the model lat,
lon should be shrinked according to the passed lat,lon.
Some times it may helpful to do statistical score in spatially
distributed in particular region among the global lat,lon.

\item[{Outputs}] \leavevmode{[}It should store the computed statistical scores in spatially{]}
distributed way for all the modelhour(s) as nc files in the
appropriate directory hierarchy structure.

\end{description}

\end{fulllineitems}



\section{Diagnosis Plots}
\label{diagnosis:diagnosis-plots}
The diagnosis plots of {\hyperref[diagnosis:diagnosis]{diagnosis}} (\autopageref*{diagnosis:diagnosis}) are listed below.
\begin{itemize}
\item {} 
{\hyperref[diagnosis:winds-plots]{Winds Plots}} (\autopageref*{diagnosis:winds-plots})

\item {} 
{\hyperref[diagnosis:iso-plots]{Iso Plots}} (\autopageref*{diagnosis:iso-plots})

\item {} 
{\hyperref[diagnosis:statistical-score-bar-plots]{Statistical Score Bar Plots}} (\autopageref*{diagnosis:statistical-score-bar-plots})

\item {} 
{\hyperref[diagnosis:statistical-score-spatial-distribution-plots]{Statistical Score Spatial Distribution Plots}} (\autopageref*{diagnosis:statistical-score-spatial-distribution-plots})

\end{itemize}


\subsection{Winds Plots}
\label{diagnosis:winds-plots}
Generate the vector plots using U and V component of the wind data.
The below script \emph{generate\_winds\_plots.py} should generates the wind plots and save it as either png or jpg or svg in the suitable directory for month and season wise.
\phantomsection\label{diagnosis:module-generate_winds_plots}\index{generate\_winds\_plots (module)}\index{editVectorPlot() (in module generate\_winds\_plots)}

\begin{fulllineitems}
\phantomsection\label{diagnosis:generate_winds_plots.editVectorPlot}\pysiglinewithargsret{\code{generate\_winds\_plots.}\bfcode{editVectorPlot}}{\emph{modelname}, \emph{processtype}, \emph{year}, \emph{monthseason}, \emph{hour=None}, \emph{level=None}, \emph{region=None}, \emph{reference=20}, \emph{scale=1}, \emph{interval=4}, \emph{svg=0}, \emph{png=1}, \emph{latlabel='lat5'}, \emph{lonlabel='lon5'}, \emph{outpath=None}}{}~\begin{description}
\item[{{\hyperref[diagnosis:generate_winds_plots.editVectorPlot]{\code{editVectorPlot()}}} (\autopageref*{diagnosis:generate_winds_plots.editVectorPlot})}] \leavevmode{[}To edit/reproduce any particular vector plots by{]}
passing modelname, processtype, year, month/season, hour, level(s),
region, reference point of vector, scale of vector arrow markers,
interval of the U \& V datasets, svg \& png options.

\item[{Inputs}] \leavevmode{[}modelname is the part of the directory structure.{]}
processtype is any one of the processes.for eg : `Mean Analysis',
`Mean Fcst' or `Anomaly' or `FcstSysErr', etc.,
year is year in string type.
monthseason is either month name or season name. It should find
out either it is month or season and make the correct path.
hour is the hour string which is the part of the directory
structure only for `FcstSysErr' and `Mean Fcst' process type.

level is either single level, or list of levels or `all'.
`all' means, it takes all the availableLevels from the variables.
level value must be int, float only. Not be string, other than
`all' keyword.
region is the region variable which should cut particular region
shape from the global data. By default it is None, i.e. takes
global data region itself.

reference is the reference points to be plotted in the vcs
vector plot. It must be float only.
scale is the length of the arrow markers in the vcs vector plot.
interval is the integer value, to split the U and V datasets,
to make clear view of the vector plot. By default it takes 4.

svg is the flag. If flag is set, then the vector plot should be
saved as svg formate. By default it is 0.

png is the flag. If flag is set, then the vector plot should be
saved as png formate. By default it is 1.

outpath is the absolute path, where the generated plots should be
stored. By default, it is None, that is it should save in the
appropirate plotsgraphs directory, which is generated by this
function.

\end{description}

Written By : Arulalan.T

Date : 11.09.2011
Updated: 11.12.2011

\end{fulllineitems}

\index{genMonthAnomalyDirs() (in module generate\_winds\_plots)}

\begin{fulllineitems}
\phantomsection\label{diagnosis:generate_winds_plots.genMonthAnomalyDirs}\pysiglinewithargsret{\code{generate\_winds\_plots.}\bfcode{genMonthAnomalyDirs}}{\emph{modelName}, \emph{availableMonths}}{}~\begin{description}
\item[{{\hyperref[diagnosis:generate_winds_plots.genMonthAnomalyDirs]{\code{genMonthAnomalyDirs()}}} (\autopageref*{diagnosis:generate_winds_plots.genMonthAnomalyDirs}): It should generate the directory hierarichy}] \leavevmode
structure of month anomaly in the plotsgraphspath. And calls the
function genVectorPlots to make vector plots and save it inside the
appropirate directory, by reading the u, v nc files of the appropirate
process month anomaly files path.

\item[{Inputs}] \leavevmode{[}modelName is the one of the directories name.{]}
availableMonths is the dictionary which is generated by fully
available months from the timeAxis.

\item[{..note:: It should takes the levels which is set in the global config}] \leavevmode
file, and generate the vector plots to those levels only.

\end{description}

Written By : Arulalan.T

Date : 11.09.2011
Updated: 11.12.2011

\end{fulllineitems}

\index{genMonthMeanDirs() (in module generate\_winds\_plots)}

\begin{fulllineitems}
\phantomsection\label{diagnosis:generate_winds_plots.genMonthMeanDirs}\pysiglinewithargsret{\code{generate\_winds\_plots.}\bfcode{genMonthMeanDirs}}{\emph{modelName}, \emph{availableMonths}}{}~\begin{description}
\item[{{\hyperref[diagnosis:generate_winds_plots.genMonthMeanDirs]{\code{genMonthMeanDirs()}}} (\autopageref*{diagnosis:generate_winds_plots.genMonthMeanDirs}): It should generate the directory hierarichy}] \leavevmode
structure of month mean in the plotsgraphspath. And calls the
function genVectorPlots to make vector plots and save it inside the
appropirate directory, by reading the u, v nc files of the appropirate
process month mean files path.

\item[{Inputs}] \leavevmode{[}modelName is the one of the directories name.{]}
availableMonths is the dictionary which is generated by fully
available months from the timeAxis.

\item[{..note:: It should takes the levels which is set in the global config}] \leavevmode
file, and generate the vector plots to those levels only.

\end{description}

Written By : Arulalan.T

Date : 11.09.2011
Updated: 11.12.2011

\end{fulllineitems}

\index{genSeasonFcstSysErrDirs() (in module generate\_winds\_plots)}

\begin{fulllineitems}
\phantomsection\label{diagnosis:generate_winds_plots.genSeasonFcstSysErrDirs}\pysiglinewithargsret{\code{generate\_winds\_plots.}\bfcode{genSeasonFcstSysErrDirs}}{\emph{modelName}, \emph{modelHour}, \emph{availableMonths}}{}~\begin{description}
\item[{{\hyperref[diagnosis:generate_winds_plots.genSeasonFcstSysErrDirs]{\code{genSeasonFcstSysErrDirs()}}} (\autopageref*{diagnosis:generate_winds_plots.genSeasonFcstSysErrDirs}): It should generate the directory hierarichy}] \leavevmode
structure of season fcstsyserr in the plotsgraphspath. And calls the
function genVectorPlots to make vector plots and save it inside the
appropirate directory, by reading the xml file of the appropirate
process season fcstsyserr files path.

\item[{Inputs}] \leavevmode{[}modelName is the one of the directories name.{]}
modelHour is the one of the directories name.
availableMonths is the dictionary which is generated by fully
available months from the timeAxis.

\item[{..note:: It should takes the levels which is set in the global config}] \leavevmode
file, and generate the vector plots to those levels only.

\end{description}

Written By : Arulalan.T

Date : 11.09.2011
Updated: 11.12.2011

\end{fulllineitems}

\index{genSeasonMeanDirs() (in module generate\_winds\_plots)}

\begin{fulllineitems}
\phantomsection\label{diagnosis:generate_winds_plots.genSeasonMeanDirs}\pysiglinewithargsret{\code{generate\_winds\_plots.}\bfcode{genSeasonMeanDirs}}{\emph{modelName}, \emph{availableMonths}}{}~\begin{description}
\item[{{\hyperref[diagnosis:generate_winds_plots.genSeasonMeanDirs]{\code{genSeasonMeanDirs()}}} (\autopageref*{diagnosis:generate_winds_plots.genSeasonMeanDirs}): It should generate the directory hierarichy}] \leavevmode
structure of season mean in the plotsgraphspath. And calls the
function genVectorPlots to make vector plots and save it inside the
appropirate directory, by reading the xml file of the appropirate
process season mean files path.

\item[{Inputs}] \leavevmode{[}modelName is the one of the directories name.{]}
availableMonths is the dictionary which is generated by fully
available months from the timeAxis.

\item[{..note:: It should takes the levels which is set in the global config}] \leavevmode
file, and generate the vector plots to those levels only.

\end{description}

Written By : Arulalan.T

Date : 11.09.2011
Updated: 11.12.2011

\end{fulllineitems}

\index{genVectorPlots() (in module generate\_winds\_plots)}

\begin{fulllineitems}
\phantomsection\label{diagnosis:generate_winds_plots.genVectorPlots}\pysiglinewithargsret{\code{generate\_winds\_plots.}\bfcode{genVectorPlots}}{\emph{uvar}, \emph{vvar}, \emph{upath=None}, \emph{vpath=None}, \emph{xmlpath=None}, \emph{outpath=None}, \emph{month=None}, \emph{date=None}, \emph{level=None}, \emph{region=None}, \emph{reference=20.0}, \emph{scale=1}, \emph{interval=4}, \emph{svg=0}, \emph{png=1}, \emph{latlabel='lat5'}, \emph{lonlabel='lon5'}, \emph{style='portrait'}}{}~\begin{description}
\item[{{\hyperref[diagnosis:generate_winds_plots.genVectorPlots]{\code{genVectorPlots()}}} (\autopageref*{diagnosis:generate_winds_plots.genVectorPlots}): It should generate the vector plots in vcs}] \leavevmode
background and save it as png(by default) inside the outpath,
with some default vector properties like reference, scale and
interval.

\item[{Inputs}] \leavevmode{[}uvar is the `u' variable name{]}
vvar is the `v' variable name
upath is the `u' nc file absolute path.
vpath is the `v' nc file absolute path.
xmlpath is the xml file absolute path which must contains the
u and v vars.
outpath is the absolute path, where the generated plots should be
stored. By default, it is None. It means, it should save in the
current working directory itself.
level is either single level, or list of levels or `all'.
`all' means, it takes all the availableLevels from the variables.
level value must be int, float only. Not be string, other than
`all' keyword.
region is the region variable which should cut particular region
shape from the global data. By default it is None, i.e. takes
global data region itself.

reference is the reference points to be plotted in the vcs
vector plot. It must be float only.
scale is the length of the arrow markers in the vcs vector plot.
interval is the integer value, to split the U and V datasets,
to make clear view of the vector plot. By default it takes 4.

svg is the flag. If flag is set, then the vector plot should be
saved as svg formate. By default it is 0.

png is the flag. If flag is set, then the vector plot should be
saved as png formate. By default it is 1.

\item[{Condition}] \leavevmode{[}If we passed xmlpath, then we no need to pass upath and vpath{]}
args. uvar and vvar must be available in the passed filepath.

\end{description}

Written By : Arulalan.T

Date : 11.09.2011
Updated: 11.12.2011

\end{fulllineitems}

\index{getProcessPath() (in module generate\_winds\_plots)}

\begin{fulllineitems}
\phantomsection\label{diagnosis:generate_winds_plots.getProcessPath}\pysiglinewithargsret{\code{generate\_winds\_plots.}\bfcode{getProcessPath}}{\emph{modelname}, \emph{processtype}, \emph{year}, \emph{monthseason}, \emph{hour=None}}{}~\begin{description}
\item[{{\hyperref[diagnosis:generate_winds_plots.getProcessPath]{\code{getProcessPath()}}} (\autopageref*{diagnosis:generate_winds_plots.getProcessPath}): By passing fewer args and get the correct and}] \leavevmode
absolute path of the process files, which generated by automated or
manual for the purpose of this diagnosis.

\item[{Inputs}] \leavevmode{[}modelname is the part of the directory structure.{]}
processtype is any one of the processes.for eg : `Mean Analysis',
or `Mean Fcst' or `Anomaly' or `FcstSysErr', etc.,
year is year in string type.
monthseason is either month name or season name. It should find
out either it is month or season and make the correct path.
hour is the hour string which is the part of the directory
structure only for `FcstSysErr' and `Mean Fcst' process type.

\item[{Outputs}] \leavevmode{[}Return the absolute path of the process files, only if that{]}
directory is exists. Other wise it raise error.

\item[{To Do}] \leavevmode{[}Need to decide about, either it should raise error, or it should{]}
return None, if wrong args passed or process directory doesnot
exists.

\end{description}

Written By : Arulalan.T

Date : 11.09.2011
Updated: 11.12.2011

\end{fulllineitems}



\subsection{Iso Plots}
\label{diagnosis:iso-plots}
Generate the iso plots for the iso variables which are all set in the `vars.txt' file.

The below script \emph{generate\_iso\_plots} should generates the following kind of plots.
\begin{itemize}
\item {} 
iso line

\item {} 
iso fill

\item {} 
iso fill line

\end{itemize}

Finally save the generated iso plots as either png or jpg or svg in the suitable directory for month and season wise.
\phantomsection\label{diagnosis:module-generate_iso_plots}\index{generate\_iso\_plots (module)}\index{genIsoFillLinePlots() (in module generate\_iso\_plots)}

\begin{fulllineitems}
\phantomsection\label{diagnosis:generate_iso_plots.genIsoFillLinePlots}\pysiglinewithargsret{\code{generate\_iso\_plots.}\bfcode{genIsoFillLinePlots}}{\emph{var}, \emph{key}, \emph{isoLevels}, \emph{xmlpath=None}, \emph{path=None}, \emph{outpath=None}, \emph{month=None}, \emph{date=None}, \emph{level='all'}, \emph{region=None}, \emph{svg=0}, \emph{png=1}}{}~\begin{description}
\item[{{\hyperref[diagnosis:generate_iso_plots.genIsoFillLinePlots]{\code{genIsoFillLinePlots()}}} (\autopageref*{diagnosis:generate_iso_plots.genIsoFillLinePlots}): It should generate the isoFillLine plots in}] \leavevmode
vcs background and save it as png(by default) inside the outpath,
with isoLevels (passed by user) and isoColors (default).

\item[{Inputs}] \leavevmode{[}var is the variable name.{]}
key to identify, it is which variable to make plot name.
isoLevels is the levels to plot isoFillLine and set the legend
levels in vcs.
xmlpath is the xml file absolute path.
path is the nc file absolute path.
pass any one (path or xmlpath)

outpath is the absolute path, where the generated plots should be
stored. By default, it is None. It means, it should save in the
current working directory itself.
level is either single level, or list of levels or `all'.
`all' means, it takes all the availableLevels from the variables.
level value must be int, float only. Not be string, other than
`all' keyword.
region is the region variable which should cut particular region
shape from the global data. By default it is None, i.e. takes
global data region itself.

svg is the flag. If flag is set, then the vector plot should be
saved as svg formate. By default it is 0.

png is the flag. If flag is set, then the vector plot should be
saved as png formate. By default it is 1.

\end{description}

Written By : Arulalan.T

Date : 21.09.2011
Updated: 12.12.2011

\end{fulllineitems}

\index{genIsoLinePlots() (in module generate\_iso\_plots)}

\begin{fulllineitems}
\phantomsection\label{diagnosis:generate_iso_plots.genIsoLinePlots}\pysiglinewithargsret{\code{generate\_iso\_plots.}\bfcode{genIsoLinePlots}}{\emph{var}, \emph{key}, \emph{xmlpath=None}, \emph{path=None}, \emph{outpath=None}, \emph{month=None}, \emph{date=None}, \emph{level='all'}, \emph{region=None}, \emph{svg=0}, \emph{png=1}}{}~\begin{description}
\item[{{\hyperref[diagnosis:generate_iso_plots.genIsoLinePlots]{\code{genIsoLinePlots()}}} (\autopageref*{diagnosis:generate_iso_plots.genIsoLinePlots}): It should generate the isoLine plots in}] \leavevmode
vcs background and save it as png(by default) inside the outpath,
with isoLevels (find out by this method) and isoColors (default).

\item[{Inputs}] \leavevmode{[}var is the variable name.{]}\begin{quote}

key to identify, it is which variable to make plot name.
xmlpath is the xml file absolute path.
path is the nc file absolute path.
pass any one (path or xmlpath)

outpath is the absolute path, where the generated plots should be
stored. By default, it is None. It means, it should save in the
current working directory itself.
level is either single level, or list of levels or `all'.
`all' means, it takes all the availableLevels from the variables.
level value must be int, float only. Not be string, other than
`all' keyword.
region is the region variable which should cut particular region
shape from the global data. By default it is None, i.e. takes
global data region itself.

svg is the flag. If flag is set, then the vector plot should be
saved as svg formate. By default it is 0.

png is the flag. If flag is set, then the vector plot should be
saved as png formate. By default it is 1.
\end{quote}
\begin{description}
\item[{..note:: This function should find out the isoLevels to set in the}] \leavevmode
isoline plot and its legend in vcs. IsoLevels is the range of min and
max of all the levels data min and max.

\end{description}

\end{description}

Written By : Arulalan.T

Date : 21.09.2011
Updated: 12.12.2011

\end{fulllineitems}

\index{genSeasonFcstSysErrDirs() (in module generate\_iso\_plots)}

\begin{fulllineitems}
\phantomsection\label{diagnosis:generate_iso_plots.genSeasonFcstSysErrDirs}\pysiglinewithargsret{\code{generate\_iso\_plots.}\bfcode{genSeasonFcstSysErrDirs}}{\emph{modelName}, \emph{modelHour}, \emph{availableMonths}, \emph{plotLevel}}{}~\begin{description}
\item[{{\hyperref[diagnosis:generate_iso_plots.genSeasonFcstSysErrDirs]{\code{genSeasonFcstSysErrDirs()}}} (\autopageref*{diagnosis:generate_iso_plots.genSeasonFcstSysErrDirs}): It should generate the directory hierarichy}] \leavevmode
structure of season fcstsyserr in the plotsgraphspath. And calls the
function genIsoFillLinePlots to make `isofillline' plots and save it
inside the appropirate directory, by reading the xml file of the
appropirate process season fcstsyserr files path.

It should plot for all the vars in the `isovars' which has set in the
global `vars.txt' file.

To plot isoFillLinePlot, this function should find out the isoLevels
for all the variables of all the levels and all the hours.
\begin{description}
\item[{isoLevels}] \leavevmode{[}It is a range of levels which is from the min (of data of{]}
all the hours and levels), to the max (of data of all the hours and
levels) to set the levels in the vcs plot and legend.

\end{description}

\item[{Inputs}] \leavevmode{[}modelName is the one of the directories name.{]}
modelHour is the one of the directories name.
availableMonths is the dictionary which is generated by fully
available months from the timeAxis.

\item[{..note:: It should takes the levels which is set in the global config}] \leavevmode
file, and generate the `IsoFillLine' plots to those levels only.

\end{description}

Written By : Arulalan.T

Date : 20.09.2011
Updated: 12.12.2011

\end{fulllineitems}

\index{genSeasonMeanDirs() (in module generate\_iso\_plots)}

\begin{fulllineitems}
\phantomsection\label{diagnosis:generate_iso_plots.genSeasonMeanDirs}\pysiglinewithargsret{\code{generate\_iso\_plots.}\bfcode{genSeasonMeanDirs}}{\emph{modelName}, \emph{availableMonths}, \emph{plotLevel}}{}~\begin{description}
\item[{{\hyperref[diagnosis:generate_iso_plots.genSeasonMeanDirs]{\code{genSeasonMeanDirs()}}} (\autopageref*{diagnosis:generate_iso_plots.genSeasonMeanDirs}): It should generate the directory hierarichy}] \leavevmode
structure of season mean in the plotsgraphspath. And calls the
function genIsoLinePlots to make `isoline' plots and save it inside the
appropirate directory, by reading the xml file of the appropirate
process season mean files path.

\item[{Inputs}] \leavevmode{[}modelName is the one of the directories name.{]}
availableMonths is the dictionary which is generated by fully
available months from the timeAxis.

\item[{..note:: It should takes the levels which is set in the global config}] \leavevmode
file and generate the `isoline' plots to those levels only.

\end{description}

Written By : Arulalan.T

Date : 20.09.2011
Updated: 12.12.2011

\end{fulllineitems}



\subsection{Statistical Score Bar Plots}
\label{diagnosis:statistical-score-bar-plots}
The script \emph{generate\_statistical\_score\_bars.py} should generate the bar plots w.r.t the statistical scores for different region \& fcst hours.
Finally save the generated bar plots as either png or jpg or svg in the suitable directory for season wise.
\phantomsection\label{diagnosis:module-generate_statistical_score_bars}\index{generate\_statistical\_score\_bars (module)}\phantomsection\label{diagnosis:module-generate_statistical_score_bars.py}\index{generate\_statistical\_score\_bars.py (module)}
Date : 04.08.2011

Updated on : 28.09.2011
\index{genBarDiagrams() (in module generate\_statistical\_score\_bars)}

\begin{fulllineitems}
\phantomsection\label{diagnosis:generate_statistical_score_bars.genBarDiagrams}\pysiglinewithargsret{\code{generate\_statistical\_score\_bars.}\bfcode{genBarDiagrams}}{\emph{var}, \emph{path}, \emph{hours}, \emph{outpath=None}, \emph{bargap=0.28}, \emph{barwidth=0.8}, \emph{yticdiff=0.25}}{}~\begin{description}
\item[{\code{genBarDiagrams()}: It should generate the least directory hierarichy}] \leavevmode
structure of season statiscore in the plotsgraphspath by score name.
It will plots score values in xmgrace as bar diagram and save it
inside the appropirate directory, by reading the nc file of the
appropirate process season Region statiscore files path.

It should plot for all the vars of that statiscore nc files.

\item[{Inputs}] \leavevmode{[}var is the variable name. If var is `all' means, then it should{]}
plot the bar diagram for all the available variables in the passed
path nc or xml file.

path is an absolute nc or xml file path.

outpath is the path to store the images. If it is None means, it
should create the least (plotname)directory in the current
directory path itself and save it.

bargap is the value of the gap ratio in between each bars of each
threshold in xaxis of score bar diagram.

barwidth is the width of the each bar in xaxis of the score
bar diagram.

yticdiff is the difference of the tic levels in y axis of the
bar diagram.

\end{description}

Written By : Dileep Kumar.R, Arulalan.T

Updated on : 28.09.2011

\end{fulllineitems}

\index{genSeasonStatiScoreDirs() (in module generate\_statistical\_score\_bars)}

\begin{fulllineitems}
\phantomsection\label{diagnosis:generate_statistical_score_bars.genSeasonStatiScoreDirs}\pysiglinewithargsret{\code{generate\_statistical\_score\_bars.}\bfcode{genSeasonStatiScoreDirs}}{\emph{modelName}, \emph{modelHour}, \emph{availableMonths}}{}~\begin{description}
\item[{\code{genSeasonStatiScoreDirs()}: It should generate the directory hierarichy}] \leavevmode
structure of season statiscore in the plotsgraphspath. And calls the
function genIsoFillPlots to make `isofill' plots and save it
inside the appropirate directory, by reading the nc file of the
appropirate process season hour statiscore files path.

It should plot for all the vars of that statiscore spatial distributed
nc files.

\item[{Inputs}] \leavevmode{[}modelName is the one of the directories name.{]}
modelHour is the one of the directories name.
availableMonths is the dictionary which is generated by fully
available months from the timeAxis.

\end{description}

Written By : Arulalan.T

Date : 27.09.2011

\end{fulllineitems}



\subsection{Statistical Score Spatial Distribution Plots}
\label{diagnosis:statistical-score-spatial-distribution-plots}
The script \emph{generate\_stati\_score\_spatial\_distribution\_plots.py} should generate the iso fill plots w.r.t the statistical scores for different fcst hours.
Finally save the generated iso fill plots as either png or jpg or svg in the suitable directory for season wise.
\phantomsection\label{diagnosis:module-generate_stati_score_spatial_distribution_plots}\index{generate\_stati\_score\_spatial\_distribution\_plots (module)}\index{genIsoFillPlots() (in module generate\_stati\_score\_spatial\_distribution\_plots)}

\begin{fulllineitems}
\phantomsection\label{diagnosis:generate_stati_score_spatial_distribution_plots.genIsoFillPlots}\pysiglinewithargsret{\code{generate\_stati\_score\_spatial\_distribution\_plots.}\bfcode{genIsoFillPlots}}{\emph{var}, \emph{path}, \emph{outpath=None}, \emph{region=None}, \emph{svg=0}, \emph{png=1}}{}~\begin{description}
\item[{{\hyperref[diagnosis:generate_stati_score_spatial_distribution_plots.genIsoFillPlots]{\code{genIsoFillPlots()}}} (\autopageref*{diagnosis:generate_stati_score_spatial_distribution_plots.genIsoFillPlots}): It should generate the directory least hierarichy}] \leavevmode
structure of season statiscore in the plotsgraphspath,by the plotname.

It should plot for all the vars of that statiscore spatial distributed
nc files.

\item[{Inputs}] \leavevmode{[}var is the variable name. If var is `all' means, then it should{]}\begin{quote}

plot the isofill for all the available variables in the passed
path nc or xml file.

path is an absolute nc or xml file path.

outpath is the path to store the images. If it is None means, it
should create the least (plotname)directory in the current
directory path itself and save it.

region to extract the region from the var data.

if svg is 1, then plot should be saved as svg.
if png is 1, then plot should be saved as png.
\end{quote}
\begin{description}
\item[{..note:: isoLevels and isoColors are set inbuilt (some default) range of}] \leavevmode
levels and colors with respect to the variable name of statistical
scores.

\end{description}

\end{description}

Written By : Dileep Kumar.R, Arulalan.T

Date : 26.09.2011

\end{fulllineitems}

\index{genSeasonStatiScoreDirs() (in module generate\_stati\_score\_spatial\_distribution\_plots)}

\begin{fulllineitems}
\phantomsection\label{diagnosis:generate_stati_score_spatial_distribution_plots.genSeasonStatiScoreDirs}\pysiglinewithargsret{\code{generate\_stati\_score\_spatial\_distribution\_plots.}\bfcode{genSeasonStatiScoreDirs}}{\emph{modelName}, \emph{modelHour}, \emph{availableMonths}}{}~\begin{description}
\item[{{\hyperref[diagnosis:generate_stati_score_spatial_distribution_plots.genSeasonStatiScoreDirs]{\code{genSeasonStatiScoreDirs()}}} (\autopageref*{diagnosis:generate_stati_score_spatial_distribution_plots.genSeasonStatiScoreDirs}): It should generate the directory hierarichy}] \leavevmode
structure of season statiscore in the plotsgraphspath. And calls the
function genIsoFillPlots to make `isofill' plots and save it
inside the appropirate directory, by reading the nc file of the
appropirate process season hour statiscore files path.

It should plot for all the vars of that statiscore spatial distributed
nc files.

\item[{Inputs}] \leavevmode{[}modelName is the one of the directories name.{]}
modelHour is the one of the directories name.
availableMonths is the dictionary which is generated by fully
available months from the timeAxis.

\end{description}

Written By : Arulalan.T

Date : 26.09.2011

\end{fulllineitems}



\section{Statistical Scores}
\label{diagnosis:statistical-scores}

\subsection{Contigency Table \& Related Statistical Scores}
\label{diagnosis:contigency-table-related-statistical-scores}
The module \emph{ctgfunction.py} should helps to calculate the contigency table and its related statistical scores.

For eg : Threat Score, Bias Score, Proabability of detection and more.
\phantomsection\label{diagnosis:module-ctgfunction}\index{ctgfunction (module)}\index{accuracy() (in module ctgfunction)}

\begin{fulllineitems}
\phantomsection\label{diagnosis:ctgfunction.accuracy}\pysiglinewithargsret{\code{ctgfunction.}\bfcode{accuracy}}{\emph{obs=None}, \emph{fcst=None}, \emph{th=None}, \emph{**ctg}}{}~\begin{description}
\item[{{\hyperref[diagnosis:ctgfunction.accuracy]{\code{accuracy()}}} (\autopageref*{diagnosis:ctgfunction.accuracy}):Hit Rate ,the most direct and intuitive measure of the}] \leavevmode\begin{quote}

accuracy of categorical forecasts is hit rate. The average correspond-
ence between individual nforecasts and the events they predict. Scalar
measures  of accuracy are meant to summarize,in a single number, the
overall quality of a set of forecasts. Can be mislead, since it is
heavily influenced by the most common category, usually ``no event''
in the case of rare weather.
\begin{quote}
\begin{description}
\item[{Accuracy= (a+d)/(a+b+c+d); `a' -hits, `b'-false alarm,}] \leavevmode
`c'-misses, \& `d'- correct negatives

\end{description}
\end{quote}
\end{quote}
\begin{description}
\item[{Inputs: obs- the observed values has to be a numpy array(or whatever}] \leavevmode\begin{quote}

you decide)
\end{quote}

fcst - the forecast values
th  - the threshold value for which the contingency table needs
\begin{quote}

to be created (floating point value please!!)
\end{quote}

By default obs, fcst, th are None. Instead of passing obs, fcst,
and th values, you can pass `ctg\_table' kwarg as 2x2 matrix value.

\item[{Outputs:}] \leavevmode
Range: 0 to 1

Perfect Score: 1

\item[{Reference: ``Statistical Methods in the Atmospheric Sciences'',}] \leavevmode
Daniel S Wilks, ACADEMIC PRESS(Page No:236-240)

\end{description}

Links : \href{http://www.cawcr.gov.au/projects/verification/}{http://www.cawcr.gov.au/projects/verification/}
\begin{description}
\item[{Written by: Dileepkumar R,}] \leavevmode
JRF, IIT Delhi

\end{description}

Date: 24/02/2011

\end{description}

\end{fulllineitems}

\index{bias\_score() (in module ctgfunction)}

\begin{fulllineitems}
\phantomsection\label{diagnosis:ctgfunction.bias_score}\pysiglinewithargsret{\code{ctgfunction.}\bfcode{bias\_score}}{\emph{obs=None}, \emph{fcst=None}, \emph{th=None}, \emph{**ctg}}{}~\begin{description}
\item[{{\hyperref[diagnosis:ctgfunction.bias_score]{\code{bias\_score()}}} (\autopageref*{diagnosis:ctgfunction.bias_score}): Bias score(frequency bias)-Measures the correspondence}] \leavevmode
between the average forecast and the average observed value of the
predictand. This is different from accuracy, which measures the
average correspondence between individual pairs of forecasts and
observations.
\begin{quote}

Bias= (a+b)/(a+c); `a' -hits, `b'-false alarm, \& `c'-misses
\end{quote}

\item[{Inputs: obs- the observed values has to be a numpy array(or whatever}] \leavevmode\begin{quote}

you decide)
\end{quote}

fcst - the forecast values
th  - the threshold value for which the contingency table needs
\begin{quote}

to be created (floating point value please!!)
\end{quote}

By default obs, fcst, th are None. Instead of passing obs, fcst,
and th values, you can pass `ctg\_table' kwarg as 2x2 matrix value.

\item[{Outputs:}] \leavevmode
Range: 0 to infinity

Perfect score: 1

\item[{Reference: ``Statistical Methods in the Atmospheric Sciences'',}] \leavevmode
Daniel S Wilks, ACADEMIC PRESS(Page No: 241)

\end{description}

Links : \href{http://www.cawcr.gov.au/projects/verification/}{http://www.cawcr.gov.au/projects/verification/}
\begin{description}
\item[{Written by: Dileepkumar R,}] \leavevmode
JRF, IIT Delhi

\end{description}

Date: 24/02/2011

\end{fulllineitems}

\index{contingency\_table\_2x2() (in module ctgfunction)}

\begin{fulllineitems}
\phantomsection\label{diagnosis:ctgfunction.contingency_table_2x2}\pysiglinewithargsret{\code{ctgfunction.}\bfcode{contingency\_table\_2x2}}{\emph{obs}, \emph{fcst}, \emph{th}}{}~\begin{description}
\item[{{\hyperref[diagnosis:ctgfunction.contingency_table_2x2]{\code{contingency\_table\_2x2()}}} (\autopageref*{diagnosis:ctgfunction.contingency_table_2x2}):Creates the 2x2 contigency table useful for}] \leavevmode
forecast verification. From 2x2 condigency table we can find thse
statistical scores such as Hit Rate(HR), Bias(BS),
Threat Score(TS), Odds Ratio(ODR)...etc

\item[{Inputs: obs- the observed values has to be a numpy array(or whatever}] \leavevmode\begin{quote}

you decide)
\end{quote}

fcst - the forecast values
th  - the threshold value for which the contingency table needs
\begin{quote}

to be created (floating point value please!!)
\end{quote}

\item[{Outputs: A list of values {[}a, b, c, d{]}}] \leavevmode\begin{description}
\item[{where a = No of values such that both observed and}] \leavevmode
predicted \textgreater{} threshold

\item[{b = No of values such that observed is \textless{} threshold and}] \leavevmode
predicted \textgreater{} threshold

\item[{c = No of values such that observed is \textgreater{} threshold and}] \leavevmode
predicted \textless{} threshold

\item[{d = No of values such that both observed and}] \leavevmode
predicted \textless{} threshold

\end{description}

\end{description}

Usage:
\begin{description}
\item[{example:}] \leavevmode
From the contingency table the following statistics can be
calculated.
\textgreater{}\textgreater{}\textgreater{} HR = (a+d)/(a+b+c+d)
\textgreater{}\textgreater{}\textgreater{} ETS = a/(a+b+c)
\textgreater{}\textgreater{}\textgreater{} BS = (a+b)/(a+c)
\textgreater{}\textgreater{}\textgreater{} ODR = (a*d)/(b*c)

\item[{Reference: ``Statistical Methods in the Atmospheric Sciences'',}] \leavevmode
Daniel S Wilks, ACADEMIC PRESS

\end{description}

Links: \href{http://www.cawcr.gov.au/projects/verification/\#Atger\_2001}{http://www.cawcr.gov.au/projects/verification/\#Atger\_2001}
\begin{description}
\item[{Written by: Dileepkumar R,}] \leavevmode
JRF, IIT Delhi

\end{description}

Date: 24/02/2011

\end{fulllineitems}

\index{eds() (in module ctgfunction)}

\begin{fulllineitems}
\phantomsection\label{diagnosis:ctgfunction.eds}\pysiglinewithargsret{\code{ctgfunction.}\bfcode{eds}}{\emph{obs=None}, \emph{fcst=None}, \emph{th=None}, \emph{**ctg}}{}~\begin{description}
\item[{:func:'eds':Extreme dependency score, converges to 2n-1 as event}] \leavevmode
frequency approaches 0, where n is a parameter describing how
fast the hit rate converges to zero for rarer events. EDS is
independent of bias, so should be presented together with the
frequency bias.
\begin{quote}

EDS=\{2Log{[}(a+c)/(a+b+c+d){]}/Log(a/(a+b+c+d))\}-1; `a' -hits,
`b'-false alarm, `c'-misses, \& `d'- correct negatives
\end{quote}

\item[{Inputs: obs- the observed values has to be a numpy array(or whatever}] \leavevmode\begin{quote}

you decide)
\end{quote}

fcst - the forecast values
th  - the threshold value for which the contingency table needs
\begin{quote}

to be created (floating point value please!!)
\end{quote}

By default obs, fcst, th are None. Instead of passing obs, fcst,
and th values, you can pass `ctg\_table' kwarg as 2x2 matrix value.

\end{description}

Outputs:
\begin{quote}

Range: -1 to 1, 0 indicate no skill.

Perfect Score: 1
\end{quote}
\begin{description}
\item[{Reference: {[}1{]}Stephenson D.B., B. Casati, C.A.T. Ferro and}] \leavevmode
C.A. Wilson, 2008: The extreme dependency score:
a non-vanishing measure for forecasts of rare events.
\begin{quote}

Meteorol. Appl., 15, 41-50.
\end{quote}

\end{description}

Link:    \href{http://www.cawcr.gov.au/projects/verification}{http://www.cawcr.gov.au/projects/verification}.
\begin{description}
\item[{Written by: Dileepkumar R,}] \leavevmode
JRF, IIT Delhi

\end{description}

Date: 24/02/2011

\end{fulllineitems}

\index{ets() (in module ctgfunction)}

\begin{fulllineitems}
\phantomsection\label{diagnosis:ctgfunction.ets}\pysiglinewithargsret{\code{ctgfunction.}\bfcode{ets}}{\emph{obs=None}, \emph{fcst=None}, \emph{th=None}, \emph{**ctg}}{}~\begin{description}
\item[{{\hyperref[diagnosis:ctgfunction.ets]{\code{ets()}}} (\autopageref*{diagnosis:ctgfunction.ets}):Equitable threat score (Gilbert skill score), the number of}] \leavevmode
forecasts of the event correct by chance, `a\_random',is determined
by assuming that the forecasts are totally independent of the obs-
ervations, and forecast will match the observation only by chance.
This is one form of an unskilled forecast, which can be generated
by just guessing what will happen.
\begin{quote}

ETS=(a-a\_random)/(a+c+b-a\_random)
a\_random={[}(a+c)(a+b){]}/(a+b+c+d); `a' -hits, `b'-false alarm,
\begin{quote}

`c'-misses, \& `d'- correct negatives
\end{quote}
\end{quote}

\item[{Inputs: obs- the observed values has to be a numpy array(or whatever}] \leavevmode\begin{quote}

you decide)
\end{quote}

fcst - the forecast values
th  - the threshold value for which the contingency table needs
\begin{quote}

to be created (floating point value please!!)
\end{quote}

By default obs, fcst, th are None. Instead of passing obs, fcst,
and th values, you can pass `ctg\_table' kwarg as 2x2 matrix value.

\end{description}

Outputs:
\begin{quote}

Range: -1/3 to 1, 0 indicate no skill.

Perfect Score: 1
\end{quote}

Links: \href{http://www.cawcr.gov.au/projects/verification/}{http://www.cawcr.gov.au/projects/verification/}
\begin{description}
\item[{Written by: Dileepkumar R,}] \leavevmode
JRF, IIT Delhi

\end{description}

Date: 24/02/2011

\end{fulllineitems}

\index{far() (in module ctgfunction)}

\begin{fulllineitems}
\phantomsection\label{diagnosis:ctgfunction.far}\pysiglinewithargsret{\code{ctgfunction.}\bfcode{far}}{\emph{obs=None}, \emph{fcst=None}, \emph{th=None}, \emph{**ctg}}{}~\begin{description}
\item[{{\hyperref[diagnosis:ctgfunction.far]{\code{far()}}} (\autopageref*{diagnosis:ctgfunction.far}):False alarm ratio(FAR)-Proportion of forecast events that fail}] \leavevmode
to materialize.
\begin{quote}

FAR= b/(a+b); `a' -hits, \& `b'-false alarm
\end{quote}

\item[{Inputs: obs- the observed values has to be a numpy array(or whatever}] \leavevmode\begin{quote}

you decide)
\end{quote}

fcst - the forecast values
th  - the threshold value for which the contingency table needs
\begin{quote}

to be created (floating point value please!!)
\end{quote}

By default obs, fcst, th are None. Instead of passing obs, fcst,
and th values, you can pass `ctg\_table' kwarg as 2x2 matrix value.

\end{description}

Outputs:
\begin{quote}

Range: 0 to 1

Perfect Score: 0
\end{quote}
\begin{description}
\item[{Reference: ``Statistical Methods in the Atmospheric Sciences'',}] \leavevmode
Daniel S Wilks, ACADEMIC PRESS(Page No: 240-241)

\end{description}

Links: \href{http://www.cawcr.gov.au/projects/verification/}{http://www.cawcr.gov.au/projects/verification/}
\begin{description}
\item[{Written by: Dileepkumar R,}] \leavevmode
JRF, IIT Delhi

\end{description}

Date: 24/02/2011

\end{fulllineitems}

\index{hss() (in module ctgfunction)}

\begin{fulllineitems}
\phantomsection\label{diagnosis:ctgfunction.hss}\pysiglinewithargsret{\code{ctgfunction.}\bfcode{hss}}{\emph{obs=None}, \emph{fcst=None}, \emph{th=None}, \emph{**ctg}}{}~\begin{description}
\item[{{\hyperref[diagnosis:ctgfunction.hss]{\code{hss()}}} (\autopageref*{diagnosis:ctgfunction.hss}):Heidke skill score (Cohen's k), the reference accuracy}] \leavevmode
measure in the Heidke score is the hit rate that would be
achieved by random forecasts, subject to the constraint that
the marginal distributions of forecasts and observations
characterizing the contingency table for the random forecasts,
P(Yi) and p(oj), are the same as the marginal distributions in
the actual verification data set.
\begin{quote}

HSS= 2.( a d - bc)/{[}(a + c)(c + d) + (a + b)(b + d){]};
\end{quote}

`a' -hits, `b'-false alarm, `c'-misses, \& `d'- correct  negatives

\item[{Inputs: obs- the observed values has to be a numpy array(or whatever}] \leavevmode\begin{quote}

you decide)
\end{quote}

fcst - the forecast values
th  - the threshold value for which the contingency table needs
\begin{quote}

to be created (floating point value please!!)
\end{quote}

By default obs, fcst, th are None. Instead of passing obs, fcst,
and th values, you can pass `ctg\_table' kwarg as 2x2 matrix value.

\end{description}

Outputs:
\begin{quote}

Range: -infinity to 1, 0 indicate no skill.

Perfect Score: 1
\end{quote}
\begin{description}
\item[{Reference: ``Statistical Methods in the Atmospheric Sciences'',}] \leavevmode
Daniel S Wilks, ACADEMIC PRESS(Page No: 248-249)

\end{description}

Links: \href{http://www.cawcr.gov.au/projects/verification/}{http://www.cawcr.gov.au/projects/verification/}
\begin{description}
\item[{Written by: Dileepkumar R,}] \leavevmode
JRF, IIT Delhi

\end{description}

Date: 24/02/2011

\end{fulllineitems}

\index{kss() (in module ctgfunction)}

\begin{fulllineitems}
\phantomsection\label{diagnosis:ctgfunction.kss}\pysiglinewithargsret{\code{ctgfunction.}\bfcode{kss}}{\emph{obs=None}, \emph{fcst=None}, \emph{th=None}, \emph{**ctg}}{}~\begin{description}
\item[{{\hyperref[diagnosis:ctgfunction.kss]{\code{kss()}}} (\autopageref*{diagnosis:ctgfunction.kss}):Hanssen and Kuipers discriminant(Kuipers Skill Score), the}] \leavevmode
contribution made to the Kuipers score by a correct ``no'' or
``yes'' forecast increases as the event is more or less likely,
respectively.A drawback of this score is that it tends to converge
to the POD for rare events, because the value of ``d'' becomes very
large.
\begin{quote}

HK= (ad-bc)/{[}(a + c)(b + d){]}; `a' -hits, `b'-false alarm,
`c'-misses, \& `d'- correct negatives
\end{quote}

\item[{Inputs: obs- the observed values has to be a numpy array(or whatever}] \leavevmode\begin{quote}

you decide)
\end{quote}

fcst - the forecast values
th  - the threshold value for which the contingency table needs
\begin{quote}

to be created (floating point value please!!)
\end{quote}

By default obs, fcst, th are None. Instead of passing obs, fcst,
and th values, you can pass `ctg\_table' kwarg as 2x2 matrix value.

\end{description}

Outputs:
\begin{quote}

Range:-1 to 1
Perfect Score: 1
\end{quote}
\begin{description}
\item[{Reference: ``Statistical Methods in the Atmospheric Sciences'',}] \leavevmode
Daniel S Wilks, ACADEMIC PRESS(Page No: 249-250)

\end{description}

Links: \href{http://www.cawcr.gov.au/projects/verification/}{http://www.cawcr.gov.au/projects/verification/}
\begin{description}
\item[{Written by: Dileepkumar R,}] \leavevmode
JRF, IIT Delhi

\end{description}

Date: 24/02/2011

\end{fulllineitems}

\index{logodr() (in module ctgfunction)}

\begin{fulllineitems}
\phantomsection\label{diagnosis:ctgfunction.logodr}\pysiglinewithargsret{\code{ctgfunction.}\bfcode{logodr}}{\emph{obs=None}, \emph{fcst=None}, \emph{th=None}, \emph{**ctg}}{}~\begin{description}
\item[{{\hyperref[diagnosis:ctgfunction.logodr]{\code{logodr()}}} (\autopageref*{diagnosis:ctgfunction.logodr}): Log Odds ratio, LOR is the logaritham of odds ratio.}] \leavevmode
When the sample is small/moderate it is better to use Log
Odds Ratio. It is a good tool for finding associations
between variables.
\begin{quote}
\begin{description}
\item[{LOR=Log(Odds Ratio); Odds Ratio=ad/bc; `a' -hits,}] \leavevmode
`b'-false alarm, `c'-misses, \& `d'- correct negatives

\end{description}
\end{quote}

\item[{Inputs: obs- the observed values has to be a numpy array(or whatever}] \leavevmode\begin{quote}

you decide)
\end{quote}

fcst - the forecast values
th  - the threshold value for which the contingency table needs
\begin{quote}

to be created (floating point value please!!)
\end{quote}

By default obs, fcst, th are None. Instead of passing obs, fcst,
and th values, you can pass `ctg\_table' kwarg as 2x2 matrix value.

\end{description}

Outputs:
\begin{quote}

Range: -infinity to infinity, 0 indicate no skill.

Perfect Score: infinity
\end{quote}
\begin{description}
\item[{Written by: Dileepkumar R,}] \leavevmode
JRF, IIT Delhi

\end{description}

Date: 24/02/2011

\end{fulllineitems}

\index{odr() (in module ctgfunction)}

\begin{fulllineitems}
\phantomsection\label{diagnosis:ctgfunction.odr}\pysiglinewithargsret{\code{ctgfunction.}\bfcode{odr}}{\emph{obs=None}, \emph{fcst=None}, \emph{th=None}, \emph{**ctg}}{}~\begin{description}
\item[{{\hyperref[diagnosis:ctgfunction.odr]{\code{odr()}}} (\autopageref*{diagnosis:ctgfunction.odr}):Odds ratio, the odds ratio is the ratio of the odds of an}] \leavevmode
event occurring in one group to the odds of it occurring in
another group.cThe term is also used to refer to sample-based
estimates of this ratio.Do not use if any of the cells in the
contingency table are equal to 0. The logarithm of the odds
ratio is often used instead of the original value.Used widely
in medicine but not yet in meteorology.
\begin{quote}
\begin{description}
\item[{OD= ad/bc; `a' -hits, `b'-false alarm, `c'-misses, \&}] \leavevmode
`d'- correct negatives

\end{description}
\end{quote}

\item[{Inputs: obs- the observed values has to be a numpy array(or whatever}] \leavevmode\begin{quote}

you decide)
\end{quote}

fcst - the forecast values
th  - the threshold value for which the contingency table needs
\begin{quote}

to be created (floating point value please!!)
\end{quote}

By default obs, fcst, th are None. Instead of passing obs, fcst,
and th values, you can pass `ctg\_table' kwarg as 2x2 matrix value.

\end{description}

Outputs:
\begin{quote}

Range: 0 to infinity, 1 indicate no skill

Perfect Score: infinity
\end{quote}

Links: \href{http://www.cawcr.gov.au/projects/verification/}{http://www.cawcr.gov.au/projects/verification/}
\begin{description}
\item[{Written by: Dileepkumar R,}] \leavevmode
JRF, IIT Delhi

\end{description}

Date: 24/02/2011

\end{fulllineitems}

\index{orss() (in module ctgfunction)}

\begin{fulllineitems}
\phantomsection\label{diagnosis:ctgfunction.orss}\pysiglinewithargsret{\code{ctgfunction.}\bfcode{orss}}{\emph{obs=None}, \emph{fcst=None}, \emph{th=None}, \emph{**ctg}}{}~\begin{description}
\item[{{\hyperref[diagnosis:ctgfunction.orss]{\code{orss()}}} (\autopageref*{diagnosis:ctgfunction.orss}): Odds ratio skill score (Yule's Q), this score was proposed}] \leavevmode
long ago as a `measure of association' by the statistician
G. U. Yule (Yule 1900) and is referred to as Yule's Q. It is
based entirely on the joint conditional probabilities,
and so is not influenced in any way by the marginal totals.
\begin{description}
\item[{ORSS= (ad-bc)/(ad+bc); `a' -hits, `b'-false alarm, `c'-misses,}] \leavevmode
\& `d'- correct negatives

\end{description}

\item[{Inputs: obs- the observed values has to be a numpy array(or whatever}] \leavevmode\begin{quote}

you decide)
\end{quote}

fcst - the forecast values
th  - the threshold value for which the contingency table needs
\begin{quote}

to be created (floating point value please!!)
\end{quote}

By default obs, fcst, th are None. Instead of passing obs, fcst,
and th values, you can pass `ctg\_table' kwarg as 2x2 matrix value.

\end{description}

Outputs:
\begin{quote}

Range: -1 to 1, 0 indicates no skill

Perfect Score: 1
\end{quote}
\begin{description}
\item[{Reference: Stephenson, D.B., 2000: Use of the ``odds ratio'' for diagnosing}] \leavevmode
forecast skill. Wea. Forecasting, 15, 221-232.

\end{description}

Links: \href{http://www.cawcr.gov.au/projects/verification/}{http://www.cawcr.gov.au/projects/verification/}
\begin{description}
\item[{Written by: Dileepkumar R,}] \leavevmode
JRF, IIT Delhi

\end{description}

Date: 24/02/2011

\end{fulllineitems}

\index{pod() (in module ctgfunction)}

\begin{fulllineitems}
\phantomsection\label{diagnosis:ctgfunction.pod}\pysiglinewithargsret{\code{ctgfunction.}\bfcode{pod}}{\emph{obs=None}, \emph{fcst=None}, \emph{th=None}, \emph{**ctg}}{}~\begin{description}
\item[{{\hyperref[diagnosis:ctgfunction.pod]{\code{pod()}}} (\autopageref*{diagnosis:ctgfunction.pod}):Probability of detection(POD), simply the fraction of those}] \leavevmode
occasions when the forecast event occurred on which it was also
forecast.
\begin{quote}

POD= a/(a+c); `a' -hits, \& `c'-misses
\end{quote}

\item[{Inputs: obs- the observed values has to be a numpy array(or whatever}] \leavevmode\begin{quote}

you decide)
\end{quote}

fcst - the forecast values
th  - the threshold value for which the contingency table needs
\begin{quote}

to be created (floating point value please!!)
\end{quote}

By default obs, fcst, th are None. Instead of passing obs, fcst,
and th values, you can pass `ctg\_table' kwarg as 2x2 matrix value.

\item[{Outputs:}] \leavevmode
Range: 0 to 1

Perfect Score: 1

\item[{Reference: ``Statistical Methods in the Atmospheric Sciences'',}] \leavevmode
Daniel S Wilks, ACADEMIC PRESS(Page No: 240)

\end{description}

Links: \href{http://www.cawcr.gov.au/projects/verification/}{http://www.cawcr.gov.au/projects/verification/}
\begin{description}
\item[{Written by: Dileepkumar R,}] \leavevmode
JRF, IIT Delhi

\end{description}

Date: 24/02/2011

\end{fulllineitems}

\index{pofd() (in module ctgfunction)}

\begin{fulllineitems}
\phantomsection\label{diagnosis:ctgfunction.pofd}\pysiglinewithargsret{\code{ctgfunction.}\bfcode{pofd}}{\emph{obs=None}, \emph{fcst=None}, \emph{th=None}, \emph{**ctg}}{}~\begin{description}
\item[{\code{podf()}:Probability of false detection (false alarm rate), measures}] \leavevmode
the fraction of false alarms given the event did not occur.
\begin{quote}

POFD=b/(d+b); `b'-false alarm \& `d'- correct negatives
\end{quote}

\item[{Inputs: obs - the observed values has to be a numpy array(or whatever}] \leavevmode\begin{quote}

you decide)
\end{quote}

fcst - the forecast values
th  - the threshold value for which the contingency table needs
\begin{quote}

to be created (floating point value please!!)
\end{quote}

By default obs, fcst, th are None. Instead of passing obs, fcst,
and th values, you can pass `ctg\_table' kwarg as 2x2 matrix value.

\end{description}

Outputs:
\begin{quote}

Range: 0 to 1

Perfect Score: 0
\end{quote}

Links: \href{http://www.cawcr.gov.au/projects/verification/}{http://www.cawcr.gov.au/projects/verification/}

\end{fulllineitems}

\index{ts() (in module ctgfunction)}

\begin{fulllineitems}
\phantomsection\label{diagnosis:ctgfunction.ts}\pysiglinewithargsret{\code{ctgfunction.}\bfcode{ts}}{\emph{obs=None}, \emph{fcst=None}, \emph{th=None}, \emph{**ctg}}{}~\begin{description}
\item[{:func:'ts':    Threat Score (Critical Success Index), a frequently used}] \leavevmode
alternative to the hit rate, particularly when the event to be
forecast (as the ``yes'' event) occurs substantially less frequently
than the non-occurrence (``no'').
\begin{quote}

TS=a/(a+b+c); `a' -hits, `b'-false alarm, \&'c'-misses
\end{quote}

\item[{Inputs: obs- the observed values has to be a numpy array(or whatever}] \leavevmode\begin{quote}

you decide)
\end{quote}

fcst - the forecast values
th  - the threshold value for which the contingency table needs
\begin{quote}

to be created (floating point value please!!)
\end{quote}

By default obs, fcst, th are None. Instead of passing obs, fcst,
and th values, you can pass `ctg\_table' kwarg as 2x2 matrix value.

\item[{Outputs:}] \leavevmode
Range: 0 to 1

Perfect Score: 1

\item[{Reference: ``Statistical Methods in the Atmospheric Sciences'',}] \leavevmode
Daniel S Wilks, ACADEMIC PRESS(Page No: 240)

\end{description}

Links: \href{http://www.cawcr.gov.au/projects/verification/}{http://www.cawcr.gov.au/projects/verification/}
\begin{description}
\item[{Written by: Dileepkumar R,}] \leavevmode
JRF, IIT Delhi

\end{description}

Date: 24/02/2011

\end{fulllineitems}



\section{More}
\label{diagnosis:more}
More utilities will be added and optimized in near future.


\chapter{Documentation of \textbf{MJO} source code}
\label{mjo:mjo}\label{mjo::doc}\label{mjo:documentation-of-mjo-source-code}

\section{MJO-LEVEL1 Diagnosis Utils}
\label{mjo:mjo-level1-diagnosis-utils}

\subsection{eof}
\label{mjo:module-eof_diag}\label{mjo:eof}\index{eof\_diag (module)}\index{genEofVars() (in module eof\_diag)}

\begin{fulllineitems}
\phantomsection\label{mjo:eof_diag.genEofVars}\pysiglinewithargsret{\code{eof\_diag.}\bfcode{genEofVars}}{\emph{infiles}, \emph{outfile}, \emph{eobjf=True}, \emph{latitude=(-30}, \emph{30}, \emph{`cob')}, \emph{NEOF=4}, \emph{season='all'}, \emph{year=None}, \emph{**kwarg}}{}
\end{fulllineitems}



\subsection{Power Spectrum Utils}
\label{mjo:power-spectrum-utils}\label{mjo:module-psutils}\index{psutils (module)}\index{areaAvg() (in module psutils)}

\begin{fulllineitems}
\phantomsection\label{mjo:psutils.areaAvg}\pysiglinewithargsret{\code{psutils.}\bfcode{areaAvg}}{\emph{varName}, \emph{fpath}, \emph{**kwarg}}{}
Returns the area averaged data by accessing the var data from the fpath
itself by extracting needed portion of data only by the following
key word arguments.

KWargs : (latitude and/or longitude) or (region) and/or level

Written by : Arulalan.T

Date : 08.01.2013

\end{fulllineitems}

\index{powerSpectrum() (in module psutils)}

\begin{fulllineitems}
\phantomsection\label{mjo:psutils.powerSpectrum}\pysiglinewithargsret{\code{psutils.}\bfcode{powerSpectrum}}{\emph{varName}, \emph{fpath}, \emph{sday}, \emph{smon}, \emph{eday}, \emph{emon}, \emph{hr=0}, \emph{nodays=None}, \emph{**kwarg}}{}
varName - variable
fpath or data :
\begin{quote}

This anomaly data should not be spatial one. It should be spatially
averaged, daily data.
It may contains many years data with proper time axis.
\end{quote}
\begin{description}
\item[{nodays}] \leavevmode{[}No of days. though its averaged data, pass the no of days of the{]}
data of each year which has averaged.
For 180 days summer/winter season averaged of multi-year data,
you have to pass nodays=180.

\end{description}

Written By : Arulalan.T

Date : 31.10.2012

\end{fulllineitems}

\index{waveNumber() (in module psutils)}

\begin{fulllineitems}
\phantomsection\label{mjo:psutils.waveNumber}\pysiglinewithargsret{\code{psutils.}\bfcode{waveNumber}}{\emph{varName}, \emph{fpath}, \emph{sday}, \emph{smon}, \emph{eday}, \emph{emon}, \emph{hr=0}, \emph{**kwargs}}{}
varName - variable
fpath or data :
\begin{quote}

This anomaly data should be meridionally averaged one.
It should be daily data.
It may contains many years data with proper time axis.
\end{quote}

KWArgs :-
nodays : No of days. though its averaged data, pass the no of days of the
\begin{quote}

data of each year which has averaged.
For 180 days summer/winter season averaged of multi-year data,
you have to pass nodays=180.
\end{quote}

window : do cosine window if True
\begin{description}
\item[{iglimit}] \leavevmode{[}ignore newlon\_limit days. By default it takes 10 days.{]}
i.e. If extracted season data is less than the (actual seasonal
days - iglimit) it will throw an error.

\end{description}

Date : 28.11.2012, 02.12.2012

\end{fulllineitems}

\index{zonalAvg() (in module psutils)}

\begin{fulllineitems}
\phantomsection\label{mjo:psutils.zonalAvg}\pysiglinewithargsret{\code{psutils.}\bfcode{zonalAvg}}{\emph{varName}, \emph{fpath}, \emph{**kwarg}}{}
Returns the zonal averaged data by accessing the var data from the fpath
itself by extracting needed portion of data only by the following
key word arguments.

KWargs : (latitude and/or longitude) or (region) and/or level

Returns : It vanishes the latitude in the input data and returns it.

Written by : Arulalan.T

Date : 08.01.2013

\end{fulllineitems}



\subsection{Variance Utils}
\label{mjo:variance-utils}\label{mjo:module-variance_utils}\index{variance\_utils (module)}\index{anomaly() (in module variance\_utils)}

\begin{fulllineitems}
\phantomsection\label{mjo:variance_utils.anomaly}\pysiglinewithargsret{\code{variance\_utils.}\bfcode{anomaly}}{\emph{data}, \emph{climatology}, \emph{climyear=1}, \emph{**kwarg}}{}
Calculating anomaly.
Anomaly = model data - climatology

In this function, it should find out either data has leap year day data
or not. Also find out either climatology data has leap year day data or
not.

Depends upon these case, it should remove the leap day data from either
model data or climatology data, when shapes are mis-match (i.e. both leap
and non leap year data has passed as arguments) to compute anomaly.
\begin{description}
\item[{KWargs :}] \leavevmode
When data and climatology shapes are mis-match, then
cregrid : if True, then climatology data will be regridded w.r.t
\begin{quote}

model/obs/data and then anomaly will be calculated.
\end{quote}
\begin{description}
\item[{dregrid}] \leavevmode{[}if True, then model/obs/data will be regridded w.r.t{]}
climatology data and then anomaly will be calculated.

\end{description}

..note:: We can not enable both cregrid and dregrid at the same time.

\end{description}

Written By : Arulalan.T

Date : 11.06.2012
Updated : 24.06.2013

\end{fulllineitems}

\index{calculateSeasonalVariance() (in module variance\_utils)}

\begin{fulllineitems}
\phantomsection\label{mjo:variance_utils.calculateSeasonalVariance}\pysiglinewithargsret{\code{variance\_utils.}\bfcode{calculateSeasonalVariance}}{\emph{varName}, \emph{fpath}, \emph{sday}, \emph{smon}, \emph{eday}, \emph{emon}, \emph{hr=0}, \emph{**kwarg}}{}~\begin{description}
\item[{calculateSeasonalVariance}] \leavevmode{[}calculate Variance for anomaly seasonal data{]}
in temporal way. It should extract the data only for the seasonal
days of the year. Even it has more years, then it should extract
only the seasonal days data of all the years, and then calculate
the variance for the seasonal days of the all the years of the data.

\item[{Inputs:}] \leavevmode
varName : anomaly variable name
fpath : anomaly (nc) file path
sday : starting day of the season {[}of all the years of the data{]}.
smon : starting month of the season {[}of all the years of the data{]}.
eday : ending day of the season {[}of all the years of the data{]}.
emon : ending month of the season {[}of all the years of the data{]}.
hr : hour for both start and end date

\item[{KWargs:}] \leavevmode\begin{description}
\item[{year}] \leavevmode{[}Its optional only. By default it is None. i.e. It will{]}
extract all the available years seasonalData.
If one interger year has passed means, then it will do
extract of that particular year seasonData alone.
If two years has passed in tuple, then it will extract
the range of years seasonData from year{[}0{]} to year{[}1{]}.
eg1 : year=2005 it will extract seasonData of 2005 alone.
eg2 : year=(1971, 2013) it will extract seasonData from
\begin{quote}

1971 to 2013 years.
\end{quote}

\end{description}

\item[{..note::    If end day and end month is lower than the start day and start}] \leavevmode
month, then we need to extract the both current and next year
data. For eg : Winter Season (November to April).
It can not be reversed for this winter season. We need to
extract data from current year november month upto next year
march month.
\begin{quote}

If you pass one year data and passed the above
\end{quote}

winter season, then it will be extracted november and december
months data and will be calculated variance for that alone.
\begin{quote}

If you will pass two year data then it will extract the data
\end{quote}

from november \& december of first year and january, feburary \&
march of next year will be extracted and calculated variance
for that.

\end{description}

Written By : Arulalan.T

Date : 26.07.2012

\end{fulllineitems}

\index{calculateVariance() (in module variance\_utils)}

\begin{fulllineitems}
\phantomsection\label{mjo:variance_utils.calculateVariance}\pysiglinewithargsret{\code{variance\_utils.}\bfcode{calculateVariance}}{\emph{varName}, \emph{fpath}, \emph{speed=True}, \emph{**kwarg}}{}
calculate Variance for anomaly data in temporal way.All Seasonal Variance.

if speed is True, then it should calculate the variance for the whole
data.

If data size is too large which cant handle by normal system, then we
should off the speed arg. speed = False. So it will take the partial data
(in latitude and/or level wise) with its full time axis. By this way, we
can calculate the variance for large size of data.

It should loop through the each and every latitude grid point, (but not by
longitude) get the full time series with full longitude wise data and
compute the statistics.variance and store it.
\begin{description}
\item[{KWargs:}] \leavevmode\begin{description}
\item[{year}] \leavevmode{[}Its optional only. By default it is None. i.e. It will{]}
extract all the available years data to compute variance over timeAxis.
If one interger year has passed means, then it will do
extract of that particular year data alone.
If two years has passed in tuple, then it will extract
the range of years data from year{[}0{]} to year{[}1{]}.
eg1 : year=2005 it will extract seasonData of 2005 alone.
eg2 : year=(1971, 2013) it will extract seasonData from
\begin{quote}

1971 to 2013 years.
\end{quote}

\end{description}

\end{description}

Written By : Arulalan.T

Date : 26.07.2012

\end{fulllineitems}

\index{computeAnomaly() (in module variance\_utils)}

\begin{fulllineitems}
\phantomsection\label{mjo:variance_utils.computeAnomaly}\pysiglinewithargsret{\code{variance\_utils.}\bfcode{computeAnomaly}}{\emph{mvar}, \emph{modelPath}, \emph{cvar}, \emph{climPath}, \emph{climyear}, \emph{outPath}, \emph{convertTI2N=False}, \emph{hour=None}, \emph{sign=1}, \emph{**kwarg}}{}~\begin{description}
\item[{computeAnomaly}] \leavevmode{[}compute the anomaly by opening the model and climatology{]}
data from the files. Finally writes the output anomaly into outpath
file.

\item[{Input :}] \leavevmode
mvar - model variable name
modelPath - model nc file absolute path
cvar - climatology variable name
climPath - climatology nc file absolute path
climyear - Year in climatology file
outPath - anomaly nc file (output) absolute path
convertTI2N - It takes either True or False. If it is True, then the
\begin{quote}

model data will be converted from Time Integrated to Normal form.
i.e. Units will be converted from Wsm\textasciicircum{}-2 to Wm\textasciicircum{}2.
\end{quote}

hour - hour of the model data (will be used in convertTI2N function)
sign - change sign of the model data (will be used in convertTI2N fn)
\begin{description}
\item[{kWargs :-}] \leavevmode
long\_name - long name for the output anomaly
comments - comments for the output anomaly
cregrid - climatology data will be regridded w.r.t model data
dregrid - model data will be regridded w.r.t climatology data
\begin{quote}

by default both cregrid \& dregrid are False
\end{quote}

\end{description}

\end{description}

Refer : anomaly, convertTimeIntegratedToNormal

Written By : Arulalan.T

Date : 11.06.2012

\end{fulllineitems}

\index{lfilter() (in module variance\_utils)}

\begin{fulllineitems}
\phantomsection\label{mjo:variance_utils.lfilter}\pysiglinewithargsret{\code{variance\_utils.}\bfcode{lfilter}}{\emph{data}, \emph{weights}, \emph{cyclic=True}, \emph{**kwarg}}{}
Lanczos Filtered
Todo : Write doc of the work flow
KWarg:
\begin{quote}
\begin{description}
\item[{returntlen}] \leavevmode{[}return time axis length. By default it takes None.{]}\begin{quote}

It can take positive or negative integer as value.
If it is +ve no, then it will do filter only to
the first no days rather than doing filter to the
whole available days of the data.
If it is -ve no, then it will do filter only to
the last no days.
Timeaxis also will set properly w.r.t return data.
if cyclic is enabled or disabled, then returntlen
index will change according to cyclic flag.

Lets assume weights length is 101.
Eg1: Lets consider data length is 365.
\begin{quote}

If cyclic is False, then filteredData length is
365-(100*2)=165 when returntlen is None.
If returntlen = 10, so filteredData length is 10
and its index is range(100, 110).
If returntlen = -10, so filteredData length is 10
and its index is range(255, 265).
\end{quote}
\begin{description}
\item[{Eg2: If cyclic is True, then filteredData length is}] \leavevmode
365 when returntlen is None.
If returntlen = 10, so filteredData length is 10
and its index is range(0, 10).
If returntlen = -10, so filteredData length is 10
and its index is range(355, 365).

\end{description}
\end{quote}

So returntlen option is useful when we need to filter
only to the certain days. But regardless of returntlen
option, we need to pass the full data to apply Lanczos
Filter to either returntlen days or full days.
This option will save a lot of time when returntlen is
very less compare to the total length of the data.

\end{description}
\end{quote}

Written By : Arulalan.T

Date : 15.09.2012
Updated : 22.09.2013

\end{fulllineitems}

\index{plotVariance() (in module variance\_utils)}

\begin{fulllineitems}
\phantomsection\label{mjo:variance_utils.plotVariance}\pysiglinewithargsret{\code{variance\_utils.}\bfcode{plotVariance}}{\emph{data}, \emph{outfile}, \emph{season}, \emph{title='`}, \emph{pdf=1}, \emph{png=0}, \emph{**kwarg}}{}
Written By : Alok Singh, Arulalan.T

Date : 28.05.2012
Updated : 08.07.2013

\end{fulllineitems}

\index{summerVariance() (in module variance\_utils)}

\begin{fulllineitems}
\phantomsection\label{mjo:variance_utils.summerVariance}\pysiglinewithargsret{\code{variance\_utils.}\bfcode{summerVariance}}{\emph{varName}, \emph{fpath}, \emph{sday=1}, \emph{smon=5}, \emph{eday=31}, \emph{emon=10}, \emph{hr=0}, \emph{**kwarg}}{}
summerVariance : summer season (May to October) variance
\begin{description}
\item[{Inputs :}] \leavevmode
varName - anomaly variable name
fpath - anomaly file path
sday : starting day of the summer season {[}of all the years of the data{]}.
smon : starting month of the summer season {[}of all the years of the data{]}.
eday : ending day of the summer season {[}of all the years of the data{]}.
emon : ending month of the summer season {[}of all the years of the data{]}.
hr : hour for both start and end date

\item[{KWargs :}] \leavevmode\begin{description}
\item[{year}] \leavevmode{[}It could be single year or range of years in tuple.{]}
By default it is None. So it will calculate all available years
summerVariance

\end{description}

\end{description}

Refer: calculateSeasonalVariance

Written By : Arulalan.T

Date : 26.07.2012

\end{fulllineitems}

\index{winterVariance() (in module variance\_utils)}

\begin{fulllineitems}
\phantomsection\label{mjo:variance_utils.winterVariance}\pysiglinewithargsret{\code{variance\_utils.}\bfcode{winterVariance}}{\emph{varName}, \emph{fpath}, \emph{sday=1}, \emph{smon=11}, \emph{eday=30}, \emph{emon=4}, \emph{hr=0}, \emph{**kwarg}}{}
winterVariance : winter season (November to April) variance
\begin{description}
\item[{Inputs :}] \leavevmode
varName - anomaly variable name
fpath - anomaly file path
sday : starting day of the winter season {[}of all the years of the data{]}.
smon : starting month of the winter season {[}of all the years of the data{]}.
eday : ending day of the winter season {[}of all the years of the data{]}.
emon : ending month of the winter season {[}of all the years of the data{]}.
hr : hour for both start and end date

\item[{KWargs :}] \leavevmode\begin{description}
\item[{year}] \leavevmode{[}It could be single year or range of years in tuple.{]}
By default it is None. So it will calculate all available years
winterVariance

\end{description}

\end{description}

Refer: calculateSeasonalVariance

Written By : Arulalan.T

Date : 26.07.2012

\end{fulllineitems}



\section{MJO-LEVEL2 Diagnosis Utils}
\label{mjo:mjo-level2-diagnosis-utils}

\subsection{combined eof utils}
\label{mjo:module-ceof_diag}\label{mjo:combined-eof-utils}\index{ceof\_diag (module)}\index{genCeofVars() (in module ceof\_diag)}

\begin{fulllineitems}
\phantomsection\label{mjo:ceof_diag.genCeofVars}\pysiglinewithargsret{\code{ceof\_diag.}\bfcode{genCeofVars}}{\emph{infiles}, \emph{outfile}, \emph{eobjf=True}, \emph{lat=(-15}, \emph{15}, \emph{`cob')}, \emph{NEOF=4}, \emph{season='all'}, \emph{**kwarg}}{}
genCeofVars : generate the combined eof variables as listed below.

Input :-
\begin{description}
\item[{infiles}] \leavevmode{[}List of tuples contains variable generic name, actual varible{]}
name (say model varName), data path. So ceof input files list
contains the tuples which contains the above three information.

For eg : ceof(olr, u200, u850)
infiles = {[}(`olr', `olrv', `olr.ctl'),
\begin{quote}

(`u200', `u200v', `u200.ctl'),
(`u850', `u850v', `u850.ctl'){]}
\end{quote}

\item[{outfile}] \leavevmode{[}output nc file path. All the supported output varibles of ceof{]}
will be stored in this nc file. This nc file will be opened in
append mode.

\item[{eobjf}] \leavevmode{[}eof obj file store. If True, the eofobj of ceof will be stored{]}
as binary (`{\color{red}\bfseries{}*}.pkl') file in the outfile path directory,
using pickle module.

\item[{lat}] \leavevmode{[}latitude to extract the data of the input file.{]}
By default it takes as (-15, 15, `cob').

\end{description}

NEOF : no of eof or no of mode. By default it takes 4.
\begin{description}
\item[{season}] \leavevmode{[}It could be `all', `sum', `win'. Not a list.{]}
all : through out the year data from all the available years
sum : only summer data will be extracted from all the available years
win : only winter data will be extracted from all the available years

\item[{Output}] \leavevmode{[}The follwing output varibles from 1 to 5 will be stored/append{]}\begin{quote}

into outfile along with individual varName, season
(where ever needed).
\end{quote}
\begin{enumerate}
\item {} 
Std of each zonal normalized varibles

\item {} 
Percentage explained by ceof input variables

\item {} 
variance accounted for ceof of each input variables

\item {} 
eof variable of each input variables

\item {} 
PC time series, Normalized PC time Series (for `all' season only)

\item {} 
Store the eofobj (into .pkl file) {[}optional{]}

\end{enumerate}

\end{description}

Written By : Dileep.K, Arulalan.T

Date :

Updated : 11.05.2013

\end{fulllineitems}

\index{genProjectedPcts() (in module ceof\_diag)}

\begin{fulllineitems}
\phantomsection\label{mjo:ceof_diag.genProjectedPcts}\pysiglinewithargsret{\code{ceof\_diag.}\bfcode{genProjectedPcts}}{\emph{infiles}, \emph{outfile}, \emph{eofobj}, \emph{lat=(-15}, \emph{15}, \emph{`cob')}, \emph{NEOF=4}, \emph{season='mjjas'}, \emph{**kwarg}}{}~\begin{description}
\item[{KWargs:}] \leavevmode\begin{description}
\item[{ogrid}] \leavevmode{[}if grid has passed then data will be regridded before{]}
normalize it w.r.t passed grid resoltution. By default it
takes None.

\item[{dtype}] \leavevmode{[}dtype should be either `anl', or fcst hours.{]}
if both season and dtype has passed then, according to dtype
season time will be calculated using xmlobj.findPartners()
method. By default it takes `Analysis'.
i.e. period from 01-05-yyyy to 30-9-yyyy for the `mjjas' season.
If dtype will be `24' means then period will be from 30-04-yyyy
to 29-09-yyyy. Likewise user can pass fcst hour.

\end{description}

\end{description}

Date : 08.08.2013

\end{fulllineitems}

\index{getZonalNormStd() (in module ceof\_diag)}

\begin{fulllineitems}
\phantomsection\label{mjo:ceof_diag.getZonalNormStd}\pysiglinewithargsret{\code{ceof\_diag.}\bfcode{getZonalNormStd}}{\emph{data}}{}
data - pass filtered data
Return : normalized data and std of the meridionally averaged data.
\begin{quote}

Once we averaged over latitude, then it will become zonal data.
\end{quote}

\end{fulllineitems}

\index{makeGenCeofVars() (in module ceof\_diag)}

\begin{fulllineitems}
\phantomsection\label{mjo:ceof_diag.makeGenCeofVars}\pysiglinewithargsret{\code{ceof\_diag.}\bfcode{makeGenCeofVars}}{\emph{rawOrAnomaly='Anomaly', filteredOrNot='Filtered', seasons={[}'all', `sum', `win'{]}, **kwarg}}{}
Written By : Arulalan.T

Date : 22.07.2013

\end{fulllineitems}

\index{makeProjectedPcts() (in module ceof\_diag)}

\begin{fulllineitems}
\phantomsection\label{mjo:ceof_diag.makeProjectedPcts}\pysiglinewithargsret{\code{ceof\_diag.}\bfcode{makeProjectedPcts}}{\emph{rawOrAnomaly='Anomaly', filteredOrNot='Filtered', seasons={[}'mjjas'{]}, obsname='MJO', **kwarg}}{}~\begin{description}
\item[{KWarg:}] \leavevmode\begin{description}
\item[{exclude}] \leavevmode{[}exclude hours. If some hours list has passed then those {]}
model hours will be omitted. eg : 01 hour.
Note : it will omit those exclude model hours directory.
So for the remaining model anl, fcst hours only calculated the
projected pcts.

\end{description}

\end{description}

Written By : Arulalan.T

Date : 08.08.2013

\end{fulllineitems}



\subsection{phase3d utils}
\label{mjo:module-trig}\label{mjo:phase3d-utils}\index{trig (module)}\index{getHalfQuadrantOfCircle() (in module trig)}

\begin{fulllineitems}
\phantomsection\label{mjo:trig.getHalfQuadrantOfCircle}\pysiglinewithargsret{\code{trig.}\bfcode{getHalfQuadrantOfCircle}}{\emph{x}, \emph{y}}{}~\begin{description}
\item[{getHalfQuadrantOfCircle}] \leavevmode{[}return the half quadrant of the circle for the{]}
given x, y points. Usually we split the circle into 4 equal quadrants,
but in this function we split circle into 8 equal parts. Thats why
we call it as half quadrants. i.e we split one quadrant into two equal
parts (two equal half quadrants).

\end{description}

Written By : Arulalan.T

Date : 02.04.2013

License : GPL V3

\end{fulllineitems}

\phantomsection\label{mjo:module-phase3d}\index{phase3d (module)}\phantomsection\label{mjo:module-phase3d}\index{phase3d (module)}\index{miso\_phase3d() (in module phase3d)}

\begin{fulllineitems}
\phantomsection\label{mjo:phase3d.miso_phase3d}\pysiglinewithargsret{\code{phase3d.}\bfcode{miso\_phase3d}}{\emph{xdata, ydata, sxyphase=None, dmin=None, dmax=None, colors={[}'violet', `green', `orange', `red'{]}, tcomment='Centre India \& North India', bcomment='Southern tp \& IO', lcomment='Peninsular \& Centre India', rcomment='FootHill \& SIO', title='Indian Monsoon Intraseasonal Oscillation Index', stitle1='`, stitle2='`, ctitle='Weak MISO', pposition1=8, pdirection='clock', plocation='out', mintick=4, **kwarg}}{}~\begin{description}
\item[{miso\_phase3d}] \leavevmode{[}Monsoon Intraseasonal Oscillation Phase 2 Dimensional{]}
Diagram of 3D Var (pcs1, pcs2 \& time)

\item[{Inputs :}] \leavevmode
xdata : Normalized PC1 or relevant data
ydata : Normalized PC2 or relevant data
sxyphase : phase number of start x,y points of npc1, npc2. By default
\begin{quote}

it takes None. For the simplicity to the user, passed the
default/correct argument to the pposition1 as 8.
\end{quote}
\begin{description}
\item[{pposition1}] \leavevmode{[}phase number of the phase position1 of the graph. For the{]}
MJO it must be 8. User can over write the sxyphase and
pposition1 args while calling this function, if needed.

\end{description}

pdirection : By default it takes `clock' to this MJO.
plocation : Phase name/string drawn inside/outside of the graph. By
\begin{quote}

default it takes `in'.
\end{quote}
\begin{description}
\item[{colors}] \leavevmode{[}list contains 6 colors to indicate 6 months dataset. User{]}
can overwrite it either by single or many color.

\end{description}

\item[{Return :}] \leavevmode
It should return the `x' of the xmgrace object which has plotted the
data with this MJO args. User can either save it into ps, pdf, etc.,
or modify/update further.

\end{description}

Refer : phase3d() for the detailed documents of all the arguments.
\begin{description}
\item[{Plot Properties Reference}] \leavevmode{[}``An Indian monsoon intraseasonal oscillations{]}
(MISO) index for real time monitoring and forecast verification'',
E. Suhas , J. M. Neena , B. N. Goswami, Clim Dyn,
DOI 10.1007/s00382-012-1462-5

\end{description}

Author : Arulalan.T

Date : 20.02.2013

License : GPL V3

\end{fulllineitems}

\index{mjo\_phase3d() (in module phase3d)}

\begin{fulllineitems}
\phantomsection\label{mjo:phase3d.mjo_phase3d}\pysiglinewithargsret{\code{phase3d.}\bfcode{mjo\_phase3d}}{\emph{npc1, npc2, sxyphase=None, dmin=0, dmax=0, colors={[}'magenta', `blue', `violet', `green', `orange', `red'{]}, tcomment='Western Pacific', bcomment='Indian Ocean', lcomment='Western Hemisphere \& Africa', rcomment='Maritime Continent', title='MJO Phase diagram of PC-1 and PC-2', stitle1='`, stitle2='`, ctitle='Weak MJO', pposition1=5, pdirection='anticlock', plocation='in', mintick=0, **kwarg}}{}~\begin{description}
\item[{mjo\_phase3d}] \leavevmode{[}Madden-Julian Oscillation Phase 2 Dimensional Diagram of{]}
3D Var (pcs1, pcs2 \& time)

\item[{Inputs :}] \leavevmode
npc1 : Normalized PC1 (xdata)
npc2 : Normalized PC2 (ydata)
sxyphase : phase number of start x,y points of npc1, npc2. By default
\begin{quote}

it takes None. For the simplicity to the user, passed the
default/correct argument to the pposition1 as 5.
\end{quote}
\begin{description}
\item[{pposition1}] \leavevmode{[}phase number of the phase position1 of the graph. For the{]}
MJO it must be 5. User can over write the sxyphase and
pposition1 args while calling this function, if needed.

\end{description}

pdirection : By default it takes `anticlock' to this MJO.
plocation : Phase name/string drawn inside/outside of the graph. By
\begin{quote}

default it takes `in'.
\end{quote}
\begin{description}
\item[{colors}] \leavevmode{[}list contains 6 colors to indicate 6 months dataset. User{]}
can overwrite it either by single or many color.

\end{description}

\item[{Return :}] \leavevmode
It should return the `x' of the xmgrace object which has plotted the
data with this MJO args. User can either save it into ps, pdf, etc.,
or modify/update further.

\end{description}

Refer : phase3d() for the detailed documents of all the arguments.
\begin{description}
\item[{Plot Properties Reference :}] \leavevmode
\href{http://climate.snu.ac.kr/mjo\_diagnostics/index.htm}{http://climate.snu.ac.kr/mjo\_diagnostics/index.htm}

\end{description}

Author : Arulalan.T

Date : 20.02.2013

License : GPL V3

\end{fulllineitems}

\index{phase3d() (in module phase3d)}

\begin{fulllineitems}
\phantomsection\label{mjo:phase3d.phase3d}\pysiglinewithargsret{\code{phase3d.}\bfcode{phase3d}}{\emph{xdata, ydata, sxyphase, dmin=None, dmax=None, colors={[}'red'{]}, tcomment='`, bcomment='`, lcomment='`, rcomment='`, title='Phase Diagram', stitle1='`, stitle2='`, ctitle='Weak', **kwarg}}{}~\begin{description}
\item[{phase3d}] \leavevmode{[}This will allow user to plot the 2 dimensional line plot in{]}
circular path among 8 phases of 3 dimensional datasets.
i.e. pcs1, pcs2 and time are 3 dimensional datasets. It as
projected into 2 dimensional line plot. But it actually reprsent
3 dimensional dataset. So we named it as phase3d.
Also named as phase space diagram.

\item[{Inputs:}] \leavevmode\begin{description}
\item[{xdata}] \leavevmode{[}single dimensional xaxis dataset.{]}
eg : normalized pc1 time series

\item[{ydata}] \leavevmode{[}single dimensional yaxis dataset.{]}
eg : normalized pc2 time series
Both xdata \& ydata should be continous time series dataset.

\end{description}

xdata.id and ydata.id will be set as xaxis \& yaxis label.
\begin{description}
\item[{dmin}] \leavevmode{[}data min - xaxis and yaxis minimum scale/label value.{]}
It must be -ve.

\item[{dmax}] \leavevmode{[}data max - xaxis and yaxis maximum scale/label value.{]}
It must be +ve.
By default both dmin, dmax takes as None. If it is None, then
this function will automatically find out the min \& max from
the xdata \& ydata. Finally set dmin = -dmax. So that we will
get squared region.

\item[{colors}] \leavevmode{[}phase line colors. It should be eithre string or list of{]}
colors. If it is string or single color list, the it will
apply that color through out the data line in the graph.
If user passed it as list of colors (more than one colors)
then it will apply those colors to the available months data
set. So make sure that your colorlist and available months
length should be same.

\end{description}

tcomment : Top Comment
bcomment : Bottom Comment
lcomment : Left Comment
rcomment : Right Comment
title : Title of the diagram
stitle1 : Sub Title 1 -
\begin{quote}

If `year' has passed then it will draw string as start year
and end year from the xdata.
\end{quote}
\begin{description}
\item[{stitle2}] \leavevmode{[}Sub Title 2 -{]}
If `month' has passed then it will draw string as season of
available months (from startmonth to endmonth).

If `date' has passed then it will draw string as season of
available dates (from startdate to enddate).

User can overwrite the stitle1 \& stitle2 strings.

\end{description}

ctitle : Centred Title or text inside the circle.
\begin{description}
\item[{sxyphase}] \leavevmode{[}start x, y phase - start x,y points of xdata, ydata phase{]}
number. So user can pass the phase number of the starting
points of the whole data (xdata, ydata). With depends upon
pdirection it will be draw the phase name over the 8
half quadrants of the graph. User no need to pass phase
numbers for the whole datasets (xdata, ydata). Just pass
the first or start day of season phase alone. User can use
pposition1 argument also instead of sxyphase argument.
No default argument takes place here. So user must pass
argument to this sxyphase arg. If user need to use
pposition1 arg, then they must pass None to this sxyphase
argument.

\item[{KWargs :-}] \leavevmode\begin{description}
\item[{pposition1}] \leavevmode{[}phase position 1 - Lets understand the distribution of{]}
8 phases. Lets consider from 0 to 45 degree region as
phase position 1 (pposition1).From 45 to 90 degree region
as phase position 2(pposition2) and so on. So if we walk
in anti-clock wise direction we will endup with pposition8
in the region of 315 to 360 degree.
\begin{quote}

User can pass argument to pposition1 as integer from
\end{quote}

1 to 8. So whatever user passed in the pposition1 phase
number, that phase no string will be draw within in range
from 0 to 45 degree.
\begin{quote}
\begin{description}
\item[{eg}] \leavevmode{[}pposition1=5 . i.e. phase name `PHASE 5' will be{]}
drawn in the 0 to 45 degree region of the graph.

\end{description}

By default it takes None argument.
\end{quote}
\begin{description}
\item[{..note:: User can not use both sxyphase and pposition1}] \leavevmode
args. Can be used either sxyphase or pposition1
only.

\end{description}

\item[{pdirection}] \leavevmode{[}phase direction - It can be either `clock' or `anticlock'{]}\begin{quote}

It will determine what is the phase number in the
pposition2, pposition3, ..., pposition8 w.r.t the input
of pposition1 argument or sxyphase and clock/anticlock
direction.
eg 1: pposition1 = 5, pdirection = `anticlock'
\begin{quote}

pposition1=5, pposition2=6, pposition3=7, pposition4=8,
pposition5=1, pposition6=2, pposition7=3, pposition8=4.
\end{quote}
\begin{description}
\item[{i.e. w.r.t pposition1 as 5 and pdirection as `anticlock'}] \leavevmode\begin{quote}

the phase names of the rest phase positions are
\end{quote}

determined as increment trend of pnames in ppositions.

\end{description}
\end{quote}
\begin{description}
\item[{eg 2: pposition1 = 5, pdirection = `clock'}] \leavevmode\begin{quote}

pposition1=5, pposition2=4, pposition3=3, pposition4=2,
pposition5=1, pposition6=8, pposition7=7, pposition8=6.
\end{quote}
\begin{description}
\item[{i.e. w.r.t pposition1 as 5 and pdirection as `clock'}] \leavevmode\begin{quote}

the phase names of the rest phase positions are
\end{quote}

determined as decrement trend of pnames in ppositions.

\end{description}

same examples we can play with sxyphase also.

\end{description}

\item[{plocation}] \leavevmode{[}It takes input as `in/inside/out/outside'. If it is{]}
`in' or `inside', then the phase name/number will be drawn
inside the squared graph itself.
If it is `out' or `outside', then the phase name/number
will be drawn outside the squared graph.

\item[{mintick}] \leavevmode{[}minor ticks count. By default it takes as 0. user can pass 4{]}
also.

\item[{timeorder}] \leavevmode{[}It takes a list of nos which representing the month order{]}
from 1 to 12. It could be any order. If user has passed any
order of nos in this, then line plots will be plotted in
the passed months order. By default it takes None. In that
case the data months wil be sorted in the correct order,
then that will be plotted.
eg : timeorder={[}11, 12, 1, 2, 3, 4{]} . This will be useful
to plot winter season when we have only one year data.
So user has to enable both cyclic=True and timeorder as
the above list. Then it will be plotted as NDJFMA.
Otherwise it will plot as JFMAND \& make looks like mess-up.
For cyclic option refer timeutils.getSeasonData().

\end{description}

\end{description}

\end{description}

Methods Used : \_sortMonths, getTimeAxisMonths, getHalfQuadrantOfCircle
\begin{description}
\item[{Return :}] \leavevmode
It should return the `x' of the xmgrace object which has plotted the
data with this MJO args. User can either save it into ps, pdf, etc.,
or modify/update further.

\end{description}

Author : Arulalan.T

Date : 20.02.2013

License : GPL V3

\end{fulllineitems}



\subsection{Wheeler Kiladis Diagram Utils}
\label{mjo:wheeler-kiladis-diagram-utils}\label{mjo:module-wk_utils}\index{wk\_utils (module)}\index{genWKVars() (in module wk\_utils)}

\begin{fulllineitems}
\phantomsection\label{mjo:wk_utils.genWKVars}\pysiglinewithargsret{\code{wk\_utils.}\bfcode{genWKVars}}{\emph{data}, \emph{outpath}, \emph{outfile}, \emph{segment=96}, \emph{overlap=60}, \emph{**kwarg}}{}
data - model anomaly
\begin{description}
\item[{output - It produces the following 6 variables \& will be written}] \leavevmode\begin{description}
\item[{into outfile.}] \leavevmode\begin{enumerate}
\item {} 
power

\item {} 
power\_S  \# symmetric power

\item {} 
power\_A  \# anti-symmetric power

\item {} 
background

\item {} 
power\_S\_bg  \# symmetric power / background

\item {} 
power\_A\_bg  \# anti-symmetric power / background

\end{enumerate}

\end{description}

\item[{outfile - user has to pass the outfile name with .nc file extension.}] \leavevmode
This outfile name will be updated along with no of days in
segment and no of days overlap. Finally it will return the
new nc file name with its outpath.

\end{description}

Return - outfile path (outfile name has modified here)

Written By : Arulalan.T

Updated : 24.06.2013

\end{fulllineitems}

\index{plotWKBackground() (in module wk\_utils)}

\begin{fulllineitems}
\phantomsection\label{mjo:wk_utils.plotWKBackground}\pysiglinewithargsret{\code{wk\_utils.}\bfcode{plotWKBackground}}{\emph{infile}, \emph{outpath}, \emph{outfile='background'}, \emph{lmin=-1}, \emph{lmax=2}, \emph{png=0}, \emph{pdf=1}}{}
figure2 : plotting background

\end{fulllineitems}

\index{plotWKPowers() (in module wk\_utils)}

\begin{fulllineitems}
\phantomsection\label{mjo:wk_utils.plotWKPowers}\pysiglinewithargsret{\code{wk\_utils.}\bfcode{plotWKPowers}}{\emph{infile}, \emph{outpath}, \emph{outfile='Powers'}, \emph{png=0}, \emph{pdf=1}}{}
figure1 : plotting power\_S \& power\_A

\end{fulllineitems}

\index{plotWK\_Sym\_ASym() (in module wk\_utils)}

\begin{fulllineitems}
\phantomsection\label{mjo:wk_utils.plotWK_Sym_ASym}\pysiglinewithargsret{\code{wk\_utils.}\bfcode{plotWK\_Sym\_ASym}}{\emph{infile}, \emph{outpath}, \emph{outfile='Sym\_ASym'}, \emph{lmin=1.0}, \emph{lmax=2.1}, \emph{ptitle='Wheeler Kiladis Diagram'}, \emph{comment\_1='`}, \emph{comment\_2='Variable (Data) :}, \emph{Period : JJAS'}, \emph{png=0}, \emph{pdf=1}}{}
figure3 : plotting power\_S\_bg \& power\_A\_bg

lmin : legend min user can pass even +ve value or -ve value
lmax : legend max must be +ve value

ptitile : Plot title
\begin{description}
\item[{comment\_1}] \leavevmode{[}If comment\_1 is `' or None, then comment will be extracted{]}
from power\_S\_bg variable (while generating this variable
using funciton genWKVars, the comment has set to all the
variables as like `96-day segment, 60-day overlapping'.
The parameters days will be set w.r.t input of that funciton).

\item[{comment\_2}] \leavevmode{[}By default it is `Variable (Data){]}{[}, Period{]}{[}JJAS'.{]}
User can change it as follows
eg : `Variable (Data) : OLR (NCMRWF), Period : JJAS 2010'

Both comment\_1 \& comment\_2 can be changed by user.

\end{description}

\end{fulllineitems}



\chapter{Documentation of \textbf{MISO} source code}
\label{miso:miso}\label{miso:documentation-of-miso-source-code}\label{miso::doc}

\section{MISO Diagnosis Utils}
\label{miso:miso-diagnosis-utils}

\subsection{harmonic climatology utils}
\label{miso:harmonic-climatology-utils}\label{miso:module-harmonic_util}\index{harmonic\_util (module)}\index{harmonic() (in module harmonic\_util)}

\begin{fulllineitems}
\phantomsection\label{miso:harmonic_util.harmonic}\pysiglinewithargsret{\code{harmonic\_util.}\bfcode{harmonic}}{\emph{data}, \emph{k=3}, \emph{time\_type='daily'}, \emph{phase\_shift=15}}{}~\begin{description}
\item[{Inputs}] \leavevmode{[}{]}
data : climatology data 
k : Integer no to compute K th harmonic. By default it takes 3.
time\_type : daily \textbar{} monthly \textbar{} full (time type of input climatology)
\begin{quote}

`daily' -\textgreater{} it returns 365 days harmonic,
`monthly' -\textgreater{} it returns 12 month harmonic,
`full' -\textgreater{} it retuns harmonic for full length of 
input data.
\end{quote}
\begin{description}
\item[{phase\_shift}] \leavevmode{[}Used to subtract `phase\_shift' days lag to adjust{]}
phase\_angle w.r.t daily or monthly. By default it takes
15 days lag to adjust phase\_angle w.r.t daily data.
User can pass None disable this option.

\end{description}

\item[{Returns :}] \leavevmode
Returns ``sum mean of mean and first K th harmonic'' of input 
climatology data.

\end{description}

Concept :

Earth science data consists of a strong seasonality component as 
indicated by the cycles of repeated patterns in climate variables such 
as air pressure, temperature and precipitation. The seasonality forms 
the strongest signals in this data and in order to find other patterns,
the seasonality is removed by subtracting the monthly mean values of the
raw data for each month. However since the raw data like air temperature,
pressure, etc. are constantly being generated with the help of satellite
observations, the climate scientists usually use a moving reference base 
interval of some years of raw data to calculate the mean in order to 
generate the anomaly time series and study the changes with respect to
that.

Fourier series analysis decomposes a signal into an infinite series of 
harmonic components. Each of these components is comprised initially of 
a sine wave and a cosine wave of equal integer frequency. These two waves
are then combined into a single cosine wave, which has characteristic 
amplitude (size of the wave) and phase angle (offset of the wave). 
Convergence has been established for bounded piecewise continuous 
functions on a closed interval, with special conditions at points of
discontinuity. Its convergence has been established for other conditions
as well, but these are not relevant to the analysis at hand.
\begin{description}
\item[{Reference: Daniel S Wilks, `Statistical Methods in the Atmospheric }] \leavevmode
Sciences' second Edition, page no(372-378).

\end{description}

Written By : Arulalan.T

Date : 16.05.2014

\end{fulllineitems}



\chapter{Documentation of \textbf{Other Utils} source code}
\label{others:other}\label{others::doc}\label{others:documentation-of-other-utils-source-code}

\section{binary2ascii convertor}
\label{others:binary2ascii-convertor}
This function used to convert any kind of climate model binary files such as netcdf, grib1, grib2, pp, ctl, etc., to ascii / csv file.
\phantomsection\label{others:binary2ascii}\phantomsection\label{others:module-binary2ascii}\phantomsection\label{others:binary2ascii}\index{binary2ascii (module)}\index{binary2ascii() (in module binary2ascii)}

\begin{fulllineitems}
\phantomsection\label{others:binary2ascii.binary2ascii}\pysiglinewithargsret{\code{binary2ascii.}\bfcode{binary2ascii}}{\emph{var}, \emph{fpath}, \emph{opath=None}, \emph{dlat=None}, \emph{dlon=None}, \emph{freq='daily'}, \emph{missing\_value='default'}, \emph{speedup='True'}}{}~\begin{description}
\item[{binary2ascii}] \leavevmode{[}Convert the binary file such as nc, ctl, pp into ascii csv{]}
files. It should create individual files for each years.
Csv file contains the month, day, lat \& lon information
along with its corresponding data.

It has optimised code to extract data and write into file
by using numpy.tofile() function. Its just extract the
particular/each lat grid, extract all the longitude values
in single dimension array and write into file object at a
time. So it is more optimised.

\item[{Inputs :}] \leavevmode
var - variable name

fpath - binary file input absolute path
\begin{description}
\item[{opath - output directory path. Inside this folder, it should create}] \leavevmode
csv files with variable name along with year. If user didnt
pass any value for this, then it should create variable name
as folder name for the output in the current working
directory path.

\end{description}

dlat - need data lat shape in ascii. eg (0, 40)
dlon - need data lon shape in ascii. eg (60, 100)
\begin{description}
\item[{freq - it takes either `daily' or `monthly'.}] \leavevmode
It is just to fastup the time dimension loop by skipping 365
days in daily and 12 months in monthly to access the another/
next year dataset.

\item[{missing\_value - if missing\_value passed by user, then that value}] \leavevmode
should be set while writing into csv file. By default it takes
`default' value, i.e. it will take fill\_value from the binary
file information itself.

\item[{speedup - This binary2ascii.py works fine only for all 12 months or }] \leavevmode
365 days data. If some months are missing in b/w means, 
it will fail to work. So in that case, you switch off this 
speedup option.

\item[{todo - to get the available years, we need to use timeutils.py module.}] \leavevmode
in that case, the above speedup option no need.

\end{description}

\end{description}

Written By : Arulalan.T

Date : 22.08.2012

\end{fulllineitems}



\section{numpy utils}
\label{others:numpy-utils}
This {\hyperref[others:numutils]{numutils}} (\autopageref*{others:numutils}) module helps us to generate our own time axis, correct existing time axis bounds and generate bounds.

Here we used inbuilt methods of cdtime and cdutil module of uv-cdat.


\subsection{numutils}
\label{others:id1}\label{others:numutils}\phantomsection\label{others:module-numutils}\index{numutils (module)}\phantomsection\label{others:module-numutis.py}\index{numutis.py (module)}\index{nextmax() (in module numutils)}

\begin{fulllineitems}
\phantomsection\label{others:numutils.nextmax}\pysiglinewithargsret{\code{numutils.}\bfcode{nextmax}}{\emph{x}, \emph{val=None}}{}~\begin{description}
\item[{\code{nextmax()}: Returns the max value next to the top max value of the}] \leavevmode
numpy x. If val doesnot passed by user, it returns the
second most max value.

\item[{Inputs}] \leavevmode{[}x, numpy array{]}
val, any value. If value passed, then it should return the max
value of the next to the passed value.

\end{description}

Usage :
\begin{quote}

\begin{Verbatim}[commandchars=\\\{\}]
\PYG{g+gp}{\PYGZgt{}\PYGZgt{}\PYGZgt{} }\PYG{n}{x} \PYG{o}{=} \PYG{n}{numpy}\PYG{o}{.}\PYG{n}{array}\PYG{p}{(}\PYG{p}{[}\PYG{p}{[}\PYG{l+m+mi}{10}\PYG{p}{,} \PYG{l+m+mi}{1}\PYG{p}{]}\PYG{p}{,} \PYG{p}{[}\PYG{l+m+mi}{100}\PYG{p}{,} \PYG{l+m+mi}{1000}\PYG{p}{]}\PYG{p}{]}\PYG{p}{)}
\PYG{g+gp}{\PYGZgt{}\PYGZgt{}\PYGZgt{} }\PYG{n}{x}
\PYG{g+go}{array([[  10,    1],}
\PYG{g+go}{       [ 100, 1000]])}
\PYG{g+gp}{\PYGZgt{}\PYGZgt{}\PYGZgt{} }\PYG{n}{nextmax}\PYG{p}{(}\PYG{n}{x}\PYG{p}{)}
\PYG{g+go}{100}
\PYG{g+go}{    we didnt pass any val, so it should return 2nd max value}
\end{Verbatim}

\begin{Verbatim}[commandchars=\\\{\}]
\PYG{g+gp}{\PYGZgt{}\PYGZgt{}\PYGZgt{} }\PYG{n}{nextmax}\PYG{p}{(}\PYG{n}{x}\PYG{p}{,}\PYG{l+m+mi}{99}\PYG{p}{)}
\PYG{g+go}{10}
\PYG{g+go}{    we passed 99 as the val. So it should return the next max value}
\PYG{g+go}{    to the 99 is 10.}
\end{Verbatim}

\begin{Verbatim}[commandchars=\\\{\}]
\PYG{g+gp}{\PYGZgt{}\PYGZgt{}\PYGZgt{} }\PYG{n}{nextmax}\PYG{p}{(}\PYG{n}{x}\PYG{p}{,}\PYG{l+m+mi}{1000}\PYG{p}{)}
\PYG{g+go}{100}
\PYG{g+go}{    we passed 1000 as the val. So it should return the next max value}
\PYG{g+go}{    to the 1000 is 100.}
\end{Verbatim}

\begin{Verbatim}[commandchars=\\\{\}]
\PYG{g+gp}{\PYGZgt{}\PYGZgt{}\PYGZgt{} }\PYG{n}{nextmax}\PYG{p}{(}\PYG{n}{x}\PYG{p}{,}\PYG{l+m+mi}{1}\PYG{p}{)}
\end{Verbatim}
\begin{quote}

we passed 1 as the val. i.e. the least number in the x (or least
number which is not even present in the x). So there is no next
max number to 1. It should return None.
\end{quote}

we can find out 3rd most max value by just calling this function two
times.
\textgreater{}\textgreater{}\textgreater{} n = nextmax(x)
\textgreater{}\textgreater{}\textgreater{} nextmax(x, n)
\textgreater{}\textgreater{}\textgreater{} 10

10 is the 3rd most number in x.
\end{quote}

Written By : Arulalan.T

Date : 27.09.2011

\end{fulllineitems}

\index{nextmin() (in module numutils)}

\begin{fulllineitems}
\phantomsection\label{others:numutils.nextmin}\pysiglinewithargsret{\code{numutils.}\bfcode{nextmin}}{\emph{x}, \emph{val=None}}{}~\begin{description}
\item[{\code{nextmin()}: Returns the min value next to the least min value of the}] \leavevmode
numpy x. If val doesnot passed by user, it returns the
second lease min value.

\item[{Inputs}] \leavevmode{[}x, numpy array{]}
val, any value. If value passed, then it should return the min
value of the next to the passed value.

\end{description}

Usage :
\begin{quote}

\begin{Verbatim}[commandchars=\\\{\}]
\PYG{g+gp}{\PYGZgt{}\PYGZgt{}\PYGZgt{} }\PYG{n}{x} \PYG{o}{=} \PYG{n}{numpy}\PYG{o}{.}\PYG{n}{array}\PYG{p}{(}\PYG{p}{[}\PYG{p}{[}\PYG{l+m+mi}{10}\PYG{p}{,} \PYG{l+m+mi}{1}\PYG{p}{]}\PYG{p}{,} \PYG{p}{[}\PYG{l+m+mi}{100}\PYG{p}{,} \PYG{l+m+mi}{1000}\PYG{p}{]}\PYG{p}{]}\PYG{p}{)}
\PYG{g+gp}{\PYGZgt{}\PYGZgt{}\PYGZgt{} }\PYG{n}{x}
\PYG{g+go}{array([[  10,    1],}
\PYG{g+go}{       [ 100, 1000]])}
\PYG{g+gp}{\PYGZgt{}\PYGZgt{}\PYGZgt{} }\PYG{n}{nextmin}\PYG{p}{(}\PYG{n}{x}\PYG{p}{)}
\PYG{g+go}{10}
\PYG{g+go}{    we didnt pass any val, so it should return 2nd min value}
\end{Verbatim}

\begin{Verbatim}[commandchars=\\\{\}]
\PYG{g+gp}{\PYGZgt{}\PYGZgt{}\PYGZgt{} }\PYG{n}{nextmin}\PYG{p}{(}\PYG{n}{x}\PYG{p}{,} \PYG{l+m+mi}{11}\PYG{p}{)}
\PYG{g+go}{100}
\end{Verbatim}
\begin{quote}

we passed 11 as the val. So it should return the next min value
to the 11 is 100.
\end{quote}

\begin{Verbatim}[commandchars=\\\{\}]
\PYG{g+gp}{\PYGZgt{}\PYGZgt{}\PYGZgt{} }\PYG{n}{nextmin}\PYG{p}{(}\PYG{n}{x}\PYG{p}{,} \PYG{l+m+mi}{101}\PYG{p}{)}
\PYG{g+go}{1000}
\end{Verbatim}
\begin{quote}

we passed 101 as the val. So it should return the next min value
to the 101 is 1000.
\end{quote}

\begin{Verbatim}[commandchars=\\\{\}]
\PYG{g+gp}{\PYGZgt{}\PYGZgt{}\PYGZgt{} }\PYG{n}{nextmin}\PYG{p}{(}\PYG{n}{x}\PYG{p}{,} \PYG{l+m+mi}{1000}\PYG{p}{)}
\end{Verbatim}
\begin{quote}

we passed 1000 as the val. i.e. the most number in the x (or most
number which is not even present in the x). So there is no next
min number to 1000. It should return None.
\end{quote}

we can find out 3rd least min value by just calling this function two
times.
\textgreater{}\textgreater{}\textgreater{} n = nextmin(x)
\textgreater{}\textgreater{}\textgreater{} nextmin(x, n)
\textgreater{}\textgreater{}\textgreater{} 100

100 is the 3rd least number in x.
\end{quote}

Written By : Arulalan.T

Date : 27.09.2011

\end{fulllineitems}

\index{permanent() (in module numutils)}

\begin{fulllineitems}
\phantomsection\label{others:numutils.permanent}\pysiglinewithargsret{\code{numutils.}\bfcode{permanent}}{\emph{data}}{}
permanent: Square Matrix permanent

It would be numpy data or list data.

Matrix permanent is just same as determinant of the matrix but change -ve
sign into +ve sign through out its calculation of determinant.
\begin{description}
\item[{eg 1:}] \leavevmode
\begin{Verbatim}[commandchars=\\\{\}]
\PYG{g+gp}{\PYGZgt{}\PYGZgt{}\PYGZgt{} }\PYG{n}{a} \PYG{o}{=} \PYG{n}{numpy}\PYG{o}{.}\PYG{n}{ones}\PYG{p}{(}\PYG{l+m+mi}{9}\PYG{p}{)}\PYG{o}{.}\PYG{n}{reshape}\PYG{p}{(}\PYG{p}{(}\PYG{l+m+mi}{3}\PYG{p}{,}\PYG{l+m+mi}{3}\PYG{p}{)}\PYG{p}{)}
\PYG{g+gp}{\PYGZgt{}\PYGZgt{}\PYGZgt{} }\PYG{n}{z} \PYG{o}{=} \PYG{n}{permanent}\PYG{p}{(}\PYG{n}{a}\PYG{p}{)}
\PYG{g+gp}{\PYGZgt{}\PYGZgt{}\PYGZgt{} }\PYG{k}{print} \PYG{n}{z}
\PYG{g+gp}{\PYGZgt{}\PYGZgt{}\PYGZgt{} }\PYG{l+m+mf}{6.0}
\end{Verbatim}

\item[{eg 2:}] \leavevmode
\begin{Verbatim}[commandchars=\\\{\}]
\PYG{g+gp}{\PYGZgt{}\PYGZgt{}\PYGZgt{} }\PYG{n}{a} \PYG{o}{=} \PYG{n}{numpy}\PYG{o}{.}\PYG{n}{ones}\PYG{p}{(}\PYG{l+m+mi}{16}\PYG{p}{)}\PYG{o}{.}\PYG{n}{reshape}\PYG{p}{(}\PYG{p}{(}\PYG{l+m+mi}{4}\PYG{p}{,}\PYG{l+m+mi}{4}\PYG{p}{)}\PYG{p}{)}
\PYG{g+gp}{\PYGZgt{}\PYGZgt{}\PYGZgt{} }\PYG{n}{z} \PYG{o}{=} \PYG{n}{permanent}\PYG{p}{(}\PYG{n}{a}\PYG{p}{)}
\PYG{g+gp}{\PYGZgt{}\PYGZgt{}\PYGZgt{} }\PYG{k}{print} \PYG{n}{z}
\PYG{g+gp}{\PYGZgt{}\PYGZgt{}\PYGZgt{} }\PYG{l+m+mf}{24.0}
\end{Verbatim}

\end{description}

Written By : Arulalan.T

Date : 01.08.2012

\end{fulllineitems}

\index{remove\_nxm() (in module numutils)}

\begin{fulllineitems}
\phantomsection\label{others:numutils.remove_nxm}\pysiglinewithargsret{\code{numutils.}\bfcode{remove\_nxm}}{\emph{data}, \emph{n}, \emph{m}}{}
remove\_nxm : Remove n-th row and m-th column from the matrix/numpy data.
zero is the starting index for the row and column.
To remove first row \& column, we need to pass 0 as args.
\begin{description}
\item[{eg:}] \leavevmode
\begin{Verbatim}[commandchars=\\\{\}]
\PYG{g+gp}{\PYGZgt{}\PYGZgt{}\PYGZgt{} }\PYG{n}{a} \PYG{o}{=} \PYG{n}{numpy}\PYG{o}{.}\PYG{n}{arange}\PYG{p}{(}\PYG{l+m+mi}{20}\PYG{p}{)}\PYG{o}{.}\PYG{n}{reshape}\PYG{p}{(}\PYG{p}{(}\PYG{l+m+mi}{4}\PYG{p}{,}\PYG{l+m+mi}{5}\PYG{p}{)}\PYG{p}{)}
\PYG{g+gp}{\PYGZgt{}\PYGZgt{}\PYGZgt{} }\PYG{k}{print} \PYG{n}{a}
\PYG{g+gp}{\PYGZgt{}\PYGZgt{}\PYGZgt{} }\PYG{p}{[}\PYG{p}{[} \PYG{l+m+mi}{0}  \PYG{l+m+mi}{1}  \PYG{l+m+mi}{2}  \PYG{l+m+mi}{3}  \PYG{l+m+mi}{4}\PYG{p}{]}
\PYG{g+go}{     [ 5  6  7  8  9]}
\PYG{g+go}{     [10 11 12 13 14]}
\PYG{g+go}{     [15 16 17 18 19]]}
\PYG{g+gp}{\PYGZgt{}\PYGZgt{}\PYGZgt{} }\PYG{n}{b} \PYG{o}{=} \PYG{n}{remove\PYGZus{}nxm}\PYG{p}{(}\PYG{n}{a}\PYG{p}{,} \PYG{l+m+mi}{2}\PYG{p}{,} \PYG{l+m+mi}{2}\PYG{p}{)}
\PYG{g+gp}{\PYGZgt{}\PYGZgt{}\PYGZgt{} }\PYG{k}{print} \PYG{n}{b}
\PYG{g+gp}{\PYGZgt{}\PYGZgt{}\PYGZgt{} }\PYG{p}{[}\PYG{p}{[} \PYG{l+m+mi}{0}  \PYG{l+m+mi}{1}  \PYG{l+m+mi}{3}  \PYG{l+m+mi}{4}\PYG{p}{]}
\PYG{g+go}{     [ 5  6  8  9]}
\PYG{g+go}{     [15 16 18 19]]}
\PYG{g+go}{\PYGZgt{}\PYGZgt{}\PYGZgt{}}
\PYG{g+go}{  ..note:: removed 2\PYGZhy{}nd row and 2\PYGZhy{}column from the matrix a.}
\end{Verbatim}

\begin{Verbatim}[commandchars=\\\{\}]
\PYG{g+gp}{\PYGZgt{}\PYGZgt{}\PYGZgt{} }\PYG{n}{c} \PYG{o}{=} \PYG{n}{remove\PYGZus{}nxm}\PYG{p}{(}\PYG{n}{a}\PYG{p}{,} \PYG{l+m+mi}{0}\PYG{p}{,} \PYG{l+m+mi}{4}\PYG{p}{)}
\PYG{g+gp}{\PYGZgt{}\PYGZgt{}\PYGZgt{} }\PYG{k}{print} \PYG{n}{c}
\PYG{g+gp}{\PYGZgt{}\PYGZgt{}\PYGZgt{} }\PYG{p}{[}\PYG{p}{[} \PYG{l+m+mi}{5}\PYG{p}{,}  \PYG{l+m+mi}{6}\PYG{p}{,}  \PYG{l+m+mi}{7}\PYG{p}{,}  \PYG{l+m+mi}{8}\PYG{p}{]}\PYG{p}{,}
\PYG{g+go}{     [10, 11, 12, 13],}
\PYG{g+go}{     [15, 16, 17, 18]]}
\PYG{g+go}{\PYGZgt{}\PYGZgt{}\PYGZgt{}}
\PYG{g+go}{ ..note:: removed 0\PYGZhy{}th row and 4\PYGZhy{}th column from the matrix a.}
\end{Verbatim}

\end{description}

Written By : Arulalan.T

Date : 01.08.2012

\end{fulllineitems}



\section{Non-rectilinear Utils}
\label{others:module-nonrect_utils}\label{others:non-rectilinear-utils}\index{nonrect\_utils (module)}\index{get1LatLonFromNonRectiLinearGrid() (in module nonrect\_utils)}

\begin{fulllineitems}
\phantomsection\label{others:nonrect_utils.get1LatLonFromNonRectiLinearGrid}\pysiglinewithargsret{\code{nonrect\_utils.}\bfcode{get1LatLonFromNonRectiLinearGrid}}{\emph{grid}, \emph{lat}, \emph{lon}, \emph{diff=0.5}}{}~\begin{description}
\item[{\emph{func}}] \leavevmode{[}get1LatLonFromNonRectiLinearGrid{]}
It is locating the input lat \& lon in the non-rectilinear grid
data and returning its corresponding first dimension index (i) \&
second dimension index (j) (of the grid which is very close to
the input lat \& lon values).

\item[{Inputs :}] \leavevmode\begin{description}
\item[{grid}] \leavevmode{[}Its the cdms2 dataset grid value. Use x.getGrid() to pass{]}
the dataset grid value, where x is cdms2 dataslab.

\end{description}

lat : latitude which you looking for.
lon : longitude which you looking for.
diff : By default it takes 0.5. It is the purpose of masking the
\begin{quote}

outer region other than (lat-diff, lat+diff) and
(lon-diff, lon+diff).
\end{quote}

\item[{Logic :}] \leavevmode
Here we are getting the lat\_vertices and lon\_vertices data as well as
lat\_slab and lon\_slab from the grid.  i.e. Using grid.getBounds(),
grid.getLatitude() \& grid.getLongitude() functions.

Do the mask operation on the lat\_vertices where ever outer than the
(lat-diff, lat+diff).

Do the mask operation on the lon\_vertices where ever outer than the
(lon-diff, lon+diff).

Multiply the resultant masked boolean array of lat \& lon gives us the
near about 10 grids location which are all with in the range of
(lat-diff, lat+diff) and (lon-diff, lon+diff) both matched together.

So from this 10 grids locations, using distance b/w two points in the
curved line equation, we can identify the minimum distance from the
input lat \& lon.

Finally we can loate the minimum distance grid cell's first dimension
index (i) and its second dimension index (j).

Here index i belongs to longitude and index j belongs to latitude.

\item[{Return :}] \leavevmode
Return the first dimension index (i) \& second dimension index (j)
value where we located the nearest grid cell of the input lat, lon
passed by the user.

\item[{Reference :}] \leavevmode
function `getArcDistance()'

\end{description}

Example :
\begin{quote}
\begin{description}
\item[{eg 1:}] \leavevmode
\begin{Verbatim}[commandchars=\\\{\}]
\PYG{g+gp}{\PYGZgt{}\PYGZgt{}\PYGZgt{} }\PYG{n}{f} \PYG{o}{=} \PYG{n}{cdms2}\PYG{o}{.}\PYG{n}{open}\PYG{p}{(}\PYG{l+s}{\PYGZdq{}}\PYG{l+s}{zos\PYGZus{}Omon\PYGZus{}ACCESS1\PYGZhy{}0\PYGZus{}rcp45\PYGZus{}r1i1p1.xml}\PYG{l+s}{\PYGZdq{}}\PYG{p}{)}
\PYG{g+gp}{\PYGZgt{}\PYGZgt{}\PYGZgt{} }\PYG{n}{x} \PYG{o}{=} \PYG{n}{f}\PYG{p}{(}\PYG{l+s}{\PYGZsq{}}\PYG{l+s}{zos}\PYG{l+s}{\PYGZsq{}}\PYG{p}{,} \PYG{n}{time}\PYG{o}{=}\PYG{n+nb}{slice}\PYG{p}{(}\PYG{l+m+mi}{1}\PYG{p}{)}\PYG{p}{,} \PYG{n}{squeeze}\PYG{o}{=}\PYG{l+m+mi}{1}\PYG{p}{)}
\PYG{g+gp}{\PYGZgt{}\PYGZgt{}\PYGZgt{} }\PYG{n}{lat}\PYG{p}{,} \PYG{n}{lon} \PYG{o}{=} \PYG{l+m+mi}{10}\PYG{p}{,} \PYG{l+m+mi}{300}
\PYG{g+gp}{\PYGZgt{}\PYGZgt{}\PYGZgt{} }\PYG{n}{latidx}\PYG{p}{,} \PYG{n}{lonidx} \PYG{o}{=} \PYG{n}{get1LatLonFromNonRectiLinearGrid}\PYG{p}{(}\PYG{n}{x}\PYG{o}{.}\PYG{n}{getGrid}\PYG{p}{(}\PYG{p}{)}\PYG{p}{,} \PYG{n}{lat}\PYG{p}{,} \PYG{n}{lon}\PYG{p}{)}
\PYG{g+gp}{\PYGZgt{}\PYGZgt{}\PYGZgt{} }\PYG{n}{val} \PYG{o}{=} \PYG{n}{x}\PYG{p}{[}\PYG{n}{latidx}\PYG{p}{]}\PYG{p}{[}\PYG{n}{lonidx}\PYG{p}{]}
\end{Verbatim}
\begin{quote}
\begin{description}
\item[{..note:: Here val is the data value of that lat, lon. Mind that}] \leavevmode
index (i/lonidx) is first dimension and index (j/latidx)
is second dimension of the data.
Though to access the data here, we need pass latidx as
1st and lonidx as 2nd. Just work out this example, you
will understand.

\end{description}
\end{quote}

\item[{eg 2:}] \leavevmode
\begin{Verbatim}[commandchars=\\\{\}]
\PYG{g+gp}{\PYGZgt{}\PYGZgt{}\PYGZgt{} }\PYG{n}{latidx}\PYG{p}{,} \PYG{n}{lonidx} \PYG{o}{=} \PYG{n}{get1LatLonFromNonRectiLinearGrid}\PYG{p}{(}\PYG{n}{x}\PYG{o}{.}\PYG{n}{getGrid}\PYG{p}{(}\PYG{p}{)}\PYG{p}{,} \PYG{n}{lat}\PYG{p}{,} \PYG{n}{lon}\PYG{p}{)}
\PYG{g+gp}{\PYGZgt{}\PYGZgt{}\PYGZgt{} }\PYG{c}{\PYGZsh{} extract the time series data points of 10N, 60S position alone}
\PYG{g+gp}{\PYGZgt{}\PYGZgt{}\PYGZgt{} }\PYG{n}{val} \PYG{o}{=} \PYG{n}{f}\PYG{p}{(}\PYG{n}{var}\PYG{p}{,} \PYG{n}{i}\PYG{o}{=}\PYG{p}{(}\PYG{n}{lonidx}\PYG{p}{)}\PYG{p}{,} \PYG{n}{j}\PYG{o}{=}\PYG{p}{(}\PYG{n}{latidx}\PYG{p}{)}\PYG{p}{)}
\PYG{g+gp}{\PYGZgt{}\PYGZgt{}\PYGZgt{} }\PYG{n}{val}\PYG{o}{.}\PYG{n}{shape}
\PYG{g+go}{(365,)}
\end{Verbatim}
\begin{quote}

..note:: Mind that i belongs to longitude \& j belongs to latitude.
\end{quote}

\item[{eg 3:}] \leavevmode
\# efficient manner.
\textgreater{}\textgreater{}\textgreater{} f = cdms.open(``zos\_Omon\_ACCESS1-0\_rcp45\_r1i1p1.xml'')
\# getting variable access alone, not the whole data.
\textgreater{}\textgreater{}\textgreater{} x = f{[}'zos'{]}
\textgreater{}\textgreater{}\textgreater{} latidx, lonidx = get1LatLonFromNonRectiLinearGrid(x.getGrid(), lat, lon)
\begin{quote}
\begin{description}
\item[{..note:: Since we can not directly use latitude, longitude values}] \leavevmode
in the non-rectilinear grid data, we are using 1st dimension
(j) for latitude and 2nd dimension (i) for longitude as
corresponding indecies to get the data.

\end{description}
\end{quote}

\end{description}
\end{quote}

Written By : Arulalan.T

Date : 19.09.2012

\end{fulllineitems}

\index{getArcDistance() (in module nonrect\_utils)}

\begin{fulllineitems}
\phantomsection\label{others:nonrect_utils.getArcDistance}\pysiglinewithargsret{\code{nonrect\_utils.}\bfcode{getArcDistance}}{\emph{x1}, \emph{y1}, \emph{x2}, \emph{y2}, \emph{radius=6371}}{}
\emph{func} : get the Arc distance
\begin{quote}

Using equation of distance between two points on arc line.

We can use this function to find out the distances between
some list of latitudes \& longitudes positions to some other
single/list of lat, lon position of the earth. For this we need
to pass the radius of the earth.
\end{quote}
\begin{description}
\item[{Inputs :}] \leavevmode
x1, x2 - single latitude / list of latitudes
y1, y2 - single longitude/ list of longitudes

But x1 \& y1 should be same shape. Also x2 \& y2 should be same shape.
\begin{description}
\item[{radius - radius of the circle. By default it takes the earth's radius}] \leavevmode
in kilometer.

\end{description}

\end{description}

Written By : Arulalan.T

Date : 23.09.2012

\end{fulllineitems}



\section{Weather Utils}
\label{others:weather-utils}\label{others:module-weatherutils}\index{weatherutils (module)}
Author : Arulalan.T

Date: 17.05.2012
\index{getHighs() (in module weatherutils)}

\begin{fulllineitems}
\phantomsection\label{others:weatherutils.getHighs}\pysiglinewithargsret{\code{weatherutils.}\bfcode{getHighs}}{\emph{data}, \emph{value=1015}}{}~\begin{description}
\item[{getHighs}] \leavevmode{[}get the centred highs values along with its latitude, longitude  {]}
of the passed data. Centred highs means the surrounded lat, lon 
values are should be less than the centre high value.

\item[{arguments:}] \leavevmode
data : cdms2 variable with latitude, longitude axis information
value : the centred highs are greate than or equal to this value.
\begin{quote}

default val is 1015.
\end{quote}

\item[{return}] \leavevmode{[}returning the list containing tuples which are containing lat, {]}\begin{quote}

lon and centred high values. If there is no centred highs
less or equal to the passed value arg, then return an empty list.
\end{quote}
\begin{description}
\item[{eg: {[}(35.5, 85.5, 1027.11), (37.5, 73.5, 1024.29),}] \leavevmode
(31.0, 83.0, 1019.28), (40.0, 91.5, 1015.39){]}

\end{description}

\end{description}

Author : Arulalan.T

Date : 17.05.2012

\end{fulllineitems}

\index{getLows() (in module weatherutils)}

\begin{fulllineitems}
\phantomsection\label{others:weatherutils.getLows}\pysiglinewithargsret{\code{weatherutils.}\bfcode{getLows}}{\emph{data}, \emph{value=1000}}{}~\begin{description}
\item[{getLows}] \leavevmode{[}get the centred lows values along with its latitude, longitude  {]}
of the passed data. Centred lows means the surrounded lat, lon 
values are should be higher than the centre low value.

\item[{arguments:}] \leavevmode
data : cdms2 variable with latitude, longitude axis information
value : the centred lows are less than or equal to this value.
\begin{quote}

default val is 1000.
\end{quote}

\item[{return}] \leavevmode{[}returning the list containing tuples which are containing lat, {]}\begin{quote}

lon and centred low values. If there is no centred lows less or
equal to the passed value arg, then return an empty list.
\end{quote}
\begin{description}
\item[{eg: {[}(30.0, 74.5, 993.28003), (27.0, 80.0, 994.07001), }] \leavevmode
(21.0, 90.0, 998.26001), (34.0, 100.0, 999.53998), 
(25.0, 95.5, 1000.3){]}

\end{description}

\end{description}

Author : Arulalan.T

Date : 17.05.2012

\end{fulllineitems}

\index{is\_surrounding\_greater() (in module weatherutils)}

\begin{fulllineitems}
\phantomsection\label{others:weatherutils.is_surrounding_greater}\pysiglinewithargsret{\code{weatherutils.}\bfcode{is\_surrounding\_greater}}{\emph{data}, \emph{lat}, \emph{lon}}{}~\begin{description}
\item[{is\_surrounding\_greater}] \leavevmode{[}Find either the surrounded lat, lon position {]}
values are greater to the centre value on passed lat, lon of the data.

i.e. check the surrounded lat, lon position values against to the 
passed lat, lon position value. If all the surrounded position values
are greater than the center value (passed lat, lon position value)
then return this center value.

In any case while checking the surrounded lat, lon position value is 
lower than the center value, then immediately come out the function
by returning False.

If any case while checking the surrounded lat, lon position value is
equal to the center value, then again check that point's (position's)
surrounded lat, lon position value and do comparison. It has some 
default limit.

\end{description}

Author : Arulalan.T

Date : 17.05.2012

\end{fulllineitems}

\index{is\_surrounding\_less() (in module weatherutils)}

\begin{fulllineitems}
\phantomsection\label{others:weatherutils.is_surrounding_less}\pysiglinewithargsret{\code{weatherutils.}\bfcode{is\_surrounding\_less}}{\emph{data}, \emph{lat}, \emph{lon}}{}~\begin{description}
\item[{is\_surrounding\_less}] \leavevmode{[}Find either the surrounded lat, lon position {]}
values are lesser to the centre value on passed lat, lon of the data.

i.e. check the surrounded lat, lon position values against to the 
passed lat, lon position value. If all the surrounded position values
are lesser than the center value (passed lat, lon position value)
then return this center value.

In any case while checking the surrounded lat, lon position value is 
higher than the center value, then immediately come out the function
by returning False.

If any case while checking the surrounded lat, lon position value is
equal to the center value, then again check that point's (position's)
surrounded lat, lon position value and do comparison. It has some 
default limit.

\end{description}

Author : Arulalan.T

Date : 17.05.2012

\end{fulllineitems}



\chapter{Indices and tables}
\label{index:indices-and-tables}\begin{itemize}
\item {} 
\emph{genindex}

\item {} 
\emph{modindex}

\item {} 
\emph{search}

\end{itemize}


\renewcommand{\indexname}{Python Module Index}
\begin{theindex}
\def\bigletter#1{{\Large\sffamily#1}\nopagebreak\vspace{1mm}}
\bigletter{b}
\item {\texttt{binary2ascii}}, \pageref{others:module-binary2ascii}
\indexspace
\bigletter{c}
\item {\texttt{ceof\_diag}}, \pageref{mjo:module-ceof_diag}
\item {\texttt{climatology\_utils}}, \pageref{diagnosis:module-climatology_utils}
\item {\texttt{collect\_season\_fcst\_rainfall}}, \pageref{diagnosis:module-collect_season_fcst_rainfall}
\item {\texttt{collect\_season\_fcst\_rainfall.py}}, \pageref{diagnosis:module-collect_season_fcst_rainfall.py}
\item {\texttt{compute\_month\_anomaly}}, \pageref{diagnosis:module-compute_month_anomaly}
\item {\texttt{compute\_month\_anomaly.py}}, \pageref{diagnosis:module-compute_month_anomaly.py}
\item {\texttt{compute\_month\_fcst\_sys\_error}}, \pageref{diagnosis:module-compute_month_fcst_sys_error}
\item {\texttt{compute\_month\_fcst\_sys\_error.py}}, \pageref{diagnosis:module-compute_month_fcst_sys_error.py}
\item {\texttt{compute\_month\_mean}}, \pageref{diagnosis:module-compute_month_mean}
\item {\texttt{compute\_month\_mean.py}}, \pageref{diagnosis:module-compute_month_mean.py}
\item {\texttt{compute\_region\_statistical\_score}}, \pageref{diagnosis:module-compute_region_statistical_score}
\item {\texttt{compute\_region\_statistical\_score.py}}, \pageref{diagnosis:module-compute_region_statistical_score.py}
\item {\texttt{compute\_season\_mean}}, \pageref{diagnosis:module-compute_season_mean}
\item {\texttt{compute\_season\_mean.py}}, \pageref{diagnosis:module-compute_season_mean.py}
\item {\texttt{compute\_season\_stati\_score\_spatial\_distribution}}, \pageref{diagnosis:module-compute_season_stati_score_spatial_distribution}
\item {\texttt{compute\_season\_stati\_score\_spatial\_distribution.py}}, \pageref{diagnosis:module-compute_season_stati_score_spatial_distribution.py}
\item {\texttt{ctgfunction}}, \pageref{diagnosis:module-ctgfunction}
\indexspace
\bigletter{e}
\item {\texttt{eof\_diag}}, \pageref{mjo:module-eof_diag}
\indexspace
\bigletter{g}
\item {\texttt{generate\_iso\_plots}}, \pageref{diagnosis:module-generate_iso_plots}
\item {\texttt{generate\_stati\_score\_spatial\_distribution\_plots}}, \pageref{diagnosis:module-generate_stati_score_spatial_distribution_plots}
\item {\texttt{generate\_statistical\_score\_bars}}, \pageref{diagnosis:module-generate_statistical_score_bars}
\item {\texttt{generate\_statistical\_score\_bars.py}}, \pageref{diagnosis:module-generate_statistical_score_bars.py}
\item {\texttt{generate\_winds\_plots}}, \pageref{diagnosis:module-generate_winds_plots}
\indexspace
\bigletter{h}
\item {\texttt{harmonic\_util}}, \pageref{miso:module-harmonic_util}
\indexspace
\bigletter{n}
\item {\texttt{nonrect\_utils}}, \pageref{others:module-nonrect_utils}
\item {\texttt{numutils}}, \pageref{others:module-numutils}
\item {\texttt{numutis.py}}, \pageref{others:module-numutis.py}
\indexspace
\bigletter{p}
\item {\texttt{phase3d}}, \pageref{mjo:module-phase3d}
\item {\texttt{plot}}, \pageref{diagnosisutils:module-plot}
\item {\texttt{psutils}}, \pageref{mjo:module-psutils}
\indexspace
\bigletter{t}
\item {\texttt{trig}}, \pageref{mjo:module-trig}
\indexspace
\bigletter{v}
\item {\texttt{variance\_utils}}, \pageref{mjo:module-variance_utils}
\indexspace
\bigletter{w}
\item {\texttt{weatherutils}}, \pageref{others:module-weatherutils}
\item {\texttt{wk\_utils}}, \pageref{mjo:module-wk_utils}
\indexspace
\bigletter{x}
\item {\texttt{xml\_data\_access}}, \pageref{diagnosisutils:module-xml_data_access}
\end{theindex}

\renewcommand{\indexname}{Index}
\printindex
\end{document}
